The second version of the FWI objective responds to the observation
that seismic sources are themselves subject to much uncertainty, and
therefore could reasonably be included in the unknowns of the inverse
problem \cite[]{SymMink:97,VirieuxOperto:09}. Thus: chose $m$ {\em
  and} $f$ to minimize
\begin{equation}
  \label{eqn:FWIvar}
  J_{\rm var}[m,f] = \frac{1}{2}\|S[m]f-\dl\|^2.
\end{equation}
It might be thought that the additional degrees of freedom represented
by the unknown source function in \ref{eqn:FWIvar} would ``relax'' the
minimization and open up more paths to the global
solution. Paradoxically, a good way to use these extra degrees of
freedom is to get rid of them: define
\begin{equation}
  \label{eqn:FWIvarvpm}
  J_{\rm var}^{\rm vpm}[m] = \min_f \frac{1}{2}\|S[m]f-\dl\|^2
\end{equation}
Replacement of $J_{\rm var}$ by $J_{\rm var}^{\rm vpm}$ is the
so-called Variable Projection Method \cite[]{GolubPereyra:03}, applied to the problem at
hand. Note that local, respectively global minimizers $m$ of $J_{\rm
  var}^{\rm vpm}$ are necessarily the first components of local,
respectively global, minimizers of $J_{\rm var}$, therefore
minimization of $J_{\rm var}$ is logically equivalent to minimization
of $J_{\rm var}^{\rm vpm}$. A number of authors have specifically
suggested that this approach be applied eliminate ``nuisance''
parameters such as seismic source models that do not represent the
main information content to be extracted from the data
\cite[]{MulderVanLeeuwen:09,Rickett:SEG12,Aravkin:12,LiRickettAbubakar:13}.

For the present problem, it is easy to explicitly compute $ J_{\rm
  var}^{\rm vpm}$. If $f[m]$ denotes the minimizer of $J_{\rm
  var}[m,f]$ over $f$, then
\[
  J_{\rm  var}^{\rm vpm}[m]=J_{\var}[m,f[m]].
\]
Since $J_{\var}[m,f]$ is quadratic in $f$, $f[m]$ is a solution of the
normal equation
\begin{equation}
  \label{eqn:defnormal}
  S[m]^TS[m] f[m] = S[m]^T\dl
\end{equation}
It follows immediately from the definition \ref{eqn:mod} that
\begin{equation}
\label{eqn:tran}
S[m]^T g (t) = \frac{1}{4\pi r}g\left(t+mr\right)
\end{equation}
so
\begin{equation}
\label{eqn:unit}
S[m]^TS[m] f (t) = \frac{1}{(4\pi r)^2} f(t).
\end{equation}
From the definition \ref{eqn:defdata} of the data and the identity \ref{eqn:tran},
\[
  S[m]^T\dl = \frac{1}{(4\pi r)^2}\fl(t-(m_*-m)r)
\]
so from the identity \ref{eqn:unit} and the normal equatioh
\ref{eqn:defnormal} obtain
\[
  f[m] = \fl(t-(m_*-m)r).
\]
Therefore $S[m]f[m] = \fl(t-m_*r)