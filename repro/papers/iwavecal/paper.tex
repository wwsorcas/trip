\title{Calibrating IWAVE}
\date{}
\author{William W. Symes}

\inputdir{project}

\lefthead{Symes}
\righthead{Calibrating IWAVE}

\maketitle
\parskip 12pt
\begin{abstract}
IWAVE simulation of wavefields involves many individual computations combined into a single command. While each computation can be tested for accuracy on its own, it is also important to ensure that the final result accurately approximates the solution of the target continuum problem by direct comparison with solutions of known accuracy. This paper presents comparison of acoustic simulations on staggered and non-staggered grids with numerical implementations of analytic solutions in homogeneous media. The comparisons suggest that IWAVE correctly solves these problems with reasonable accuracy.
\end{abstract}

\section{Introduction}
This note concerns verification of the IWAVE family of finite
difference wave solvers \cite[]{GeoPros:11}. In the numerics QC vernacular, verification
means ``show that the equations are solved correctly'', as opposed to
validation, ``show that the correct equations are solved''. In the
context of discrete methods for partial differential equations,
``correctly'' means: with an adequate rate of convergence as grids or
meshes are refined.

Several types of verification tests exist, none of
them entirely satisfactory (see for example \cite{FehlerKeliher:11},
Chapter 3). The approach used here is comparison with
analytic (or nearly analytic) closed-form solutions. Such solutions
exist only in special cases, for example, wave equations with constant
coefficients, the case considererd here. While verification in such
special cases does not necessarily predict the errors encountered
under fewer constraints (for example, heterogenous media), it does
indicate whether overall scaling of the numerical solution is correct.

The example reported below, and similar examples not explicitly shown,
verify the correct solution of homogeneous medium acoustics problems
with isotropic point sources using IWAVE finite difference
simulators. This task requires first that the
analytic solution for a homogeneous medium be derived. While such
solutions are found in many publications, most instances take
advantage of restrictions on the problem parameters to simplify the
results. Since the point of this exercise is to verify correct
dependence on parameters, I begin this note with a complete derivation
of the homogeneous medium point source solution (the ``radiation
problem'' of \cite{CourHil:62}, Chapter 6), with dependence on
parameters explicitly described. The solution for the 3D version of
this problem is actually analytic. Since 2D finite difference
simulation is much less expensive than 3D, that case is also
important, but the result involves quadrature, so is not really
analytic. I pose a 2D homogeneous medium point source problem, and
solve it using IWAVE's staggered grid finite difference simulator, and
a trapezoidal rule realization of the 2D quasi-analytic
solution. Comparison shows convincingly that the finite difference
simulator is functioning correctly for this class of problems.

\section{Theory}
Linear acoustics describes small amplitude motions of a compressible fluid in terms of the evolution of (excess) pressure $p(\bx,t)$ and particle velocity $\bv(\bx,t)$: 
\begin{eqnarray}
\label{eqn:aws}
\frac{\partial p}{\partial t}(\bx,t) & = &\kappa(\bx) (-\nabla \cdot \bv(\bx,t) + f(\bx,t))\nonumber\\
\frac{\partial \bv}{\partial t}(\bx,t) & = & -\beta(\bx) \nabla p(\bx,t)\nonumber\\
p, \bv &=& 0,\, t << 0.
\end{eqnarray}
(\cite{gur81}, section 19, or \cite{Frie:58}, Chapter 1). The linear acoustic constitutive law makes relative change of volume and pressure proportional by the the bulk modulus $\kappa$, which therefore has units of pressure = force/area. The first of the equations in \ref{eqn:aws} therefore represents input of energy into the system via a defect in the acoustic consitutive relation: the forcing (or source) term $f$ has units of 1/time, that is, relative rate of change. The second equation in \ref{eqn:aws} expresses Newton's law relating acceleration and force for the acoustic model in which stress is scalar and its divergence (by definition) equal to the negative pressure gradient. The buoyancy $\beta$ is the reciprocal of density, $\beta=1/\rho$. The final condition in \ref{eqn:aws} specifies causal solutions.

The system \ref{eqn:aws} is equivalent to a single second order PDE for pressure: differentiate the first equation with respect to time, then use the second equation to eliminate $\bv$ and obtain
\begin{eqnarray}
\label{eqn:awe}
\frac{\partial^2 p}{\partial t^2}(\bx,t) & = &\kappa(\bx)\left(\nabla \cdot \beta(x) \nabla p(\bx,t) + \frac{\partial f}{\partial t}(\bx,t)\right) \nonumber\\
p & = & 0, \,t<<0
\end{eqnarray}

The systems \ref{eqn:aws} and \ref{eqn:awe} describe linear acoustodynamics in any dimension, in particular in 2D or 3D, and far from physical boundaries, or in Euclidean space without boundaries. Appropriate boundary conditions must be supplied to complete the description of acoustics in  domains with boundaries. 

In this section I will develop analytic expressions for the pressure field $p$ in \ref{eqn:aws} or \ref{eqn:awe} for 2D and 3D, assuming that $\kappa$ and $\beta$ are constant and $f$ has the form of an {\em isotropic point radiator} located at the origin:
\begin{equation}
\label{eqn:ipr}
f(\bx,t) = w(t) \delta(\bx).
\end{equation}
The source pulse, or wavelet, $w$ is smooth and of compact support.
Since $f$ has units of rate (1/time), $w$ must have units of volume rate.

Note that in this case \ref{eqn:awe} simplifies to 
\begin{eqnarray}
\label{eqn:aweccipr}
\frac{1}{c^2}\frac{\partial^2 p}{\partial t^2}(\bx,t) & = &\nabla^2 p(\bx,t) + \rho\frac{dw}{dt}(t)\delta(\bx) \nonumber\\
p & = & 0, \,t<<0 
\end{eqnarray}
using the wave velocity $c=\sqrt{\kappa\beta}$. Note that the forcing term has units of density rate, or force/distance. The same for would be obtained starting with a version of \ref{eqn:aws} with a body force, that is, inhomogenous term in Newton's law, in which case the source would be the divergence of a force.

\subsection{3D}
The solution of the point radiator problem is well-known, see for example \cite{CourHil:62}, Chapter 6 or \cite{GuiSte:79}, Chapter 1. I repeat it here for completeness. Since the isotropic point radiator \ref{eqn:ipr} is rotationally symmetric, it is natural to expect the same of the pressure field, i.e. that $p(\bx,t) = P(r,t)$ with $r=|\bx|$. The simple identity
\begin{equation}
\label{eqn:calc}
\nabla^2 \frac{F(r)}{r} = \left(\frac{\partial^2}{\partial r^2} + \frac{2}{r}\frac{\partial}{\partial r}\right) \frac{F(r)}{r} = \frac{1}{r}\frac{\partial^2 F}{\partial r}^2(r)
\end{equation}
shows that $rP(r,t)$ obeys the homogeneous 1D wave equation for $r>0$. To achieve causality, necessarily $rP(r,t)$ is a function of $t-r/c$. Set
\[
P(r,t) = \frac{g(t-r/c)}{r}
\]
and assume $g$ to be smooth and of compact support. Then $P$ has an integrable singularity at $r=0$, and is a distribution solution of \ref{eqn:aweccipr} exactly if
\[
g(t) = \frac{\rho\frac{dw}{dt}(t)}{4\pi}
\]
as one sees by writing the equation in weak form, expressing the integration against at test function image under the wave operator as the limit of integrals over the exterior of the ball of radius $\epsilon >0$, integrating by parts, and taking the limit $\epsilon \rightarrow 0$. Therefore we obtain for $p=p_{3D}$ the analytic expression
\begin{equation}
\label{eqn:awsol3D}
p_{3D}(\bx,t) = \frac{\rho\frac{dw}{dt}(t-r/c)}{4\pi r}
\end{equation}

As stated in the introduction, the IWAVE discretizations of systems \ref{eqn:aws} and \ref{eqn:awe} normalize the source as a black-box right-hand side. The IWAVE acoustic constant density (ACD) solver (code in {\tt iwave/acd}) approximates the solution of 
\begin{eqnarray}
\label{eqn:acd}
\frac{\partial^2 p_{ACD}}{\partial t^2}(\bx,t) & = &c^2\nabla^2 p_{ACD}(\bx,t) + f_{ACD}(\bx,t)\nonumber\\
p_{ACD} & = & 0, \,t<<0 
\end{eqnarray}
The point source version of this problem has
\begin{equation}
\label{eqn:acdipr}
f_{ACD} = w_{ACD}(t)\delta(\bx)
\end{equation}
Comparing equations \ref{eqn:aweccipr}, \ref{eqn:acd} and \ref{eqn:acd}, one sees that $p_{ACD} = p$ if 
\[
w_{ACD}(t) = \rho c^2 \frac{dw}{dt}(t) 
\]
so for 3D one obtains the solution of \ref{eqn:acd}, \ref{eqn:acdipr}:
\begin{equation}
\label{eqn:acdsol3D}
p_{ACD3D}(\bx,t) = \frac{ w_{ACD}(t-r/c)}{4\pi c^2 r}.
\end{equation}

The IWAVE acoustic staggered grid (ASG) solver (code in {\tt iwave/asg}) approximates the solution of
\begin{eqnarray}
\label{eqn:asg}
\frac{\partial p_{ASG}}{\partial t}(\bx,t) & = &-\kappa(\bx) \nabla \cdot \bv_{ASG}(\bx,t) + f_{ASG}(\bx,t)\nonumber\\
\frac{\partial \bv_{ASG}}{\partial t}(\bx,t) & = & -\beta(\bx) \nabla p_{ASG}(\bx,t)\nonumber\\
p_{ASG}, \bv_{ASG} &=& 0,\, t << 0.
\end{eqnarray}
This system differs from \ref{eqn:aws} only by having the right-hand side vector scaled by $1/\kappa$ - that is, $p_{ASG} = p$ if $f_{ASG} = \kappa f$. Thus for the 3D point source problem, that is, the solution of \ref{eqn:asg} with 
\[
f_{ASG}(\bx,t) = w_{ASG}(t) \delta(\bx),
\]
the source pulse $w_{ASG}=\kappa w$ has units of pressure $\cdot$ volume / time, and equation \ref{eqn:awsol3D} implies
\[
p_{ASG3D}(\bx,t) = \frac{\rho\frac{dw_{ASG}}{dt}(t-r/c)}{4\pi \kappa r}
\]
\begin{equation}
\label{eqn:asgsol3d}
= \frac{\frac{dw_{ASG}}{dt}(t-r/c)}{4\pi c^2 r}.
\end{equation}

An interesting consequence of that for any space dimension, isotropic point source wavelets used in \ref{eqn:asg} and \ref{eqn:acd} will produce the same pressure field if 
\begin{equation}
\label{eqn:acdasg}
w_{ACD} = \frac{\partial w_{ASG}}{\partial t}.
\end{equation}
Equation \ref{eqn:acdasg} is consistent with $w_{ASG}$ having units of force $\cdot$ velocity, and $w_{ACD}$ force $\cdot$ acceleration, as follows from equations \ref{eqn:asg} and \ref{eqn:acd} respectively.

\subsection{2D}
I will regard 2D solutions of equations \ref{eqn:aws} and
\ref{eqn:awe} as line source solutions of the 3D equations. In this
subsection, I use the notation $\bx=(x,z)$ for the 2D position vector
in the ``inline'' plane perpindicular to the $y$ axis, and $\bx_{3D} =
(x,y,z)$. The isotropic line radiator appropriate for the ASG version
of the acoustic system
\ref{eqn:asg} is 
\begin{equation}
\label{eqn:ilr}
f(\bx_{3D},t) = w_{ASG2D}(t)\delta(\bx).
\end{equation}
For constant $\kappa$ and $\beta$, the weak solution $(p_{ASG2D}, \bv_{ASG2D})$ of system \ref{eqn:asg} is also independent of $y$. Since $\delta(\bx)$ has units of 1/area, $w_{ASG2D}$ must have units of pressure $\cdot$ area / time. 
 
To find an analytic form for the pressure field of an isotropic line radiator, use D'Alembert's method of descent (\cite{CourHil:62}, Ch. 6). This approach begins with the observation that
\[
\delta(\bx) = \int dy \delta(\bx_{3D})
\]
(with the right-hand side interpreted as a pull-back). Therefore the solution of system \ref{eqn:asg} with $f$ given by equation \ref{eqn:ilr} is  
\begin{equation}
\label{eqn:descent}
p_{ASG2D}(\bx,t) = \int dy \tilde{p}_{ASG3D}(\bx_{3D},t).
\end{equation}
where $\tilde{p}_{3D}$ is the solution of the isotropic point radiator problem given by equation \ref{eqn:asgsol3d}, but with $w_{ASG}$ replaced by $w_{ASG2D}$, so that $\tilde{p}_{ASG3D}$ has units of pressure/length and $p_{ASG2D}$ is a pressure.

From equation \ref{eqn:asgsol3d},
\[
p_{ASG2D}(\bx,t) = \int dy \tilde{p}_{ASG3D}(\bx_{3D},t).
\]
\[
= \int dy \frac{\frac{dw_{ASG2D}}{dt}(t-r/c)}{4\pi c^2 r}
\]
\[
= \int_{-\infty}^{\infty} dy \frac{\frac{dw_{ASG2D}}{dt}(t-\sqrt{|\bx|^2 + y^2}/c)}{4\pi c^2 \sqrt{|\bx|^2 + y^2}}
\]
\[
= 2 \int_0^{\infty} dy \frac{\frac{dw_{ASG2D}}{dt}(t-\sqrt{|\bx|^2 + y^2}/c)}{4\pi c^2 \sqrt{|\bx|^2 + y^2}}
\]
\begin{equation}
\label{eqn:green2dpart1}
= 2 \lim_{\epsilon \rightarrow 0}\int_{\epsilon}^{\infty} dy \frac{\frac{dw_{ASG2D}}{dt}(t-\sqrt{|\bx|^2 + y^2}/c)}{4\pi c^2\sqrt{|\bx|^2 + y^2}}
\end{equation}
since the integrand is continuous, of compact support, and even in $y$. Set $\tau = \sqrt{|\bx|^2 + y^2}/c$. Then $y \rightarrow \tau$ is a nonsingular diffeomorphism $[\epsilon,\infty) \rightarrow [\sqrt{|\bx|^2+\epsilon^2}/c,\infty)$. Changing variables in the \ref{eqn:green2dpart1} gives
\[
= 2 \lim_{\epsilon \rightarrow 0} \int_{\sqrt{|\bx|^2+\epsilon^2}/c}^{\infty} \frac{c^2 \tau d\tau}{\sqrt{c^2 \tau - |\bx|^2}} \frac{\frac{dw_{ASG2D}}{dt}(t-\tau)}{4\pi c^3 \tau} 
\]
\[
=\lim_{\epsilon \rightarrow 0} \frac{1}{2\pi c}\int_{\sqrt{|\bx|^2+\epsilon^2}/c}^{\infty} d\tau \frac{\frac{dw_{ASG2D}}{dt}(t-\tau)}{\sqrt{c^2 \tau^2 -|\bx|^2}}
\]
so taking the limit $\epsilon \rightarrow 0$ and introducing the
Heavyside function to express the integral as a convolution, obtain
\[
p_{ASG2D}(\bx,t) = \frac{1}{2\pi c}\int d\tau
\frac{dw_{ASG2D}}{dt}(t-\tau)\frac{H(c\tau-|\bx|)}{\sqrt{c^2 \tau^2
    -|\bx|^2}}
\]
\begin{equation}
\label{eqn:asgsol2d}
= \frac{d}{dt}(w * g(\cdot,r))
\end{equation}
where
\begin{equation}
  \label{eqn:green2dsing}
  g(t,r) = \frac{1}{2\pi c^2}\frac{H(t-r/c)}{\sqrt{t^2 -r^2/c^2}}
\end{equation}

The convolution kernel \ref{eqn:green2dsing} is singular, so simple
regular grid 
discretizations
of convolution (for instance trapezoidal rule) have sub-optimal accuracy. 
To produce an equivalent expression involving less singular
integrands, note that
\[
  g(t,r) = \frac{1}{2\pi c^2}
  \left(\frac{H(t-r/c)}{\sqrt{t+r/c}}2\frac{d}{dt}\sqrt(t-r/c)\right)
\]
\[
  =\frac{1}{\pi c^2}\left(\frac{d}{dt} H(t-r/c)
    \sqrt{\frac{t-r/c}{t+r/c}}
 - H(t-r/c)\sqrt{t-r/c}
    \frac{d}{dt}\frac{H(t-r/c)}{\sqrt{t+r/c}}\right)
\]
Since $H(t-r/c)\sqrt{t-r/c}$ is continuous and vanishes at $t=r/c$, 
\[
  = \frac{d}{dt} \left(\frac{1}{\pi c^2}H(t-r/c)
    \frac{(t-r/c)^{1/2}}{(t+r/c)^{1/2}} \right)+
  \left(\frac{1}{2\pi c^2} H(t-r/c)
    \frac{(t-r/c)^{1/2}}{(t+r/c)^{3/2}} \right)
\]
Therefore, the relation \ref{eqn:asgsol2d} is equivalent to
\begin{equation}
  \label{eqn:asg2dsolbis}
  p_{ASG2D}(\bx,t) = \frac{d^2}{dt^2}(g_1(\cdot, r) *w)(t) + \frac{d}{dt}(g_2(\cdot,r)*w)(t),
\end{equation}
in which
\begin{eqnarray}
  \label{eqn:green2dt1}
  g_1(t,r) &=& \frac{1}{\pi c^2}H(t-r/c)
               \frac{(t-r/c)^{1/2}}{(t+r/c)^{1/2}},\\
  \label{eqn:green2dt2}
  g_2(t,r) &=& \frac{1}{2\pi c^2} H(t-r/c)
               \frac{(t-r/c)^{1/2}}{(t+r/c)^{3/2}}.
\end{eqnarray}

\section{Example}
In this section, I present a 2D example showing that pressure traces
computed via convolution with Green;'s functions, as developed in the
last section, closely approximate
those computed via staggered
grid finite difference approximation of the system \ref{eqn:asg}. I
present only the 2D case, to avoid computational expense. This example
verifies that calibration of the finite difference implementation.

Numerical evaluation of the right-hand side in equation
\ref{eqn:asg2dsolbis} requires a choice of quadrature rule. The
Green's function components $g_1$ and $g_2$ are continuous but have a
$t^{1/2}$ singularity at the wavefront $t=r/c$. Sampling on a uniform
grid effectively zeros $g_1$ and $g_2$ in the sample interval next
to the wavefront, resulting in a time shift of a fraction of the time
step. The trapezoidal rule
applied to such integrands is therefore only first-order accurate. A
specialized quadrature rule adapted to this singularity would yield
more accurate results. Alternatively, further
integration-by-parts manipulations like those that led to the
expression \ref{eqn:asg2dsolbis} would shift more derivatives onto the
wavelet, thus restoring second-order accuracy of the trapezoidal rule
approximate convolution. I will not explore these options, since the
main aim (calibration of the finite difference code) is achieved
without them.

The finite difference scheme used to approximate the solution of the
system \ref{eqn:asg}
is the staggered grid scheme
introduced by \cite{GauTarVir:86}, based on leapfrog centered
difference formulae. Any even formal order of accuracy in space and
time can be achieved by higher-order leapfrog formulae in space and
Lax-Wendroff refinement in time \cite[]{moczoetal:06}. The scheme used
here is 2nd order accurate in time, 4th order in space. I will show
computational results from three levels of grid refinement, by a
factor of 2 in both space and time, so the formal order of accuracy is
2. The use of a higher order of accuracy in the space derivatives
leads to less grid dispersion, that is, a more accurate result at each
level of grid refinement.

IWAVE source input assumes that right-hand sides in the wave equation
are represented by time traces, and that each time trace is the
coefficient of a spatial delta function. Spatial delta functions are
discretized by adjoint interpolation, mapping amplitudes at
(arbitrary) sample points
to nearby grid points, and scaling by the inverse of the spatial grid
cell volume. This adjoint sampling and scaling is built into
IWAVE. Only the time trace - in other words, the wavelet in the
context of the present problem - need be passed to the simulator, in
the form of a SEGY trace file.

In this example, $\kappa = $ 4 GPa and $\rho = 1$ g/cm$^3$, so that $c = 2$
m/ms. A point source is placed at at $x_s=$ 3500 m, $z_s=$ 3000 m with
respect to an arbitrarly chosen origin, and a receiver at $x_r=$ 3500
m, $z_r=$ 1000 m, so the source-receiver distance is $r=$
2061.552... Note that the the wavefront position will not lie on any
time grid point, which are integral multiples of a
millisecond, for the sampling used below.
The source wavelet is a trapezoidal bandpass filter
with corner frequencies at 1.0, 2.5, 7.5 and 12 Hz, depicted in Figure
\ref{fig:ptsrc2}.

\plot{ptsrc2}{width=0.8\textwidth}{Zero-phase bandpass filter  with
  corner frequencies 1, 2.5, 7.5, 12 Hz, delayed by 1 s.}

The computations use three time sample rates for source wavelet and
data traces, with $\Delta t = 0.008, 0.004, $ and
$0.002$ s. The finite difference computations use corresponding square spatial grids
with $\Delta x = \Delta z = $ 20, 10, and 5 m. The finite difference 
time step is computed internally, and is smaller than the time sample
rates. Cubic spline interpolation and adjoint interpolation
transfers data between internal and sample time grids.

Figures \ref{fig:plot0}, \ref{fig:plot1}, and \ref{fig:plot2} show
the finite difference and convolutional traces computed with these choices for time steps $\Delta t = $
0.008, 0.004, and 0.002 s respectively, along with the difference
traces. The first order error is visually obvious, but so is
convergence. The $L^2$ errors (relative to the finite difference
traces) are 29\%, 14.8\%, and 7.6\% respectively. Almost all of this
error is the effective time shift error in the trapezoidal rule evaluation of
the integrations in equation \ref{eqn:asg2dsolbis}: the relative $L^2$
error between the finite difference traces for coarsest and finest
grids is 2.7\%.

\plot{plot0}{width=0.8\textwidth}{Finite difference trace (blue)
  vs. convolution with Green's function (equation
  \ref{eqn:asg2dsolbis}) (red) and difference trace (black), coarsest grid
  ($\Delta t$ = 0.008 s).}

\plot{plot1}{width=0.8\textwidth}{Finite difference trace (blue)
  vs. convolution with Green's function (equation
  \ref{eqn:asg2dsolbis}) (red) and difference trace (black), medium grid
  ($\Delta t$ = 0.004 s).}

\plot{plot2}{width=0.8\textwidth}{Finite difference trace (blue)
  vs. convolution with Green's function (equation
  \ref{eqn:asg2dsolbis}) (red) and difference trace (black), finest grid
  ($\Delta t$ = 0.002 s).}

\section{Conclusion}
The calculations presented here accomplished two objectives. First,
the Green's functions constructed here are correctly scaled for linear
acoustics in homogeneous fluids with arbitrary bulk modulus and
density. While many references give expressions for these Green's
functions, most choose a particular reference scale (frequently
$\kappa = \rho = 1$) hence do not explicitly give results needed for
comparison to numerical solutions. Second, the problem is formulated
consistently with the IWAVE finite difference codes for both the
second-order wave equation and the first-order pressure-velocity
system. With these tools, verification of correct calibration for the
finite difference codes is straightforward. The example presented in
the last section verifies correct calibration of the staggered grid
scheme for the pressure-velocity system in 2D. Similar verifications
are possible for the other cases covered by the theory presented above.

\bibliographystyle{seg}
\bibliography{../../bib/masterref}


