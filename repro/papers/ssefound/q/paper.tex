\title{Computation of $Q$ for surface source extension}
\author{William W. Symes}

\begin{abstract}
  Abstract goes here.
\end{abstract}

\section{Cauchy problem}

Suppose that $K \in OPS^1$ is skew-adjoint, at least in principal
part: $K_1^*=-K_1$. Then there is unitary $U \in OPS^0$ so that
$U(x,\xi)^*K_1(x,\xi)U(x,\xi) = i\Lambda(x,\xi) =
i\mbox{diag}(\lambda_1(x,\xi),...\lambda_k(x,\xi))$, with scalar
$\lambda_j \in OPS^1, j=1,...,k$ (in fact, homogeneous of degree
1). Also,
\begin{equation}
  \label{eqn:ortho}
  U(x,D)^*U(x,D) = U(x,d)U(x,D)^* = I \mbox{ mod } OPS^{-1}
\end{equation}
Define $\cU(t)$ to be the one-parameter group satisfying
\[
  \partial_t \cU(t) = -K \cU(t), \cU(0)=I
\]
Note that $\cU(-t)$ is inverse to $\cU(t)$, and 
\[
  \partial_t \cU(-t) = -\cU(-t) \partial_t \cU(t) \cU(-t) = \cU(-t)K
\]
Hence
\[
  \partial_t \cU(t) = -iU \Lambda U^*\cU(t) + R\cU(t), R \in OPS^0 
\]
Write $\tilde{\cU} = U^*\cU U$. Then 
\[
  \partial_t \tilde{\cU}(t) = -i\Lambda \tilde{\cU}(t) + \mbox{ mod } OPS^0 
\]
If $u^0 = (u^0_1,...,u^0_k)$, then $\tilde{u}(t) = \tilde{\cU}(t)u^0$ satisfies
\[
  \partial_t \tilde{u}(t) = -i\Lambda \tilde{u}(t) + ...
\]
so $\tilde{u}_i$ solves
\[
  \partial_t \tilde{u}_i(t) = -i\lambda_i \tilde{u}_i(t), \, i=1,...,k
\]
Let $\tilde{\cU}_i$ solve
\begin{equation}
  \label{eqn:stilde}
  \partial_t \tilde{\cU}_i(t) = - i\lambda_i \tilde{\cU}_i(t)
\end{equation}
Then
\[
  \tilde{u}_i(t) = \tilde{\cU}_i(t) u^0_i = (\tilde{\cU}(t)u^0)_i 
\]
whence
\[
  \tilde{\cU} = \mbox{diag}(\tilde{\cU}_1,....,\tilde{\cU}_k).
\]
and
\begin{equation}
  \label{eqn:spec}
  \cU(t) = \sum_{i=1}^k U_i \tilde{\cU}_i(t)U_i^*
\end{equation}

Define
\[
  F = \phi_r \cU(T) \phi_s
\]
in which $\phi_{s,r} \in C_0^{\infty}(\bR^{n+1})$ and $T>0$. Suppose that $K$
is perturbed by $\delta K \in OPS^1$, also essentially skew. Then
$\Lambda$, $U$, $\cU$, $\tilde{\cU}$ and $F$ undergo perturbations
$\delta \Lambda$, $\delta U$, $\delta \cU$, $\delta \tilde{\cU}$, and
$\delta F$. Our object is to understand the structure of $\delta F$, under the
assumption that $\phi_{s,r} \delta K \in OPS^{-\infty}$, i.e. that
$\delta K$ essentially vanishes on the supports of $\phi_{s,r}$.

This assumption on $\delta K$ implies that $\phi_{s,r} \delta U_i \in
OPS^{-\infty}, i=1,...,k$. 
\[
  \delta F =  \phi_r \delta \cU(T) \phi_s  = \phi_r \left(\sum _{j=1}^k \delta U_j \tilde{\cU}_j(T) U_j^*
\right.
\]
\[
 +  \left. \sum_{j=1}^k U_j \delta\tilde{\cU}_j(T)U_j^* + \sum_{J=1}^k U_j
    \tilde{\cU}_j(T)\delta U_j^*\right) \phi_s
\]
Every term in the first and last sums on the RHS is smoothing, since
$\phi_r \delta U_i, \delta U_i^* \phi_s\in OPS^{-\infty}$:
\[
  \delta F = \phi_r \left(\sum_{j=1}^k U_j
    \delta\tilde{\cU}_j(T)U_j^*\right)\phi_s \mbox{ mod }OPS^{-\infty}
\]
Denote by ${\cal B}$ operators on $H^{\infty}$ that are bounded on
each $H^s$, all $s \in \bR$ (so $OPS^0 \subset {\cal B}$

Since $[\phi_{s,r},U_i] \in OPS^{-1}$, this is
\[
  = \sum_{j=1}^k U_j \phi_r \delta \tilde{\cU}_j(T) U_J^*\phi_s 
  \mbox{ mod }{\cal B}
\]

The first order perturbation of the definition \ref{eqn:stilde} is
\begin{equation}
  \label{eqn:dstilde}
  \partial_t \delta \tilde{\cU}_i(t) = - i\lambda_i \delta
  \tilde{\cU}_i(t) - i\delta \lambda_i \tilde{\cU}_i(t)
\end{equation}
Since $\tilde{\cU_i}(-t)$ is inverse to $\tilde{\cU}_i(t)$,
\[
  \partial_t \tilde{\cU}_i(-t)= -\tilde{\cU}_i(-t) (\partial_t \tilde{\cU}_i(t)) \tilde{\cU}_i(-t)
\]
\begin{equation}
  \label{eqn:stildeinv}
= -\tilde{\cU}_i(-t)(-i\lambda_i \tilde{\cU}_i(t)) \tilde{\cU}_i(-t) = i\tilde{\cU}_i(-t)\lambda_i 
\end{equation}
So
\begin{equation}
  \label{eqn:dstildeconj}
  \partial_t\tilde{\cU}_i(-t)\delta \tilde{\cU}_i(t) = -i\tilde{\cU}_i(-t)\delta \lambda_i \tilde{\cU}_i(t).
\end{equation}
From \cite{Tay:81}, Egorov's Theorem (Section VIII.1), for $p \in OPS^s(\bR^n)$,
\begin{equation}
  \label{eqn:egorov}
   \tilde{\cU}_i(-t) p \tilde{\cU}_i(t)=  p \circ C_i(t),\, p\tilde{\cU}_i(t) =
   \tilde{\cU}_i(t)p \circ C_i(t)
\end{equation}
in which $C_i$ is the one-parameter family of symplectic maps given by
\begin{equation}
  \label{eqn:canon}
  C_i(t)(x,\xi)=(y_i(t;x,\xi),\eta_i(t;x,\xi))
\end{equation}
Here $t \rightarrow y_i(t;x,\xi),\eta_i(t;x,\xi) $is the integral curve of the
Hamiltonian vector field
\[
  H_i(x,\xi) = (\partial_{\xi}\lambda_i(x,\xi), - \partial_x\lambda_i(x,\xi)) 
\]
with $y_i(0;x,\xi)=x, \eta_i(0;x,\xi)=\xi$.

For $p \in OPS^{\infty}$, define
\[
  \Gamma_i[p](T) = \int_0^T dt \, p \circ C_i(t)
\]
Then from the identity \ref{eqn:dstildeconj} follows
\[
  \delta\tilde{\cU}_i(T) = \tilde{\cU}_i(T) \Gamma_i[ -i \delta \lambda_i](T)
\] 
so
\[
  \delta F = \sum_{i=1}^k U_i \phi_r \delta \tilde{\cU}_i \phi_s U_i^*
  =\sum_{i=1}^k U_i \phi_r \tilde{\cU}_i(T) \Gamma_i[ -i \delta
  \lambda_i](T) U_i^*\phi_s
  \mbox{ mod }{\cal B}
\]
\[
  =\left(\sum_{i=1}^k U_i \phi_r \tilde{\cU}_i(T)U_i^*\right)
  \left(\sum_{j=1}^k U_j  \Gamma_j[ -i \delta \lambda_j](T) U_j^*\right)
  \phi_s   \mbox{ mod }{\cal B}
\]
\[
  =\phi_r\left(\sum_{i=1}^k U_i \tilde{\cU}_i(T)U_i^*\right)
  \left(\sum_{j=1}^k U_j  \Gamma_j[ -i \delta \lambda_j](T) U_j^*\right)
  \phi_s   \mbox{ mod }{\cal B}
\]
Define
\begin{equation}
  \label{eqn:qdef}
  \cQ(T) = \sum_{j=1}^k U_j  \Gamma_j[ -i \delta \lambda_j](T) U_j^*.
\end{equation}
Then
\begin{equation}
  \label{eqn:qident}
  \phi_r (\delta \cU(T) - \cU(T)
  \cQ(T)) \phi_s \in {\cal B}
\end{equation}

\section{Composite Map}

Define $R:\bR \times (C_0^{\infty}(\Sigma_s\times \bR))^k \rightarrow
({\cal D}'(\bR^2))^k$ by $R\bff = \bu(T)$, where
\begin{eqnarray}
  \label{eqn:stoc}
  \partial_t \bu + K \bu & = & \bff \delta_{\Sigma_s},\\
  \bu & = & 0, t \ll 0
\end{eqnarray}

Define $\tilde{R}: \bR \times ({\cal D}(\bR^2))^k \rightarrow
(C_0^{\infty}(\Sigma_r\times \bR))^k$ by $\tilde{R}\bw = P_r \bu$, where
\begin{eqnarray}
  \label{eqn:ctot}
  \partial_t \bu + K \bu & = & 0,\\
  \bu(T) & = & \bw.
\end{eqnarray}

Define ${\cal U}: \bR \times   ({\cal D}(\bR^2))^k
\rightarrow  ({\cal D}(\bR^2))^k$ by ${\cal U}(t)\bw = \bu(t)$
for a solution $\bu$ of the system \ref{eqn:ctot} with $T-0$.

Finally, define ${\cal S}: (C_0^{\infty}(\Sigma_s\times \bR))^k
\rightarrow (C_0^{\infty}(\Sigma_r\times \bR))^k$ by ${\cal S}\bff =
P_r\bu$, where $\bu$ solves the system \ref{eqn:stoc} and $P_r$ is the
restriction map to $\Sigma_r \times \bR$, that is the pull-back by the
injection into $\bR^3$.

Let $T_{r,s}$ be the minimum travel time between the compact surfaces
$\Sigma_r$ and $\Sigma_s$. Suppose that $\bff \in
(C_0^{\infty}(\Sigma_s\times \bR))^k$, $\mbox{supp }\bff \subset
\{(\bx,t):0 <t < T\}$ with $T < T_{r,s}$, also $T' \in (T,T_{r,s})$.
Let $\bu$ be the solution of the system $\ref{eqn:stoc}$.
Then $\bu(t)=0$ in a neighborhood of $\Sigma_r$ for $t<T'$, whence $\mbox{supp }R(t)\bff \cap \Sigma_r \times \{t\} = \emptyset$ for
$t<T'$, so ${\cal S}\bff = \tilde{R}(T')R(T')\bff$. Since the Cauchy
data for $\bu$ at $t=T$ is $R(T)\bff$, and is $R(T')\bff$ for $t=T'$,
$R(T')\bff = U(T,T')R(T)\bff$. So 
\begin{equation}
  \label{eqn:comprep}
  {\cal S}\bff = \tilde{R}(T'){\cal U}(T'-T)R(T)\bff.
\end{equation}

Suppose that $\phi_{s,r}$ are chosen as described above, and $T, T'$ and
$\Phi_{s,r}(\bx,D) \in OPS^0$ so that
\[
  (1-\phi_s) R(T)\Phi_{s}, \, \Phi_r\tilde{R}(T') (1-\phi_r)
\]
are smoothing. Then
\[
  \Phi_r {\cal S} \Phi_s \approx \Phi_r \tilde{R} (T') \phi_r {\cal U}(T'-T)
  \phi_s R(T)\Phi_s
\]
and equation \ref{eqn:qident} implies
\[
  \Phi_r {\delta \cal S} \Phi_s \approx \Phi_r \tilde{R} (T') \phi_r
  \delta {\cal U}(T'-T)
  \phi_s R(T)\Phi_s
\]
\[
  \approx \Phi_r \tilde{R} (T') \phi_r
   {\cal U}(T'-T)\tilde{{\cal Q}}(T'-T)
  \phi_s R(T)\Phi_s
\]
Now suppose that $\tilde{{\cal Q}}(T'-T) R(T) \Phi_s = R(T) {\cal Q}
 \Phi_s$, with ${\cal Q} \in OPS^1$. Since $[\tilde{{\cal Q}}, \phi_s]
 \in OPS^0$, this
implies that
\begin{equation}
  \label{eqn:qident2}
  \Phi_r {\delta \cal S} \Phi_s \approx \Phi_r \tilde{R} (T') \phi_r
   {\cal U}(T'-T)
  \phi_s R(T) {\cal Q}\Phi_s = \Phi_r {\cal S} {\cal
    Q}\Phi_s \mbox{ mod } {\cal B}
\end{equation}


\section{Example: 2D Acoustics}
Source and receiver surfaces are linear antennae $\Sigma_{s,r} \subset
\{x_2=z_{s,r}\}$. For convenience, $z_s < z_r$, so incoming rays have
positive $x_2$ velocity components.

$\tilde{\bu} = (p,v_1,v_2)$, $\bx = (x_1,x_2)$,  $ \xi = (\xi_1,\xi_2)$,
\[
  M\partial_t \tilde{\bu} + \tilde{K}(\bx, \partial_{\bx})\tilde{\bu} = 0
\]
$M=\mbox{diag }(\kappa^{-1},\rho,\rho)$, 
\[
  \tilde{K}(\bx,\xi) =
 -i\left(
    \begin{array}{ccc}
      0 & \xi_1 & \xi_2 \\
      \xi_1 & 0 & 0 \\
      \xi_2 & 0 & 0
    \end{array}
  \right)
\]
If $\bu = M^{-1/2}\tilde{\bu}$, then
\[
  \partial_t \bu - K \bu = 0
\]
\[
  K = M^{-1/2}\tilde{K} M^{-1/2} = c\tilde{K}
\]
$c = \kappa^{1/2}\rho^{-1/2} $ = sound velocity.

Spectrum
\[
  \det (\lambda I + K) = \det
  \left(
    \begin{array}{ccc}
      \lambda & c\xi_1 & c\xi_2 \\
      c\xi_1 & \lambda & 0 \\
      c\xi_2 & 0 & \lambda
    \end{array}
  \right)
  = \lambda(\lambda^2 - c^2(\xi_1^2 + \xi_2^2))
\]
so eigenvalues are $\lambda_0=0, \lambda_1 = c\sqrt{\xi_1^2 +
  \xi_2^2}, \lambda_2=-c\sqrt{\xi_1^2 +  \xi_2^2}$. Corresponding unit
eigenvectors are
\[
  U_0(\xi) = \left(
    \begin{array}{c}
      0\\
      -\frac{\xi_2}{\sqrt{\xi_1^2 + \xi_2^2}}\\
      \frac{\xi_1}{\sqrt{\xi_1^2 + \xi_2^2}}
    \end{array}
  \right),\,
  U_1(\xi) = \frac{1}{\sqrt{2}}\left(
    \begin{array}{c}
      1\\
      \frac{\xi_1}{\sqrt{\xi_1^2 + \xi_2^2}}\\
      \frac{\xi_2}{\sqrt{\xi_1^2 + \xi_2^2}}
    \end{array}
  \right),\,
  U_2(\xi) = \frac{1}{\sqrt{2}}\left(
    \begin{array}{c}
      -1\\
      \frac{\xi_1}{\sqrt{\xi_1^2 + \xi_2^2}}\\
      \frac{\xi_2}{\sqrt{\xi_1^2 + \xi_2^2}}
    \end{array}
  \right).                     
\]
Note that in this case, the eigenvectors are independent of both $\bx$
and of the material parameters - in particular, $\delta U_i \equiv 0$.

So
\[
  H_i(\bx,\xi) = (-1)^{i-1}\left(c(\bx)\frac{\xi}{|\xi|}, - \nabla
    c(\bx)|\xi|
  \right),\,i=1,2
\]
\[
\Gamma_j[ -i \delta \lambda_j](T;\bx,\xi) = \int_0^T dt \, \delta \lambda_j(C_j(t;\bx,\xi))
\]
\[
= (-1)^{j-1}\int_0^T dt \, \delta c(\by(t;\bx,\xi))|\eta(t;\bx,\xi)|
= (-1)^{j-1}\int_0^T dt \, \frac{\delta c(\by(t;\bx,\xi))}
{c(\by(t;\bx,\xi))}\lambda_j(C_j(t;\bx,\xi))
\]
Since $\lambda_j$ is constant along $t \mapsto C_j(t;\bx,\xi)$, this
is
\[
  = (-1)^{j-1}c(\bx)|\xi|\int_0^T dt \, \frac{\delta c(y(t;\bx,\xi))}
{c(\by(t;\bx,\xi))} = = (-1)^{j-1}c(\bx)|\xi|\int_0^T ds \, \frac{\delta c(\by(t;\bx,\xi))}
{c(\by(t;\bx,\xi))^2} 
\]
where $ds= c(y(t;\bx,\xi))dt $ is the element of arc length $s$ along the
curve $t \mapsto y(t;\bx,\xi)$. Writing $y$ as a function of $s$, and
using $s(T;\bx,\xi)$ as the limit of integration, this becomes
\[
  = (-1)^{j-1} c(\bx)|\xi| \int_0^{s(T;\bx,\xi)} ds\, \left(\delta
    \frac{1}{c}\right) (\by(t(s;\bx,\xi);\bx,\xi))
\]
For any $C^1$ path $\gamma: [0,s(T;\bx,\xi)] \rightarrow \bR^2$ for
which $\gamma(0) = \bx, \gamma(s(T;\bx,\xi))=\by(T;\bx,\xi)$, define the
traveltime along $\gamma$ as usual:
\[
  {\cal T}(\gamma,c) = \int_0^{s(T;\bx,\xi)} ds \frac{1}{c(\gamma(s))}
\]
Then (Fermat)
\[
  \delta_{\gamma} {\cal T}(\gamma,c)_{\gamma=\by(\cdot;\bx,\xi)} = 0
\]
so
\[
  \delta_{c} {\cal T}(\by(\cdot;\bx,\xi),c)=\int_0^{s(T;\bx,\xi)} ds\, \left(\delta
    \frac{1}{c}\right) (\by(t(s;\bx,\xi);\bx,\xi))
\]


Which rays are incoming on $\{z=x_2 > 0\}$ The first Hamilton
equation:
\[
  \frac{dx_2}{dt} = \partial_{\xi_2} \lambda_i(\bx,\xi) = (-1)^{i-1}
c(\bx)\frac{\xi_2}{|\xi|}
\]
So, it depends: if $\xi_2 >$, then components parallel to $U_1$
propagate in positive $x_2$, positive $t$, whereas if $\xi_2<0$, it's
components parallel to $U_2$. A $\Psi$DO projector onto such
components is constructed as follows: choose $\epsilon>0$, and choose
convolutional 
$\Gamma_1\in S^0$ for which
\[
  \Gamma_1 \sim 1 \mbox{ for } \xi_2/|\xi| > \epsilon, \sim 0
  \mbox{ for } \xi_2/|\xi|< \epsilon/2.
\]
and $\Gamma_2(\xi) =\Gamma_1(-\xi)$. Then any $\lambda_i$ bicharacteristic passing over
$x_2=0$ in $ES(\Gamma_i)$, $i=1,2$ is incoming to $x_2>0$. 

Define
\[
  P_+(\bx,\xi) = \chi(\bx) (\Gamma_1(\xi)U_1(\xi)U_1(\xi)^* +
  \Gamma_2(\xi)U_2(\xi)U_2(\xi)^*)
\]
Choose $\chi \in C_0^{\infty}(\bR^2)$ so that any $\lambda_j$ ray with initial
conditions in $ES(\Gamma_j) \cap \mbox{supp }\chi$ exits $\{x_2 <
0\}$ as an incoming ray. Similarly,
\[
  P_-(\bx,\xi) = \chi(\bx) (\Gamma_2(\xi)U_1(\xi)U_1(\xi)^* +
  \Gamma_1(\xi)U_2(\xi)U_2(\xi)^*)
\]
projects onto outgoing rays.

Incoming GO solution: $\xi \in ES(\Gamma_1)$, $\tau(\bx;\xi)$ solution
of eikonal $c(\bx)|\nabla \tau(\bx;\xi_1)|=1$ with $\tau(x_1,0,\xi_1)
= x_1\xi_1$. Necessary condition: $c(x_1,0)|\xi_1| < 1$ for all
$x_1$. Then 

Set
\[
  \xi^{\pm}_2(\bx,\xi_1) = \pm \sqrt{\frac{1}{c(\bx)^2} - \xi_1^2},\,\,
  \xi^{\pm}(\bx,\xi_1) = (\xi_1,\xi_2^{\pm}(\bx,\xi_1)).
\]
Then the pair $(t,x_1,\xi_1)$ indexes two ray families
\[
  s \rightarrow (t +
s,\by(\pm s;(x_1,0),\xi^{\pm}((x_1,0),\xi_1)),
\eta(\pm s;(x_1,0),\xi^{\pm}((x_1,0),\xi_1)))  = C^{\pm}(s;t,x_1,\xi_1)
\]
From Hamilton's equations, follows that
$C^{-}(s;t,x_1,\xi_1)=C^{+}(-s;t,x_1,\xi_1)$, so both ray families
generate the same solution of the eikonal, $\tau(\bx,\xi_1)$

GO solutions:
\[
  \bu_i(t,(x_1,0)) =\exp(i\omega(t+x_1\xi_1))U_i((\xi^{\pm}((x_1,0),\xi_1)))
\]
incoming:
\[
 \bu(t,\bx;\xi_1)=a(\bx)
 \exp(i\omega(t-\tau(\bx,\xi_1)))U_1(\xi^+(\bx,\xi_1))
\]
From definition of $\xi^+(\bx,\xi_1)$,
\[
  \frac{\xi^+_i(\bx,\xi_1)}{|\xi^+(\bx,\xi_1)|} = c(\bx) \xi^+_i(\bx,\xi_1)
\]
%%%%%%%%
GO:
\[
  \partial_t u + \lambda(\bx,D)u = 0
\]
\[
  u(x,\xi) = a(x,\xi)e^{i(\omega(t-\phi(\bx,\xi)))},\,\, \phi(\bx,\xi) = |\xi|\phi(\bx,\hat{\xi})
\]
Then (\cite{Tay:81}, VIII.7)
\[
  \lambda(\bx,D)u(\bx,\xi) =a(\bx,\xi)\lambda(\bx,\omega
  D_x\phi(\bx,\xi)) e^{i(\omega(t-\phi(\bx,\xi)))} \mbox{ mod } S^0
\]
so assuming that $\lambda$ is homogeneous of degree 1 in $\xi$
\[
  \partial_t u + \lambda(\bx,D)u = i \omega a(\bx,\xi) (1 -
  \lambda(\bx, D_x\phi(\bx,\xi))) e^{i(\omega(t-\phi(\bx,\xi)))}
  \mbox{ mod } S^0
\]
Eikonal:
\[
  \lambda(\bx, D_x\phi(\bx,\xi))=1
\]

\bibliographystyle{seg}
\bibliography{../../../bib/masterref}
