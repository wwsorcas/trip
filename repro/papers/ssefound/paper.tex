\title{Mapping Properties of Surface Source Extension}
\author{William W. Symes}

\begin{abstract}
  Abstract goes here.
\end{abstract}

\section{Local Analysis of Surface Source Field}
$k \times  k$ symmetric hyperbolic system: in $\bR^{n+1}$, $\Omega
\subset \bR^n, \Gamma_r, \Gamma_s \subset \partial \Omega$. $\Gamma_r$
is receiver locus, $\Gamma_s$ is sourcee locus. Assume that $\Gamma_s$
is a subset of a coordinate neighborhood in  which coordinates
$(z,\bx) \in \bR^{n-1} \times \bR$ are
chosen so that $\Gamma_i\subset \{(z,\bx): z=z_i\}, i=s,r$ with $z_s <
z_r$.

Assume $M, A_1,...,A_{n-1}, L \in C^{\infty}(\bR^n, \bR^{k\times k})$
are symmetric and $M$ is positive-definite.

Let $\chi \in C^{\infty}(\Gamma_s \times \bR, \bR^k)$.
Develop an asymptotic solution of
\begin{equation}
  \label{eqn:shs}
  (M\partial_t + \sum_{i=1}^{n-1}A_i \partial_{x_i} + L \partial_z)
  u = 0,\, u(t,z,\bx_s) = \chi(t,\bx)e^{i\omega(t + \bx \cdot \xi_0)}
\end{equation}
of the form
\begin{equation}
  \label{eqn:hfa}
  u(t,z,\bx) \approx e^{i\omega(t + \psi(z,\bx))} \sum_{n=0}^{\infty}
  (i\omega)^{-n}a^n(t,z,\bx) U^n(z,\bx)
\end{equation}
in which $a^n, n=0,...$ is scalar and $U^n, n=0,...$ is in $\bR^k$. 

Substitute first term in equation \ref{eqn:shs}, obtain
\[
(M\partial_t + A\cdot \partial_x + L \partial_z)u(t,z,\bx) =
\]
\[
 e^{i\omega(t + \psi(z,\bx))}\left[ i\omega (M + A
\cdot \partial_x \psi(z,\bx) + L \partial_z
\psi(z,\bx))U(z,\bx)a^0(t,z,\bx) \right.
\]
\[
+ M(z,\bx) U^0(z,\bx) \partial_t a^0(t,z,\bx) + A(z,\bx)\cdot \partial_x
(a^0(t,z,\bx) U^0(z,\bx))
\]
\[
 \left.   + L(z,\bx)\partial_z (a^0(t,z,\bx)U^0(z,\bx))\right]
\]
\[
+  e^{i\omega(t + \psi(z,\bx))}\left[ (M + A
\cdot \partial_x \psi(z,\bx) + L \partial_z
\psi(z,\bx))U^1(z,\bx)a^1(t,z,\bx) \right.
\]
\[
  + (i\omega)^{-1} (M(z,\bx) U^1(z,\bx) \partial_t a^1(t,z,\bx)
    + A(z,\bx)\cdot \partial_x  (a^1(t,z,\bx) U^1(z,\bx))
\]
\begin{equation}
  \label{eqn:series}
  \left.   + L(z,\bx)\partial_z (a^1(t,z,\bx)U^1(z,\bx)))\right]
+...
\end{equation}

Assume $\Sigma$ is an open neighborhood of the zero section in
$T^*\Gamma_s$, so that for each $(\bx,\xi) \in \Sigma$, $|\xi|$ is
sufficiently small that $M + A\cdot \xi$ is positive
definite. Further, for $z$ sufficiently small,
there are $l$ independent solutions of 
$(\zeta_1(z,\bx,\xi),U^0_1(z,\bx,\xi)),...,(\zeta_l(z,\bx,\xi), U^0_l(z,\bx,\xi))$ of
\begin{equation}
  \label{eqn:nullsp}
 (M + A\cdot \xi + L \zeta) U^0=0, \, \zeta\ne 0.
\end{equation}
The $U^0$s are orthogonal, and might as well normalize by
$|U^0|=1$. For each $(\bx,\xi) \in \Gamma$, construct by method of characteristics solutions
$\psi_1,...,\psi_l$ of 
\begin{equation}
  \label{eqn:eik}
\partial_z \psi_i(z,\bx,\xi) = \zeta_i(z,\bx,\partial_x\psi(z,\bx,\xi))
\end{equation}
with $\psi_l(\bx,0,\xi) =\xi \cdot x$. The corresponding Hamiltonian
system for $(\bx_i(z;\bx_0,\xi_0),\xi_i(z;\bx_0,\xi_0))$ with
$(\bx_0,\xi_0) \in \Gamma$ is
\begin{equation}
  \label{eqn:zetahe1}
  \frac{d\bx_i}{dz} = - \partial_{\xi}\zeta_i(z,\bx_i,\xi_i),\,
  \bx_i(0;\bx_0,\xi_0) = \bx_0;
\end{equation}

\begin{equation}
  \label{eqn:zetahe2}
  \frac{d\xi_i}{dz}
 = \partial_x\zeta_i(z,\bx_i,\xi_i),\, \xi_i(0;\bx_0,\xi_0) = \xi_0.
\end{equation}

From the relation \ref{eqn:hsol}, 
\[
  \psi_l(\bx_i(z;\bx_0,\xi_0),z,\xi_i(z;\bx_0,\xi_0))= \bx_0\cdot\xi_0 
\]
\begin{equation}
  \label{eqn:psihe}
+ \int_{z_s}^z ds [\zeta(\bx_i(s;\bx_0,\xi_0),s, \xi(s;\bx_0,\xi_0)) - \xi(s;\bx_0,\xi_0)\cdot \partial_{\xi}\zeta_i(\bx_i(s;\bx_0,\xi_0),s,\xi_i(s;\bx_0,\xi_0))]
\end{equation}

As previously noted, $\zeta_i \ne 0$ if $\Sigma$ is chosen
sufficiently small. Enumerate the $\zeta$s so that $zeta_1,.../\zeta_m
< 0, \zeta_{m+1},...,\zeta_l > 0$. Then from the eikonal equation
\ref{eqn:eik}, the time $t$ along equal phase surfaces $t +
\psi_i(z,\bx_i(z,\bx_0,\xi_0),\xi_i(z,\bx_0,\xi_0)) = \mbox{ const}$
increases for $i=1,...,m$, decreases for $i=m+1,...,l$.

% Suppose that $supp \chi \subset \{(\bx,t): t > 0\}$.
So much for the first term in the RHS of equation
\ref{eqn:series}. The second and third terms are of order zero in
$i\omega$, hence should cancel. However the matrix in the third term
is not invertible in view of the relation \ref{eqn:nullsp}. Therefore
the second term must be orthogonal to the kernel of this matrix, which
since the matrix is symmetric simply means in its orthogonal to $U^0$.
That is,
\[
U^0(z,\bx) \cdot (M(z,\bx) U^0(z,\bx) \partial_t a^0(t,z,\bx) + A(z,\bx)\cdot \partial_x
(a^0(t,z,\bx) U^0(z,\bx))
\]
\[+ L(z,\bx)\partial_z
  (a^0(t,z,\bx)U^0(z,\bx))) = 0,
\]
or
\[
 (b_t \partial_t + b_x \cdot \partial_x + b_z \partial_z)
 a^0 + c a^0 = 0
\]
where
\[
  b_t = U^0 \cdot M U^0, b_{x,i} = U^0 \cdot A_i U^0, b_z = U^0 \cdot
  L U_0
\]
and $c$ is the rest, not involving derivatives of $a^0$.



\section{2D Acoustics}

\begin{eqnarray}
  \frac{1}{\kappa}\partial_t p + \nabla \cdot \bv & = & 0, \nonumber \\
  \rho \partial_t \bv + \nabla p & = & 0.
  \label{eqn:awe}                                     
\end{eqnarray}

Set $u=(p,v_x,v_z), \bx=x, x_2=z. \xi=\xi_x$. Then \ref{eqn:awe} takes the form \ref{eqn:shs} with
\[
  L =
  \left(
    \begin{array}{ccc}
      0 & 0 & 1\\
      0 & 0 & 0\\
      1 & 0 & 0
    \end{array}
  \right),
  M ={\rm diag}\left(\frac{1}{\kappa},\rho,\rho\right),
  A =
 \left(
    \begin{array}{ccc}
      0 & 1 & 0\\
      1 & 0 & 0\\
      0 & 0 & 0
    \end{array}
  \right),
\]
\[
  M + A\xi + L\zeta =
  \left(
    \begin{array}{ccc}
      \frac{1}{\kappa} & \xi & \zeta \\
      \xi & \rho & 0 \\
      \zeta & 0 & \rho
    \end{array}
  \right)
\]
Determinant is
\[
  {\rm det} = \rho\left(\frac{\rho}{\kappa} - \xi^2 - \zeta^2\right)
\]
Therefore choose $\epsilon > 0$ and define $\Gamma = \{(\bx,\xi): |\xi| < \mbox{ inf}\{\sqrt{\rho(\bx,0)/\kappa(\bx,0)}, \mbox{ all } |z| < Z\}$. For $\xi
\in \Gamma$, there are two roots $\zeta_{\pm} = \pm \sqrt{\rho/\kappa - \xi^2}.$
Accordingly there are two solutions of equation \ref{eqn:eik},
\[
  \psi_{\pm}(\bx_{\pm}(z;\bx_0,\xi_0),z,\xi_{\pm}(z;\bx_0,\xi_0))= \bx_0\cdot\xi_0
\]
\[
 +\int_0^z ds \zeta_{\pm}(\bx_{\pm}(s;\bx_0,\xi_0),s,\xi_{\pm}(s;\bx_0,\xi_0))-\xi_{\pm}(s;\bx_0,\xi_0)\cdot \partial_{\xi}\zeta_{\pm}(\bx_{\pm}(s;\bx_0,\xi_0),s,\xi_{\pm}(s;\bx_0,\xi_0))
\]
\[
  =\bx_0 \cdot \xi_0 \pm \int_0^z
  \frac{ds}{c(\bx_{\pm}(s;\bx_0,\xi_0),z)
    \sqrt{(1-\xi_{\pm}(s;\bx_0,\xi_0),z)^2
    c(\bx_{\pm}(s;\bx_0,\xi_0),z)^2}}
\]
where $c = \sqrt{\kappa/\rho}$ is the sound velocity.









\section{2D Elasticity}

Velocity, stress, state vector:
\[
  \bv = \left(
    \begin{array}{c}
      v_1 \\
      v_2
    \end{array}
  \right), \,
  \sigma = \left(
    \begin{array}{c}
      \sigma_{11} \\
      \sigma_{12} \\
      \sigma_{21} \\
      \sigma_{22}
    \end{array}
  \right), \, 
  u = \left(
    \begin{array}{c}
      \sigma\\
      \bv
    \end{array}
  \right)
\]
Symmetry of stress: $\sigma_{12}=\sigma_{21}$. Compliance tensor (inverse of Hooke tensor):
\[
  C =\left(
    \begin{array}{cccc}
      c_{1111} &  c_{1112} &  c_{1121} &  c_{1122} \\
      c_{1211} &  c_{1212} &  c_{1221} &  c_{1222} \\
      c_{2111} &  c_{2112} &  c_{2121} &  c_{2122} \\
      c_{2211} &  c_{2212} &  c_{2221} &  c_{2222} \\
    \end{array}
  \right)
\]
Symmetries: $C_{ijkl}=C_{klij}, C_{ijkl}=C_{jikl}=C_{ijlk}$. Density = $\rho$

Elastodynamics:
\begin{eqnarray}
  \label{eqn:const}
  C \partial_ t \sigma & = & \frac{1}{2}(\nabla \bv + \nabla \bv^T) \\
  \label{eqn:mombal}
  \rho \partial_t \bv & = & \nabla \cdot \sigma
\end{eqnarray}

Matrix representation: set
\[
  B_1=\left(
    \begin{array}{cccc}
      1 & 0 & 0 & 0 \\
      0 & \frac{1}{2} & \frac{1}{2} & 0
    \end{array}
  \right),\,
  B_2=\left(
    \begin{array}{cccc}
      0 & \frac{1}{2} & \frac{1}{2} & 0 \\
      0 & 0 & 0 & 1
    \end{array}
  \right)
\]
and
\[
  A_0 = \left(
    \begin{array}{cc}
      C & 0 \\
      0 & \rho I
    \end{array}
  \right), \,
  A_j = -\left(
    \begin{array}{cc}
      0 & B_j^T \\
      B_j & 0
    \end{array}
  \right), \, j=1,2
\]

Then the elastodynamics equations \ref{eqn:const}, \ref{eqn:mombal} are equivalent to
\[
  (A_0\partial_0 + A_1 \partial_1 + A_2 \partial_2)u =0.
\]

$u = (\sigma, \bv)^T \in \bR^6$ is a null vector for $A_2$ iff $B_2\bv = 0$ and $B_2^T \sigma=$, that is, $\bv=0$ and $\sigma_{12}=\sigma_{21}=\sigma_{22} = 0$. Thus $U_0= \mbox{ker }A_2 = \mbox{span } \{\be_1\}$ and  $U_1 = \mbox{rng }A_2 = \mbox{span }\{\be_j: 2\le j \le 6\}$, and $\Pi_0 = \mbox{diag}(1,0,0,0,0,0)$, $\Pi_1 = I-\Pi_0$.

Eliminating zero rows and columns, equation \ref{eqn:shsnull1} for $u_1 = (\sigma_{12}, \sigma_{21}, \sigma_{22}, v_1, v_2)^T$ reads
\begin{eqnarray}
  \frac{1}{2}\partial_2v_1 &=& -((C \partial_0 \sigma)_{21} + \frac{1}{2}\partial_1v_2)\\
  \partial_2 v_2 &=& -(C \partial_0 \sigma)_{41}\\
  \partial_2 \sigma_{12} &=& -(\rho \partial_0v_1)
\end{eqnarray}

Thus
\begin{eqnarray}
  A_{000} &=& \Pi_0 A_0 \Pi_0 = \mbox{ diag}(c_{1111},0,0,0,0,0) \nonumber \\
  A_{100} &=& \Pi_0 A_1 \Pi_0 = 0 \nonumber \\
  A_{001} &=& \Pi_0 A_0 \Pi_1 =
              \left(
              \begin{array}{cccccc}
                0 & c_{1112} & c_{1121} & c_{1122} & 0 & 0 \\
                0 & 0 & 0 & 0 & 0 & 0 \\
                0 & 0 & 0 & 0 & 0 & 0 \\                
                0 & 0 & 0 & 0 & 0 & 0 \\
                0 & 0 & 0 & 0 & 0 & 0 \\
                0 & 0 & 0 & 0 & 0 & 0
              \end{array}
                                    \right) \nonumber \\
  A_{101} &=& \Pi_0 A_1 \Pi_1 =
                            \left(
              \begin{array}{cccccc}
                0 & 0 & 0 & 0 & 1 & 0 \\
                0 & 0 & 0 & 0 & 0 & 0 \\
                0 & 0 & 0 & 0 & 0 & 0 \\                
                0 & 0 & 0 & 0 & 0 & 0 \\
                0 & 0 & 0 & 0 & 0 & 0 \\
                0 & 0 & 0 & 0 & 0 & 0
              \end{array}
                                    \right)
                                    \label{eqn:ela0}
\end{eqnarray}
So equation \ref{eqn:rel01} reduces to
\[
  c_{1111}\sigma_{11} = -(c_{1112}\sigma_{12} + c_{1121}\sigma_21 + c_{1122}\sigma_{22} + \partial_1 \int_{-\infty}^t v_1)
\]
This is just the first row of equation \ref{eqn:const}, integrated in time! Now apply $\partial_2$ to both sides to obtain the missing equation.

\append{G\r{a}rding Interlude}

In the following $U$ is an open set in $\bR^n$.

\begin{definition}
  A $k \times k$ valued symbol $M(\bx,\xi) \in OPS^m(U)$ is {\em elliptic} iff $|\xi|^{-m}M(\bx,\xi)$ is uniformly invertible for $(\bx,\xi) \in T^*(U) \setminus \{0\}$. That is, there exists $C>0$ so that for all  $(\bx,\xi) \in T^*(U) \setminus \{0\}$, $v \in \bR^k$,
  \[
    |M(\bx,\xi)v| \ge C|\xi|^m|v|.
  \]
\end{definition}

Connection to defintion V.1.1 in \cite{Tay:81}: concerns the first order system
\[
  (\frac{\partial}{\partial y} - K(y,x,D_x))u = 0
\]
where $u=u(y,x) \in \bR^k$, $K(y,\cdot,\cdot) \in C^{\infty}(\bR,OPS^1(U))$. According to the cited definition, this system is elliptic iff the principal symbol $K_1(y,x,\xi)$ has no purely imaginary eigenvalues. The symbol of the operator on the LHS above is $i\eta I - K(y,x,\xi)$, which is not the symbol of a $\Psi$DO unless $K$ is differential. However $K$ having no purely imaginary eigenvalues iz equivalent to  $i\eta I - K(y,x,\xi)$ being invertible.


\begin{lemma} \label{thm:lema1}
``Interpolation'' inequality: for any $m, N \in {\bf Z}_+, 0 < \epsilon < 1$,
\[
  \lxr^{\frac{m}{2}-1} \le \epsilon^2 \lxr^{\frac{m}{2}} + C^2_{\epsilon,N,m}\lxr^{-N}. 
\]
\end{lemma}

\begin{proof}
 Equivalent to
\[
  \lxr^{-1} \le \epsilon^2 + C^2_{\epsilon,N,m}\lxr^{-N-\frac{m}{2}}. 
\]
Holds trivially if $\lxr \ge \epsilon^{-2}$. Otherwise $\lxr^{N+\frac{m}{2}-1} \le \epsilon^{-2(N+\frac{m}{2}-1)} \equiv C_{\epsilon,N,m}$.
\end{proof}

\begin{proposition}
\label{thm:propa1}(clone of V.6.2 in \cite{Tay:81}): Suppose that $p \in OPS^0(U,\bR^k)$ has a SPD principal symbol $p_0$: there is $C > 0$ so that for sufficiently large $|\xi|$,
  \[
    p_0(x,\xi) = p_0(x,\xi)^T, p_0(x,\xi) \ge C
  \]
  Then there existss $B \in OPS^0(U,\bR^k)$ so that $P = B^*B \mbox{ mod } OPS^{-\infty}$.
\end{proposition}

\begin{proof}
  From the spectral theorem, for each $(x,\xi) \in T^*(U)$, can construct SPD $b_0(x,\xi)$ for which $p_0(x,\xi) = b_0(x,\xi)^2$. Suppose that we have built $b_0,...,b_{j-1}$ with $b_i \in S^{-i}$ homogeneous of degree $-i$ for large $|\xi|$ so that
  $r_{j}(x,D) = p(x,D) - (b_0(x,D) +...+ b_{j-1}(x,D))^2 \in OPS^{-j}$. Want $b_j \in S^{-j}$ homogeneous of degree $j$ so that
  \[
    p(x,D)-(b_0 +...+ b_{j-1}+b_j)^2 = r_{j-1} -  (b_0 +...+ b_{j-1})b_j - b_j((b_0 +...+ b_{j-1}) - b_j^2
  \]
  \[
    = r_{j-1} - (b_0 b_j+b_j b_0) \mbox{ mod } OPS^{-j-1}
  \]
  is in $OPS^{j-1}$. This will be the case if the principal symbols match: denoting the principal symbol of $r_{j-1}$ by $r_{j-1,0}$, this requires that
  \[
    r_{j-1,0}(x,\xi) = b_0(x,\xi)b_j(x,\xi) + b_j(x,\xi)b_0(x,\xi).
  \]
  Since $b_0$ is SPD, the map $a \mapsto b_0a+ab_0$ on $k \times k$ symmetric matrices $a$ is invertible, as its eigenvalues are sums of the eigenvalues of $b_0$. So the above equation has a unique symmetric solution, and the inductive step in the construction of $b \in OPS^0$ is finished.
\end{proof}
  
\begin{theorem}
  \label{thm:garding}
  ``G\r{a}rding for systems'': Suppose that $P \in OPS^m(U,\bR^k)$ is elliptic, that is, its principal symbol $p_0$ satisfies 
\[
  p_0^*(x,\xi)p_0(x,\xi) \ge C|\xi|^{2m}, \, (x,\xi) \in T^*(U).
\]
Then for any $K \subset \subset U$, $N \ge 1$ there exists $C_{K,N} > 0$ so that for $u \in C^{\infty}_0(K)$, 
\[
  \|Pu\|^2_0 \ge \frac{C}{2}\|u\|^2_m - C_{K,N}\|u\|^2_{-N}.
\]
\end{theorem}

\begin{proof}
$\ldr^{-m}P^*P\ldr^{-m} - \frac{C}{2} \in OPS^0$ is self-adjoint  and its principal symbol $\lxr^{-m}p_0^Tp_0\lxr^{-m} - \frac{C}{2} \in S^0$ is positive semi-definite. Therefore from Proposition \ref{thm:propa1} there exists $B \in OPS^0$ for which
  \[
    \ldr^{-m}P^*P\ldr^{-m} - \frac{C}{2} = B^*B \mbox{ mod } OPS^{-\infty}
  \]
  or
  \[
    P^*P -\frac{C}{2} \ldr^{2m} = \ldr^m B^*B \ldr^m \mbox{ mod } OPS^{-\infty}
  \]
  So for every $N  \in {\bf Z}_+$ there is $C_{K,N} > 0$ so that for $u \in C_0^{\infty}(K)$,  
  \[
    \langle u,P^*P u \rangle \ge \frac{C}{2}\|u\|_m^2 -   C_{K,N}\|u\|^2_{-N}
  \]
  
\end{proof}

\begin{definition} $P \in OPS^m(U,\bR^k)$ is (microlocally) elliptic in the open conic set $\Gamma \subset T^*(U)$ iff for some $C>0$ and conic neighborhood $\Gamma_1 \supset \bar{\Gamma}$, its principal symbol $p_0$ satisfies
  \[
    p(x,\xi)^*p(x,\xi) \ge C|\xi|^m, \, \xi \in \Gamma_1, \, |\xi| \ge 1.
  \]
\end{definition}

Some results from \cite{BaoSy:91b}. Here $0 \le m \le n$, ${\cal N}$ is the normal bundle of the foliation $\bR^n = \bR^{n-m} \times \bR^m$, typically $x \in \bR^{n-m}$ is the parameter, $y \in \bR^m$ is the trace coordinate.

\begin{lemma}
\label{thm:prebao}
Suppose that $k \in {\bf Z}_+$, $n > 0$. Then there exists $C_{k} \ge 0$ so that for $\xi, \eta \in \bR^n$,
\begin{equation}
  \label{eqn:pre}
  (1 + |\xi + \eta|)^{k} \le (1 + |\xi|)^{k} (1 + |\eta|)^{k}
\end{equation}
\end{lemma}
\begin{proof}
  Obvious for $k = 0$. If equation \ref{eqn:pre} holds for $k \in {\bf Z}_+$, then
  \[
    (1 + |\xi + \eta|)^{k+1} \le (1 + |\xi)^{k}(1+ \eta|)^{k}(1+|\xi| + |\eta|)
  \]
  Since $1+|\xi| + |\eta| \le (1+|\xi|)(1 + |\eta|)$, the induction step is done.
\end{proof}

\begin{proposition}
  \label{thm:bao1}
  Suppose that $B \in \Ci(\bR^{n-m},OPS^0(\bR^m)), H \in OPS^0(\bR^n)$, $H$ is a convolution operator, $\phi \in \Ciz(\bR^n)$,
  \[
    ES(H) \cap {\cal N} = \emptyset.
  \]
  Then
  \[
    B\phi H \in OPS^0(\bR^n)
  \]
  and
  \[
    ES(B\phi H) \subset \Pi_2^{-1}(ES(B)) \cap ES(H)
  \]
\end{proposition}

\begin{proof}
  For $u \in \Ciz(\bR^n)$,
  \[
    {\cal F}_x(Hu)(\xi',y) = \int  d\eta H(\xi',\eta')\hat{u}(\xi',\eta)e^{ i y \eta}
  \]
  So
  \[
    (P\phi H)u(x,y) = \int d \xi P(x,y,\xi) {\cal F}_x(\phi Hu)(\xi,y)e^{i x\xi}
  \]
  \[
    =\int d \xi' \int d \xi P(x,y,\xi) {\cal F}_x\phi(\xi-\xi',y) {\cal F}_x(Hu)(\xi',y)e^{i x\xi}
  \]
  \[
   =\int d \xi' \int d \xi \int d\eta P(x,y,\xi) {\cal F}_x\phi(\xi-\xi',y) H(\xi',\eta)\hat{u}(\xi',\eta)e^{i x\xi + i y\eta }
  \]
  \[
    =\int d \xi' \int d \xi \int d\eta P(x,y,\xi) {\cal F}_x\phi(\xi-\xi',y) e^{ix(\xi-\xi')}H(\xi',\eta)\hat{u}(\xi',\eta)e^{i x\xi' + i y\eta }
  \]
  So the symbol of $Q = P \phi H$ is
  \[
    Q(x,y,\xi,\eta) = \int d\xi' P(x,y,\xi') {\cal F}_x\phi(\xi'-\xi,y) e^{i x(\xi'-\xi)}H(\xi,\eta)
  \]
  \[
    = \left(\int d\xi' P(x,y,\xi+\xi') {\cal F}_x\phi(\xi',y) e^{i x\xi'}\right) H(\xi,\eta)
  \]
  From the definition of $S^0_{1,0}$, for $(x,y)$ ranging over a compact $K \subset \bR^n$ , and any $\alpha \in {\bf Z}_+^m$, there is $C_{K,|\alpha|} > 0$ so that
  \[
    |\partial^{\alpha}_{\xi} P(x,y,\xi)| \le C_{K,|\alpha|}(1 + |\xi|)^{-|\alpha|}
  \]
  so for $(x,y) \in K, \alpha=(\alpha_1,\alpha_2) \in {\bf Z}_+^m \times {\bf Z}_+^{n-m}$,
  \[
    |\partial_{\xi,\eta}^{\alpha} Q(x,y,\xi,\eta)|
  \]
  \[
    \le \sum_{\alpha_1' \le \alpha_1} \left|\left(\partial_{\xi}^{\alpha_1'}  \int d\xi' P(x,y,\xi+\xi') {\cal F}_x\phi(\xi',y) e^{i x\xi'}\right)\partial_{\xi}^{\alpha_1-\alpha_1'} \partial_{\eta}^{\alpha_2}H(\xi,\eta)       \right|
  \]
  \[
    \le C \sum_{\alpha_1' \le \alpha_1} \left|\left(  \int d\xi' (1 + |\xi+\xi'|)^{-|\alpha_1'|} {\cal F}_x\phi(\xi',y) e^{i x\xi'}\right)\partial_{\xi}^{\alpha_1-\alpha_1'} \partial_{\eta}^{\alpha_2}H(\xi,\eta)       \right|
  \]
  Rearranging the conclusion of Lemma \ref{thm:prebao},
  \[
    (1 + |\xi + \xi'|)^{-k} \le (1+|\xi'|)^k(1+|\xi|)^{-k},
  \]
  so
  \begin{equation}
    \label{eqn:qbound}
    |\partial_{\xi,\eta}^{\alpha} Q(x,y,\xi,\eta)| \le C \sum_{\alpha_1' \le \alpha_1} (1 + |\xi|)^{-|\alpha_1'|}|\partial_{\xi}^{\alpha_1-\alpha_1'}\partial_{\eta}^{\alpha_2}H(\xi,\eta) |
  \end{equation}
  Recall that $ES(H) \cap {\cal N} = \emptyset$. That is, there is $s>0$ so that for
  $k,N \in {\bf Z}_+, \beta \in {\bf Z_+}^n$ with $|\beta| \le k$, there are $C_{k,N} \ge 0$ so that for $|\xi| \le s|\eta|$,
  \[
    |\partial_{\xi,\eta}^{\beta}H(\xi,\eta)| \le C_{k,N} |(\xi,\eta)|^{-N}
  \]
  In particular, if $k, N = |\alpha|$, then the integrand in each summand on the RHS of inequality \ref{eqn:qbound} is bounded by an $|\alpha|$-dependent multiple of $(1+|(\xi,\eta)|)^{-|\alpha|}$ in the conic neighborhood of ${\cal N}$ defined by $|\xi| \le s|\eta|$. 

  On the other hand, note that
  \[
    s|\eta| \le |\xi| \Rightarrow 1 + |(\xi,\eta)| \le 1 + |\xi| + |\eta| 
    \le \frac{1+s}{s} (1+ |\xi|) 
  \]
  hence
  \[
    (1 + |\xi|)^{-|\alpha_1'|} \le \left(\frac{s}{1+s}(1+|(\xi,\eta)|)\right)^{-|\alpha_1'|}   
  \]
  In the complement defined by $|\xi|\ge s|\eta|$, the $\alpha_1'$  summand on the RHS of inequality \ref{eqn:qbound} is bounded by a multiple of
  \[
    \left(\frac{s}{1+s}(1 + |(\xi,\eta)|)\right)^{-|\alpha_1'|}|\partial_{\xi}^{\alpha_1-\alpha_1'}\partial_{\eta}^{\alpha_2}H(\xi,\eta) |
  \]
  which is in turn bounded by a multiple of
  \[
    \left(\frac{s}{1+s}\right)^{-|\alpha|} (1+|(\xi,\eta)|)^{-|\alpha|}
  \]
  because $H \in S^0(\bR^n)$.

  These bounds together cover all of $T^*(\bR^n)$, so imply the bounds necessary to conclude that $Q \in S^0(\bR^n)$.
    
  The assertion about the essential support is equivalent to existence of $C>0$ for any $\beta \in {\bf Z}_+^n, N \in {\bf Z}_+$ so that
  \[
    |(\partial_{\xi,\eta}^{\beta} Q)(x,y,\xi, \eta)| \le C(1+|(\xi,\eta)|)^{-N}
  \]
  for $\xi \notin ES(P)$ or $(\xi,\eta) \notin ES(H)$.
\end{proof}

\section{Computing Q}

\subsection{Cauchy problem}
\[
  \partial_t S(t) = -K S(t), S(0)=I
\]
\[
  \partial_t S(-t) = -S(-t) \partial_t S(t) S(-t) = S(-t)K
\]
\[
  \partial_t \delta S(t) = - \delta K S(t) - K \delta S(t)
\]
\[
  \partial_t S(-t)\delta S(t) = S(-t)K\delta S(t) - S(-t) \delta K S(t) - S(-t)K \delta S(t)
\]
\begin{equation}
  \label{eqn:qevol1}
  =  -S(-t)\delta K S(t)
\end{equation}

\subsection{PsiDO RHS}
Suppose that $K \in OPS^1$ is skew-adjoint, at least in principal
part: $K_1^*=-K_1$. Then there is unitary $U \in OPS^0$ so that
$U(x,\xi)^*K_1(x,\xi)U(x,\xi) = i\Lambda(x,\xi) =
i\mbox{diag}(\lambda_1(x,\xi),...\lambda_k(x,\xi))$, with scalar
$\lambda_j \in OPS^1, j=1,...,k$ (in fact, homogeneous of degree
1). Also,
\begin{equation}
  \label{eqn:ortho}
  U(x,D)^*U(x,D) = U(x,d)U(x,D)^* = I \mbox{ mod } OPS^{-1}
\end{equation}
Define $S(t)$ to be the one-parameter group satisfying
\[
  \partial_t S(t) = -K S(t), S(0)=I
\]
Note that $S(-t)$ is inverse to $S(t)$, and 
\[
  \partial_t S(-t) = -S(-t) \partial_t S(t) S(-t) = S(-t)K
\]
Hence
\[
  \partial_t S(t) = -iU \Lambda U^*S(t) + RS(t), R \in OPS^0 
\]
Write $\tilde{S} = U^*SU$. Then 
\[
  \partial_t \tilde{S}(t) = -i\Lambda \tilde{S}(t) + \mbox{ mod } OPS^0 
\]
If $u^0 = (u^0_1,...,u^0_k)$, then $\tilde{u}(t) = \tilde{S}(t)u^0$ satisfies
\[
  \partial_t \tilde{u}(t) = -i\Lambda \tilde{u}(t) + ...
\]
so $\tilde{u}_i$ solves
\[
  \partial_t \tilde{u}_i(t) = -i\lambda_i \tilde{u}_i(t), \, i=1,...,k
\]
Let $\tilde{S}_i$ solve
\begin{equation}
  \label{eqn:stilde}
  \partial_t \tilde{S}_i(t) = - i\lambda_i \tilde{S}_i(t)
\end{equation}
Then
\[
  \tilde{u}_i(t) = \tilde{S}_i(t) u^0_i = (\tilde{S}(t)u^0)_i 
\]
whence
\[
  \tilde{S} = \mbox{diag}(\tilde{S}_1,....,\tilde{S}_k).
\]
and
\begin{equation}
  \label{eqn:spec}
  S(t) = \sum_{i=1}^k U_i \tilde{S}_i(t)U_i^*
\end{equation}

Define
\[
  F = \phi_r S(T) \phi_s
\]
in which $\phi_,r \in C_0^{\infty}(\bR^{n+1})$ and $T>0$. Suppose that $K$
is perturbed by $\delta K \in OPS^1$, also essentially skew. Then
$\Lambda$, $U$, $S$, $\tilde{S}$ and $F$ undergo perturbations
$\delta \Lambda$, $\delta U$, $\delta S$, $\delta \tilde{S}$, and
$\delta F$. Our object is to understand the structure of $\delta F$, under the
assumption that $\phi_r \delta K \in OPS^{-\infty}$. 

This assumption on $\delta K$ implies that $\phi_r \delta U \in
OPS^{-\infty}$. 
\[
  \delta F =  \phi_r \delta S(T)  = \phi_r \left(\sum _{j=1}^k \delta U_j \tilde{S}_j(T) U_j^*
\right.
\]
\[
 +  \left. \sum_{j=1}^k U_j \delta\tilde{S}_j(T)U_j^* + \sum_{J=1}^k U_j
    \tilde{S}_j(T)\delta U_j^*\right) 
\]
Every term in the first sum on the RHS is bounded $H^s \rightarrow H^{s-1}$ since
$\phi_r \delta U_i \in OPS^{-\infty}$.  Since $[\phi_rU_i], \in OPS^{-1}$, 
\[
  \phi_r  U_j \delta \tilde{S}_j(T)  U_j^*  = U_j \phi_r
  \tilde{S}_j(T) U_j^*
  +...
\]
where the elided terms are operators mapping $H^s$ to $H^{s-1}$, any
$s$. So,
\[
 \delta F = \sum_{j=1}^k U_j \phi_r \left( \delta\tilde{S}_j(T) U_i^*
   + \tilde{S}_j(T) \delta U_j^*\right)
  + ...
\]
and
\[
 S(-T)\delta F =\left(\sum_{i=1}^k U_i \tilde{S}_i(T)
    U_i^*\right) \left(\sum_{j=1}^k U_j \phi_r ( \delta\tilde{S}_j(T) U_i^*
    + \tilde{S}_j(T) \delta U_j^*)\right)
\]  
Using the relation \ref{eqn:ortho},
The first order perturbation of the definition \ref{eqn:stilde} is
\begin{equation}
  \label{eqn:dstilde}
  \partial_t \delta \tilde{S}_i(t) = - i\lambda_i \delta
  \tilde{S}_i(t) - i\delta \lambda_i \tilde{S}_i(t)
\end{equation}
Since $\tilde{S_i}(-t)$ is inverse to $\tilde{S}_i(t)$,
\[
  \partial_t \tilde{S}_i(-t)= -\tilde{S}_i(-t) (\partial_t \tilde{S}_i(t)) \tilde{S}_i(-t)
\]
\begin{equation}
  \label{eqn:stildeinv}
= -\tilde{S}_i(-t)(-i\lambda_i \tilde{S}_i(t)) \tilde{S}_i(-t) = i\tilde{S}_i(-t)\lambda_i 
\end{equation}
So
%\[
% \partial_t (\tilde{S}_i(-t) \phi_r \delta \tilde{S}_i(t) \phi_s) 
%\]
%\[
%  = i\tilde{S}_i(-t)\lambda_i \phi_r \delta \tilde{S}_i(t) \phi_s +
%  \tilde{S}_i(-t) \phi_r (- i\lambda_i \delta
%  \tilde{S}_i(t) - i\delta \lambda_i \tilde{S}_i(t)) \phi_s
%\]
%\[
%  = -i\tilde{S}_i(-t)[\lambda_i,\phi_r] \delta \tilde{S}(t) \phi_s -i
%  \tilde{S}_i(-t)\phi_r\delta\lambda_i\tilde{S}_i(t) \phi_s
%\]
%\begin{equation}
%  \label{eqn:qevol1}
%  = -i\tilde{S}_i(-t)[\lambda_i,\phi_r]\tilde{S}_i(t) \tilde{S}_i(-t)\delta \tilde{S}(t) \phi_s %-i
%  \tilde{S}_i(-t)\phi_r\delta\lambda_i\tilde{S}_i(t) \phi_s.
%\end{equation}
%Similarly,
\begin{equation}
  \label{eqn:dstildeconj}
  \partial_t\tilde{S}_i(-t)\delta \tilde{S}_i(t) = -i\tilde{S}_i(-t)\delta \lambda_i \tilde{S}_i(t).
\end{equation}
From \cite{Tay:81}, Egorov's Theorem (Section VIII.1), for $p \in OPS^s(\bR^n)$,
\begin{equation}
  \label{eqn:egorov}
   \tilde{S}_i(-t) p \tilde{S}_i(t)=  p \circ C_i(t),\, p\tilde{S}_i(t) =
   \tilde{S}_i(t)p \circ C_i(t)
\end{equation}
in which $C_i$ is the one-paraimeter family of symplectic maps given by
\begin{equation}
  \label{eqn:canon}
  C_i(t)(x,\xi)=(y_i(t;x,\xi),\eta_i(t;x,\xi))
\end{equation}
Here $t \rightarrow y_i(t;x,\xi),\eta_i(t;x,\xi) $is the integral curve of the
Hamiltonian vector field
\[
  H_i(x,\xi) = (\partial_{\xi}\lambda_i(x,\xi), - \partial_x\lambda_i(x,\xi))
\]
with $y_i(0;x,\xi)=x, \eta_i(0;x,\xi)=\xi$.

For $p \in OPS^{\infty}$, define
\[
  \Gamma_i[p] = \int_0^T dt \, p \circ C_i(t)
\]
Then from the identity \ref{eqn:dstildeconj} follows
\[
  \delta\tilde{S}_i(T) = \tilde{S}_i(T) \Gamma_i[ -i \delta \lambda_i]
\] 
so
\[
  \delta F = \sum_{i=1}^k U_i \phi_r \delta \tilde{S}_i \phi_s U_i^*
\]
\[
  =\sum_{i=1}^k U_i \phi_r \tilde{S}_i(T) \Gamma_i[ -i \delta \lambda_i]\phi_s U_i^*
\]
\[
 Q = -i\sum_{i=1}^k U_i \tilde{S}_i(-T) \phi_r \tilde{S}_i(T) \Gamma_i[\delta \lambda_i] U_i^*\phi_s
  + ...
\]
\[
  = -i \sum_{i=1}^k U_i (\Gamma_i[\phi_r] \Gamma_i[\delta \lambda_i] )
  U_i^*
\]

 
The right hand side of equation \ref{eqn:qevol1} contains three
expressions of the form appearing on the LHS of \ref{eqn:egorov}.

\subsection{Generic H-J}
Generic 1st order PDE:
\begin{equation}
  \label{eqn:gen}
  F(x, \partial_x u) = 0.
\end{equation}
Suppose that $(x,p)$
is curve in phase space, and
\begin{equation}
  \label{eqn:defp}
  p(s) = \partial_x u(x(s)).
\end{equation}
Then
\[
  \frac{d}{ds}u(x(s))=p(s) \cdot \frac{dx}{ds}(s)
\]
Also
\[
  \frac{dp}{ds}(s) = \partial_x \partial_x u(x(s))\cdot
  \frac{dx}{ds}(s).
\]
Note that
\[
  0 = \partial_x( F(x, \partial_x u(x))) = (\partial_x F)(x,
  \partial_x u(x)) + 
  (\partial_p F)(x, \partial_x u(x)) \partial_x \partial_x u(x)
\]
So if
\begin{equation}
  \label{eqn:H1}
  \frac{dx}{ds} = \partial_p F(x,p)
\end{equation}
then
\begin{equation}
  \label{eqn:H2}
  \frac{dp}{ds} = -\partial_x F(x,p)
\end{equation}
That is: if $u$ is a solution of $F(x,\partial_x u)=0$ and
$(x(s),p(s))$ satisfies
\[
  p(s)=\partial_x u(x),\, \frac{dx}{ds} = (\partial_p
  F)(x(s),\partial_x u(x(s)),
\]
then $(x(s),p(s))$ is a trajectory of the system \ref{eqn:H1},
\ref{eqn:H2}, and
\begin{equation}
  \label{eqn:sol}
  u(x(s))=u(x(0)) + \int_0^t ds p(s)\cdot (\partial_p F)(x(s),p(s)).
\end{equation}
Because the system the system \ref{eqn:H1},
\ref{eqn:H2} is uniquely solvable, the result \ref{eqn:sol} holds for
any solution $u$ of \ref{eqn:gen}, for $x(0)$ in the domain of
definition of $u$, $p(0)=\partial_xu(x(0))$, and a trajectory $(x,p)$
solving the system \ref{eqn:H1}, \ref{eqn:H2}, so long as $x(s)$
remains in the domain of $u$.

This observation implies a similar result for the H-J system
\begin{equation}
  \label{eqn:hjdef}
  0 = \partial_t u + H(t,\bx,\partial u).
\end{equation}
Set $X=(t,x)$, $P=(q,p)$ (with $q$ dual to $t$, $F(X,\partial_X u) =
\partial_t u + H(t,x,\partial_x u)$, so the Hamiltonian is $F(X,P) =
q+H(t,x.p)$.  Thus
\begin{eqnarray*}
  \frac{dt}{ds} & = & \partial_q F = 1 \\
  \frac{dx}{ds} & = & \partial_p F = \partial_p H \\
  \frac{dq}{ds} & = & -\partial_t F = -\partial_t H \\
  \frac{dp}{ds} & = & -\partial_x F = -\partial_x H
\end{eqnarray*}
From the first equation, $t=s$ up to a constant shift. From the third,
\[
  q(t) = -\int_0^t ds \partial_tH(s,x(s),p(s))
\]
\[
  = -\int_0^t ds \frac{d}{ds}H(s,x(s),p(s)) - \frac{dx}{ds}(s)\cdot
  \partial_x H(s,x(s),p(s)) -\frac{dp}{ds}(s)\cdot \partial_p
  H(s,x(s),p(s))
\]
\[
  = -H(t,x(t),p(t)) + H(0,x(0),p(0))
\]
in view of the 2nd and 4th equations.
Note that these are the Hamiltonian system with Hamiltonian $H$.

The result \ref{eqn:sol} becomes
\begin{equation}
  \label{eqn:hsol}
  u(t,x(t))=u(0,x(0)) +tH(0,x(0),p(0))+ \int_0^t ds (-H(s,x(s),p(s)) + p(s)\cdot (\partial_p H)(s,x(s),p(s))).
\end{equation}

\section{Symmetric Hyperbolic Systems}

If $A_j$ is positive definite, then $x_j$ can play the role of time, and the Cauchy problem with initial data $u_0 \in L^2(\bR^n,\bR^k)$  on $x_j=$ constant has a unique weak solution $u \in L^2(\bR^{n+1})^k$ with well-defined $L^2$ traces on any $x_j$ level set (Lax, Taylor). Want to understand traces on other hypersurfaces, to begin with coordinate hypersurfaces, wlog $x_0=0$.

Notation:$x_0=t$, $\partial_t=\partial_0$, $\bx = (x_1,...,x_n)$, $D=(\partial_1,...\partial_n)$
\[
  K(t,\bx,D)= \sum_{i=1}^n A_i(t,\bx)\partial_i, \, M(t,\bx) = A_0(t,\bx).
\]
If $M$ is positive definite, trace properties coverd by standard theory (above). Let $\Pi_0(t,\bx)$ = projection on null space $V_0$ of $M(t,\bx)$, $\Pi_1=I-\Pi_0$ = projection on range $V_1$. Denote the rank of $M$ by $l={\rm dim} V_1$ Then
\[
  M\partial_t u + Ku = \Pi_1 M \partial_t \Pi_1 u + \sum_{\alpha,\beta=0,1} \Pi_{\alpha}K\Pi_{\beta}u + \mbox{ l.o.t. }
\]
Since the ranges are orthogonal,
\begin{eqnarray}
  \Pi_1 M \partial_t \Pi_1 u + \Pi_1 K \Pi_1 u + \Pi_1 K \Pi_0 u & \in & L^2_{\rm loc}(\bR^{n+1},V_1),
  \label{eqn:decomp1} \\
  \Pi_0 K \Pi_0 u + \Pi_0 K \Pi_1 u & \in & L^2_{\rm loc}(\bR^{n+1},V_0)
  \label{eqn:decomp0}
\end{eqnarray}
Set $K_{\alpha,\beta} = \Pi_{\alpha}K\Pi_{\beta}$.

Main assumption: for all $t \in \bR, (\bx,\xi) \in T^*(\bR^n)\setminus \{0\}$, $K_{0,0}(t,\bx,\xi)$ is invertible on $V_0$.

% Suppose that $E \in OPS^1(\bR^{n+1})$ is scalar, $Eu \in H^1_{\rm loc}(\bR^{n+1})$. Denote by $i(t):\{t\} \times \bR^n \rightarrow \bR^{n+1}$ the embedding. contained in $i(0)^*(ES(E))$, and that and

% Suppose that $\Gamma_0, \Gamma \subset T^*(\bR^n)$ are conic open sets with $\overbar{\Gamma_0 \subset \Gamma$ and $\delta t >0$. Assume that  that for $|t|\le \delta t$ and $(\bx,\xi) \in Gamma$, $K_{00}(0,\bx,\xi)$ is invertible as a map $V_0 \rightarrow V_0$.
Let  $L_{00}(t,\bx,\xi) \in C^{\infty}(\bR, S^{-1}(\bR^n))$ be inverse to $K_{0,0}(t,\bx,\xi)$ on $V_0$: that is,
\[
  K_{0,0}(t,\bx,\xi)L_{0,0}(t,\bx,\xi) v=v=L_{0,0}(t,\bx,\xi)K_{0,0}(t,\bx,\xi)v
\]
for all $v \in V_0$. Construct $L_{0,0}$ to be homogeneous of degree -1 outside of a neighborhood of $\xi=0$, with $V_1$ an invariant subspace for $L_{0,0}(t,\bx,\xi)$ on which it vanishes. It follows that $L_{0,0}$ is symmetric, and that
\[
  \Pi_0 u - L_{0,0}K_{0,0}\Pi_u = \Pi_0 u - L_{0,0}K_{0,0}u = \Pi_0 u + L_{0,0}K_{0,1}u \in H^{1}_{\rm loc}(\bR^{n+1})^k
\]
so
\[
  K_{1,0}u + K_{1,0}L_{0,0}K_{0,1}u \in L^2 _{\rm loc}(\bR^{n+1})^k
\]
Define
\[
  \tilde{K} = K_{11} - K_{10}L_{00}K_{01}.
\]
Then equation \ref{eqn:decomp1} can be rewritten
\begin{equation}
  \label{eqn:shs1}
  M\partial_t u_1 + \tilde{K}u_1 \in L^2_{\rm loc}(\bR^{n+1},V_1)
\end{equation}

Note that for each $t \in \bR, (\bx,\xi) \in T^*(\bR^n)$, both $V_0$ and $V_1$ are invariant subspaces of $\tilde{K}(t,\bx,\xi)$, and $V_0 \subset {\rm ker}\tilde{K}(t,\bx,\xi)$. Also, $M$ is invertible when restricted to $V_1 = {\rm Rng} M = {\rm Rng } \Pi_1$ , but not (necessarily) positive definite.
  
Restricted determinant: if $V$ is an invariant subspace of $A$, then ${\rm det}_{V} A$ is the determinant of the matrix of $A$ in any orthonormal basis of $V$.
For $(t,\bx,\tau,\xi) \in T^*(\bR^n)$, define the restricted characteristic polynomial
\[
  {\cal C}(t,\bx,\tau,\xi) = {\rm det}_{V_1} (M(t,\bx)\tau + \tilde{K}(t,\bx,\xi) )= 0
\]
Since $V_0$ and $V_1$ are invariant subspaces of the matrix in parentheses, and $V_0$ is (contained in) its null space, there are at most $l={\rm dim}V_1$ non-zero roots in $\tau$ for $\xi \ne 0$.

Choose a conic submanifold $\gamma \subset T^*(\bR^n)$, so that for $|t| \le \delta t, (\bx,\xi) \in \gamma$ there are exactly $m \le l$ distinct real roots $\tau$ of ${\cal C}(t,\bx,\tau,\xi)$. Denote these by $\tau_i(t,\bx,\xi), i=1,...,m$.

Choose a conic $\gamma_1 \subset T^*(\bR^n)$ so that $\bar{\gamma_1} \subset \gamma$, and $Q \in OPS^0(\bR^n)$ with $ES(Q) \subset \gamma$, $Q(\bx,\xi) =1$ for $(\bx,\xi)\in \gamma_1$.

Let ${\cal N} = $ normal bundle of $x_0=0$ and $\Gamma \subset T^*(\bR^n)$ for which $\Gamma \cap {\cal N} = \emptyset$.

Assume that $u \in L^2(\bR^{n+1}) \cap H^1_{\rm loc}(\Gamma \cap \Pi_{\cal N}^{-1}(\gamma_1^C))$.



For each $\tau_i, i =1,...m$, choose $U_i(t,\bx,\xi) \in C^{\infty}(\bR \times T^*(\bR^n),V_1(t,\bx))$,  so that
\[
  \tau_i M(\bx) U_i(t,\bx,\xi) + \tilde{K}(t,\bx,\xi) U_i(t,\bx,\xi)=0, \, i=1,..,m.
\]
$U_1,...,U_m$ span a subspace $\tilde{V}_1(t,\bx,\xi)$. Scale $U_i$ so that $|U_i(t,\bx,\xi)| = 1$.

Note that $\tau_i(t,\bx,\xi), i=1,...,m$ are homogeneous of order 1 in $\xi$ and smooth in $\bx$, $U_i(t,\bx,\xi), i=1,...,m$ are homogeneous of order 0 in $\xi$, smooth in $\bx$. So both are symbol-valued functions, of orders 1 and 0 respectively. Since the $U$'s must be orthogonal, $U_i^TU_j = \delta_{ij}$.

Suppose first that $m=l$, that is, that $\tilde{V}_1=V_1$. Then $\{U_i(t,\bx,\xi):i=1,..,l\}$ is a basis for $V_1(t,\bx)$ for each $(t,\bx,\xi \in \Gamma$. Set $u_{1,i}(t,\bx) = U_i(t,\bx,D_x)^Tu_1$. Then for any $w \in L^2(\bR^{n+1},V_1)$
\[
  \sum_{i=1}^l U_i(t,\bx,D)U_i(t,\bx,D)^Tw(t,\bx) \approx
\]
\[
  \int \, d\xi \, e^{i \bx \cdot \xi} U_i(t,\bx,\xi)U_i(t,\bx,\xi)^T\hat{w}(t,\xi) \approx
\]
\[
  \int \, d\xi \, e^{i \bx \cdot \xi}
\]
\[
  \tilde{K}(t,\bx,D)u_{1,i} = \tilde{K}(t,\bx,D)U_i(t,\bx,D)^T
\]
\[
  M(\bx)\partial_t u_{1,i}(t,\bx) = -\tilde{K}(t,\bx,D)
\]

\subsection{2D Acoustics}
%###################
So $V_0={\rm span}(\{e2\}),\, V_1={\rm span}(\{e1,e3\}), \Pi_0 = {\rm diag}(0,1,0)$ 
\[
  K(t,\bx,\xi) =  A_1(\bx) \xi_1 + A_2(\bx) \xi_2 =
  \left(
    \begin{array}{ccc}
      \frac{1}{\kappa(\bx)}\omega & \xi_x & 0 \\
      \xi_x & \rho(\bx) \omega & 0 \\
      0 & 0 & \rho(\bx) \omega
    \end{array}
  \right)
\]
whence $K_{0,0} = {\rm diag}(0, \rho(\bx) \omega, 0)$.
\[
  K_{1,1}(t,\bx,\xi) =
  \left(
    \begin{array}{ccc}
      \frac{1}{\kappa(\bx)}\omega & 0 & 0 \\
      0 & 0 & 0 \\
      0 & 0 & \rho(\bx) \omega
    \end{array}
  \right),
\]
\[
  K_{0,1}(t,\bx,\xi) =
  \left(
    \begin{array}{ccc}
      0 & \xi_x & 0 \\
      0 & 0 & 0 \\
      0 & 0 & 0
    \end{array}
  \right)
\]
and $K_{1,0} = K_{0,1}^T$.
  
Choose $\omega_0>0$, $\phi \in C^{\infty}(\bR)$ so that $\phi(\omega) = 1$ for $\omega> 2\omega_0$, $=0$ for $\omega < \omega_0$. Set
\[
  L_{0,0}= {\rm diag}\left(0, \frac{\phi(|\omega|)}{\rho(\bx) \omega}, 0\right)
\]
Then $L_{0,0}K_{0,0} = I$ restricted to $V_0$ and away from zero frequency, and
\[
  \tilde{K} = K_{1,1} - K_{1,0}L_{0,0}K_{0,1} =
  \left(
    \begin{array}{ccc}
      \frac{\omega}{\kappa} - \frac{\xi_x^2\phi(|\omega|)}{\rho \omega} & 0 & 0\\
      0 & 0  & 0 \\
      0 & 0 & \rho \omega
    \end{array}
  \right).
\]
Since $V_1={\rm span}(e_1,e_3)$, the determinant of the restricted characteristic polynomial is obtained by dropping the second row and column:
\[
  \det
  \left(
    \begin{array}{cc}
      \frac{\omega}{\kappa} - \frac{\xi_x^2\phi(|\omega|)}{\rho \omega} &  \xi_z\\
      \xi_z &  \rho \omega
    \end{array}
  \right).
\]
\[
  = \frac{\rho \omega^2 }{\kappa} - \xi_x^2\phi(|\omega|) - \xi_z^2
\]
\[
  = \frac{\rho \omega^2 }{\kappa} - \xi_x^2 - \xi_z^2 \mbox{ mod }  S^{-\infty}
\]  
as expected. So a maximal choice for $\Gamma$ would be
\[
  \Gamma = \{(z,(t,x),(\omega,\xi_x)): c(z,x)|\xi_x| < |\omega|\}.
\]



\bibliographystyle{seg}
\bibliography{../../bib/masterref}
