%% This file is automatically generated. Do not edit!
\documentclass[notimes]{georeport}

\usepackage{hyperref}
\usepackage{natbib}
\usepackage{graphicx}
\usepackage{color}
\usepackage{listings}
\usepackage{amsmath}
\usepackage{amsthm}
\usepackage{amssymb}
\usepackage{amsbsy}
\usepackage{float}
\usepackage{wasysym}
\usepackage{setspace}
\usepackage{bm}


\newcommand{\mb}{\mathbf}
\DeclareMathAlphabet{\mathcal}{OMS}{cmsy}{m}{n}
\newcommand{\bv}{\mathbf{v}}
\newcommand{\bx}{\mathbf{x}}
\newcommand{\tp}{\tilde{p}}
\newcommand{\td}{\tilde{d}}
\newcommand{\te}{\tilde{e}}
\newcommand{\tw}{\tilde{w}}
\newcommand{\tf}{\tilde{f}}
\newcommand{\tc}{\tilde{c}}
\newcommand{\tv}{\tilde{v}_z}
\newcommand{\tu}{\tilde{u}}
\newcommand{\tS}{\tilde{S}}
\newcommand{\tk}{\tilde{\kappa}}
\newcommand{\tA}{\tilde{A}}
\newcommand{\wl}{w_{\lambda}}
\newcommand{\dl}{d_{\lambda}}
\newcommand{\tdl}{\tilde{d}_{\lambda}}
\newcommand{\tul}{\tilde{u}_{\lambda}}
\newcommand{\bR}{\mathbf{R}}
\newcommand{\cF}{{\cal F}}
\newtheorem{exmp}{Example}
\newtheorem{theorem}{Theorem}


\setfigdir{Fig}

\begin{document}
\address{email: {\tt symes@rice.edu}}
\title{Responses to reviews of ``Wavefield Reconstruction Inversion: an example'', submitted to Inverse Problems}
\author{William. W. Symes}

\lefthead{Symes}

\righthead{1D WRI}

\section{Responses to Referee Comments}
I very much appreciate the careful and constructive criticisms provided by the referees, and have attempted to address every point they have made, at the cost of lengthening the paper somewhat.

\section{Referee 1}

\noindent Comment: (1)The theory is credible, however, I also suggest that you’d better supply some numerical examples;

\noindent Response: I have added a short section with a numerical example.

\noindent Comment: (2)In the section of abstract, I suggest that you should focus on our new method;

\noindent Response: Indeed, I would love to do so, but you have the advantage of me: I don't know who you are, and even more don't know what your ``new method'' is. Also note that abstracts should not include references.

\noindent Comment: (3)Equations (1),(4)-(7), (9), (12)-(14), (19), (21), (23)-(26), (28)-(32): lack of punctuation marks;

\noindent Response: thank you - I have attempted to properly punctuate every equation, and appreciate you calling these out.

\noindent Comment: (4)Pg. 2, line 12: “((Bamberger et al.,” Please delete a “(”.

\noindent Response: thank you for pointing this out.

\noindent Comment: (5)DISCUSSION: operator-valiued-->operator-valued;

\noindent Response: Fixed, thanks. 

\noindent Comment: (6)In the section of Introduction, please supply the researches in the last three years.

\noindent Response: I have added more references, to make clear tha work on WRI continues right up to the present. This paper is not meant to be a review of WRI - that is for someone more involved in it to write. I just want to make the point that the claims offered for WRI as a cycle-skip remedy are clearly wrong, by showing a counter-example to which these claims should also apply.

\section{Referee 2}

\noindent Comment: on the relation between WRI and MSWI.

\noindent Response: You have made me aware that my first draft was too glib about the relation between these various approaches. I have added a full description of the relation between ``traditional'' WRI as formulated by van Leeuwen and Herrmann, and the variant of MSWI that I call WRI in this paper. In fact they are completely equivalent under conditions that hold for the problem I consider, in that (1) the original formulation of the vL-H GJI 2013 paper and the VPM reduction as in their 2016 IP paper have stationary points and local minima in 1-1 correspondence, and (2) the VPM-WRI of the 2016 IP paper and the VPM reduction of the problem stated in display 2 of my paper also have stationary points and local minima in 1-1 correspondence - it could not be otherwise, as these two reduced objectives are the same function! The equivalence (1) is a consequence of the separable part of the objective (that is, the dependence on the state variable) being strictly convex quadratic. In any case, while this discussion bulks up the paper a bit, I agree that it is really necessary to fully support my conclusions - namely, it would not matter what version of WRI you use, you will wind up with the same behaviour.

I also gave a more detailed (but still rough) account of the theoretical results of van Leeuwen and Herrmann from their 2016 IP paper, and tried to identify clearly the reason advanced in most papers on the topic to think WRI might overcome cycle-skipping - namely bigger search space, that is, merely the fact that it uses extended modeling. Further, I pointed out that the theory developed in my paper applies precisely to the 1D transmission problem, and not to any other WRI implementation (necessarily). However the arguments in favor of WRI also apply, and therefore are shown to be (at least) incomplete.

\noindent Comment: on other approaches to convexifying FWI.

\noindent Response: I have added a few sentences in the introduction, with some recent references, on the use of the Wasserstein metric. Also, I identified WRI as being based on extended modeling, and gave a reference which includes a discussion of MVA and related ideas (and some historical stuff, as extended modeling is really an old story). You are right that these things should be mentioned, and of course more could be said, eg. about tomography, but I hesitate to do muich more than this, as the paper begins already to become a bit top-heavy.

\noindent Detailed Comments:

\noindent Response: many thanks! I believe I have addressed them all.

\noindent [Except for the first: ``...requires that the intial estimate predict...'' is a weird bit of English grammar called a modal verb phrase - ``requires that'' is the modal verb (a compound, unlike ``must'' or ``should'') and ``predict'' is the auxiliary verb, which is always used in its base (infinitive) form. So no ``s''. Another example: ``he must dig a hole'' vs. ``he digs a hole''. This is really obscure, sorry, there is nothing more confusing than English grammar.]

\end{document}
