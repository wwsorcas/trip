\title{Wavefield Reconstruction Inversion: an example}
\author{William. W. Symes \thanks{The Rice Inversion Project,
Department of Computational and Applied Mathematics, Rice University,
Houston TX 77251-1892 USA, email {\tt symes@caam.rice.edu}.}}

\lefthead{Symes}

\righthead{1D WRI}

\begin{abstract}
  Full waveform inversion, or nonlinear least squares data-fitting
  driven by wave equation simulation, has proven its ability to
  extract detailed maps of earth structture from near-surface seismic
  observations. However, the least squares error function at the heart
  of this method tends to develop a high degree of nonconvexity, so
  that local optimization methods (the only numerical methods feasible
  for field-scale problems) tend to fail, unless provided with initial
  estimates of a quality not always available. A number of alternative
  optimization principles have been advanced that promise some degree
  of release from the multimodality of Full Waveform Inversion,
  amongst them Wavefield Reconstruction Inversion, the focus of this
  paper. Applied to a simple 1D acoustic transmission problem, both
  Full Waveform and Wavefield Reconstruction Inversion methods reduce
  to minimization of explicitly computable functions, in an asymptotic
  sense. The analysis presented here shows that Wavefield
  Reconstruction Inversion can be vulnerable to the same
  ``cycle-skipping'' failure mode as Full Waveform Inversion.
\end{abstract}

\section{Introduction}

Full waveform inversion (FWI) is the current nomenclature in the
seismology literature for data-fitting earth structure estimation
driven by wavefield modeling. The earth properties to be estimated
(material densities, stiffnesses, attenuation rates,...) form a vector
$c$ of spatially fields that appear as coefficients in systems of hyperbolic
partial differential equations, modeling seismic wave
propagation. The wavefields used in structure estimation (``imaging'')
are small motion disturbances of the earth's equilibrium state, so the
equations of motion are typically 
linear(ized). Right-hand side vectors $f$ in these systems model energy
input that initiates waves (earthquakes, man-made sources such as
explosives or mechanical vibrators). Data vectors $d$ are simulated by
sampling the solution fields at the locations of measurement devices
(accelerometers, microphones,...)  over appropriate time
intervals. The relation between the energy source $f$ and the
simulated data is linear in $f$, but nonlinear in the coefficient
vector $c$, so is naturally represented by a family of linear
operators $S[c]$ parametrized by the coefficient vector $c$.

The objective of FWI is to find $c$ and $f$, given $d$, so that
$S[c]f \approx d$. A typical method for enforcing this requirement is
the minimization of an objective misfit measure (objective, for
short), the most common choice being the square norm of a Hilbert
space in which the data is presumed to reside:
\begin{equation}
  \label{eqn:deffwi}
  \mbox{Given } d, \mbox{ find }c \mbox{ and }f \mbox{ to
    minimize }
  \|d -  S[c]f\|^2
\end{equation}
in which $\|\cdot\|$ is the norm in a suitable Hilbert space. This
approach was first suggested in the 1980's
(\cite{BamChavLai:79,Tara:84a,KolbColLai:86,Crasetal:90}, and many
other papers since then). Usually some form of regularization is
applied, to compensate for poorly determined aspects of $c$ and/or
$f$, as is explained in Tarantola's influential book
\cite[]{Tarantola:05}. Also, $f$ may be constrained in one way or
another to embody characteristics of field energy sources, or even
regarded as known (an example of this is given below).

Within a few years of its introduction into quantitative seismology,
FWI was understood to suffer from a severe limitation. Because of the
typical dimensions of earth models, and consequent cost of accurate
computation of $S$, iterative local optimization provides the only
feasible route to estimation of $c$ via solution of the optimization
problem \ref{eqn:deffwi}. However the objective function of this
optimization problem (the mean-square residual appearing in display \ref{eqn:deffwi}) has many local minima in general, most
having nothing to do with a usable estimate of earth structure. [An
explicit example of this multi-modal behaviour appears below.]
Reliable estimation of $c$ via iterative local optimization requires
that the initial estimate predict the correct arrival time of waves,
as they appears in the data, within a wavelength at dominant
frequencies \cite[]{GauTarVir:86,VirieuxOperto:09}.

Despite this severe constraint, FWI has shown enough promise as a tool
for both industrial and academic seismology that it is now a
mainstream research topic, and to some extent a commercial
product. Estimation of sufficiently accurate initial models via
non-FWI methods is common practice, though what ``sufficiently
accurate'' means may be difficult to discern
\cite[]{Plessix:10}. The wavelength criterion mentioned in the last
paragraph may be made less onerous by collection of relatively
low-frequency data \cite[]{Wolfspar:16}. Finally, many alternatives to
straightforward least-squares data fitting have been suggested, some
of which appear to exhibit less tendency to develop local minima than
does the problem described in \ref{eqn:deffwi} \cite[]{Symes:09}.

This topic of this paper is one of these alternative approaches,
Wavefield Reconstruction Inversion (``WRI''). WRI was introduced by \cite{LeeuwenHerrmannWRI:13}, and further
developed by \cite{LeeuwenHerrmann:16,WangYingst:SEG16} and other
authors. It is based on the presumption that the correct source (right
hand side in the equations of motion) $q$ is
known, and combines a penalty for data misfit with a penalty for
failing to solve the equations of motion with the correct right hand side:
\begin{equation}
  \label{eqn:defwri}
  \mbox{Given } d \mbox{ and }q, \mbox{ find }c \mbox{ and }f \mbox{ to
    minimize }
  \|d -  S[c]f\|^2+\alpha^2\|f-q\|^2.
\end{equation}
The penalty parameter $\alpha$ may be chosen small, and examples given
by \cite{LeeuwenHerrmannWRI:13} and others suggest that the problem
defined in display \ref{eqn:defwri} is less likely to develop
uninformative local minima if $\alpha$ is small. This might be so for
several reasons: one is that when less weight is
put on making the residual in the wave equation ($f-q$) small, 
$f$ may be chosen to make the data residual $d-S[c]f$ small
instead. One aspect of failure to predict the arrival times of waves
accurately is that small data residual is then difficult to
achieve. By maintaining fidelity to
the data, local minimization of the WRI objective might provide a
spurious-minimum-free path to a satisfactory estimate of $c$. 

The main result of this paper is that in one simple case, in which all
of the necessary computations can be carried out by hand, this hope is
not realized: minimization of the WRI objective is just as likely to
be trapped in a spurious local minimizer as is the FWI objective.  The
context of this conclusion is a simple transmission inverse problem
for the 1D acoustic wave system, which models pulse transmission along
a 1D continuum from a source point to a receiver point. The pulses
used in this thought experiment are short, so the main information
content of the data is the time of transit from source to
receiver. The predominant information about the material model, in
this case reduced to the wave velocity (a scalar function of
position), is just this transit time, so I constrain both the target
wave velocity $c_*$ generating the data and the trial wave velocity
$c$ to be constant, that is, independent of position along the 1D
continuum. I introduce a family of inverse problems, parametrized by a
parameter $\lambda$ playing the role of wavelength. For sufficiently small
$\lambda$, it is possible to show explicitly that the FWI problem
possesses local minimizers far from the global minimizer at the target
$c_*$, and that initiating a local iterative optimization, such as
steepest descent or Newton's method, at a distance from $c_*$ bounded
below by a multiple of the ``wavelength'' $\lambda$ will result in
convergence to these spurious local minima. That is, FWI 
behaves in exactly the manner described in much of the literature on
this topic. However, an analysis of WRI applied to the same context
yields the same result: spurious local minima exist for sufficiently
small $\lambda$, and will be found by local optimization unless the
starting point is within $O(\lambda)$ of the target. That is, WRI
behaves in a manner qualitatively indistinguishable from FWI. In
particular, its ability to allow good fit to data for small $\alpha$
does not safeguard it from failure to converge globally to a ``good''
local minimum.

This paper begins with a description of the 1D inverse transmission
problem and explicit computation of various components of the FWI
approach, based on explicit solution of the 1D acoustic system
presented in Appendix A. In the third section I introduce the $\lambda-$dependent
family of problems, and establish the asymptotic properties of FWI as
$\lambda \rightarrow 0$. The fourth section develops the algebraic
structure of WRI, culminating in a remarkable identity revealing WRI
to be equivalent to minimization of a weighted norm of the data
residual, with a weight operator depending on the coefficient vector
$c$. This result is quite general, applying to essentially any
realization of WRI, and already shows that its behaviour must be
closely related to that of FWI. In the fifth section I return to 
1D acoustics problem and the $\lambda$-dependent family of inverse
problems, compute that weight operator explicitly, and deduce the
global behaviour of WRI in this instance. The paper ends with a
discussion of the relation of the analysis presented here to a
previous analysis of optimization formulations of wave inversion
problems based on parameter-dependent quadratic forms
\cite[]{StolkSymes:03}, and a brief discussion of some other
alternatives to FWI.

\section{FWI for 1D Acoustics}
The example of FWI to be explored in this paper is one of the simplest possible, based on the 1D acoustics system connecting excess pressure $p$, particle velocity $v$, sonstitutive law defect (``source'') $f$, density $\rho$, and wave velocity $c$:
\begin{eqnarray}
\label{eqn:awe1d}
\frac{\partial p}{\partial t} + \rho c^2\frac{\partial 
  v}{\partial z} &=& f \nonumber\\
\rho \frac{\partial v}{\partial t} + \frac{\partial p}{\partial 
  z}&=&0\nonumber\\
 p,v&=&0, t \ll 0. 
\end{eqnarray}
The fields $p,v,f$ are functions of spatial position $z \in \bR$ and time $t \in \bR$, whereas $c, \rho$ are functions of $z$ alone, so that the system \ref{eqn:awe1d} is autonomous.

The system \ref{eqn:awe1d} has classical (smooth) solutions $(p,v)$
when $c, \rho,$ and $f$ are smooth, and $\log c$ and $\log \rho$ are
bounded on $\bR$, as is well-established \cite[]{Lax:PDENotes}. In
this paper, for reasons to be discussed below, $c$ and $\rho$ are
constrained to be constant ($z$-independent) in which case solutions
may be constructed by elementary methods (Appendix A). 

In fact, I shall assume $\rho>0$ to be fixed for the remainder of this
paper, that is, not updated in the inversion process. The wave
velocity $c$ will range over an interval: it is the parameter to be inverted.
To be specific, choose $c_{\rm max} > c_{\rm min} > 0$, and require
that $c$ satisfy $c_{\rm min} \le c \le c_{\rm max}$.

Limit observations to the time interval $[0,T]$, at the spatial (``receiver'')  location $z_r$. The modeling operator outputs the pressure trace $p(z_r,t)$ over the time interval $[0,T]$:
\begin{equation}
  \label{eqn:defmod}
  S[c]f = p|_{\{z_r\}\times [0,T]}
\end{equation}

The formulation of the inverse problem via least-squares requires a choice of Hilbert space structure for the domain and range of $S[c]$. A natural choice is
\begin{equation}
  \label{eqn:defmoddom}
  S[c]: L^2([z_{\rm min},z_{\rm max}] \times \bR) \rightarrow L^2[0,T]
\end{equation}
As it will turn out, the choice of spatial interval $[z_{\rm min},z_{\rm max}]$ is arbitrary, so long as it has positive length and contains $z_s$ in its interior. 

For homogeneous ($z$-independent) $\kappa$, Appendix A provides an explicit expression:
\begin{equation}
  \label{eqn:defmodhom}
  S[c]f(t) = \frac{1}{2c}\int_{z_{\rm min}}^{z_{\rm max}} dz f\left(z,t - \frac{|z_r-z|}{c}\right) 
\end{equation}
From this expression it is simple to see that $S[c]$ is a bounded
operator with the domain and range described in
\ref{eqn:defmoddom}. The role of compact support in $z$ in ensuring
boundedness is also evident. % It is also evident from that limiting source to have bounded support in $z$ is necessary make $S[c]$ bounded.

With this framework, the data is presumed to lie in the range space of $S[c]$: $d \in L^2[0,T]$.

The version of FWI discussed here presumes that the source field corresponding to the data $d$ is supported at a point $z_s \ne z_r$, and is known. This field depends on a function of time (``wavelet'') $w \in L^2[0,T]$. Formally, the resulting acoustic field satisfies the system \ref{eqn:awe1d} with $f=w(t)\delta(z-z_s)$. Inserting this expression in the explicit expression \ref{eqn:defmodhom}, obtain
\begin{equation}
  \label{eqn:defmodpt}
  S[c](w\delta(\cdot-z_s)) = \frac{1}{2c}w\left(t - \frac{|z_r-z_s|}{c}\right) = S_p[c]w(t)
\end{equation}
Thus for this restricted class of source fields, the FWI objective can be redefined as
\begin{equation}
  \label{eqn:deffwipt}
  J_{\rm FWI}[c,d.w] = \frac{1}{2}\|S_p[c]w -d\|^2.
\end{equation}
$S_p[c]$ is a bounded operator with domain $L^2(\bR)$ and range $L^2([0,T])$. 

To end this section, it is necessary to address an irritating technical point: the point source defined above is not a member of the domain of $S[c]$, as it was defined in display \ref{eqn:defmoddom}, so the left-hand side of equation \ref{eqn:defmodpt} does not actually make sense. The fix for this incompatibility actually elucidates the relation between $S[c]$ and $S_p[c]$. Appendix B describes the construction of a family of bounded injective operators  $:L^2(\bR) \rightarrow L^2([z_{\rm min},z_{\rm max}] \times \bR)$ for which
\begin{equation}
  \label{eqn:ext}
  S_p[c] = S[c] \circ E[c]
\end{equation}
This relation exhibits $S[c]$ as an {\em extension} of $S_p[c]$ as
described by \cite{Symes:09}, and identifies WRI as an extended
inversion method.

The construction described in Appendix B requires that $z_s \in
(z_{\rm min},z_{\rm max})$, as mentioned above.

\section{Global Asymptotics of FWI}
The FWI objective is well-known to exhibit non-convexity unless
data frequency content is limited to a small range near 0 Hz, how
small being determined by other scales and by the extent to which the
initial wave velocity differs from the target.

To understand the non-convexity phenomenon and the
relation of the various scales, it is advantageous to introduce a
family of source wavelets, depending on a parameter $\lambda$ playing
the role of wavelength:
\begin{equation}
  \label{eqn:wfam}
  \wl(t) = \frac{1}{\sqrt{\lambda}}w_1\left(\frac{t}{\lambda}\right).
\end{equation}
In the definition \ref{eqn:wfam}, the ``mother wavelet'' $w_1\in
C_0^{\infty}(0,T)$, and the scaling is chosen so fhat
$\|\wl\|_{L^2(\bR)}$ is independent of $\lambda>0$.

Evidently there is no control of $c$ at all in the data if the time
interval of the observation, namely $[0,T]$, is so short that no
signal arrives within it. Accordingly, add to the other assumptions
made so far
the requirement that the transit time between source and receiver at
the slowest permitted velocity is less than $T$:
\begin{equation}
  \label{eqn:time}
  \frac{|z_s-z_r|}{c_{\rm min}} < T
\end{equation}

Choose a target wave velocity $c_* \in [c_{\rm min},c_{\rm
  max}]$. Introduce a family of consisten data $\dl$, generated by
$c_*$ and the wavelet family $\wl$:
\begin{equation}
  \label{eqn:dfam}
  \dl = S_p[c_*]\wl
\end{equation}
and a corresponding family of FWI objectives:
\[
  J_{\rm FWI}[c,\dl,\wl] = \frac{1}{2}\|\dl-S_p[c]\wl\|^2
\]
\begin{equation}
  \label{eqn:fwiexpl}
  =
  \int_0^T\,dt\,\left|\frac{1}{2c_*}\wl\left(t-\frac{z_s-z_r}{c_*}\right)
    - \frac{1}{2c}\wl\left(t-\frac{z_s-z_r}{c}\right)\right|^2
\end{equation}
Note that $\mbox{supp }\wl \subset [0,\lambda T]$, so
\begin{equation}
  \label{eqn:wlsupps}
  \mbox{supp }S_p[c]\wl \subset \left[\frac{z_s-z_r}{c},\lambda
    T+\frac{z_s-z_r}{c}\right] \cap [0,T]
\end{equation}
The transit time condition \ref{eqn:time} implies that there exists
$\lambda_*>0$ so that for $\lambda < \lambda_*$,
\[
  \lambda T+\frac{z_s-z_r}{c} < T
\]
for all admissible $c$. That is, 
for $\lambda <\lambda_*$
\begin{equation}
  \label{eqn:constnorm}
  \|S_p[c]\wl\|^2 = \frac{1}{4c^2}\int_{-\infty}^{\infty}\,dt\,\left|
    \wl\left(t-\frac{z_s-z_r}{c}\right)\right|^2 =
  \frac{\|w_1\|^2}{4c^2}.
\end{equation}

Recall that the object of this study is the global behaviour of
objective functions for velocity estimation: in this context, that
means the behaviour for $c$ far from $c_*$.  Pick $\delta c >0$ for
which $[c_*-\delta c, c_* + \delta c] \subset [c_{\rm min},c_{\rm
  max}]$, that is, that the set of admissible $c$ further than $\delta
c$ from $c_*$ is non-void.

Then there is $\lambda_{**}$ depending on $\delta c, T, and
|z_s-z_r|$ so that if $|c-c_*|>\delta c$ and $\lambda<\lambda_{**}$, then
\[
  \left|\frac{|z_s-z_r|}{c}-\frac{|z_s-z_r|}{c_*}\right| > 2\lambda T,
\]
that is, the infima of the supports of $S_p[c]\wl, \dl=S_p[c_*]\wl$
are further apart than the lengths of these supports. In that case,
necessarily
\[
  \mbox{supp }S_p[c]\wl \cap \mbox{supp }S_p[c_*]\wl = \emptyset.
\]
Therefore $S_p[c]\wl$ and $S_p[c_*]\wl$ are orthogonal in $L^2[0,T]$,
and
\begin{equation}
  \label{eqn:fwiscrewed}
  J_{\rm FWI,\lambda}[c] =
  \frac{1}{2}\left(\frac{1}{4c^2}+\frac{1}{4c_*^2}\right)\|w_1\|^2.
\end{equation}
Amongst other consequences, one immediately deduces from the
expression \ref{eqn:fwiscrewed} the non-convexity result:

\begin{theorem}
  \label{thm:thm1}
  There exists $C>0$, $\lambda_0$ so that for $\lambda <
  \lambda_0$, the minimizer of $J_{\rm FWI, \lambda}[c]$ on the
  complement of $[c_*-C\lambda, c_*+C\lambda]$ is $c=c_{\rm max}$.
\end{theorem}

That is, outside of a neighborhood of width proportional to a
wavelength, minimization of $J_{\rm FWI}$ yields a local minimizer far from
the target $c_*$ which generates the (noise-free) data.

For this 1D problem, a happy 1D accident occurs: a descent
minimization starting at $c_0 < c_*$ will at least initially proceed
in the right direction. With sufficiently small steps, it is possible
that an interation might land in the (small) domain of attraction
around $c_*$. However neither this nor various other accidental
advantages stemming from the very special form of this problem should
be regarded as of any importance.
 
\section{Wavefield Reconstruction Inversion}
This section will describe Wavefield Reconstruction Inversion (WRI)
and develop some of its formal algebraic properties. The notation
$S[c]$ will represent a modeling operator based on wave dynamics of
some sort, depending on a vector $c$ of material parameters. The
conclusions developed here in fact apply to WRI in any such
setting. These conclusions will be applied to 1D acoustics in the
following section, with $c$ specialized to a scalar (wave velocity).

In the application below, the target source $q$ will be a point source
as in the previous section, however for the development of the basic
properties of WRI, that is immaterial.

Note that in general $\alpha$ is a dimensional parameter, having the
same dimensions as $S$.

\cite{LeeuwenHerrmannWRI:13} posed this problem slightly differently:
instead of the first order acoustics system (equation \ref{eqn:awe1d}
for the 1D case), they pose the wave dynamics in terms of the second
order wave equation for the pressure wavefield $p$. From this
viewpoint, $\partial(f-q) /\partial t$ is the residual, that is, the
difference between the image $\partial f/\partial t$ of the 2nd order
wave operator on $p$, and the assumed right-hand side
$\partial q/\partial t$. So in this form, the second term penalizes
the failure of $p$ to solve the wave equation with the assumed
source. The formulation presented here is equivalent, and was
introduced by \cite{WangYingst:SEG16}.

\cite{LeeuwenHerrmann:16} used the variable projection method
\cite[]{GolubPereyra:03}, eliminating the source $f$ in the inner step
and updating the bulk modulus (or an equivalent quantity) in the outer
step. That is, the problem \ref{eqn:defwri} is equivalent to
minimization of
\begin{equation}
\label{eqn:defvpm}
  J_{\rm WRI}[c,d,q] =
  \mbox{min}_f \frac{1}{2}(\| d-S[c]f\|^2+\alpha^2\|f-q\|^2)
\end{equation}
over $c$. This is the approach that I shall pursue here.

In any penalty method, control of the penalty
parameter has a large influence on the speed of
convergence. \cite{Aghamiry:19} use an augmented Lagrangian algorithm
to minimize the influence of the penalty weight choice. Alternatively, one can use a version of the discrepancy principle to adjust $\alpha$ dynamically \cite[]{FuSymes2017discrepancy}, as the WRI problem has the necessary features described in that paper.

``Most'' source fields $f$ are non-radiating,
that is, $S[c]f=0$. The orthocomplement of the subspace of
non-radiating sources is the range of the adjoint operator
$S[c]^T$. This observation suggest that the sources may be re-parametrized as images of
$S[c]^T$. Write $f = g + q$: then the second term in
the objective displayed in \ref{eqn:defwri} is simply $\|g\|^2$.
Decompose $g = S[c]^Te + n$, in
which $e$ is the same type of object as $d$ and $S[c]n=0$ (that
is, $n$ is a non-radiating
source), and note that the decomposition is orthogonal. Then
\[
  J_{\rm WRI}[c,d,q] =
  \mbox{min}_{e,n} \frac{1}{2}(\| d-S[c](S[c]^Te+q)\|^2+\alpha^2(\|S[c]^Te\|^2+\|n\|^2))
  \]
\begin{equation}
  \label{eqn:defvpmred}
   \mbox{min}_{e} \frac{1}{2}(\|r-S[c]S[c]^Te\|^2+\alpha^2\|S[c]^Te\|^2)
 \end{equation}
 in which $r[c]=d-S[c]q$ is the data residual.
 
This reformulation has some computational advantages \cite[]{WangYingst:SEG16,Herrmann:SEG19}, but also leads to a useful analytic transformation of the WRI problem. The minimizer on the RHS of equation \ref{eqn:defvpmred} is the solution $e=e_{\alpha}[c]$ of the normal equation
\[
 ( (S[c]S[c]^T)^2 + \alpha^2S[c](S[c]^T)e = S[c]S[c]^Tr[c]
\]
whence
\[
  S[c]S[c]^Te_{\alpha}[c] = S[c]S[c]^T(S[c]S[c]^T+\alpha^2I)^{-1}r[c]
\]
Assume that $S[c]S[c]^T$ is invertible - this is in fact for the problem outlined in the introduction, as will be verified shortly. Then
\begin{equation}
  \label{eqn:norsol}
  e_{\alpha}[c]=(S[c]S[c]^T+\alpha^2I)^{-1}r[c]
\end{equation}
Consequently
\[
  J_{\rm WRI}[c,d,q] =
  \frac{1}{2}(\|r[c]-S[c]S[c]^Te_{\alpha}[c]\|^2+\alpha^2\|S[c]^Te_{\alpha}[c]\|^2)
\]
\[
  = \frac{1}{2}\left(\|r[c]-S[c]S[c]^T
    (S[c]S[c]^T+\alpha^2I)^{-1}r[c]\|^2 \right.
\]
\[
  \left.+\langle (S[c]S[c]^T+\alpha^2I)^{-1}r[c],
  S[c]S[c]^T (S[c]S[c]^T+\alpha^2I)^{-1}r[c]
  \rangle\right)
\]
\[
  =\frac{1}{2}\left(\|\alpha^2 (S[c]S[c]^T+\alpha^2I)^{-1}r[c]\|^2\right.
\]
\[
  \left.
  +\alpha^2\langle (S[c]S[c]^T+\alpha^2I)^{-1}r[c],
  -\alpha^2 (S[c]S[c]^T+\alpha^2I)^{-1}r[c]\rangle\right.
\]
\[
  + \left.\alpha^2\langle (S[c]S[c]^T+\alpha^2I)^{-1}r[c],r[c]\rangle\right)
\]
\[
  =  \frac{\alpha^2}{2}
\langle (S[c]S[c]^T+\alpha^2I)^{-1}r[c],r[c]\rangle 
\]
That is,
\begin{equation}
  \label{eqn:defwrialt}
  J_{\rm WRI}[c,d,q] = \frac{1}{2}\langle r[c], W_{\alpha}[c] r[c]\rangle
\end{equation}
with
\begin{equation}
  \label{eqn:defwriwt}
  W_{\alpha}[c] = \frac{\alpha^2}{2}(S[c]S[c]^T+\alpha^2I)^{-1}
\end{equation}
This remarkable identity shows that the WRI objective function is a {\em weighted norm of the data residual $r[c]$}. Note that this conclusion depends only on the presumption that the operator $S[c]S[c]^T$ is invertible (or, more precisely, injective), a condition that will be verified below for the problem considered in this paper, and which holds in many other settings in which WRI has been advocated.

\section{WRI for 1D Acoustics}
The preceding section provides the necessary ingredients for an assessment of the relation between WRI and FWI. While the conclusion reached below applies to many wave propagation settings, the 1D acoustic setting is particularly simple and yet illustrates clearly the nature of this relation.

The first task is to give an explicit expression for the operator $S[c]S[c]^T$ appearing repeatedly in the expression \ref{eqn:defwriwt}. From the definition \ref{eqn:defmodhom}, it follows immediatlely that 
\begin{equation}
  \label{eqn:modhomtrans}
S[c]^Te(z,t)=
\left\{
    \begin{array}{c}
      \frac{1}{2c}e\left(t +  \frac{|z_r-z|}{c}\right), \,z_{\rm min} \le z \le z_{\rm max};\\
      0, else.
    \end{array}
  \right.
\end{equation}
whence
\[
  S[c]S[c]^Te(t) = \frac{z_{\rm max}-z_{\rm min}}{4c^2}e(t),
\]
that is,
\begin{equation}
  \label{eqn:modhomnormal}
  S[c]S[c]^T = \frac{z_{\rm max}-z_{\rm min}}{4c^2}I
\end{equation}
Thus the weight operator $W[c]$ appearing in \ref{eqn:defwrialt} takes
the form
\[
  W_{\alpha}[c] = u(c) I,
\]
\begin{equation}
  \label{eqn:wriwthom}
u(c) = \frac{\alpha^2}{2}\left(\frac{z_{\rm max}-z_{\rm min}}{4c^2}+\alpha^2\right)^{-1}.
\end{equation}

Next, suppose that $q = w\delta(z-z_s)$, that is, the target source is
a point source, so that $S[c]q = S_p[c]w$ in the notation used in
the discussion of FWI. Thus \ref{eqn:defwrialt} can be re-written as
\begin{equation}
  \label{eqn:wriscrewed}
  J_{\rm WRI}[c,d,q]=u[c]J_{\rm FWI}[c,d,w]
\end{equation}

Recall the wavelength-dependent family of problems introduced in the derivation and
statement of Theorem \ref{thm:thm1}: target wave velocity $c_*$, wavelength parameter
$\lambda$, parametrized family of wavelets $\wl$ and corresponding
data $\dl$.

Define
\begin{equation}
  \label{eqn:defbeta}
\beta = \frac{z_{\rm max}-z_{\rm min}}{c_*^2} - 4 \alpha^2
\end{equation}
\begin{theorem}
  \label{thm:thm2}
  There exist $C>0$, $\lambda_0$ so that for $\lambda <
  \lambda_0$, the minimizer of $J_{\rm WRI}[c,\dl,\wl \delta(\cdot-z_s)]$ on the
  complement of $[c_*-C\lambda, c_*+C\lambda]$ is
  \begin{itemize}
  \item $c=c_{\rm max}$ if $\beta<0$;
  \item $c=c_{\rm min}$ if $\beta>0$;
  \item any $c < c_*-C\lambda$ or $>c_*+C\lambda$ if $\beta=0$.
  \end{itemize}
\end{theorem}
\begin{proof}
  From \ref{eqn:wriscrewed}, \ref{eqn:wriwthom}, and
  \ref{eqn:fwiscrewed},
  \[
    J_{\rm WRI}[c,\dl,\wl\delta(\cdot-z_s)] =
    \frac{\alpha^2}{2}\left(\frac{z_{\rm max}-z_{\rm
          min}}{4c^2}+\alpha^2\right)^{-1}\frac{1}{2}\left(\frac{1}{4c^2}+\frac{1}{4c_*^2}\right)\|w_1\|^2
  \]
  \[
    =
    \frac{\alpha^2}{4}\frac{1+\frac{c^2}{c_*^2}}{z_{\rm 
        max}-z_{\rm min}+4c^2\alpha^2}
  \]
  This linear fractional function of $c^2$ is increasing, decreasing,
  or constant if $\beta>0, \beta<0$ or $\beta=0$, respectively.
\end{proof}

In other words, $J_{\rm WRI}$ has local minima far from the target
velocity $c_*$, in the same way as does $J_{\rm FWI}$. One of the
local minima will be the result of a local optimization almost surely,
unless the initial estimate of $c$ is ``within a wavelength'' of the
target velocity.

\section{Discussion}

(TBA)

\section{Conclusion}

(TBA)

%%%%%%%%%%%%%%%%%%%%%%%%%%%%%%%%%%%%%%%%%%%%%%%%%%%%%%%%%%

\append{1D Radiation Problem}
Begin with the 1D acoustics point source system. 
\begin{eqnarray}
\label{eqn:awe1dptsrc}
\frac{\partial p}{\partial t} +\kappa\frac{\partial 
  v}{\partial z} &=& w(t)\delta(z-z_s) \nonumber\\
\rho \frac{\partial v}{\partial t} + \frac{\partial p}{\partial 
  z}&=&0\nonumber\\
 p,v&=&0, t \ll 0. 
\end{eqnarray}
Since the right hand side is singular, so is the solution, so it must
be a solution in the weak sense. It follows from the weak solution
conditions that the pressure is continuous at $z=z_s$, whence $v$ must
have a discontinuity. 

Now assume that $\kappa=\kappa_0 \in \bR^+, \rho=\rho_0 \in \bR^+, c_0
= \sqrt{\kappa_0/\rho_0}$ are constant. 
In $z \ne z_s$, the right hand side 
vanishes, so the solution must be locally a combination of plane
waves; causality implies that
\[
p(z,t)=a\left(t -\frac{|z-z_s|}{c_0}\right), \, v(z,t)=\mbox{sgn}(z-z_s) b\left(t -
  \frac{|z-z_s|}{c_0}\right)
\]
From the second dynamical equation (Newton's law) it follows that $b =
a/(\rho_0 c_0)$. The singularity on the LHS of the first dynamical
equation (constitutive law) is
\[
\kappa_0 [v]_{z=z_s}\delta(z-z_s) =
2\kappa_0 b\delta(z-z_s) = 2c_0 a\delta(z-z_s).
\] 
This must in turn equal the RHS of the constitutive law, whence
$a=w/(2c_0)$. Thus
\begin{eqnarray}
\label{eqn:sol1dptsrc}
p(z,t) &=& \frac{1}{2c_0}w\left(t - \frac{|z-z_s|}{c_0}\right) \nonumber \\
v(z,t) &=& \mbox{sgn}(z-z_s)\frac{1}{2\kappa_0}w\left(t -\frac{|z-z_s|}{c_0}\right)
           \nonumber \\
\end{eqnarray}
This result (computation of the Green's function for the acoustic
system) permits an explicit expression for the system with a
space-time source:
\begin{eqnarray}
\label{eqn:awe1d}
\frac{\partial p}{\partial t} +\kappa_0\frac{\partial 
  v}{\partial z} &=& f(z,t) \nonumber\\
\rho_0 \frac{\partial v}{\partial t} + \frac{\partial p}{\partial 
  z}&=&0\nonumber\\
 p,v&=&0, t \ll 0. 
\end{eqnarray}
Since
\[
  f(z,t) = \int dz_1\,f(z_s,t)\delta(z-z_1)
\]
obtain
\begin{eqnarray}
\label{eqn:sol1dp}
p(z,t) &=& \frac{1}{2c_0}\int dz_1 f\left(z_1,t -
           \frac{|z-z_1|}{c_0}\right) \\
  \label{eqn:sol1dv}
v(z,t) &=& \frac{1}{2\kappa_0} \int dz_1 \mbox{sgn} (z-z_1) f\left(z_1,t - \frac{|z-z_1|}{c_0}\right)
\end{eqnarray}

%%%%%%%%%%%%%%%%%%%%%%%%%%%%%%%%%%%%%%%%%%%%%%%%%%%%
\append{Equivalence of Force and Defect}

(TBA)

\bibliographystyle{seg}
\bibliography{../../bib/masterref}