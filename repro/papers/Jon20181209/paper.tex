
\title{Perturbations by a Pool}
\date{}
\address{
        \footnotemark[1]{West Sound, Orcas Island}
}
\author{William Symes\footnotemark[1]}

\righthead{Perturbations}

\maketitle
\parskip 12pt

It may be that my take on this misses the point, but here is a try at an answer.

In this horizontally layered business, every trace depends on the same reflectivity $r(t_0)$.

The trace $d(t,x)$ at offset $x$ has an amplitude $a(t,x)$ and a moveout function $\tau_0(t,x) = t_0$ corresponding to arrival time $t$ at offset $x$ - ignore the noise series and the wavelet.

Officially, 
\[d(t,x) = a(t,x) r(\tau_0(t,x))\]

I think of the last factor as a non-stationary convolution - the moveout function stretches $t$ to $t_0$.

introduce velocity perturbation $\delta v$ and reflectivity perturbation $\delta r$. $\delta v$ results in perturbations $\delta a$ in amplitude and $\delta \tau_0$ in moveout function.

by Taylor, Leibnitz, chain rule, etc., the resulting data perturbation $\delta d$  is to first order
\[
\delta d(t,x) = \delta a(t,x) r(\tau_0(t,x)) + a(t,x) \delta \tau_0(t,x) \frac{dr}{dt_0} (\tau_0(t,x)) + a(t,x) \delta r(\tau_0(t,x))
\]
So, yes, the effect of small velo and refl changes is a sum of three nonstationary convolutions.

However note that $\delta v$ appears implicitly - notably through $\delta \tau_0$ - rather than as the explicit target of a non-stationary filter. It is the reflectivity and its perturbation that are subject to non-stationary filter.

The middle term is "blue", that is, preferentially emphasizes short wavelengths in the reflectivity. The other two are "white".

But maybe I have gone completely off the rails, so will stop here.

