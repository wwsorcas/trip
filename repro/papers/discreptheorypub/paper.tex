\title{Error Bounds for Extended Source Inversion applied to an Acoustic Transmission Inverse Problem}
\author{William W. Symes\\
  Department of Computational and Applied Mathematics, \\
  Rice University, Houston, Texas 77251, USA; {\tt symes@rice.edu}}

\lefthead{Error Bounds for Extended Source Inversion}
\righthead{Symes}

\begin{abstract}
  A simple inverse problem for the wave equation requires
  determination of both the wave velocity in a homogenous acoustic
  material and the transient waveform of an isotropic point radiator,
  given the time history of the wavefield at a remote point in
  space. The duration (support) of the source waveform and the
  source-to-receiver distance are assumed known. A least squares
  formulation of this problem exhibits the ``cycle-skipping''
  behaviour observed in field scale problems of this type, with many
  local minima differing greatly from the global minimizer. An
  extended formulation, dropping the support constraint on the source
  waveform in favor of a weighted quadratic penalty, eliminates this
  misbehaviour. With proper choice of the weight operator, the
  velocity component at {\em any} local minimizer of this extended
  objective function differs from the global
  minimizer of the least-squares formulation by less than a linear
  combination of the source waveform support radius and data
  noise-to-signal ratio.

\end{abstract}

\section{Introduction}
Inverse problems based on wave equations (acoustic, elastic,
Maxwell's...) are commonplace in geophysics, nondestructive materials
testing, medical imaging, and other areas of science and engineering
in which wave motion plays an important role. These problems may be
formulated as nonlinear least squares problems, requiring optimization
of mean square misfit between observed and predicted data. Iterative
local numerical optimization methods (relatives of Newton's method)
are feasible solution approaches for the computationally large
variants of these nonlinear least squares problems that occur in
seismology and medical imaging. However, local optimization produces
only approximate stationary points, and these exhibit a tendency to
stagnate at physically irrelevant solutions, under typical conditions
of data acquisition. This ``cycle-skipping''
pathology can sometimes be overcome through use of information other than
measured wave data, however it remains a serious obstacle to effective
solution of field-scale inverse problems in wave propagation
\cite[]{GauTarVir:86,VirieuxOperto:09,Fichtner:10,Plessix:10,Schuster:17}.

Since numerical feasibility limits practical optimization approaches
to search for stationary points, it is natural to explore alternatives
to nonlinear least squares formulations for which approximate
stationary points might perforce be acceptable solutions. Many such
alternatives have been suggested and applied in both synthetic and
field data tests, in some cases with excellent results (see
\cite{HuangNammourSymesDollizal:SEG19},
\cite{PladysBrossierLiMetivier:GEO21} for recent reviews). However
numerical tests are inevitably ``a look in the rear view mirror'',
with no guarantee that the next example will not behave quite
differently. See \cite{Symes:2020} for an example of such an unpleasant
surprise. Theoretical guarantees that stationary points are
satisfactory solutions of specific inverse problems, in some specific
sense and under specified conditions, are almost entirely absent from
the literature.

This paper provides such a theoretical guarantee for a modification of
least-squares data fitting via {\em modeling operator extension},
applied to a very simple example: recovery of the (spatially
homogeneous) wave speed of an acoustic material model, together with
the waveform of an isotropic point radiator (``wavelet'') with
specified support, from the time history (``trace'') of the resulting
pressure field at a single remote location. This is perhaps the
simplest inverse wave propagation problem that exhibits
cycle-skipping. While far too simple to have immediate
practical application, it is a special case, or subproblem, of many
field-scale inverse problems in industrial and academic seismology. It
is also routinely used to illustrate cycle-skipping (see for example
\cite{VirieuxOperto:09}, Figure 7). Modeling operator extension ideas underlie
widely used techniques in seismic data processing, and have been
applied to inversion for several decades
\cite[]{geoprosp:2008}. The particular approach used in this work
is an example of {\em source extension}.
\cite{HuangNammourSymesDollizal:SEG19} review recent source extension
contributions, some of which are closely related to the approach to
single-trace transmission inversion studied here.

Model extension is only one of several ideas proposed to remedy
cycle-skipping. For instance, measuring data misfit with versions of the Wasserstein
metric from optimal transport theory, rather than the $L^2$ norm, also
shows promise in mitigating cycle-skipping
\cite[]{Metivier:GEO18,EngquistYang:GEO18,Ramos-Martinez:SEG18,Wang:SEG19,EngquistYang:CPAM21}. \cite{MahankaliYang:21}
have established conditions under which the Wasserstein misfit is
convex in a few parameters of simple models. Aside from the present
paper, their work is one of the very few to give a complete mathematical
justification (albeit also in very
restricted circumstances) for application of local optimization to a nonlinear
least squares alternative.

The next section precisely defines the single-trace transmission
inverse problem, and the following section confirms the
cycle-skipping behaviour of the obvious nonlinear least squares
formulation, that is, that stationary points exist at arbitrarily
large distance from any global minimizer (Theorem \ref{thm:fwi}). The
modeling operator extension used here simply consists in dropping the
support constraint, thus allowing fit to any data with any wavespeed,
by suitable choice of wavelet. The link between wave speed and data is
re-established by supplementing the mean-square data misfit function
with a suitable quadratic penalty on nonzero values of the wavelet far
from the specified support. I show that any stationary point of the
resulting penalty function deviates from the global solution of the
original, constrained nonlinear least squares problem by an amount
proportional to a combination of the support diameter and the relative
data error, and this bound is sharp
(Theorems \ref{thm:rampreallygood} and \ref{thm:mnoiseres}). A
solution of the original inverse problem (with wavelet support
constraint) is recovered by truncating the wavelet obtained in the
course of minimizing the penalty function (Theorem
\ref{thm:ipnoisesuf}).

The elementary
analysis developed here leans on the special features of the problem
under study to achieve sharper results than will likely be possible in
more general contexts. The Appendix recasts some of these
constructions in a form that applies, at least formally, to more
prototypical, larger-scale problems
\cite[]{Symes:IPTA14,tenKroode:IPTA14}, and suggests a route to
achieving a similar, mathematically complete understanding of
extension-based approaches to applied wave propagation inverse
problems.

\section{An Acoustic transmission Inverse Problem}
According to linear acoustics, the causal pressure field $p(\bx,t)$ due to an
isotropic point radiator at $\bx_s \in \bR^3$ with time-varying
intensity (``wavelet'') $w(t)$, propagating in an elastic fluid with
slowness (reciprocal velocity) $m$ and density $1$ (in appropriate
units), is the solution of the the initial value problem for the wave equation \cite[]{Frie:58}:
\begin{eqnarray}
  \label{eqn:awe}
  \left(m^2\frac{\partial^2 p}{\partial t^2} - \nabla^2\right) p(\bx,t) &=&
                                                                         w(t)\delta(\bx-\bx_s), \nonumber\\
  p(\bx,t)&=&0, t\ll 0.
\end{eqnarray}
The solution is well-known, see for instance
\cite{CourHil:62}, Chapter VI, section 12, equation 47:
\begin{equation}
  \label{eqn:homsol}
  p(\bx,t) = \frac{1}{4\pi |\bx-\bx_s|}w\left(t-m|\bx-\bx_s|\right).
\end{equation}
This trace of $p$ at $\bx_r \in \bR^3$ can be viewed as the result of
applying an m-dependent linear operator $F[m]$ to the wavelet $w$:
\begin{equation}
\label{eqn:mod}
F[m]w(t)  = p(\bx_r,t) = \frac{1}{4\pi r}w\left(t-mr\right).
\end{equation}
The slowness $m$ must be positive, as follows from basic acoustics,
and in fact reside in a range characteristic of the
material model: for crustal rock, a reasonable choice would be
$m_{\rm min}=0.125, m_{\rm max}=0.6$ s/km. The pressure trace
$p(\bx_r,\cdot)$ is square-integrable if and only if the same is true
of the wavelet $w$. Since the square-integral of the pressure trace is
proportional to the accumulated energy transferred from the fluid to
the sensor \cite[]{SantosaSymes:00}, assume that $w \in L^2(\bR)$.

Natural choices for domain and
range of $F$ are thus
\begin{itemize}
\item $M=(m_{\rm min}, m_{\rm max}),\,0 < m_{\rm min} \le m_{\rm
    max}$;
\item $W = L^2(\bR)$;
\item $D=L^2([t_{\rm min},t_{\rm max}]),\, t_{\rm min}<t_{\rm max}$;
\item $F: M \times W \rightarrow D$ as specified in \ref{eqn:mod}.
\end{itemize}
It is immediately evident from these choices and from the definition
\ref{eqn:mod} that
\begin{equation}
  \label{eqn:mapprop}
  \mbox{for }m \in M, F[m] \mbox{ is bounded, and }\|F[m]\| =
  \frac{1}{4\pi r}.
\end{equation}

Since
all possible data lie in the range of $F[m]$ for any $m \in M$, some
restriction of the domain of $F$ is necessary in order that fitting
the data constrain $m$. The constraint employed in this work is the specification
of a maxium support radius $\lambda_{\rm max} >0$ (see the companion
paper \cite{SymesChenMinkoff:21} for a justification of this choice).

For
$\lambda \in (0,\lambda_{\rm max}]$, define
\begin{itemize}
\item $\lW = \{w \in W:
  \mbox{supp }w \subset [-\lambda,\lambda]\}$;
\item $\lF = F|_{M \times \lW}$.
\end{itemize}

In terms of this infrastructure, the inverse problem studied in 
this paper may be stated as

\begin{quote}
\noindent {\bf Inverse Problem:}
  given data $d \in D$, relative error level $\epsilon \in
  [0,1)$, and support radius $\lambda \in (0, \lambda_{\rm
    max}]$, find $(m,w) \in M \times \lW$ for which 
\begin{equation}
  \label{eqn:probstat0}  \|\lF[m]w-d\| \le \epsilon\|d\|,
\end{equation}
\end{quote}

Define the relative mean-square error $\lerr: M \times \lW \times D
\rightarrow \bR^+$ by
\begin{equation}
  \label{eqn:redms}
  \lerr[m,w;d]=\frac{1}{2}\|\lF[m]w-d\|^2/\|d\|^2,
\end{equation}
so that inequality \ref{eqn:probstat0} is equivalent to
\begin{equation}
  \label{eqn:probstat1}
  \lerr[m,w;d] \le \frac{1}{2}\epsilon^2.
\end{equation}
Minimization of $\lerr$ is the standard least-squares formulation of
the Inverse Problem defined above: if the global minimum value of
$\lerr$ is less than $\frac{1}{2}\epsilon^2$, then the global
minimizer is a solution of the Inverse Problem.

\noindent{\bf Remark:} The constraint $\epsilon < 1$ imposed on the
target noise level eliminates the obvious choice $(m,0)$, which
satisfies the data misfit constraint for any $m \in M$ if $\epsilon
\ge 1$. 

\noindent{\bf Remark:} I shall refer to the minimization of $\lerr$ as
``Full Waveform Inversion'' or ``FWI'', as this is the
terminology used in the seismology literature to identify this and
similar optimization problems.

The best case for data fitting
is clearly the one in which the data can be fit precisely: that is,
there exists $(m_*,w_*) \in M\times \lW$ so that
\begin{equation}
  \label{eqn:defdatanonoise}
  d=\lF[m_*]w_*.
\end{equation}
Such data $d$ is {\em noise-free}, in the range of the map $\lF$. For
such data a solution of the Problem Statement \ref{eqn:probstat0}
exists with arbitrarily small $\epsilon>0$.


\section{Full Waveform Inversion}
While $F$ is surjective, as noted above, it is
very far from injective. On the other hand, under a constraint that
will be assumed throughout, $\lF[m]$ is injective for each $m \in M$ (in fact, $4 \pi r
\lF[m]$ is an isometry):
\begin{proposition}
  \label{thm:fullrec}
  Suppose that 
  \begin{equation}
    \label{eqn:fullrec}
    [ m_{\rm min}r-\lambda_{\rm max}, m_{\rm max}r+\lambda_{\rm max}]
    \subset [t_{\rm min},t_{\rm max}].
  \end{equation}
  Then $\lF[m]$ is coercive for every $m \in M, \lambda \in
  (0,\lambda_{\rm max}]$.
\end{proposition}

\noindent{\bf Remark:} A useful consequence of the condition 
\ref{eqn:fullrec}: for every $m \in M$, 
\begin{equation}
  [-\lambda_{\rm max}, \lambda_{\rm max}] \subset[ t_{\rm min}-mr , 
  t_{\rm max}+mr]. 
  \label{eqn:zeroinc}
\end{equation}

The first main result establishes the existence of large (100 \%)
residual local minimizers for the basic FWI objective $\lerr$, even
for noise-free data.
\begin{theorem}
  \label{thm:fwi}
  Suppose that $0 <\lambda\le \lambda_{\rm max}$,  $m_* \in M, w_*
  \in \lW, d=\lF[m_*]w_*$ is noise-free data per definition \ref{eqn:defdatanonoise},
  Under assumption \ref{eqn:fullrec}, for any $m \in M$ with $|m-m_*|r>2\lambda$,
\begin{equation}
  \label{eqn:isovpm}
 \min_w \lerr[m,w;d]=\lerr[m,0;d] = \frac{1}{2},
\end{equation}
and any such $(m,0)$ is a local minimizer of $\lerr$ with relative RMS
error = 1.0.
\end{theorem}

\begin{proof} From the definition \ref{eqn:mod},
\[
 \lerr[m,w;d] =  \frac{1}{32\pi^2
    r^2\|d\|^2}\int\,dt\,\left|w\left(t-mr\right)-w_*\left(t-m_*r\right)\right|^2.
\]
Since $w_*, w$ vanish for $|t|>\lambda$,
$\lF[m_*]w_*(t)$ vanishes if $|t-m_*r|>\lambda$ and $\lF[m]w$ vanishes if $|t-mr|>\lambda$. So if $|mr-m_*r|
= |m-m_*|r > 2\lambda$, then $|t-mr|+|t-m_*r| \ge |mr-m_*r| >
2\lambda$ so either $|t-mr|>\lambda$ or $|t-m_*r|>\lambda$, that is,
either $\lF[m]w(t)=0$ or $\lF[m_*]w_*=0$. Therefore $\lF[m]w$ and
$\lF[m_*]w_*$ are orthogonal in the sense of the $L^2$ inner product
$\langle \cdot,\cdot \rangle_D$ on $D$:
\begin{equation}
  \label{eqn:ortho}
  |m- m_*|r > 2\lambda \,\, \Rightarrow \,\, \langle F[m]w,
  F[m_*]w_*\rangle_D = 0.
\end{equation}
But $d = \lF[m_*]w_*$, so this is the same as saying that $d$ is
orthogonal to $F[m]w$. So conclude after a minor manipulation that
\[
  |m- m_*|r > 2\lambda \,\, \Rightarrow \,\, \lerr[m,w;d]=\frac{1}{32\pi^2 
    r^2\|d\|^2}(\|w\|^2 + \|w_*\|^2)
\]
\begin{equation}
  \label{eqn:iso}
  = \frac{1}{2}\left(\frac{\|w\|^2}{\|w_*\|^2} + 1 \right).
\end{equation}
That is, for slowness $m$ in error by more than $2\lambda/r$ from the 
target slowness $m_*$, the means square error (FWI objective) $\lerr$ is independent of
$m$, and its minimum over $w$ is attained for $w=0$.
\end{proof}

Therefore local minimizers of $\lerr$ abound, as far as you like from the
global minimizer $(m_*,w_*)$. Local exploration of the FWI objective
$e$ gives no useful information whatever about constructive search
directions, and descent-based optimization tends to fail if the
initial estimate $m_0$ is in error by more than $2\lambda/r$. In fact the actual behaviour of FWI itererations is worse
(failure if $m_0$ is in error by ``half a wavelength''), as follows
from a more refined analysis of the cycle-skipping local behaviour of $\lerr$ near its
global minimizer.

\section{Extended Source Inversion}
The phenomenon explained in the last section can be avoided by
reformulating the inverse problem via an extended modeling operator
and a soft (penalty) constraint to replace the support
requirement. The extension simply amounts to dropping the support
constraint, and replacing $\lW$ and $\lF$ by $W$ and $F$.

The hard
support constraint implicit in the choice of $\lW$ as domain for the
modeling operator is replaced by a soft constraint in the form of a
quadratic penalty, with weight operator $A:W \rightarrow W$.
The choices for the penalty operator $A$ considered here are scalar 
multiplication operators on $W$ defined by a choice of multiplier $a \in L^{\infty}(\bR)$:
\begin{equation}
  \label{eqn:annmult}
  A w(t)= a(t)w(t), \, t\in \bR.
\end{equation}
Explicit choices
for $a$ are discuss below.

With these choices, define
\begin{eqnarray}
  \label{eqn:edef}
  e[m,w;d] & = & \frac{1}{2}\|F[m]w-d\|^2/\|d\|^2;\\
  \label{eqn:gdef}
  g[w;d] & = & \frac{1}{2}\|Aw\|^2/\|d\|^2;\\
  \label{eqn:jdef}
  \Ja[m,w;d] & = & e[m,w;d] + \alpha^2g[w;d].
\end{eqnarray}

The main theoretical device used in the proofs of our main results on 
extended inversion is reduction of the penalty objective $\Ja$
(equation \ref{eqn:jdef}) to a function $\tJa$ of 
$m$ alone, by minimization over $w$:
\begin{equation}
  \label{eqn:redexp}
  \tJa[m;d] = \inf_w \Ja[m,w;d].
\end{equation}
Reduction of this type is often termed ``Variable Projection'', after \cite{GolubPereyra:73,GolubPereyra:03}.

A minimizer $w$ on the right-hand side of definition
\ref{eqn:redexp} must solve the {\em normal equation}
\begin{equation}
  \label{eqn:norm}
  (F[m]^TF[m]+\alpha^2A^TA)w= F[m]^Td, 
\end{equation}
(here and in the following, the superscript $T$ signifies adjoint in
the sense of domain and range Hilbert structures).

With $A$ of the form \ref{eqn:annmult}, $\tilde{J}_{\alpha}$ is explicitly
computable. First observe that apart from amplitude, $F[m]$ is
unitary: for $g \in D$,
\begin{equation}
\label{eqn:tran}
F[m]^T g (t) =
\left\{
  \begin{array}{c}
    \frac{1}{4\pi r}g\left(t+mr\right), \, t \in [t_{\rm min}-mr,
    t_{\rm max}-mr],\\
    0, \mbox{ else.}
  \end{array}
\right.
\end{equation}
so
\begin{equation}
  \label{eqn:unit}
  F[m]^TF[m] = \frac{1}{(4\pi r)^2}{\bf 1}_{[t_{\rm min}-mr,  
    t_{\rm max}-mr]}
\end{equation}
in which ${\bf 1}_{S}$ denotes
multiplication by the characteristic function of a measurable 
$S \subset \bR$.

Therefore the normal equation for the minimizer on the RHS of equation \ref{eqn:redexp} is
\begin{equation}
  \label{eqn:norm1}
  \left(\frac{1}{(4\pi r)^2} {\bf 1}_{[t_{\rm min}-mr,  
      t_{\rm max}-mr]} + \alpha^2 A^TA\right)w= F[m]^Td.
\end{equation}

With these choices, the normal equation \ref{eqn:norm1} becomes
\begin{equation}
\label{eqn:norm2}
\left(\frac{1}{(4\pi r)^2}  {\bf 1}_{[t_{\rm min}-mr,  
      t_{\rm max}-mr]} + \alpha^2a^2\right)w= F[m]^Td.
\end{equation}
Evidently, $F^TF$ (the first summand in equation \ref{eqn:norm2}) is
not coercive. If $A$ is to yield small output for input localized near
$t=0$, as it must to play its desired role in replacing the hard
support constraint, then $A^TA$ must not be coercive either, that is,
its spectrum must have zero as its infimum. Therefore an additional
condition on $A$ is required to enable application of the Lax-Milgram
constrruction to the normal
equation \ref{eqn:norm}. This condition is an hypothesis of 

\begin{proposition}
  \label{thm:norminvexp}
  Assume the conditions \ref{eqn:mod}, \ref{eqn:fullrec},
  \ref{eqn:annmult}. Also assume that $\lambda \in (0,\lambda_{\rm
    max}], \alpha > 0,$ and that  $C>0$ exists so that $a \in L^{\infty}(\bR)$
  mentioned in equation \ref{eqn:annmult} satisfies the condition
  \begin{equation}
    \label{eqn:abnd} 
    a \ge 0; \, a(t) \ge C\mbox{ for }|t| \ge \lambda_{\rm max}.
  \end{equation}
%  \begin{equation}
%   \label{eqn:abnd}
%    |t| > \lambda \Rightarrow a(t) \ge C,
%  \end{equation}
%  in which $C>0$ may depend on $\lambda$.
  Then
  \begin{itemize}
  \item[1. ]the normal operator $F[m]^TF[m] + \alpha^2A^TA$ is
    invertible for any $m \in M$, $\alpha > 0$;
  \item[2. ]the solution $\aw[m;d]\in W$ of the normal equation
    \ref{eqn:norm} is given by
    \begin{equation}
      \label{eqn:normsol}
      \aw[m;d](t) = \left\{
        \begin{array}{c}
          \left(\frac{1}{(4\pi r)^2} + \alpha^2
          a^2(t)\right)^{-1}\frac{1}{4 \pi r}d(t+mr), t \in [t_{\rm
          min}-mr, t_{\rm max}-mr];\\
          0, \mbox{ else.}
        \end{array}
      \right.
    \end{equation}
  \item[3. ]if in addition $d=F[m_*]w_*, w_* \in \lW$ is noise-free, as in equation
    \ref{eqn:defdatanonoise},
    \begin{equation}
      \label{eqn:solnonoise}
      \aw[m,d](t)= \left(1+ (4\pi r)^2\alpha^2 a(t)^2\right)^{-1}w_*\left(t+(m-m_*)r\right).
    \end{equation}
  \end{itemize}
\end{proposition}

\begin{proof}
  The idea here is that $F$ fails to be coercive just where $A$ is
  coercive, and vis-versa.
  \begin{itemize}
  \item[1. ]Note that thanks to \ref{eqn:zeroinc}, if $|t|\le
    \lambda \le \lambda_{\rm max}$, then ${\bf 1}_{[t_{\rm min}-mr,  
      t_{\rm max}-mr]}(t) = 1$, whereas if $|t|>\lambda$,
    then $a(t) \ge C$, whence
    \[
      \frac{1}{(4\pi r)^2}  {\bf 1}_{[t_{\rm min}-mr,  
        t_{\rm max}-mr]} + \alpha^2a^2  \ge \min\{(4\pi r)^2,
      \alpha^2\min\{1/(4\pi r)^2,C^2\} \}> 0.
    \]
    Therefore the normal operator is invertible under the stated
    conditions.

  \item[2. ]From the identity \ref{eqn:tran}.
    \[
      \mbox{supp }F[m]^Td \subset [t_{\rm min}-mr,t_{\rm max}-mr].
    \]
    Define $w_{\rm tmp}$ to be the right-hand side of equation \ref{eqn:normsol}. Then
    from the previous observation and identity \ref{eqn:tran},
    \[
      \mbox{supp }w_{\rm tmp} \subset [t_{\rm min}-mr,t_{\rm max}-mr].
    \]
    From the identity \ref{eqn:unit}, for any $w \in W$,
    \[
      t \in [t_{\rm min}-mr,t_{\rm max}-mr] \Rightarrow F[m]^TF[m]w(t)
      = \frac{1}{(4 \pi r)^2}w(t).
    \]
    It follows from this and the previous two observations that
    $w_{\rm tmp}$ solves the normal equation \ref{eqn:norm}, and
    therefore that $\aw[m;d]=w_{\rm tmp}$.

  \item[3. ]Follows by inserting the definition
    \ref{eqn:defdatanonoise} of $d$ in \ref{eqn:normsol} and
    rearranging.
  \end{itemize}
\end{proof}

\begin{theorem}
  \label{thm:norminv}
  Assume the condition \ref{eqn:fullrec}, $C>0$, and suppose that $A$ is
  given by equation \ref{eqn:annmult} for $a \in L^{\infty}(\bR)$
  satisfying condition \ref{eqn:abnd}.  Then
  \begin{itemize}
  \item[1. ]the reduced objective $\tJa$ is given by
    \begin{equation}
      \label{eqn:redexp1}
      \tJa[m;d] = \Ja[m,\aw[m;d];d],
    \end{equation}
    in which $\aw[m;d] \in W$ is the unique solution of the normal
    equation \ref{eqn:norm}.
  \item[2. ]The following are equivalent:
    \begin{itemize}
    \item[i. ]$(m,w) \in M \times W$ is a local minimizer of
      $\Ja[\cdot,\cdot;d]$, and
    \item[ii. ]$m$ is a local minimizer of $\tJa[\cdot;d]$ and
      $w=\aw[m;d]$.
    \end{itemize}
  \end{itemize}
\end{theorem}

\begin{proof} These conclusions follow immediately from Proposition
  \ref{thm:norminvexp}.
\end{proof}

If $\Ja[\cdot,\cdot;d]$ and $\tJa[\cdot;d]$
were differentiable, then ``local minimizer'' in the conclusion of the
preceding theorem could be replaced by ``stationary point''. However,
for the problem addressed in this paper, $\Ja[\cdot,\cdot;d]$ {\em is
  not} differentiable without added smoothness constraints on $w$,
whereas $\tJa[\cdot;d]$ {\em is} differentiable for proper choice of
penalty operator $A$.
This conclusion follows from properties of the modeling operator $F$
shared with many other inverse problems in wave propagation, as
explained in the Appendix. Here, I derive differentiability from explicit
expressions for $\tJa$ and its components.

\begin{proposition}
  \label{thm:epjgen}
  Assume the hypotheses of Proposition \ref{thm:norminvexp}. Then
  \begin{equation}
  \label{eqn:residnormgen}
  e[m,\aw[m,d];d] = \frac{1}{2\|d\|^2}\int_{t_{\rm min}}^{t_{\rm max}} \,dt\,(4\pi r \alpha a(t-mr))^4(1 +
  (4\pi r \alpha a(t-mr))^2)^{-2}d(t)^2,
\end{equation}
\begin{equation}
  \label{eqn:anninormgen}
  p[m,\aw[m,d];d] = \frac{1}{2\|d\|^2}\int_{t_{\rm min}}^{t_{\rm max}} \,dt\,(4\pi r a(t-mr))^2(1 +
  (4\pi r \alpha a(t-mr))^2)^{-2}d(t)^2,
\end{equation}
and
\begin{equation}
  \label{eqn:expjgen}
\tJa[m;d] = \frac{1}{2\|d\|^2}\int_{t_{\rm min}}^{t_{\rm max}}\,dt\,(4\pi r \alpha a(t-mr))^2(1+(4\pi r \alpha 
a(t-mr))^2)^{-1}d(t)^2. 
\end{equation}
\end{proposition}

\begin{proof}
  From equation \ref{eqn:normsol},
  \[
    F[m]\aw[m;d](t) = 
    \frac{1}{4 \pi r}\left(\frac{1}{(4\pi r)^2} + \alpha^2
      a^2(t-mr)\right)^{-1}\frac{1}{4 \pi r}d(t),
  \]
  so
  \[
    (F[m]\aw[m;d]-d)(t) = (1 + (4 \pi r\alpha
    a(t-mr))^2)^{-1}-1)d(t)
  \]
  \[
    = -(4 \pi r\alpha a(t-mr))^2(1 + (4 \pi r\alpha
    a(t-mr))^2)^{-1}d(t).
  \]
  Half the integral of the square of this data residual is
  $e[m,\aw[m;d],d]$, which proves identity \ref{eqn:residnormgen}.

  To compute $p[m,\aw[m;d],d]$, note that
  \[
    A\aw[m;d](t)=a(t) \left(\frac{1}{(4\pi r)^2} + \alpha^2
      a^2(t)\right)^{-1}\frac{1}{4 \pi r}d(t+mr)
  \]
  \[
    = 4\pi r a(t) (1 + (4\pi r \alpha a(t))^2)^{-1}d(t+mr)
  \]
  for $ t \in [t_{\rm min}-mr, t_{\rm max}-mr]$, so squaring,
  integrating, and changing integration variables $t \mapsto t-mr$
  gives the result \ref{eqn:anninormgen}

  That the reduced objective $\tJa$ is given by \ref{eqn:expjgen} follows from equations \ref{eqn:redexp1}, \ref{eqn:residnormgen}, and
  \ref{eqn:anninormgen}.
\end{proof}

\begin{theorem}
  \label{thm:diffobj}
  Suppose that in addition to the hypotheses of Theorem
  \ref{thm:norminv}, $a \in W^{1,\infty}_{\rm loc}(\bR)$, then $\tJa[\cdot;d]
  \in C^1(M)$.
\end{theorem}

\begin{proof}
Suppose first that $a \in C^1(\bR)$. Differentiation under the integral sign  
  yields the expression for its derivative:
\begin{equation}
  \label{eqn:dexpjgen}
  \frac{d}{dm}\tJa[m;d] = -\frac{(4 \pi r \alpha)^2}{\|d\|^2} \int_{t_{\rm min}}^{t_{\rm max}} \,dt \, 
  \left(a\frac{da}{dt}\right)(t-mr)(1+(4\pi r \alpha 
  a(t-mr))^2)^{-2}d(t)^2. 
\end{equation}
For $a \in W^{1,\infty}_{\rm loc}(\bR)$ a limiting argument shows that the
same expression gives the derivative of $\tJa$.
\end{proof}

It will be useful to record expressions for the various components of
$\tJa$ when the data is noise-free, that is, the context of
Proposition \ref{thm:norminvexp}, item 3:
\begin{corollary}
  \label{thm:epjnonoise}
  Assume the hypotheses of Proposition \ref{thm:norminvexp}, item
  3. Then noting that $\|d\| = \|w_*\|/(4 \pi r)$
\begin{equation}
  \label{eqn:residnorm}
  e[m,\aw[m,d];d] 
= \frac{\alpha^4}{2\|w_*\|^2}\int\,dt\,a(t-(m-m_*)r)^4(1+(4\pi r)^2 \alpha^2 
    a(t-(m-m_*)r)^2)^{-2}w_*(t)^2.
\end{equation}
\begin{equation}
  \label{eqn:anninorm}
  p[m,\aw[m,d];d] = \frac{(4\pi r)^2}{2\|w_*\|^2}\int \,dt\,  
  \frac{a(t-(m-m_*)r)^2}{(1+ (4\pi r)^2\alpha^2
    a(t-(m-m_*)r)^2)^{2}}w_*(t)^2.
\end{equation}
so 
\begin{equation}
\label{eqn:expjnonoise}
\tJa[m;d] = \frac{(4\pi r \alpha)^2}{2\|w_*\|^2}\int\,dt\,a(t-(m-m_*)r)^2(1+(4\pi r)^2 \alpha^2 
  a(t-(m-m_*)r)^2)^{-1}w_*(t)^2. 
\end{equation}
Finally, if $a \in W^{1,\infty}(\bR)$, then $\tJa[\cdot;d]$ is differentiable, and 
\begin{equation}
  \label{eqn:dexpjnonoise}
  \frac{d}{dm}\tJa[m;d] = -\frac{r (4\pi r \alpha)^2}{\|w_*\|^2} \int \,dt \, 
  \frac{\left(a\frac{da}{dt}\right)(t-(m-m_*)r)}{(1+(4\pi r)^2 \alpha^2 
  a(t-(m-m_*)r)^2)^{2}}w_*(t)^2. 
\end{equation}
\end{corollary}


\section{Stationary Points}

Recall that the purpose of the penalty operator is to penalize energy away from 
$t=0$. A simple multiplier of class $W^{1,\infty}$ that accomplishes
this goal (and produces a differentiable reduced objective $\tJa$,
thanks to Theorem \ref{thm:diffobj}) is
\begin{equation}
  \label{eqn:ann}
  a(t) = \min(|t|, \tau). 
\end{equation}
for suitable $\tau>0$. In fact, 
the cutoff $\tau$ will be chosen large enough to be effectively inactive: 
hindsight suggests 
\begin{equation}
  \label{eqn:taudef}
  \tau = \max\{|t_{\rm min}-m_{\rm min}r|,|t_{\rm min}-m_{\rm max}r|, |t_{\rm max}-m_{\rm min}r|, |t_{\rm max}-m_{\rm max}r|\}. 
\end{equation}
Essentially this 
particular annihilator has been employed in earlier work on extended 
source inversion 
\cite[]{Plessix:00a,LuoSava:11,Warner:14,HuangSymes:SEG15a,Warner:16,HuangSymes:Geo17}.

\begin{proposition}
  \label{thm:rampgood}
  Suppose that
  \begin{enumerate}
  \item $m_* \in M$;
  \item $0 < \mu \le \lambda$, and $w_* \in W_{\mu}$;
  \item $d_* = F[m_*]w_*$;
  \item $a(t)=\min\{|t|,\tau\}$ in the definition \ref{eqn:annmult},
    with $\tau$ given by equation \ref{eqn:taudef}; and
  \item $\alpha > 0$.
  \end{enumerate}
  Then for any $m \in M$, 
  \begin{equation}
    | (m - m_*)r| > \lambda  \Rightarrow  \left|\frac{d}{dm}\tJa[m;d_*]\right| >  
    \frac{r(4 \pi r \alpha)^2(\lambda-\mu)}{(1+(4\pi r\alpha)^2 
      (\lambda+\mu)^2)^{2}}.
    \label{eqn:gradbndnonoise}
  \end{equation}
\end{proposition}
\begin{proof}
  As observed before, $\mbox{supp }\aw[m;d_*] \subset [t_{\rm
    min}-mr,t_{\rm max}-mr]\subset [-\tau,\tau]$, with $\tau$ defined
  in \ref{eqn:taudef}. Therefore, $a(t) = |t|$, $a a'(t) = t$ in the
  support of the integrand on the RHS of equation
  \ref{eqn:dexpjnonoise}, which therefore 
  becomes (after change of integration variable)
  %%%%%%%%%%%%%%%%%%%%%%%%%
  \begin{equation}
    \label{eqn:gradfinal}
    \frac{d}{dm}\tJa[m;d_*] = -\frac{r (4\pi r\alpha)^2}{\|w_*\|^2} \int \,dt \, 
  t(1+(4\pi r)^2 \alpha^2 
  t^2)^{-2}w_*(t+(m-m_*))^2.
  \end{equation}
  Recall that $w_*(t+(m-m_*)r)$
  vanishes if $|t+(m-m_*)r| > \lambda$. Therefore the integral on the
  RHS of equation \ref{eqn:gradfinal} can be re-written
  \[
    = -\frac{r(4 \pi r \alpha)^2}{\|w_*\|^2}\int_{-(m-m_*)r-\lambda}^{-(m-m_*)r+\lambda}
    \,dt\, t(1+(4\pi r)^2\alpha^2 t^2)^{-2}w_*\left(t+(m-m_*)r\right)^2.
  \]
  Suppose that $\mu \le \lambda$ and $w_* \in W_{\mu}$. 
  If $m > m_*+\lambda/r$, then $t+(m-m_*)r \in \mbox{supp }w_*$
  implies $-\mu - \lambda < t < \mu-\lambda<0$, so 
  \[
    t(1+(4\pi r)^2\alpha^2 t^2)^{-2} < (\mu-\lambda)(1+(4\pi r)^2\alpha^2 (\mu+\lambda)^2)^{-2}<0
  \]
  in the support of the integrand in equation
  \ref{eqn:gradfinal}. Arguing similarly for $m<m_*-\lambda/r$, obtain
  a similar inequality, implying the conclusion \ref{eqn:gradbndnonoise}.
\end{proof}

\begin{theorem}
  \label{thm:rampreallygood}
  Suppose that
  \begin{enumerate}
  \item $m_* \in M$;
  \item $0 <  \lambda$, and $w_* \in \lW$;
  \item $d_* = F[m_*]w_*$;
  \item $a(t)=\min\{|t|,\tau\}$ in the definition \ref{eqn:annmult},
    with $\tau$ given by equation \ref{eqn:taudef}; 
  \item $\alpha > 0$; and
  \item$m \in M$ is a stationary point of $\tJa[\cdot;d_*]$.
  \end{enumerate}
  Then $|m-m_*| < \lambda /r$.
\end{theorem}

\begin{proof} Follows directly from Proposition \ref{thm:rampgood} by
  taking $\mu=\lambda$.
\end{proof}

The preceding theorem established that a proper choice of annihilator
leads to a reduced penalty objective all of whose stationary points
are within $O(\lambda)$ of the target slowness $m_*$, provided that
the data are noise-free in the sense of equation
\ref{eqn:defdatanonoise}. This result leaves open two questions:
\begin{itemize}
\item how does one use this reduced penalty minimization to produce
  a solution of the inverse problem as in problem statement
  \ref{eqn:probstat0}? 
\item how does one answer the same question for noisy data?
\end{itemize}

The next result answers the first question, in the case of noise-free data:
\begin{proposition}
  \label{thm:ipnonoisesuf}
  Suppose that $a$ is given by definition \ref{eqn:ann}, $\alpha$,
  $\mu \in (0,\lambda_{\rm max}]$,
  $d$ is given by
  \ref{eqn:defdatanonoise} with $w_* \in W_{\mu}$, and  $m$ is a stationary
  point of $\tJa[\cdot;d]$. Then $(m,\aw[m;d])$ is a
  solution of the inverse problem \ref{eqn:probstat0} for any $\lambda
  \ge 2\mu$ and
  \begin{equation}
    \label{eqn:estresidnorm}
    \epsilon \ge \frac{(8\pi r \mu \alpha)^2}{1 + (8\pi r \mu\alpha)^2}.
  \end{equation}
\end{proposition}

\begin{proof}
  From the assumption $w_* \in W_{\mu}$ and Theorem
  \ref{thm:rampreallygood}, $|(m-m_*)r|\le \mu$. From the
  identity \ref{eqn:solnonoise},
  $\mbox{supp }\aw[m;d] \subset
  [(m-m_*)r-\mu,(m-m_*)r+\mu] \subset
  [-2\mu,2\mu]$. Because of the support limitation, $a(t)=|t|$ in the
  interval of integration appearing in \ref{eqn:residnorm}, so
\[
  e[m,\aw[m,d];d] 
= 8 \pi^2 r^2 \alpha^4\int^{\mu}_{-\mu}\,dt\,\frac{|t-(m-m_*)r|^4}{(1+(4\pi r)^2 \alpha^2 
|t-(m-m_*)r|^2)^{2}}w_*(t)^2
\]
\[
  \frac{1}{2} (4\pi r \alpha)^4\int^{\mu}_{-\mu}\,dt\,\frac{|t-(m-m_*)r|^4}{(1+(4\pi r)^2 \alpha^2 
|t-(m-m_*)r|^2)^{2}}d(t+m_*r)^2
\]
\[
  \le \frac{1}{2} \|d\|^2  \left(\frac{(8\pi r \alpha \mu)^4}{(1+(8\pi
      r \alpha \mu)^2)^2}\right).
  \]
\end{proof}

The inequality \ref{eqn:estresidnorm} can be interpreted as a bound 
on $\alpha$, given $\epsilon$ and $\lambda$, for a
stationary point of $\tJa$ to yield a solution of the inverse
problem: one obtains a solution, provided that $\alpha$ is
sufficiently small. On the other hand, it is clear that $\alpha$
cannot be too large if stationary points of $\tJa$ are to yield
solutions: the integrand in \ref{eqn:residnorm} is increasing in
$\alpha$ for every $t$ and $m$, and the multiplier
\[
t \mapsto (4\pi r \alpha(t-(m-m_*)r))^4(1+(4\pi r)^2 \alpha^2 
|t-(m-m_*)r|^2)^{-2}
\]
tends monotonically to $1$ as $\alpha \rightarrow \infty$, uniformly
on the complement of any open interval containing
$t=(m-m_*)r$. Therefore
\begin{equation}
  \label{eqn:elimit}
  \lim_{\alpha \rightarrow \infty}e[m,\aw[m;d];d] = \frac{1}{2}.
\end{equation}
Consequently, there exists $\alpha_{\rm max}(\epsilon,\lambda,d)$ so
that
\[
  e[m,\aw[m;d];d]  \le \frac{1}{2}\epsilon^2
  \Rightarrow \alpha \le \alpha_{\rm max}(\epsilon,\lambda,d).
\]
The existence of this limiting penalty weight has been inferred
indirectly. \cite{FuSymes2017discrepancy} describe a constructive
algorithm for its approximation, which is used in the companion paper
\cite{SymesChenMinkoff:21} to dynamically adjust $\alpha$ in the
course of optimization of $\tJa$.

I turn now to the second issue identified above, the effect of
noise. Suppose that the data trace $d$ takes the form
\begin{equation}
  \label{eqn:defdatanoisy}
  d = F[m_*]w_* + n = d_*+n,
\end{equation}
with $m_* \in M, w_* \in W_{\mu}$, $0<\mu<\lambda$, and noise trace $n \in
D$. Since no support assumptions can be made about $n$, equation
\ref{eqn:normsol} does not imply that $\aw[m;d] \in \lW$ for any values of
$\alpha$ or $\lambda$.  Therefore minimization of $\tJa$ cannot by itself yield a
solution of the inverse problem as defined in the problem statement
\ref{eqn:probstat0}. %\textcolor{red}{In this section, I explain how a solution may
%nonetheless be constructed from a stationary point of $\tJa$.
%First, examine the effect of additive noise on the estimation of the
%slowness $m$. In expressing the result, use the dimensionless
%relative data error}

%\textcolor{blue}
{To see how a solution may
  nonetheless be constructed from a stationary point of $\tJa$,
  examine the effect of additive noise on the estimation of the
slowness $m$, in terms of the dimensionless
relative data error}
\begin{equation}
  \label{eqn:defeta}
  \eta = \frac{\|n\|}{\|d_*\|}. 
\end{equation}

\begin{proposition}
  \label{thm:mnoise}
  Assume the hypotheses of Proposition \ref{thm:rampgood}, and that $d$ is
  given by definition \ref{eqn:defdatanoisy}. Suppose that $m \in M$
  is a stationary point of $\tJa[\cdot;d]$, and that
  \begin{equation}
    \label{eqn:mnoisebnd}
    \eta(1+\eta) \le \frac{16}{3\sqrt{3}}\frac{4\pi r \alpha
      (\lambda-\mu)}{(1+(4\pi r\alpha(\lambda+\mu))^2)^2}.
  \end{equation}
  Then
  \begin{equation}
    \label{eqn:mnoisebndfin}
    |m-m_*| \le \frac{\lambda}{r}.
  \end{equation}.
\end{proposition}

\begin{proof}
  From equation \ref{eqn:dexpjgen}, $d \tJa / dm$ is the
  value of a quadratic form in $d$
%  \textcolor{red}{with}
%  \textcolor{blue}
  {defined by the}
  (indefinite) symmetric operator
$B = $ multiplication by
\[
  b(t;m,\alpha)  = -\frac{(4 \pi r \alpha)^2 (t-mr)}{(1+(4\pi r \alpha (t-mr))^2)^{2}}.
\]
Therefore
\begin{equation}
  \label{eqn:gradlip}
  \left|\frac{d}{dm}\tJa[m;d]-\frac{d}{dm}\tJa[m;d_*]\right| =
  |\langle (d+d_*),B(d-d_*)\rangle| \le \max_{t \in
    \bR}|b(t;m,\alpha)|\eta(1+\eta)\|d_*\|^2
\end{equation}
A straightforward calculation shows that
\[
  \max_{t \in \bR} b(t;m,\alpha) = \frac{3\sqrt{3}}{16} 4\pi r\alpha.
\]
For a stationary point $m$ of
$\tJa[\cdot;d]$, the inequality \ref{eqn:gradlip} implies
\[
  \left|\frac{d}{dm}\tJa[m;d_*]\right| \le \frac{3\sqrt{3}}{16} 4\pi
  r\alpha \eta(1+\eta)\|d_*\|^2.
\]
On the other hand, the conclusion \ref{eqn:gradbndnonoise} of Proposition
\ref{thm:rampgood} implies that if also
\[
  \frac{3\sqrt{3}}{16} 4\pi r\alpha \eta(1+\eta)\|d_*\|^2 \le (4 \pi r
  \alpha)^2 \frac{\lambda-\mu}{(1+(4\pi r)^2\alpha^2
    (\lambda+\mu)^2)^{2}} \|d_*\|^2,
\]
then $|m-m_*|\le \lambda/r$. Rearranging, obtain the conclusion.
\end{proof}

\begin{corollary}
  \label{thm:mnoisecor}
  Asumme the hypotheses of Proposition \ref{thm:mnoise}, in particular
  that $m$ is a stationary point of $\tJa[\cdot;d]$, $d=d_*+n$. Then
  \begin{equation}
    \label{eqn:mnoisecor}
    |m-m_*| \le \frac{\mu}{r} + \frac{\eta}{\alpha} \left(\frac{3\sqrt{3}(1+\eta)}{64\pi r^2}(1+(8\pi r \alpha
      \lambda_{\rm max})^2)^2\right).
  \end{equation}
\end{corollary}

\begin{proof} Assume that $\lambda$ is chosen to obtain equality in
  the condition \ref{eqn:mnoisebnd}, substitute the bound $2
  \lambda_{\rm max}$ for $\lambda + \mu$ in the denominator, solve for
  $\lambda$ and substitute in inequality \ref{eqn:mnoisebndfin}.
\end{proof}

\noindent {\bf Remark:} This bound suggests that error in the estimate
of $m$ due to data noise is a decreasing function of $\alpha$, at
least for small $\alpha$. This result is intuitively appealing, and is
supported by numerical evidence. It may be taken as a justification
for the discrepancy-based algorithm for adjusting $\alpha$ during
inversion \cite[]{FuSymes2017discrepancy}, used in the companion paper
\cite[]{SymesChenMinkoff:21}. 

\begin{theorem}
  \label{thm:mnoiseres}
  Asumme that
  \begin{itemize}
  \item[1. ] $\alpha, \mu> 0$,
  \item[2. ] $m_* \in M, w_* \in W_{\mu}$,
  \item[3. ] $d_* = F[m_*]w_*$,
  \item[4. ] $n \in D$ and $d = d_* + n$.
  \end{itemize}
  Set $\eta = \|n\|/\|d_*\|$. Assume that $\eta$ satisfies inequality \begin{equation}
  \label{eqn:mnoisecond}    
  \eta < \frac{\sqrt{5}-1}{2},
  \end{equation}
  and that $m$ is a stationary point of $\tJa[\cdot;d]$.
  Then
  \begin{equation}
    \label{eqn:mnoisesuff}
    |m-m_*| \le \left(1+\frac{2\eta(1+\eta)}{1-\eta(1+\eta)}\right)\frac{\mu}{r}.
  \end{equation}
\end{theorem}

\begin{proof}.
  Write $\lambda = (1+\delta)\mu$, and $x=4 \pi r \alpha \mu$. Then
  the right-hand side of equation \ref{eqn:mnoisebnd} may be written as
  \begin{equation}
    \label{eqn:mnoisebndrev}
    \frac{16}{3\sqrt{3}}\frac{4\pi r \alpha
      (\lambda-\mu)}{(1+(4\pi r\alpha(\lambda+\mu))^2)^2} = D
    \frac{x}{(1+C^2 x^2)^2},
  \end{equation}
  where
  \[
    D=\frac{16}{3\sqrt{3}}\delta,\,C=2+\delta.
  \]
  The positive stationary point of the quantity on the right-hand side
  of \ref{eqn:mnoisebndrev} is a maximum, and occurs at
  $x=1/(\sqrt{3}C)$, that is
  \[
    4 \pi r \alpha \mu = \frac{1}{\sqrt{3}(2+\delta)}.
  \]
  Thus
  \[
    1+C^2x^2 = \frac{4}{3},
  \]
  hence the maximum value is
  \[
    \frac{D3\sqrt{3}}{16C} = \frac{\delta}{2+\delta}.
  \]
  This maximum value must be larger than the left hand side of inequality
  \ref{eqn:mnoisebnd}, that is,
  \[
    \eta(1+\eta) \le \frac{\delta}{2+\delta},
  \]
  in order that there be any solutions at all, but the right hand side
  is less than $1$. This observation establishes the necessity of
  hypothesis \ref{eqn:mnoisecond} of the
  theorem. Solving this above inequality for $\delta$ and unwinding
  the definitions, one finds that the right-hand side of inequality
  \ref{eqn:mnoisebnd} is bounded by \ref{eqn:mnoisesuff}, so appeal to
  Proposition \ref{thm:mnoise} finishes the proof.
\end{proof}

\noindent {\bf Remark.} That is, {\em with the choice of penalty
  multiplier $a$ given in equation \ref{eqn:ann}, and support radius
  $\mu$ of the ``noise-free'' wavelet $w_*$, $\Ja$ has no local
  minima with slownesses further than $(1+O(\eta))\mu/r)$ from the slowness used to
  generate the data}.

\noindent {\bf Remark.} The estimate $|m-m_*|r<\mu(1+O(\eta))$ for
local minima of $\Ja$ is sharp: it is possible to choose $w_* \in W_{\mu}$
so that $\mu - |m-m_*|r$ is as small as you like. In particular,
the ``exact'' or ``true'' slowness $m_*$ is not necessarily the only
slowness component of a local minimizer, or even the slowness
component of any local minimizer, and in particular
is not (necessarily) the slowness component of a global minimizer of $\Ja$.

\noindent {\bf Remark.} No similar bound could hold for much larger
noise levels than specified in condition \ref{eqn:mnoisecond}, the
right-hand side of which is a bit larger than 0.6. For example, if the noise is
the predicted data for the same wavelet $w_*$ with a substantially different
slowness $m_{\flat}$, that is, $n=F[m_{\flat}]w_*$, then a simple
symmetry argument shows that if there is a local minimizer of $\Ja[\cdot,\cdot;d_*+n]$. with
slowness near $m_*$,
there must also be a minimizer with slowness near $m_{\flat}$.
so that the difference with $m_*$ is not constrained at
all by the assumed support radius of $w_*$. So for this example with 100\% noise, no
bound of the type given by conclusion 2 could possibly hold. The
companion paper \cite[]{SymesChenMinkoff:21}
illustrates this phenomenon numerically.

\noindent{\bf Remark.} Note that $\alpha$ plays no role in the
conclusions of this theorem. It is only required that $\alpha >0$.

\noindent {\bf Remark:} I emphasize that Theorem \ref{thm:mnoiseres} states {\em sufficient} conditions for a bound on
the slowness error $|m-m_*|$ in terms of the relative data noise level $\eta$,
giving an additional ``fudge factor'' beyond the support size $\mu$
of the noise-free wavelet $w_*$ for an interval within which the slowness error is
guaranteed to lie.

Conclusion 1 in Theorem \ref{thm:mnoiseres} constrains the range of
noise level to which these results apply to a bit more than 60\%. That
is, the bound given by conclusion 2 is useful only for small noise. In
the limit as $\eta \rightarrow 0$, conclusion 2 becomes
$\lambda/\mu \ge 1 + 2\eta$, that is, the ``fudge factor'' beyond
the noise-free bound is approximatly twice the noise level.

On the other hand, stronger bounds than given by Theorem
\ref{thm:mnoiseres} are possible, given additional constraints on the
noise $n$. A natural example is uniformly distributed random noise,
filtered to have the same spectrum as the source. The expression
\ref{eqn:dexpjgen} implies that the interaction of noise $n$ and
signal $d_*$ in the derivative of $\tJa$ is local, so that the
coefficient of $\eta$ on the left-hand side of inequality
\ref{eqn:mnoisebnd} is effectively much less that 1, resulting in a
larger range of allowable $\eta$. While I will not formulate such a
result, one of the numerical examples in the companion paper
\cite[]{SymesChenMinkoff:21} suggests its feasibility.

Unless the data is noise-free, there is no reason to suppose that the
estimated wavelet $\aw[m;d]$ (Theorem \ref{thm:norminv}) will lie in
$\lW$, unless the support of the noise $n$ is restricted. In order
to construct a solution of the inverse problem \ref{eqn:probstat0}, 
project $\aw[m;d]$ onto $\lW$. For sufficiently large $\lambda,
\epsilon$, the result is a solution of the inverse problem:

\begin{theorem}
  \label{thm:ipnoisesuf}
  Assume the hypotheses of Theorem \ref{thm:mnoiseres}, and that
  inequality \ref{eqn:mnoisecond} holds, and $\mu \in
  (0,\lambda_{\rm max}]$. Then the pair
  \[
    (m,{\bf 1}_{[-\lambda,\lambda]}\aw[m,d])
  \]
  solves the inverse problem as stated in \ref{eqn:probstat0} if
  \begin{equation}
    \label{eqn:ipnoiselam}
    \left( 2+\frac{2\eta(1+\eta)}{1-\eta(1+\eta)}\right)\mu \le \lambda
    \le \lambda_{\rm max}, 
  \end{equation}
  and
  \begin{equation}
    \label{eqn:ipnoiseeps}
    \epsilon \ge \frac{(8 \pi r \alpha\lambda)^2}{1 + (8 \pi r \alpha\lambda)^2}+\eta. 
  \end{equation}    
\end{theorem}

\begin{proof}:
From Theorem \ref{thm:norminv}, 
\[
  \aw[m;d](t) = (1+ (4 \pi r \alpha t)^2)^{-1}(w_*(t+(m-m_*)) + 4\pi r 
  n(t+mr)) 
\]
\[
  = \aw[m;d_*] + (1+ (4 \pi r \alpha t)^2)^{-1}4\pi r n(t+mr).
\]
From Theorem \ref{thm:mnoiseres},
\[
  |m-m_*| \le \left(1+\frac{2\eta(1+\eta)}{1-\eta(1+\eta)}\right)\frac{\mu}{r},
\]
which from assumption \ref{eqn:ipnoiselam} is
\[
  =\left(2+\frac{2\eta(1+\eta)}{1-\eta(1+\eta)}\right)\frac{\mu}{r}
  -\frac{\mu}{r} \le \left(\frac{\lambda}{\mu}-1\right)\frac{\mu}{r}.
\]
That is,
\[
  |m-m_*|r \le \lambda-\mu.
\]
From Theorem \ref{thm:norminv},
\[
  \mbox{supp }\aw[m;d_*] \subset [-\mu-(m-m_*)r, \mu-(m-m_*)r]
  \subset [-\lambda,\lambda],
\]
so
\[
  {\bf 1}_{[-\lambda,\lambda]}\aw[m;d](t) = \aw[m;d_*](t) + {\bf 1}_{[-\lambda,\lambda]}(1+ (4 \pi r \alpha
  t)^2)^{-1}4\pi r n(t+mr).
\]
From the definition of $F[m]$, for any $t_1<t_2$, $w \in W$,  
\[
  F[m]{\bf 1}_{[t_1.t_2]}w = {\bf 1}_{[t_1+mr,t_2+mr]}F[m]w.
\]
Thus the data residual after projection is
\[
  F[m]{\bf 1}_{[-\lambda,\lambda]}\aw[m,d](t) -d(t) = F[m]\aw[m,d_*](t) -d_*(t)  
\]
\[
  + {\bf 1}_{[-\lambda+mr,\lambda+mr]}(4 \pi r \alpha (t-mr))^2 (1+ (4 \pi
  r \alpha (t-mr))^2)^{-1} n(t)
\]
\[
  -(1- {\bf 1}_{[-\lambda+mr,\lambda+mr]})n(t).
\]
From \ref{eqn:residnorm} and the bound on $m-m_*$,
\[
  \| F[m]{\bf 1}_{[-\lambda,\lambda]}\aw[m,d_*] -d_*\|^2 = (4 \pi r \alpha)^4
  \int_{-\lambda+(m-m_*)r}^{\lambda+(m-m_*)r}\,dt\, (t-(m-m_*)r)^4
\]
\[
  \times (1+(4\pi r \alpha)^2 
  (t-(m-m_*)r)^2)^{-2}d_*(t)^2
\]
\[
  \le (4 \pi r \alpha \lambda)^4\|d_*\|^2.
\]
Similarly, the norm squared of the sum of the last two terms is 
\[
  \le (4 \pi r \alpha \lambda)^2 \|{\bf 1}_{[-\lambda+mr,\lambda+mr]}
  n\|^2 + \|(1 - {\bf 1}_{[-\lambda+mr,\lambda+mr]}n\|^2.
\]
Without additional hypotheses to outlaw the accumulation of $n$ near
$t=mr$, all that can be said is that this is
\[
  \le \max \{(4 \pi r \alpha \lambda)^2, 1\} \|n\|^2.
\]
Putting this all together,
\[
  \|F[m]{\bf 1}_{[-\lambda,\lambda]}\aw[m,d]-d\| \le (4 \pi r \alpha
  \lambda)^2\|d_*\| + \max \{4\pi r \alpha \lambda, 1\}\|n\|
\]
\[
  \le ((4 \pi r \alpha \lambda)^2 +\max \{4\pi r \alpha \lambda, 1\}
  \eta)\|d_*\|.
\]
If the right-hand side is to be less than $\|d_*\|$ as required by the
definition \ref{eqn:probstat0} of the inverse problem, then
necessarily $4\pi r\alpha \lambda < 1$, so the right hand side in the
preceding inequality is bounded by the right hand side of assumption
\ref{eqn:ipnoiseeps} of the theorem. Therefore this assumption implies
that the relative residual is $\le \epsilon$.
\end{proof}

\noindent {\bf Remark:} Note that the sufficient condition \ref{eqn:ipnoiselam} for
$\lambda$ is independent of $\alpha$. It follows that for any choice
of $\lambda$ consistent with this bound, $(m,{\bf
  1}_{[-\lambda,\lambda]}\aw[m,d])$ is a solution of the inverse
problem for any $\epsilon > 0$ provided that $\alpha$ is chosen
sufficiently small ($O(\sqrt{\epsilon})$).


\bibliographystyle{seg}
\bibliography{../../bib/masterref}<

%\title{Solution of an Inverse Problem for the Wave Equation by means of the Source-Receiver Extension}
%\author{Guanghui Huang and William. W. Symes}

%\address{The Rice Inversion Project,
%Rice University,
%Houston TX 77251-1892 USA, email {\tt symes@caam.rice.edu}.}

\author{
Guanghui Huang\thanks{Department of Computational and Applied Mathematics, Rice University,
Houston, TX, 77005, USA,
{\tt ghhuang@rice.edu}} \ and \ William Symes\thanks{Department of Computational and Applied Mathematics, Rice University,
Houston, TX, 77005, USA,
{\tt symes@rice.edu}}
}

\lefthead{Huang and Symes}

\righthead{Source-Receiver Extension}

\maketitle
\parskip 12pt

\begin{abstract}
  Iterative gradient-based minimization of mean-square misfit often
  results in suboptimal estimation of sound speed from remote
  recordings of waves. Model extension, that
  is, enlargement of the model space (domain of the simulation
  operator), coupled with a penalty for deviation from the original
  domain, permits descent methods to converge to a global minimizer of
  mean-square misfit. This paper applies the model extension concept
  to a simple transmission inverse problem for the wave equation modeled on crosswell
  tomography. The model extension studied here permits the energy
  source to depend source and receiver coordinates, whereas the
  modeled physics dictates that all energy sources should be the same
  (and given a priori). Minimization of a suitable penalty for
  deviation from the prescribed source model drives the sound velocity
  towards a global solution of the least squares data fit problem. An
  analysis based on geometric optics explains why the straightforward
  least-squares formulation fails to yield a useful algorithm, and why the
  extended formulation succeeds. The analysis requires the absence of
  caustics; we show that a substantial modification of the algorithm
  is required if these are present, and suggest such a modification
  for which numerical experiments provide some evidence of
  effectiveness.
\end{abstract}

\section{Introduction}
Estimation of sound speed in the wave equation from by least-squares
fitting of time-like seismic trace data (``full waveform inversion''
or FWI, in the seismic literature) has demonstrated potential to
produce highly detailed maps of the earth's sound speed
distribution. As an optimization problem, however, FWI is notoriously
difficult. Many realistic examples, using both synthetic and field
data, appear to show that iterative descent methods tend to stagnate
quite far from any global minimizer, at points in parameter space
having little to do with any accurate rendition of earth
structure. This commonly-observed stagnation has led many authors to
infer the existence of local minimizers of the least-squares
objective, caused by serendipitous partial fits of oscillatory data,
better than those produced by any nearby model, and preventing
improvement of fit by local optimization. A combination of data accuracy
at sufficiently low frequency and sufficiently accurate initial
velocity estimates will overcome this obstacle, but such combinations
are not always available, and in any case difficult to verify.
% This ``cycle
%skipping'' phenomenon has been conclusively demonstrated in only
%trivial cases, but is widely believed to be generic.  In any case,
%premature stagnation of iterative optimization methods is the biggest
%impediment to routine use of least-squares data fitting. It may be
%overcome by use of either sufficiently low-frequency data that the
%correct cycles in the simulated data overlap those in the measured
%data for reasonable initial estimate of sound velocity, or
%sufficiently accurate initial estimate that the correct cycles overlap
%for the measured frequency range. However frequency content of seismic
%data is limited by experimental design, and a priori knowledge of
%earth structure on the required scales is limited. It is not
%necessarily possible to achieve the right combination of frequency
%content and initial sound velocity accuracy, or even to know whether
%it has been achieved. 
See \cite{VirieuxOperto:09} for a good overview
%of FWI as currently construed by industrial and academic
seismologists.

This paper explores another solution of the FWI stagnation problem,
via extending the modeling operator to a larger  {\em infeasible} domain, so that the
data is easy to fit (in principle perfectly) by local optimization. We discuss the structure of FWI, and of its
{\em source-receiver} extension, for a simple acoustic transmission
problem, idealizing crosswell waveform tomography. Extended
(infeasible) solutions must be driven towards feasibility somehow.
We introduce measures of infeasibility, in the form of {\em
  annihilation} operators, the kernels of which are the original (feasible, physical)
domain of the modeling operator. We use these operators to define
auxiliary constrained minimization problems, and show that these
problems are convex over much larger domains than is the feasible
least squares problem (FWI).

The data of the idealized crosswell tomography problem
are functions of time, representing the acoustic
fields recorded at receivers position on a line in the plane. Source
processes generating these fields lie on another line in the plane,
and the sound velocity of the region between the lines is to  be
recovered from the data. It is assumed a priori that source processes
differ only by position on the line, and that this unique source
process is known. This assumption is an idealization: the effective
source process in actual crosswell surveys may well change from one
source position to another, even if the energy source apparatus
remains the same, due to coupling with the rock formation and other
near-field effects. Also, this source is seldom measured directly in
any usable way. For the purpose of this paper, however, we set aside
these issues.

The {\em source-receiver} extension discussed here models each
function of time, recording the acoustic field for each source and
receiver posiiton, as resulting from an independent source
process. That is, the source processes are permitted to depend on both
source and receiver coordinates, whereas physically they should be
independent of these coordinates. The larger model space consists of
the sound velocity field together with a set of source/receiver
dependent filters, one for each time function (``trace'') in the
data), which modify the output of the simulation operator at each
source/receiver combination. Unsurprisingly, the addition of so many
parameters to the domain of the modeling operator makes data fitting
considerably easier - in fact, for essentially any distribution of
sound velocity, it is possible to adjust the filters provided by the
extension to fit the data. Of the myriad possible measures of
non-physicality, we single out two for careful study - the mean square
of the image under multiplication by $t$, and the mean square of the derivative with respect to source and
receiver coordinates. Forcing either
quantity toward zero, while maintaining data fit, implicitly recovers a minimizer of the original
mean-square misfit function, for model-consistent data, provided that
none of the velocity fields encountered along the way produce
conjugate points, or caustics. Our main result in this case is that
the objective functions described above closely approximate objectives
for the traveltime inverse problem (``tomography''), hence inherit
their good properties, including in some cases global convergence.  

In the presence of caustics, the close relation between tomography and
the source-receiver extended source inverse problems breaks down. We describe the
mechanism of this failure: it results from implicit attempts to match
different branches of traveltime. We show how to overcome this
problem, at least to some extent, via exponential weighting of data, a
technique introduced by Pratt (REF) to force emphasize fitting of
first-arriving waveforms in FWI.  

\section{A model problem based on crosswell tomography'}

Crosswell seismic tomography  aims to  estimate acoustic structure between
two wells from seismic signals (``traces'') recorded in one of the wells due to seismic
energy sources in the other. The simplest useful realization of this
inverse problem uses linear acoustics to describe the generation and
propagation of seismic waves in the earth between the source and
receiver wells. In this paper we adopt the further simplifying
assumption that material density is constant and can be absorbed in
other quantities. The remaining parameter of linear acoustics is the
sound velocity, which varies with position, to reflect the
spatial heterogeniety of sedimentary rocks. With these assumptions,
crosswell tomography seeks to recover the sound velocity, or an
equivalent quantity, in the region
between the wells from measurements of acoustic fields at the wells.

Linear acoustics is a gross approximation to actual seismic wave
physics. We will make quite a few other approximations and
idealizations. The discussion section will offer some thoughts about
how important or limiting these assumptions are, and what prospects
exist for transcending them. 

To describe the problem setting more precisely, we introduce
notation that will be used throughout the discussion. Coordinates in
$\bR^3$ are $\bx=(x,y,z)$, $z$ representing depth. We assume that the
source and receiver wells are 
perfectly straight and vertical line segments in $\bR^3$(certainly an
idealization!), lying in the plane $y=0$: the
source ``well'' $W_s$and receiver ``well'' $W_r$ are defined by the
limit depths $z_s^{\rm min}$,$z_s^{\rm max}$,$z_r^{\rm min}$,$z_r^{\rm
  max}$: 
\begin{eqnarray}
\label{eqn:wells} 
W_s & = & \{\bx_s = (x_s,0,z_s): z_s^{\rm min} \le z_s \le z_s^{\rm 
          max}\} \nonumber \\
W_r & = & \{\bx_r = (x_r,0,z_r): z_r^{\rm min} \le z_r \le z_r^{\rm 
          max}\} 
\end{eqnarray}
As we will frequently need to refer to the depth ranges in the wells,
introduce
\begin{eqnarray}
\label{eqn:depths} 
Z_s & = & [z_s^{\rm min}, z_s^{\rm max}] \nonumber \\
Z_r & = & [z_r^{\rm min},z_r^{\rm max}] 
\end{eqnarray}
We idealize energy sources as isotropic point radiators. A radiator at
$\bx=\bx_s$ appears as the right-hand side in the acoustic wave
equation, of the form $\delta(\bx-\bx_s)f(t)$. 
The function $f(t)$ is causal (vanishes on a negative half-axis, that
is,  for $t<<0$), and has units of
mass/time/time. The idealized model posits that every
source is has the same mechanism, modeled by the
isotropic point radiator with the same time dependence $f$, differing only in
depth $z=z_s$ in the source well.

Slowness (reciprocal velocity) turns out to be a more convenient
choice than velocity to represent the spatially varying acoustic
properties of the interwell material. We denote the slowness by $m$.
For physical reasons, $m$ should be positive and bounded uniformly above and 
below, say by $0 < m_{\rm min} < m_{\rm max}$.

We assume that the
material is {\em acoustically transparent}, that is, generates no
reflected waves of significant energy in the seismic band. The geometric
optics approximation, to be summarized below, shows that a convenient
mathematical proxy for transparency is the assumption that $m$ is
smooth, with derivatives up to
order 2 bounded by another constant $\mu > 0$. These conditions are
summarized in the definition of the feasible set of slowness models $M$:
\begin{equation}
\label{eqn:vfeas}
M=\{m \in C^{\infty}(\bR^3): m_{\rm min} < m(\bx) < m_{\rm max},
|D^{\alpha}m(\bx)| \le \mu,
\bx \in \bR^3, |\alpha| \le 2\}
\end{equation}

The pressure field 
$p(\bx,t;\bx_s)$ due to the isotropic point radiator source at 
$\bx_s=(x_s,0,z_s)$, with time dependence $f(t)$, is the solution of the initial value problem 
\begin{eqnarray}
\left(m^2 \frac{\pa^2 p}{\pa t^2} - \nabla^2 p\right)(\bx,t) &=&
     \delta(\bx-\bx_s) f(t), \nonumber \\
 p&=&0, \, t<<0 \label{eqn:pressure}
\end{eqnarray}
We will establish the existance, uniqueness, and essential properties
of solutions $p$ in the next section. Proceeding formally for the moment,
define the modeling operator $S[m]$ to relate the slowness $m$ and
source function $f$ to the pressure field at the receiver locations
$\bx_r$ and source locations $\bx_s$, for a finite time interval $0
\le t \le t^{\rm max}$. Since the horizontal coordinates of sources and
receivers are fixed as part of the idealized cross-well acquisition
geometry, we regard the output of $S[m]$ as a function of $z_s,z_r,$
and $t$, with domain
\begin{equation}
\label{eqn:omdef}
\Omega = Z_r\times Z_s \times \bR
\end{equation}
 A smooth cutoff factor provides
convenient way to incorporate the limited time duration of measured
signals, and to smoothly taper the measured data to zero near the
boundary of the measurement domain, with favorable signal-processing
consequences. We denote this factor by
\begin{equation}
\label{eqn:wdef}
w \in C_0^{\infty}( \Omega): \,{\rm supp}\,w \subset 
Z_r\times Z_s \times (0,t^{\rm max}).
\end{equation}
Proceeding formally, define
\begin{equation}
\label{eqn:sdef}
S[m]f(z_r,t;z_s) = w(z_r,t;z_s) p(\bx_r,t;\bx_s), 
\end{equation}
The crosswell waveform inversion problem, in the highly idealized form 
studied here, is: given data $d$, a function of on $\Omega$,
find slowness $m \in M$ and source function $f$ such that 
$S[m]f \approx wd$. 


\section{Properties of the Modeling Operator}

The problem formulation just stated begs several questions. For
example, what sort of functions $d$, $f$, and $m$ should be admitted?
Under what circumstances is the restriction of the pressure field $p$
to the submanifold $W_s \times W_r \times \bR$ well-defined?

For numerical convenience, and on physical grounds
\cite[]{SantosaSymes:00}, $f$ and $d$ should be square-integrable and
the natural (and conventional) measure for the data misfit $S[m]f-d$
is $L^2$. Indeed, iff $m \in M$ is constant, $f \in C_0^{\infty}(\bR)$,
$\bx \ne \bx_s$, then
\be\label{eqn:cv} 
p(\bx,t;\bx_s) = 
\frac{f(t-mr)}{4\pi r}, \,\,r=|\bx-\bx_s|.  
\ee 
The distribution on the right-hand side of equation \ref{eqn:cv} has a
well-defined restriction to $\bx=\bx_r$, for reasons to be reviewed
shortly. so that $S[m]$ given by
equation \ref{eqn:sdef} is well-defined, and extends continuously to a
map $f \in L^2(\bR) \mapsto S[m]f \in L^2(\bar{\Omega})$. 

More generally, 

\begin{theorem}
\label{thm:trace}
For any $m \in M$, $f \in {\cal E}'(\bR)$, there exists a unique solution $p \in {\cal D}'(\bR^4)$ of problem
\ref{eqn:pressure}. Define $\chi_{\bx_r}: \bR \rightarrow \bR^4$ by
$\chi_{\bx_r}(t) = (\bx_r,t)$. For any $\bx_r \in \bR^3\setminus
\{\bx_s\}$ , the map $f\mapsto \chi_{\bx_r}^*p$ is continuous: ${\cal E}'(\bR) \rightarrow
{\cal D}'(\bR)$. Denote by $G(\cdot,\cdot;\bx_s) \in {\cal D}'(\bR^4)$
the solution of the initial value problem
\ref{eqn:pressure} with $f = \delta$. Then $\chi_{\bx_r}^*p =
(\chi_{\bx_r}^*G(\cdot,\cdot;\bx_s))*f =
\chi_{\bx_r}(G(\cdot,\cdot;\bx_s)*(\delta_{\bR^3}\otimes f))$. 
\end{theorem}
\begin{proof}
  A proof of the existence and uniqueness of causal distribution
  solutions to \ref{eqn:pressure} may be found in \cite{Lax:PDENotes},
  Chapter 6. Alternatively, since $\delta(\cdot-\bx_s)\otimes f \in
  H^s(\bR^4)$ for some $s < -2$, \cite{Tay:81} 2.1-2.2 shows
  how to construct local solutions, which can be patched together
  taking advantage of the uniformly bounded speed of propagation
  implicit in $m \in M$. From the definition (\cite{Tay:81}, Chapter VI), the wave front set
  $WF(p)$ is contained in the set
\[
C \cup \Pi^{-1}({\{\bx_s\} \times \bR})
\]
in which $C$ is the characteristic variety of the wave operator,
\begin{equation}
\label{eqn:charvar}
C[m] = \{(\bx,t,\xi,\omega) \in T^*(\bR^4): m^2(\bx)\omega^2 =
|\xi|^2\}
\end{equation}
and  $\Pi: T^*(\bR^4) \rightarrow \bR^4$ is the fiber projection.

The normal bundle of $\{\bx_r\} \times \bR$ consists of covectors with
coordinates $(\bx_r,t,\xi,0)$, hence has trivial intersection with
$C[m]$. Therefore the restriction of $p$ to $\{\bx_r\} \times \bR$ is
well-defined (\cite{Dui:95}, Proposition 1.3.3), as is the pull-back
by $\chi_{\bx_r}$. Convolution in time
with $f$ is by definition convolution in space-time with
$\delta_{\bR^3} \otimes f \in {\cal E}'(\bR^4)$, a continuous map on ${\cal
  D'}(\bR^4)$,  and this latter convolution commutes with
the wave operator in \ref{eqn:pressure}. This convolution also maps
the right-hand side for $G$ to the right-hand-side for $p$.
\end{proof}

\begin{rem} $G$ is of course the causal Green's function, or distribution
  kernel of the retarded fundamental solution \cite[]{CourHil:62}.
\end{rem}

An immediate consequence of this result is a proper definition of the
map $S[m]$.

\begin{theorem}
\label{thm:gather}
Denote by $\Delta = \{(\bx,\bx): \bx \in \bR^3\}$ the diagonal. With
notation as in Theorem \ref{thm:trace}, the map $(\bR^3 \times \bR^3)
\setminus \Delta \times {\cal E}'(\bR) \rightarrow {\cal D}'(\bR)$ defined by $(\bx_r,\bx_s,f)
\mapsto \chi_{\bx_r}^*p(\cdot,\cdot;\bx_s)$ is continuous. 
\end{theorem}
\begin{proof}
The right-hand side of \ref{eqn:pressure} is a continuous
${\cal E}'(\bR^4)$- valued function of $\bx_s$. Denote by $N$ the
normal bundle of the fibration $\{\{\bx\} \times \bR: \bx \in \bR^3\}$.
Denote by $\Gamma$ a closed conic submanifold of
$T^*(\bR^4)$ containing $C$ in its interior. chosen so that $\Gamma
\cap N=\{0\}$. The solution of the wave equation is a continuous
${\cal D}'(\bR^4)$-valued function of the ${\cal E}'(\bR^4)$- valued
right-hand side. Therefore solution of \ref{eqn:pressure} is a continuous
${\cal D}'(\bR^4)$-valued function of $\bx_s\in \bR^3$ and $f\in {\cal
  E}'(\bR)$, and also continous in the sense of 
${\cal D}_{\Gamma}'(\bR^4\setminus \{\bx_s\})$ as a consequence of the
propagation of singularities theorem. Since $N \cap \Pi^{-1}(\bx_r) =\{\bx_r\} \times \bR$,
$\chi_{\bx_r}^*$ is continuous: $\bR^3 \times {\cal D_{\Gamma}}'(\bR^4) \rightarrow
{\cal D}'(\bR)$. Putting all of this together, obtain the 
statement of the theorem. 
\end{proof}

Theorem \ref{thm:gather} permits a precise rewrite of the
formal definition \ref{eqn:sdef}: define $S[m]: {\cal E}'(\bR)
\rightarrow C(Z_r \times [z_x^{\rm min},z_s^{\rm max}], {\cal E}'(\bR))$ by 
\[
S[m]f(z_r,\cdot;z_s) = w(z_r,\cdot;z_s) \chi_{\bx_r}^*p(\cdot,\cdot;\bx_s)
\]

To formulate the inverse problem as envisioned in the last section, an
analogous result is needed with $L^2$ replacing ${\cal E}'$. This step
requires the introduction of geometric optics.

Recall that the rays of geometric optics are trajectories
$(X(t),P(t))$ in $\bR^3 \times (\bR^3 \setminus \{0\})$ satisfy 
\begin{eqnarray}
\label{eqn:ham}
\frac{dX}{dt} &=&m(X)^{-2}P,\nonumber \\
\frac{dP}{dt} &=&\nabla \log m(X), \nonumber\\
|P| &=& m(X).
\end{eqnarray}
The condition $m \in M$ implies that trajectories exist that solve the
first two conditions in \ref{eqn:ham} for all $t>0$, for any initial
values of $X$ and $P$. The third condition is compatible with the
other two: if the initial $X$, $P$ satisfy it, then it is satisfied
along the entire ray.

Define the {\em geodesic exponential map}
\[
\exp(t,P_s; \bx_s) = X(t)
\]
where $(X(t),P(t))$ solves \ref{eqn:ham} with $X(0)=\bx_s$,
$P(0)=P_s, |P_s|=m(\bx_s)$: $(t,P_s)$ are {\em
  geodesic normal coordinates} centered at $\bx_s$. Denote by
$S_m(\bR^3)$ the sphere bundle defined by the Riemannian metric $m (dx
\otimes dx + dy \otimes dy + dz \otimes dz)$ defined by the slowness
$m \in M$. It is a standard
result that $t^*:S_m(\bR^3) \rightarrow \bR_+$ exists for which
$\exp$ is a diffeomorphism from 
\begin{equation}
\label{eqn:prenorm}
\tilde{{\cal N}} = \{(\bx_s,t,P_s) \in
\bR^3 \times \bR_+ \times m(\bx_s)S^2: 0 < t < t^*(\bx,P_s)\}
\end{equation}
 onto a punctured
neighborhood ${\cal N}(\bx_s) \setminus \{\bx_s\}$ of $\bx_s$. A {\em normal neighborhood}
${\cal N}$ is an open subset of $\bR^3$ for which ${\cal N} \subset
{\cal N}(\bx)$ for every $\bx \in {\cal N}$. It is also a standard
result that every point $\bx \in \bR^3$ has a normal neighborhood (\cite{Friedlander:75}, Chapter 2). 

In a normal neighborhood of $\bx_s$, the inverse of the exponential
map expresses $t$ as a function of $\bx$: this function is the
{\em travel time} $\tau(\bx,\bx_s)$.
The travel time is differentiable and satisifies the eikonal equation in the punctured normal
neighborhood (and in a suitable generalized sense in the entire normal
neighborhood):
\begin{eqnarray}
\label{eqn:eik}
|\nabla_{\bx} \tau(\bx,\bx_s)|&=&m(\bx), \bx \in {\cal N}, \nonumber \\
\tau(\bx,\bx_s) & \sim & m(\bx_s)|\bx-\bx_s| \mbox{ as } \bx \rightarrow 
\bx_s. 
\end{eqnarray}

A slowness field $m \in M$ satisfies the {\em simple ray geometry}
hypothesis iff for every $\bx_s \in W_s$, the
the receiver well $W_r$ is contained in a
normal neighborhood ${\cal N}(\bx_s)$ centered at $\bx_s$. In this case, there is a common
neighborhood ${\cal N}$, normal for all points $\bx_s$ in the source
well and containing the closure of the receiver well.


A slowness fails to satisfy simple ray geometry hypothesis if the
exponential map at some source point $\bx_s$ fails to be a
diffeomorphism in a neighborhood of the receiver well, by failing to
be injective, that is, more than one ray (``path'') connects $\bx_s$
with another point,

In the simple ray geometry case, the Green's function has a useful
decomposition. 

\begin{theorem}
\label{thm:simplegreen}
Suppose that $m$ satisfies the simple ray geometry assumption. Then there exists $t^{\rm max} >
\max\{\tau(\bx_r,\bx_s): \bx_r \in W_r, \bx_s \in W_s\}$, and 
functions $a \in C^{\infty}(W_r\times W_s), b\in
C^{\infty}(W_r,W_s,(0,t^{\rm max})$ so that for
$\bx_r \in W_r, \bx_s \in W_s, t <t^{\rm max}$,
\begin{equation}
\label{eqn:simplegreen}
G(\bx_r,t;\bx_s) = a(\bx_r,\bx_s)\delta(t-\tau(\bx_r,\bx_s)) + b(\bx_r,t;\bx_s)H(t-\tau(\bx_r,\bx_s)).
\end{equation}
\end{theorem}
\begin{proof}
The starting point is the parametrix contruction due to Hadamard, as described
by \cite{Friedlander:75}, Theorem 4.3.2. In the
notation used here, for $\bx,\bx_s \in {\cal N}$ (a normal
neighborhood), there exist $U, V \in C^{\infty}({\cal N}\times{\cal N}
\times \bR)$ so that
\begin{equation} 
\label{eqn:parametrix}
\tilde{G}(\bx,t;\bx_s) = U(\bx,t;\bx_s)\delta^+(t^2-\tau^2(\bx,\bx_s))
+ V(\bx,t;\bx_s)H^+(t^2-\tau(\bx,\bx_s))
\end{equation}
Here $\delta^+(t^2-\tau^2)$ is defined as the limit
\[
\delta^+(t^2-\tau^2) = \lim_{\epsilon \rightarrow
  0^+}H(t-\tau)\delta(t^2-\tau^2-\epsilon)
\]
The product on the right is well-defined, as the singular supports of
the factors are disjoint, and picks out the positive-time sheet of the
distorted hyperboloid $t^2 = \tau^2$. The limit exists in the sense of
${\cal D}'$.

$\tilde{G}$ is a $C^{\infty}$ parametrix, in the sense that 
\[
\left(m\frac{\partial^2}{\partial
    t^2}-\nabla^2\right)\tilde{G}(\bx,t;\bx_s) =
\delta(\bx-\bx_s) + R(\bx,t;\bx_s), \,\,R \in C^{\infty}({\cal
  N}\times{\cal N}\times \bR),\,R=0 \mbox{ for } t<\tau.
\]
Define $\tilde{R}$ to be the solution of the initial value problem
\[
\left(m\frac{\partial^2}{\partial
    t^2}-\nabla^2\right)\tilde{R} = -R; \,\,\tilde{R}=0, t<0
\]
This problem has a unique $C^{\infty}$ solution. The
domain-of-dependence property implies that $\tilde{R}=0$ for $t <
\tau$ also. 

Any singularity of $G$ in ${\cal N}\times{\cal N}\times \bR$ lies
either on the light cone $t=\tau$ or must occur in $t > t^{\rm
  max}$. So on ${\cal N}\times{\cal N}\times (0,t^{\rm max})$, 
\[
G = \tilde{G} + \tilde{R}.
\]

Finally, since 
\[
\delta^+(t^2-\tau^2) = \frac{\delta(t-\tau)}{2\tau}
\]
for $t>0$, or equivalently, $\tau>0$, set $a = U/2\tau, b = V +
\tilde{R}$ to obtain \ref{eqn:simplegreen}.
\end{proof}

\begin{rem}
In the notation used here, 
\[
U(\bx,\bx_s) = \frac{m(\bx_s)}{2\pi} + O(|\bx-\bx_s|)
\]
so for constant $m$,
\[
a(\bx_r,\bx_s) = \frac{m}{4\pi\tau(\bx,\bx_s)} +
O(1)=\frac{1}{4\pi|\bx-\bx_s|} + O(1)
\]
which recovers \ref{eqn:cv} approximately for $\bx_r$ close to $\bx_s$
- of course, the error term actually vanishes in this case.
\end{rem}

\begin{cor}
\label{thm:gosimple}
Under the assumptions of Theorem \ref{thm:simplegreen}, 
for $\bx_r \in W_r, \bx_s \in W_s$, and $t < t^{\rm max}$, set $A(z_r,z_s) = a(\bx_r,\bx_s),
T(z_r,z_s)=\tau(\bx_r,\bx_s), B(z_r,t;z_s) = b(\bx_r,t;\bx_s)$. Then
$A,T \in C^{\infty}(Z_r \times 
Z_s)$,
$B \in \in C^{\infty}(Z_r \times Z_s\times (-\infty,t^{\rm max}))$, $T>0$. For $f \in
L^2(\bR)$,
\begin{equation}
\label{eqn:gosimple} 
S[m]f (z_r,t;z_s) = w(z_r,t;z_s)\left(A(z_r,z_s)f(t-T(z_r,z_s)) + \int_{T(z_r,z_s)}^{\infty}
\,ds\,B(z_r,s;z_s)f(t-s)\right). 
\end{equation}
\end{cor}

For highly oscillatory $f$, the first term in (\ref{eqn:gosimple}) is much
larger than the second, and can be taken as a high frequency
asymptotic approximation of $S[m]f$. 

Analysis of the inverse problem requires some information on 
the smooth dependence of $S$ on $m$. 

\begin{theorem}
\label{thm:smdep}
Denote by $M^k$ be the interior of the closure of $M$ in $C^k({\cal
  N})$; $M^k$ is defined by the same conditions as $M$ but with
$C^{\infty}$ replaces by $C^k$.  Then
\begin{itemize}

\item[1. ] If $N$ is a normal neighborhood for $m_0 \in M^k$,
  $k \ge 2$, and $\tilde{\cal N}$ is its pre-image under the
  exponential map for $m_0$ as defined in \ref{eqn:prenorm}, then there is an open neighborhood ${\cal U}$ of $m_0$ in
  $M^k$ so that ${\cal N}$ is a normal neighborhood for every
  $m \in {\cal U}$.
\item[2. ] For ${\cal N}, \tilde{\cal N}$, and ${\cal U}$ as defined
  in item 1, 
\begin{equation}
\label{eqn:expreg}
\exp \in C^0({\cal U}, C^k(\tilde{\cal N},{\cal N})) \cap C^1({\cal
  U}, C^{k-1}(\tilde{\cal N},{\cal N}))
\end{equation}
\item[3. ] Suppose that the simple ray geometry assumption holds for
an open set ${\cal U} \subset M^k$ (that is, for every $m \in {\cal
  U}$ and that ${\cal N}$ is a normal neighborhood for every $m \in
{\cal U}$ containing $W_r$ and $W_s$. Then $T, A,$ and
$B$ extend to functions on ${\cal U}$ for $k \ge 2$ so that
\begin{itemize}
\item[3.1 ]$T \in C^1({\cal U},C^{k-2}({\cal N}\times{\cal N}))$ 
\item[3.2 ]$A \in C^1({\cal U},C^{k-2}({\cal N}\times{\cal N}))$ 
\item[3.3 ]$B \in C^1({\cal U},,C^{k-2}({\cal N}\times{\cal N}\times(-\infty,t^{\rm
    max})))$
\item[3.4 ]$S \in C^0({\cal U},{\cal L}_s(L^2(\bR),
C^0(Z_r\times Z_s))),
  L^2(\bR))))$
\item[3.5 ]$S \in G^1({\cal U},{\cal L}_s(H^1_0(\bR),
C^0(Z_r\times Z_s))),
  L^2(\bR))))$
\end{itemize} 
\end{itemize}
\end{theorem}
\begin{proof}
The items 1., 2., 3.1, 3.2, and 3.3 follow from the standard theorem
  on continuous dependence of solutions of ordinary differential
  equations on parameters applied to the Hamiltonian system
  \ref{eqn:ham}, and to the transport equations defining the functions
  $U$ and $V$ appearing in the Hadamard construction, and the inverse
  function theorem applied to the exponential map. Note that $B$ is
  implicitly extended smoothly to $t < \tau(\bx,\bx_s)$, possible
  because the intersection of the light cone $t=\tau(\bx,\bx_s)$ with
  $W_r$ is a smooth curve. In items 3.4 and
  3.5, ${\cal L}_s({\cal B}_1,{\cal B}_2)$ denotes the topological
  vector space of continuous linear maps from Banach space
  ${\cal B}_1$ to Banach space $ {\cal B}_2$, endowed with the strong
  operator topology, that is, the topology of pointwise
  convergence. That is, if $m_j \rightarrow m$ in
  ${\cal U} \subset M^k$, then for each $f \in L^2(\bR)$,
  $S[m_j]f \rightarrow S[m]f$ in the sense indicated in item
  3.4. In item 3.5, $G^1$ indicates the class of
  G\^{a}teaux-differentiable functions: that is, for each $m \in {\cal
    U}$, there exists a linear map 
\[
DS[m]: M^k \rightarrow {\cal L}_s(H^1_0(\bR),
C^0(Z_r\times Z_s,L^2(\bR))),
\]
so that for each $\delta m \in M^k$, $f \in L^2(\bR)$,
\[
\|(S[m + h\delta m]-S[m]-hDS[m]\delta m) f \| = o(h) \mbox{ as }
h\rightarrow 0.
\]
Items 3.4 and 3.5 follow directly from the the other
  conclusions and Corollary \ref{thm:gosimple}.
\end{proof}

\begin{rem}
\label{rem:nounif}
  It is not possible to replace ${\cal L}_s$ with ${\cal L}$, the same
  vector space of continuous linear maps with the (Banach) topology of
  uniform convergence, in the statements of items 3.4 and 3.5. To see
  this, suppose that $m_j \rightarrow m$ in $M^k$. Assume that $m$
  satisfies the simple ray geometry condition with normal neighborhood
  ${\cal N}$ containing $W_r$ and $W_s$, and that the sequence
  $\{m_j\}$ lies inside an open ball ${\cal U}$ sufficiently small
  that ${\cal N}$ is a normal neighborhood for every member of this
  sequence. Finally suppose that
\[
\inf\{\tau[m_j](z_s,z_r) -\tau[m](z_s,z_r)\}=\epsilon_j > 0.
\]
This condition can be arranged, for example, by taking
$m_j=(1+c2^{-j})m$ for sufficiently small $c>0$. For each $j$, choose
$f_j \in L^2(\bR)$ so that 
\[
\|f_j\|=1,\,\,{\rm supp}\,f_j \subset [0,\epsilon_j/2].
\]
Then ${\rm supp}\,f_j(\cdot-\tau[m_j])$ and ${\rm supp}\,f_j(\cdot-\tau[m])$ are
disjoint. It is easy to see that the integrated term in the expansion
\ref{eqn:gosimple} tends to zero,  and item 2 above implies that
$A[m_j] \rightarrow A[m]$, so 
\[
\|S[m_j]f_j-S[m]f_j\|^2\rightarrow \|A[m_j]f_j(\cdot-\tau[m_j])\|^2 + \|A[m]f_j(\cdot-\tau[m])\|^2 
\]
\[
\rightarrow 2 \int_{z_r^{\rm min}}^{z_r^{\rm
    max}}\,dz_r\,\int_{z_s^{\rm min}}^{z_s^{\rm
    max}}\,dz_s\,|A[m](z_r,z_s)|^2 > 0
\]
Thus $S[m_j]$ does not tend to $S[m]$ in operator norm. 

This observation captures a very important characteristic of all
inverse problems in wave propagation in which wave velocity is an
unknown: the data prediction map is not uniformly continuous in
material parameters and source parameters jointly. This fact explains
the character of the full waveform inversion problem, to be analyzed
in the next section.
\end{rem}

\begin{rem}
\label{rem:derivloss}
Note that for $m, \delta m \in M^k$, $DS[m]\delta m$ is continuous
on a smaller domain than is the case for $S[m]$. In fact, it follows
from Corollary \ref{thm:gosimple} that for $f \in H^1_0(\bR)$,
\[
D(S[m]f)\delta m  = w\left(-A[m](DT[m]\delta m)\frac{df}{dt}(t-T[m]) +
((DA[m]\delta m)\right.
\]
\begin{equation}
\label{eqn:preq}
\left. -B[m](DT[m]\delta m))f(t-T[m]) + \int_{T(z_r,z_s)}^{\infty} \,ds\,
(DB[m]\delta m)(s)f(t-s)\right)
\end{equation}
This observation shows that the statement of item 5 is sharp.
\end{rem}

\begin{rem}
\label{rem:multipath}
In general, geometric optics predicts more than one traveltime between
each source and receiver (the multipath case), hence a principal term
involving multiple shifted copies f. It is possible to show that in a
subset of $W_r \times W_s$ of second Baire category, smooth travel times
$T_i$, amplitudes $a_i$, and remainders $B_i$ exist so that
\begin{equation}
\label{eqn:gomultipath} 
S[m]f (z_r,t;z_s) = w(z_r,t;z_s)\left(\sum_{i=0}^NA_i(z_r,z_s)f(t-T_i(z_r,z_s)) + \int_{T_i(z_r,z_s)}^{\infty} 
\,ds\,B_i(z_r,s;z_s)f(t-s)\right). 
\end{equation}
We will not give a detailed justification of this expansion, but will
note one of its consequences later.
\end{rem}

\section{Full Waveform Inversion (FWI)}

In view of the mapping property described in the previous section, a
natural explicit formulation of the inverse problem is via least
squares: given $d\in L^2(\Omega)$ and $f\in L^2(\bR)$, find $m \in M$ to minimize 
\be\label{eqn:ls}
 J_{\rm LS}[m;f,d] =
\frac{1}{2}\|(S[m]f-d)\|^2.  
\ee 

The first consequence to draw from the expansion \ref{eqn:gosimple}
is a very strong limitation on the possible convexity of $J_{\rm
  LS}$. To state this fact conveniently, introduce a family of source
functions based on a ``mother'' function $f_1 \in
C^{\infty}_0(\bR)$, for which
\begin{equation}
\label{eqn:mutha}
\|f_1\|_{L^2(\bR)} = 1.
\end{equation}
For $\epsilon>0$, define
\be\label{eqn:srcfam}
f_{\epsilon} =
\frac{1}{\sqrt{\epsilon}}f_1\left(\frac{t}{\epsilon}\right).
\ee

\begin{theorem}\label{thm:FWIconv}
Suppose that $k \ge 2$, ${\cal U} \subset M^k$ is open and convex and that the
simple ray geometry hypothesis (Definition \ref{def:srg}) is satisfied
at every $m \in {\cal U}$.  Suppose that $m_*, m_0 \in {\cal U}$, and
for some $\rho < 1$,
\be\label{eqn:ttgap}
T[m_*](z_s,z_r) \le \rho T[m_0](z_s,z_r).
\ee
 for every $\bx_s \in W_s, \bx_r \in W_r$,  Define $d_{\epsilon} = S[m_*]f_{\epsilon}$. Then there exists
$\epsilon_*>0$ so that for $0 <\epsilon \le \epsilon_*$, $J_{\rm LS}(\cdot,d_{\epsilon},f_{\epsilon})$
is {\em not} convex on ${\cal U}$.
\end{theorem}
\begin{rem}
\label{rem:scale}
The relation \ref{eqn:ttgap} is easy to achieve. If $m_0 \in {\cal U}$ and
$1-\rho$ is sufficiently small that $m_* = \rho m_0 \in {\cal U}$, then
\ref{eqn:ttgap} holds.
\end{rem} 
\begin{proof}
It sufficies to show that the function $j:[0,1] \rightarrow \bR$ defined by 
\[
j(\sigma) = J_{\rm LS}[\sigma m_* + (1-\sigma)m_0;d_{\epsilon},f_{\epsilon}]
\]
is not convex.

The crux of the matter is the representation \ref{eqn:gosimple}. To signify the dependence of the quantities $A, T,$ and $B$ from
equation \ref{eqn:gosimple} on the slowness field, throughout this
argument and subsequent discussion we write $A=A[m]$ and so on.

Theorem \ref{thm:smdep} that implies that given $\alpha>0$, there
exists $\sigma_0 \in [0,1)$ for which 
\[
(1+\alpha) \int_{z_r^{\rm min}}^{z_r^{\rm max}}\,dz_r\, \int_{z_s^{\rm
    min}}^{z_s^{\rm max}}\,dz_s\,|A[m_*](z_r,z_s)|^2.
\]
\[
\ge \int_{z_r^{\rm min}}^{z_r^{\rm max}}\,dz_r\, \int_{z_s^{\rm
    min}}^{z_s^{\rm max}}\,dz_s\,|A[\sigma m_* +
(1-\sigma)m_0](z_r,z_s)|^2 
\]
\[
\ge (1-\alpha) \int_{z_r^{\rm min}}^{z_r^{\rm max}}\,dz_r\, \int_{z_s^{\rm
    min}}^{z_s^{\rm max}}\,dz_s\,|A[m_*](z_r,z_s)|^2.
\]
for $\sigma \in [\sigma_0,1]$.

Define $S_0$ to be the first term in the expansion \ref{eqn:gosimple},
\[
S_0[m]f(z_r,t;z_s) = A[m](z_r,z_s)f(t-T(z_r,z_s)).
\]
Condition \ref{eqn:ttgap} implies that there exists $\epsilon_*>0$,
depending on $\alpha$, so that if $0 <\epsilon\le \epsilon_*$, then the supports of $S_0[\sigma m_* + (1-\sigma)m_0]f_{\epsilon}$ and
$S_0[m_*]f_{\epsilon}$ are disjoint if $\sigma_0 \le \sigma \le \frac{1}{2}(1+\sigma_0)$, so
\[
\|S_0[\sigma m_* + (1-\sigma)m_0]f_{\epsilon}-S_0[m_*]f_{\epsilon}\|^2 =
\|S_0[\sigma m_* + (1-\sigma)m_0]f_{\epsilon}\|^2 + \|S_0[m_*]f_{\epsilon}\|^2 
\]
\[
= \int_{z_r^{\rm min}}^{z_r^{\rm max}}\,dz_r\, \int_{z_s^{\rm
    min}}^{z_s^{\rm max}}\,dz_s\,|A[\sigma m_* + (1-\sigma)m_0](z_r,z_s)|^2 +
|A[m_*](z_r,z_s)|^2 
\]
\begin{equation}
\label{eqn:orthominus}
\ge (2-\alpha)\left(\int_{z_r^{\rm min}}^{z_r^{\rm max}}\,dz_r\, \int_{z_s^{\rm
    min}}^{z_s^{\rm max}}\,dz_s\,|A[m_*](z_r,z_s)|^2\right) 
\end{equation}
Similarly, for $\sigma \in [\sigma_0, \frac{1}{2}(1+\sigma_0)]$,
\begin{equation}
\label{eqn:orthoplus}
\|S_0[\sigma m_* + (1-\sigma)m_0]f_{\epsilon}-S_0[m_*]f_{\epsilon}\|^2 
\le (2+\alpha)\left(\int_{z_r^{\rm min}}^{z_r^{\rm max}}\,dz_r\, \int_{z_s^{\rm
    min}}^{z_s^{\rm max}}\,dz_s\,|A[m_*](z_r,z_s)|^2\right) 
\end{equation}

Since $ B[\sigma m_* + (1-\sigma)m_0]$ is uniformly bounded in
$C^{k-2}$,
\[
\sup_{t \in (0,t^{\rm max})} \left|\int_{T(z_r,z_s)}^{\infty}
  \,ds\,B[\sigma m_* + (1-\sigma)m_0](z_r,s;z_s)f_{\epsilon}(t-s)\right| = O(\sqrt{\epsilon})
\]
If $\epsilon_*>0$ is chosen possibly smaller yet, again depending on
$\alpha$, then for $0 <\epsilon\le\epsilon_*$,
\[
\|S[\sigma m_* + (1-\sigma)m_0]f_{\rm epsilon}-S_0[\sigma m_* +
(1-\sigma)m_0]f_{\rm epsilon}\|^2 
\]
\begin{equation}
\label{eqn:lot}
\le \alpha \int_{z_r^{\rm min}}^{z_r^{\rm max}}\,dz_r\,
  \int_{z_s^{\rm min}}^{z_s^{\rm max}}\,dz_s\,|A[m_*](z_r,z_s)|^2
\end{equation}
Combining \ref{eqn:orthoplus} and \ref{eqn:lot} for $\sigma=\sigma_0$, obtain
\begin{equation}
\label{eqn:ub}
\|S[\sigma_0 m_* + (1-\sigma_0)m_0]f_{\epsilon}-S[m_*]f_{\epsilon}\|^2 
\le (\sqrt{2+\alpha}+2\sqrt{\alpha})^2 \int_{z_r^{\rm min}}^{z_r^{\rm max}}\,dz_r\,
  \int_{z_s^{\rm min}}^{z_s^{\rm max}}\,dz_s\,|A[m_*](z_r,z_s)|^2 
\end{equation}
Similarly, combining \ref{eqn:orthominus} and \ref{eqn:lot} for
$\sigma=\frac{1}{2}(1+\sigma_0)$, 
\[
\|S[\frac{1}{2}(1+\sigma_0) m_* + (1-\frac{1}{2}(1+\sigma_0))m_0]f_{\epsilon}-S[m_*]f_{\epsilon}\|^2 
\]
\begin{equation}
\label{eqn:lb}
\ge (\sqrt{2-\alpha}-2\sqrt{\alpha})^2 \int_{z_r^{\rm min}}^{z_r^{\rm max}}\,dz_r\,
  \int_{z_s^{\rm min}}^{z_s^{\rm max}}\,dz_s\,|A[m_*](z_r,z_s)|^2 
\end{equation}
Since 
\[
\frac{\sqrt{2-\alpha}-2\sqrt{\alpha}}{\sqrt{2+\alpha}+2\sqrt{\alpha}}
\rightarrow 1 \mbox{ as } \alpha \rightarrow 0,
\]
inequalities \ref{eqn:ub} and \ref{eqn:lb} imply that for $\alpha$
small enough (and consequenty $\epsilon_*$ small enough,
\[
\frac{1}{2}j(\sigma_0) = \frac{1}{2}(j(\sigma_0) + j(1)) <
j\left(\frac{1}{2}(1+\sigma_0)\right),
\]
\end{proof}

\begin{rem} Since 
\[
\frac{\int d\omega \omega^2 |\hat{f}_{\epsilon}(\omega)|^2}{\int
  d\omega |\hat{f}_{\epsilon}(\omega)|^2}
= \frac{\int dt \left(\frac{df_{\epsilon}}{dt}(t)\right)^2}{\int dt
  \left(f_{\epsilon}(t)\right)^2}
\]
\[
=\epsilon^{-2}\frac{\int dt \left(\frac{df_{1}}{dt}(t)\right)^2}{\int dt
  \left(f_{1}(t)\right)^2}
\]
and the left-hand side can be interpreted as mean square frequency in
$f_{\epsilon}$, $\epsilon$ can be interpreted as relative wavelength
(relative to the root mean square wavelength in $f_1$). Thus the maxiumum
diameter of an inscribed sphere in the domain of convexity of $J_{\rm
  LS}$ is proportional to a wavelength. This fact is often expressed
in the literature by stating that to ensure success of FWI, the
initial estimate of slowness must be sufficient to predict traveltimes
within a half-wavelength. Of course, Theorem \ref{thm:FWIconv} does not show that
local optimization would not succeed, that is, that stationary points
other than the global minimizer exist, merely that the domain of
convexity has diameter on the order of a mean wavelength. 

Note that the conclusion of Theorem \ref{thm:FWIconv} holds regardless of whether
$f_1$ is oscillatory or mean-zero, so the common description of this
phenomenon as ``cycle-skipping'' is somewhat misleading.
\end{rem}

\section{Source-Receiver Extension}

As illustrated in the previous section, mean-square data misfit tends
to be large, and instensitive to small
velocity changes, except in the immediate vicinity of its global
minimizer. This misfit saturation is the root cause of the poor
performance of local optimization methods for $J_{\rm LS}$..

Several cures for this pathology have been explored. The state of the
art relies on small enough error in the initial velocity estimate,
relative to the longest wavelengths reliably present in the data. The non-overlapping
signal supports of the example given in the last section are
impossible if enough low-frequency signal exists in the
data. Unfortunately frequency content is limited by acquisition
technology. Sometimes it is possible to supply a sufficiently
accurate initial model in view of spectral limitations, and sometimes
it is not, or even to know
whether one has been successfully supplied

Another class of alternatives extends the simulation map
$(m,f) \mapsto S[m]f$ to a larger domain, so chosen that the ``relaxed''
inverse problem set in this larger domain always has a small-residual
solution. Some alternative objective must then be supplied, for which
a minimizer (at least for model-consistent data) is a model of the
original type, which by construction also fits the data.
This section introduces and analyzes such a data-fitting {\em
  extension} for the idealized crosswell tomography problem.

Before we introduce the extended simulation map, we redefine $S$ so
that $S[m], m\in M$ is injective, and the linear least squares problem
for $f$ coercive.

Suppose that ${\cal U} \subset M$ consists of slowness fields
satisfying the simple ray geometry hypothesis, having a common normal
neighborhood ${\cal N}$ containing $W_s$ and $W_r$. Suppose also that
${\cal U}$ contains at least one constant.
For typical $\bx_s \in W_s, \bx_r \in W_r$, denote by $r$ and $r_0$
the ray generated by the exponential map connecting $\bx_s,\bx_r$, respectively
the straight line segment between $\bx_s$ and $\bx_r$. Note that there
may be other connecting rays, but any such rays must exit ${\cal N}$ and be associated to larger
travel times than $\tau[m](\bx_s,\bx_r)$, which is the travel time
along $r$. Also, Fermat's principle holds: the travel time along $r$
is minimum amongst travel time along all paths from $\bx_s$ to
$\bx_r$, lying entirely inside of ${\cal N}$. Write
\[
\tau[m](\bx_s,\bx_r) = \int_r \,dl\, m,
\]
with $dl$ denoting the Euclidean arc length element. it follows that
\[
\tau[m](\bx_s,\bx_r) \le \int_{r_0} \,dl, m \le m_{\rm max}|\bx_r -\bx_s|
\bx_s|.
\]
Also, 
\[
\tau[m](\bx_s,\bx_r) \ge \int_r \,dl m_{\rm min} \ge m_{\rm min}
|\bx_r-\bx_s|
\]
since $r_0$ is geodesic for the Euclidean metric. Since we have
presumed that ${\cal U}$ contains at least one constant slowness
field, both bounds can be approximated arbitrarily well.
Set
\begin{eqnarray}
\label{eqn:rdef}
r_{\rm max} &=& \sup \{|\bx_r-\bx_s|: \bx_s \in W_s, \bx_r \in
                W_r\},\nonumber\\
r_{\rm min} &=& \inf \{|\bx_r-\bx_s|: \bx_s \in W_s, \bx_r \in W_r\}.
\end{eqnarray}
Putting these simple results together, it follows that
\begin{eqnarray}
\label{eqn:tauglob}
m_{\rm max}r_{\rm max} & = & \sup \{\tau[m](\bx_s,\bx_r): \bx_s \in W_s, \bx_r \in 
  W_r, m \in {\cal U}\} \nonumber\\
m_{\rm min}r_{\rm min} |& = & \inf \{\tau[m](\bx_s,\bx_r): \bx_s \in W_s, \bx_r \in 
  W_r, m \in {\cal U}\}
\end{eqnarray}
As these quantities will recur throughout the arguments to come,
define
\begin{eqnarray}
\label{eqn:tauextr}
\tau_{\rm max}& = & m_{\rm max}r_{\rm max} \nonumber\\
\tau_{\rm min}& = & m_{\rm min}r_{\rm min}
\end{eqnarray}

We now add the only hypothesis on ${\cal U}$ pertaining to the
behaviour of slowness fields outside of the common normal
neighborhood:

The maximum measurement time $t^{\rm max}$ satisfies the {\em
  no-return hypothesis}:
\begin{itemize}
\item
\begin{equation}
\label{eqn:tmaxbd}
t^{\rm max} >  \tau_{\rm max}-\tau_{\rm min}
\end{equation}
\item for any $m \in {\cal U}$ no ray with initial $X=\bx_s \in W_s$ passes over $\bx_r \in W_r$
at any time in the interval $(\tau[m](\bx_s,\bx_r), t^{\rm max}+\tau_{\rm max})$.
\end{itemize}
Note that the second condition means that $t^{\rm max}$ as specified 
here will may be used in the statement of Theorems \ref{thm:gosimple}
and \ref{thm:smdep}. In fact, the larger limit $t^{\rm max} +\tau_{\rm max})$ may be so used. The extra margin will be needed to
make some later estimates uniform over large sets of
slowness fields.

Since rays passing over $W_r$ a second time necessarily leave ${\cal
  N}$, this hypothesis actually concerns the global behaviour of the
slowness field. It is satisfied for any constant slowness in ${\cal
  U}$, and is clearly open. Therefore beginning with an open set of
slownesses in $M^k$, $k \ge 2$, satisfying the simple ray geometry
hypothesis and containing a constant slowness, there must be a subset
satisfying also the no-return hypothesis.

One might naturally ask how a slowness field may be
extended outside the image domain ${\cal N}$ of the exponential map to
satisfy the no-return hypothesis with the largest possible $t^{\rm
  max}$, however we shall not address this question here. 

Choose $\Delta > 0$ and define
\begin{equation}
\label{eqn:fwddomdef}
D_{\Delta} = \{f \in L^2(\bR): {\rm supp}\,f \subset (-\tau_{\rm min} +
\Delta, t^{\rm max} -\Delta - \tau_{\rm max})\}
\end{equation}

\begin{theorem}
\label{thm:fwdinj}
Suppose that ${\cal U} \subset M$, $t^{\rm max}$ satisfy the simple 
ray geometry and no-return hypotheses, $\Delta >0$, and $w=1$ on 
$[\Delta, t^{\rm max}-\Delta]$. Then for $m \in {\cal U}$,
$S[m]:D_{\Delta} \rightarrow L^2(\Omega)$ is coercive, and 
\[
f \in D_{\Delta} \,\Rightarrow \, S[m]f (z_r,t;z_s) =
A(z_r,z_s)f(t-T(z_r,z_s)) 
\]
\begin{equation}
\label{eqn:gosimplemod} 
+ w(z_r,t;z_s)\int_{T(z_r,z_s)}^{\infty}
\,ds\,B(z_r,s;z_s)f(t-s). 
\end{equation}
\end{theorem}
\begin{proof}
Inequalities \ref{eqn:tauglob} and \ref{eqn:tmaxbd} imply that if
$t \in (-\tau_{\rm min} +
\Delta, t^{\rm max} -\Delta - \tau_{\rm max})$,  then $t+\tau[m](\bx_s,\bx_r) \in (\Delta ,t^{\rm
  max}-\Delta)$. Thus if $f \in D_{\Delta}$, then $w=1$ on ${\rm
  supp}\,f(\cdot-\tau[m](\bx_s,\bx_r))$, establishing
\ref{eqn:gosimplemod}. The injectivity of $S[m]$ follows from Picard
iteration, applied to \ref{eqn:gosimplemod}.
\end{proof}

\begin{definition} (Source-receiver extension)
Suppose that ${\cal U} \subset M$, $t^{\rm max}$ satisfy the simple
ray geometry and no-return hypotheses. Define $\bar{w}_0$ to be the
characteristic function of $\{t<t^{\rm max}\}$ in $\Omega$.  

Define the
{\em source-receiver} extended source domain
\[
\oD_0 = \{\of \in L^2(\Omega): {\rm supp}\,f \subset 
Z_r\times Z_s 
\]
\begin{equation}
\label{eqn:extdomdef}
\times (-\tau_{\rm max}, t^{\rm max} - \tau_{\rm  min})\}
\end{equation}
and extended modeling operator 
\be\label{eqn:extdom}
\oS[m]: \oD_{0} \rightarrow L^2(\Omega) 
\ee
by
\be\label{eqn:extmap}
\oS[m]\of (z_r,t;z_s) = \bar{w}_0(z_r,t;z_s)\bar{p}(\bx_r,t;\bx_s).
\ee
\end{definition}
in which $\bar{p}$ is the solution of the wave equation
\ref{eqn:pressure} with $f(t) = \of(z_r,t;z_s)$.  That is, $\oS$ acts
by computing each data trace $\bar{p}(\bx_r,t;\bx_s)$ as if it were
generated by an indepedent source function $\of(z_r,\cdot;z_s)$ for
each source and receiver location.

Note that the time interval in the definition of $\oD_0$ strictly
contains that in the definition of $D_{\Delta}$.
Define the extension map $E: D_{\Delta} \rightarrow \oD_0$ by 
$(Ef)(z_t,t;z_s) = f(t)$, with $f$ on the RHS being regarded as
a constant function of $z_sz_r$. Then $\oS$ is an
extension of $S$, in the sense that
\be\label{eqn:extprop}
\oS[m](Ef) = S[m]f,\,f \in D_{\Delta}.
\ee

At first blush, it would appear that computing $\oS$ would be a very
expensive proposition, even after discretization in $z_s$ and $z_r$,
as each trace would have to be simulated independently. However this
is not the case, as \be\label{eqn:extgreen} \oS[m]\of(z_r,t;z_s) =
\bar{w}_0(z_r,t;z_s(G[m](\bx_r,\cdot;\bx_s) *_t \of(,z_r,\cdot;z_s))(t), \ee
with $G[m]$ being the Green's function of the problem
(\ref{eqn:pressure}), defined above. $G[m]$ can be approximated
numerically, one numerical simulation for each value of $z_s$. Once
this is done, the evaluation of $\oS[m]\of$ is a relatively
inexpensive convolution for each trace.

The main advantage of the extended model is that any data can be fit -
for any slowness model (within the class specified in the last
section)! More precisely, define 
\begin{equation}
\label{eqn:extrange}
\oF_0 = \{d \in L^2(\Omega): d=0 \mbox{ if } t<0 \mbox{ or } t>t^{\rm 
  max} \}. 
\end{equation}
\begin{theorem}
\label{thm:extsur}
Suppose that ${\cal U} \subset M$, $t^{\rm max}$ satisfy the simple 
ray geometry and no-return hypotheses
Then for $m \in {\cal U}$,
\begin{equation}
\label{eqn:extsur}
\oF_0 \subset \oS[m]\oD_0.
\end{equation}
Define $\oS_0[m]: L^2(\Omega) \rightarrow L^2(\Omega)$ and its inverse by
\begin{eqnarray}
\label{eqn:approxrinv}
\oS_0[m]\of(z_r,t;z_s) & = &
A[m](z_r,z_s)\of(z_r,t-T[m](z_r,z_s);z_s),\nonumber\\
\oS_0^{-1}[m]d(z_r,t;z_s) &=&
A[m]^{-1}(z_r,z_s)d(z_r,t+T[m](z_r,z_s);z_s),
\end{eqnarray}
and $\bar{R}_0[m]: L^2_{\rm comp}(\Omega) \rightarrow L^2_{\rm comp}$
by
\[
\bar{R}_0[m]d(z_r,t;z_s) 
\]
\begin{equation}
\label{eqn:approxres}
= \frac{\bar{w}_0(z_r,t+T[m](z_r,z_s);z_s)}{A(z_r,z_s)}
\int_0^t\,ds\,B[m](z_r,t+T[m](z_r,z_s)-s;z_s)d(z_r,s;z_s)
\end{equation}
Then 
\begin{equation}
\label{eqn:rinvdef}
\oS^R[m] = \oS_0^{-1}
\sum_{k=0}^{\infty}(-\bar{R}_0[m])^k \bar{w}_0
\end{equation}
converges in norm to a bounded operator $\oS^R[m]: \oF_0
\rightarrow \oD_0$, and $\oS^R[m]$ is a right inverse
for $\oS[m]$ on $\oF_0$, that is, for $d \in \oF_0$,
\begin{equation}
\label{eqn:invreln}
\oS[m]\oS[m]^Rd = d.
\end{equation}
\end{theorem}

\begin{proof} 
Observe that  the definition \ref{eqn:extmap} can be re-written using
the Green's function components $A, T,$ and $B$ and the cutoff function
$\bar{w}_0$ in the form
\[
\oS[m]\of(z_r,t;z_s) = \bar{w}_0(t) \left(
A[m](z_r,z_s)\of(z_r.t-T[m](z_r,z_s);z_s) \right.
\]
\begin{equation}
\label{eqn:extmapexp0}
\left.+ \int_{T[m](z_r,z_s)}^{\infty}\,ds\,B[m](z_r,s;z_s)\of(z_r,t-s;z_s)\right)
\end{equation}
According to the no-return hypothesis, $B[m]$ is well-defined and
smooth for $t \le t^{\rm max} + \tau_{\rm max}$. For $\of \in
\oD_0$, the upper limit of integration may be replaced by $t + \tau_{\rm max}$. Since only
$t\le t^{\rm max}$ is significant on the right-hand side of
\ref{eqn:extmapexp0}, the integrand in the last summand is
well-defined, and the integral defines an
operator $\bar{B}[m]: \oD_0  \rightarrow L^2(\Omega)$,
\begin{equation}
\label{eqn:barb}
\bar{B}[m]\of(z_r,t;z_s) = 
\int_{T[m](z_r,z_s)}^{t^{\rm max}+\tau_{\rm max}}\,ds\,B[m](z_r,s;z_s)\of(z_r,t-s;z_s)
\end{equation}
Then equation \ref{eqn:extmapexp0} can be abbreviated as
\begin{equation}
\label{eqn:extmapexpabbrev}
\oS[m] = \bar{w}_0 (\oS_0[m] + \bar{B}[m]).
\end{equation}
From the definitions of $\oS_0,\bar{R}_0$,
\[
\bar{w}_0\bar{B}[m] = \bar{R}_0[m]\oS_0[m]
\]
whence
\begin{equation}
\label{eqn:extmapexp}
\oS[m]\of = \bar{w}_0(I + \bar{R}_0[m]) \oS_0[m]\of.
\end{equation}


Note that $\bar{R}_0$ preserves supports of the form $t_0 \le t \le
t^{\rm max}$, in particular maps $\oF_0$ into itself.
Straightforward estimates using a weighted $L^2$ norm with decreasing
exponential weight in $t$ shows that as an operator on $\oF_0$, $I + \bar{R}_0[m]$ is invertible,
and its inverse is given by the Neumann series appearing in the
definition \ref{eqn:rinvdef} of $\oS^R[m]$.

For $d \in \oF_0$, then, $\bar{w}_0 d = d $, and
$\sum_{k=0}^{\infty}(-\bar{R}_0[m])^kd  \in \oF_0$. The definition of
$\oD_0$ is concocted to guarantee that if $d \in \oF_0$, then
$\oS[m]^{-1}d \in \oD_0$ for any $m \in {\cal U}$. Thus equation
\ref{eqn:rinvdef} defines a map from $\oF_0$ to $\oD_0$. While in
general $\oS_0[m]\oD_0$ is not a subset of $\oF_0$,
$\oS_0[m]\oS_0[m]^{-1} = I $ so membership in $\oF_0$ is preserved:
\[
\oS_0[m]\oS_0[m]^{-1}\sum_{k=0}^{\infty}(-\bar{R}_0[m])^kd  \in \oF_0.
\]
Tacking on the other factors in the expansion \ref{eqn:extmapexp} of
$\oS[m]$,  the conclusion \ref{eqn:invreln} follows, which in turn
implies the inclusion \ref{eqn:extsur}.
\end{proof}

\section{Extended Full Waveform Inversion}
Theorem \ref{thm:extsur} shows that data fitting with $\oS$, rather
than $S$, cannot suffer from the difficulty identified in Theorem
\ref{thm:FWIconv}, that is, failure to fit the data until the slowness
field predicts the right kinematics within a wavelength or so: in
fact, data can be fit via $\oS$ with essentiallly any slowness, however
kinematically inaccurate.. However
this pleasant property does not in itself yield a solution of the
inverse problem, because it is achieved at the price of adding a large
number of parameters to the models space. This enlargement of the
model space has two negative consequecnes:
\begin{itemize}
\item[1. ] The extended modeling map $\oS$ is not
injective - it is not possible to define the domain and range so that
this map is bijective for an open set ${\cal U}$ of slowness
models. 
\item[2. ] More fundamentally, since data
can be fit with any slowness model at all, data fitting no longer
constrains the slowness model.
\end{itemize}
Item 2 is a
consequence of the unphysical nature of the extended source, with each
data trace modeled by an independent source trace. The large number of
parameters added in extending the model must somehow be controlled, and 
forced to assume trivial values in order to recover  a slowness $m$
and physical source $f$ that together explain the data.

Compensation for the null space of $\oS[m]$ is straightforward via
Tihonov regularization. That is, choose a regularization parameter
$\alpha >0$ and define the inverted extended source $\ofa[m,d], m \in
M, d \in L^2(\Omega)$, by
\begin{equation}
\label{eqn:regextinv}
\ofa[m,d] = \mbox{ argmin}_{\of}\,\frac{1}{2}(\|\oS[m]\of - d\|^2 + \alpha^2\|\of\|^2).
\end{equation}
If ${\cal U}$ and $t^{\rm max}$ are chosen as in the preceding
discussion, that is, to satisfy the simple ray geometry and no-return
hypotheses, and $m \in {\cal U}, d \in \oF_0$, then
as $\alpha \rightarrow 0$, $\ofa[m,d] \rightarrow \oS[m]^Rd$.

A hint about how the second issue might be addressed comes from
further examination of the synthetic example defined earlier. Recall
that $m_*$ is the target slowness 
(corresponding to the velocity depicted in Figure \ref{fig:velbig})
used to generate the data $d=S[m_*]f$ (Figure \ref{fig:shot6kmbig}).
Figures \ref{fig:shot6kmbig_bigdecon} and \ref{fig:shot6kmbig_decon} display
$\ofa[m,d]$ for $d=S[m_*]f$ and $m=m_*$ and $m=m_0 = 0.5 s/km$
respectively. In both cases, the regularization parameter $\alpha$ is
set = $10^{-3}$. The non-extended or
physical source $f$ is localized near $t=0$. The inverted extended
source for correct slowness (Figure \ref{fig:shot6kmbig_bigdecon})
yields a result very close to the extension $Ef$, and has the same
quality (energy focused at $t=0$, independent of $z_r$). Inversion with the incorrect slowness
shifts the extended source into $t \ne 0$. Also, the level surfaces of
$\ofa[m_0,d]$ (Figure \ref{fig:shot6kmbig_decon}) are not
constant in $z_r$, but vary significantly. If this were not true, then
any $z_r=const.$ trace could be taken as the source function $f$, and
the inverse problem would be solved. 

These observations suggest two alternative methods for measuring the
correctness of $m$ from the properties of $\ofa[m,d]$. First,
concentration near $t=0$ may be measurred by multplying $\ofa[m,d]$ by $t$ and taking the $L^2$ norm of the result. If the velocity
is correct, {\em and} the target source function actually is supported
near $t=0$, this norm will be small, as the support of $\ofa[m,d]$
will be close to $t=0$ also (see Figure \ref{fig:shot6kmbig_bigdecon}). Second, the variance with respect to $z_r$ may
be measured by taking the norm of the $z_r$ derivative, which will be
zero (not just small) for an extended model in the range of the
extension operator. That is, the range of the extension operator (the
``physical'' extended models is the null space of the $z_r$ derivative.
Either case may be captured by small values of the objective function
\be\label{eqn:dsdef}
J_{\rm DS}[m,d] = \frac{1}{2}\|A\ofa[m,d]\|_{L^2(\Omega)}^2,
\ee
in which
\begin{itemize}
\item $A\of(z_s,z_r,t) = t\of(z_s,z_r,t)$, or
\item $A\of(z_s,z_r,t) = \nabla_{z_s,z_r}\of(z_s,z_r,t)$.
\end{itemize}
Operators $A$ which indirectly detect the kinematic correctness of a
wave propagation model via qualities of auxiliary data generated with
that model (in this case, the extended source estimate $\ofa[m,d]$)
have come to be known as {\em annihilators}, since in many cases the
auxiliary data corresponding to correct models is in (or near) the
null space. Model estimation via minimizations of functions such as
$J_{\rm DS}$ is sometimes called {\em differential semblance}
optimization, after a name introduced in some of the early papers on
this concept \cite[]{SymesCar:91,KerSy:94} to reflect the use of
differential operators as annihilators. A differential annihilator
(the second option above) was used in the first papers known to us on the
source-receiver extension \cite[]{SongSymes:94b,Symes:94c}. A brief
survey of the literature on this topic appeared above, in the
Introduction.

Both choices have drawbacks. The first choice (multiplication by $t$)
defines a bounded operator on $D_0$, however without a null space;
without some further compactness-inducing constraint on $\of$, a
minimizer of $J_{\rm DS}$ may not exist. For computational purposes,
this defect appears to be minor, as discretization effectively forces
the search for $\ofa$ into a compact subspace of
$L^2(\Omega)$. However this effect has not been analyzed; the applied
literature uses it without mathematical justification (but with
considerable success - see for example \cite{Warner:16}). The second
choice has a complementary problem: the derivative with respect to
$z_r,z_s$ is of course unbounded, rendering $J_{\rm DS} $ only densely
defined. The obvious remedy is to use a subspace of $H^1(\Omega)$ as
the domain of $\oS$ - see \cite{KerSy:94} for more discussion of this
option, and implementation details. 

We will accept the drawbacks of the first option above (multiplication
by $t$) in the remainder of our discussion, simply because the
implementation is simpler and is the approach taken in the
computational examples accompanying this paper.

\section{Gradient and Hessian}

Throughout this section we will assume without further comment that
${\cal U}$ and $t^{\rm max}$ are chosen to satisfy the simple ray
geometry and no-return hypotheses, as explained earlier.

Recall from Theorem \ref{thm:smdep}, and surrounding discussion,
especially Remark \ref{rem:derivloss}, that $(m,f) \mapsto S[m]f$ is
only differentiable in $m$ if $f$ has one square integrable derivative
or better. The same is true fore $\oS$. The local
behavioiur of $J_{\rm DS}$ follows from unpacking the directional derivative of
$\oS$ at $m \in {\cal U}$ in a direction $\delta m \in
C^{\infty}(\bR^3)$ into a leading term with explicit dependence on $\partial
\of/ \partial t$, plus an $L^2$-continuous remainder.
The leading term factors as $\oS[m]Q[m]\delta m$, in which $Q$ is a
first order differential operator:
\begin{equation}
\label{eqn:qdef}
Q[m]\delta m =\frac{1}{A[m](z_r,z_s)}\left(-(DT[m]\delta
m) (z_r,z_s)\frac{\partial}{\partial t} + (D A[m]\delta m)(z_r,z_s)\right)
\end{equation}
A suitable domain for $Q[m]\delta m$ is
\begin{equation}
\label{eqn:d1def}
\oD_1 = H^1_0(Z_r \times Z_s \times [-\tau_{\rm max},t^{\rm
  max}-\tau_{\rm min}])
\end{equation}
Note that $Q[m]\delta m$ is {\em essentially skew-symmetric}, that is
\begin{equation}
\label{eqn:qess}
(Q[m]\delta m) + (Q[m]\delta m)^T =2 DA[m]\delta m.
\end{equation}
The right-hand side of equation \ref{eqn:qess} is a differential
operator of order zero, in particular bounded on $L^2(\Omega)$,
whereas $Q$ is first order.
\begin{theorem}
\label{thm:q}
There exists continuous map $P: {\cal U} \times C^{\infty}(\bR^3)
\rightarrow {\cal L}(\oD_0,\oD_0)$, linear in its second argument, 
so that for $\of \in \oD_1$,
\begin{equation}
\label{eqn:paramderiv}
D(\oS[m]\of)\delta m = \oS[m]((Q[m]\delta m) + (P[m]\delta
m))\of
\end{equation}
$P$ is a Volterra convolution operator of the first kind. In
particular, the operators $P$ and $Q$ commute.
\end{theorem}
\begin{proof}
Suppressing $z_r,z_s$ for the moment, recall the definition of $\oS_0[m]$
\[
(\oS_0[m]\of)(t) = A[m]\of(t-T[m]).
\]
Also, from the defintion \ref{eqn:barb}, 
\[
(\bar{B}[m]\of)(t) = 
\int_{T[m]}^{\infty}\,ds\,B[m](s)\of(t-s) 
\]
\[
=\int_{-\infty}^{t-T[m]}\,ds\,B[m](s)(t-s)\of(s)
\]
\begin{equation}
\label{eqn:barbalt}
= (\oS_0[m]\tilde{B}[m])\of(t),
\end{equation}
where
\begin{equation}
\label{eqn:tildeb}
(\tilde{B}[m])\of(t) = \int_{-\infty}^{t}\,ds\,B[m](t+T[m]-s)\of(s) 
\end{equation}
In order to view $\tilde{B}[m]$ as a Volterra integral operator on $D_0$ with
smooth kernel, it is necessary to extend $B[m]$ smoothly to $t >
t^{\rm max} + \tau_{\rm max}$. The extended values do not play
any role in the result of the computation \ref{eqn:barbalt} for $t >
t^{\rm max}$. 

These manipulations combine to yield a re-write of the expression
\ref{eqn:extmapexpabbrev} for $\oS[m]$:
\begin{equation}
\label{eqn:extmapexpabbrevalt}
\oS[m] = \bar{w}_0 \oS_0[m](I + \tilde{B}[m]).
\end{equation}
Its directional derivative at $m \in {\cal U}$ in the direction
$\delta m \in C^{\infty}(\bR^3)$ is
\[
D(\oS[m]\of)\delta m = \bar{w}_0( (D(\oS_0[m])\delta m) (I +
\tilde{B}[m])\of+\oS_0[m]D(\tilde{B}[m]\of)\delta m)
\]
Note that 
\[
(D(\oS_0[m]\of)\delta m)(t) = (DA[m]\delta m)\of(t-T[m]) -A[m](DT[m]\delta
m)\frac{\partial \of}{\partial t}(t-T[m])
\]
\[
=\oS_0[m]\left(-\frac{(DT[m]\delta m)}{A[m]}
  \frac{\partial \of}{\partial t} + (D\log A[m]\delta m)\of\right)(t)
\]
so 
\[
\bar{w}_0(D(\oS_0[m]\of)\delta m) =  \bar{w}_0\oS_0[m]Q[m]\delta m
\]
whence
\[
D(\oS[m]\of)\delta m = \bar{w}_0\oS_0[m] (Q[m]\delta m)(I+\tilde{B}[m])\of 
\]
\[
+ \bar{w}_0\oS_0[m]D(\tilde{B}[m]\of)\delta m 
\]
Since $\tilde{B}$ is a convolution operator in $t$, it commutes with
multiplication by functions of $z_s,z_r$, and with $\partial/\partial
t$, so this is actually equation \ref{eqn:paramderiv}, provided that
$P$ is defined as 
\begin{equation}
\label{eqn:pdef}
P[m]\delta m = (I + \tilde{B}[m])^{-1}D\tilde{B}[m] \delta m
\end{equation}
Since both factors on the right-hand side of equation \ref{eqn:pdef}
are Volterra convolution operators (of second and first types
respectively), so their product is a Volterra convolution operator of
the first kind, and $P$ commutes with $Q$ as claimed.
\end{proof}

In the remainder of this section we will use the equivalence between $D(\oS[m]\of)\delta
m$, the directional derivative of the vector-valued function $m
\mapsto \oS[m]\of$,  and $(D\oS[m]\delta m)\of$, the directional
derivative of the operator-valued function $m \mapsto \oS[m]$,
evaluated at the vector $\of$, as is valid if $\of \in \oD_1$.

Just as existence of a derivative with respect to $m$ of $\oS[m]\of$
requires that $\of$ have a square-integrable time derivative, so
existence of a derivative with respecto $m$ of $\oS[m]^Td$ requires
that $d$ have a time derivative. Accordingly, define
\[
\oF_1 = \oF_0 \cap H^1(\Omega)
\]
and metrize $\oF_1$ with the Sobolev 1-norm.

\begin{theorem}
\label{thm:qcomm}
For $d \in \oF_1$, $m \in {\cal U}$,
\[
DJ_{\rm DS}[m,d]\delta m =\alpha^2 \langle ((P[m]\delta m)^T +(P[m]\delta
m))\ofa[m,d],\oga[m,d]\rangle 
\]
\begin{equation}
\label{eqn:qderiv}
-\langle (P[m]\delta
m)\ofa[m,d],A^TA\ofa[m,d]\rangle + \langle [Q[m]\delta m,A^TA]\ofa[m,d],\ofa[m,d]\rangle.
\end{equation}
For $m \in {\cal U}$, $\delta m \in C^{\infty}(\bR^3)$, the linear
functional on $\oF_1$ defined by $d \mapsto DJ_{\rm DS}[m,d]\delta m$
extends by continuity to $\oF_0$.
\end{theorem}
\begin{proof}
The normal equation equivalent to \ref{eqn:regextinv} are:
\begin{eqnarray}
\label{eqn:normal}
N_{\alpha}[m]\of &=& \oS[m]^Td\nonumber\\
N_{\alpha}[m] & = & \oS[m]^T\oS[m] + \alpha^2 I
\end{eqnarray}
So
\begin{equation}
\label{eqn:ofanorm}
\ofa[m,d] = N_{\alpha}[m]^{-1}\oS[m]^Td
\end{equation}
and 
\[
J_{\rm DS}[m,d] = \frac{1}{2}\|A N_{\alpha}[m]^{-1}\oS[m]^Td\|^2
\]
For $d \in \oF_1$,
\[
DJ_{\rm DS}[m,d] \delta m = \langle A (DN_{\alpha}[m]^{-1}\delta m)
\oS[m]^Td + AN_{\alpha}[m]^{-1}D(\oS[m]^Td)\delta m,A\ofa[m,d]\rangle
\]
\[
=\langle - N_{\alpha}[m]^{-1}(DN_{\alpha}[m]\delta m)\ofa[m,d]
  +N_{\alpha}[m]^{-1}D(\oS[m]^Td)\delta m,A^TA\ofa[m,d]\rangle
\]
Introduce $\oga[m,d]$, defined by
\begin{equation}
\label{eqn:gdef}
\oga[m,d] = N_{\alpha}[m]^{-1}A^TA\ofa[m,d].
\end{equation}
Then
\[
DJ_{\rm DS}[m,d] \delta m = \langle - (DN_{\alpha}[m]\delta m)\ofa[m,d] +
  D(S[m]^Td)\delta m, \oga[m,d]\rangle
\]
\begin{equation}
\label{eqn:pregrad1}
= \langle (DS[m]^T\delta m)(d-S[m]\ofa[m,d]) - S[m]^T (DS[m]\delta m)\ofa[m,d],\oga[m,d]\rangle
\end{equation}
From Theorem \ref{thm:q}, this is
\[
=\langle (Q[m]\delta m + P[m]\delta m)^TS[m]^T(d-S[m]
\ofa[m,d]) - S[m]^TS[m]
(Q[m]\delta m + P[m]\delta m)\ofa[m,d],\oga[m,d]\rangle
\]
which because of the normal equation \ref{eqn:normal} is
\[
= \langle (Q[m]\delta m + P[m]\delta m)^T\alpha^2\ofa[m,d],\oga[m,d]\rangle 
-\langle (Q[m]\delta m + P[m]\delta m)\ofa[m,d], (N_{\alpha} -
\alpha^2I)\oga[m,d]\rangle
\]
\[
= \alpha^2 \langle (Q[m]\delta m + P[m]\delta
m)^T\ofa[m,d],\oga[m,d]\rangle 
+ \alpha^2 \langle (Q[m]\delta m + P[m]\delta
m)\ofa[m,d],\oga[m,d]\rangle
\]
\[
-\langle (Q[m]\delta m + P[m]\delta
m)\ofa[m,d],A^TA\ofa[m,d]\rangle
\]
\[
=\alpha^2 \langle ((P[m]\delta m)^T +(P[m]\delta
m))\ofa[m,d],\oga[m,d]\rangle 
\]
\[
-\langle (Q[m]\delta m + P[m]\delta
m)\ofa[m,d],A^TA\ofa[m,d]\rangle
\]
by virtue of the defintion \ref{eqn:gdef}. Finally, the essential skew-symmetry of
$Q[m]\delta m$ (equation \ref{eqn:qess}) implies that
\[
-\langle (Q[m]\delta m)\ofa[m,d],A^TA\ofa[m,d]\rangle
\]
\[
=-\langle [A^TA,Q[m]\delta m]\ofa[m,d],\ofa[m,d]\rangle- \langle A^TA
\ofa[m,d],(Q[m]\delta m)^T\ofa[m,d]\rangle
\]
\[
=-\langle [A^TA,Q[m]\delta m]\ofa[m,d],\ofa[m,d]\rangle +\langle
(Q[m]\delta m - 2DA[m]\delta m)\ofa[m,d],A^TA\ofa[m,d]\rangle
\]
whence
\begin{equation}
\label{eqn:qcomm}
-\langle (Q[m]\delta m)\ofa[m,d],A^TA\ofa[m,d]\rangle =
\langle\left(\frac{1}{2} [Q[m]\delta m,A^TA] + A^TA DA[m]\delta m\right)\ofa[m,d],\ofa[m,d]\rangle,
\end{equation}
and equation \ref{eqn:qderiv} is established.

Recall that $Q[m]\delta m$ is a $t$-independent multiple of
$\partial/\partial t$, and $A = $ multiplication by $t$. Therefore
\begin{equation}
\label{eqn:qcommdt}
[Q[m]\delta m,A^TA] = 2t\frac{DT[m]\delta m}{A[m]}
\end{equation}
defines a bounded multiplier on $\oD_0$. Since all of the other operators
appearing in equation \ref{eqn:qderiv} are bounded on $\oD_0$, and
$\ofa[m,d]$ and $\oga[m,d]$ are continuous in $d \in \oF_0$, 
continuous extension of $DJ_{\rm DS}[m,d]\delta m$ to $d \in \oF_0$ follows.
\end{proof}

To understand how the term involving the commutator $[Q,A^TA]$ may
dominate the derivative of $J_{\rm DS}$, we introduce a measure of
oscillation in time:

\begin{definition} For $u \in L^2(\Omega)$, $a < b$,
\begin{equation}
\label{eqn:lamdef}
\lambda_{[a,b]}(u) =
\left(\frac{\int_{Z_r}\,dz_r\,\int_{Z_s}\,dz_s\,\int_a^b\,dt,\,\left(\int_a^t\,ds\,u(z_r,s;z_s)\right)^2}{\int_{Z_r}\,dz_r\,\int_{Z_s}\,dz_s\,\int_a^b\,dt\,u^2}\right)^{\frac{1}{2}}.
\end{equation}
\end{definition}
Essentially, $\lambda$ is the ratio of the partial Sobolev $W^{-1,2}$ norm to
the $L^2$ norm, and measures oscillation in $t$. If $u(t) = \sin \pi \omega
t$, then $\lambda_{[a,b]}(u) = \frac{C}{ \omega} + O(\omega^{-2})$, so
$\lambda$ is more or less wavelength, hence the notation. Its
importance for the analysis of the gradient lies in a simple lemma,
the proof of which we leave to the reader:

\begin{lem}
\label{lem:lamlem}
Suppose that $k \in C^1(\Omega)$, $a < b$, and define for $u \in L^2(\Omega)$
\[
Ku(z_r,z_s,t) = \int_a^t\,ds\,k(z_r,t-s;z_s) u(z_r,s;z_s).
\]
Then 
\[
\|Ku\|_{L^2(Z_r\times Z_s \times [a,b])} \le
(2+(b-a)^2)^{\frac{1}{2}}\|k\|_{C^1(\Omega)}\lambda_{[a,b]}(u)\|u\|.
\]
Also, if ${\rm supp} \,u \subset [a,b]$ and $\int_a^bu = 0$, then $a' \le a < b \le b'$
implies that
\[
\lambda_{[a',b']}(u) = \lambda_{[a,b]}(u)
\]
and
\[
\|K^Tu\|_{L^2(Z_r\times Z_s \times [a,b])} \le
(2+(b-a)^2)^{\frac{1}{2}}\|k\|_{C^1(\Omega)}\lambda_{[a,b]}(u)\|u\|.
\]
\end{lem}

\begin{theorem}
\label{thm:qerr}
Under the hypotheses of Theorem \ref{thm:qcomm}, for $d \in \oF_0$ for
which
\[
\int_0^{t^{\rm max}}\,d(z_r,t;z_s) = 0, z_r \in Z_r, z_s \in Z_s,
\]
there is $C >0$
depending on ${\cal U}$ and $t^{\rm max}$ so that
\[
|DJ_{\rm DS}[m,d]\delta m - \langle [Q[m]\delta
m,A^TA]\ofa[m,d],\ofa[m,d]\rangle| 
\]
\[
\le C \alpha^{-2}\lambda_{[0,t^{\rm max}]}(d)(1+\alpha^{-2}\lambda_{[0,t^{\rm max}]}(d))\|d\|^2
\]
\end{theorem}
\begin{proof}
Every operator appearing in the following expressions depends on $m
\in {\cal U}$, and $\ofa$ and $\oga$ depend on $d$ as well. Since these
dependencies are universal, we suppress them from the notation.

Using the expansion \ref{eqn:extmapexpabbrevalt}, rewrite the normal
equation \ref{eqn:normal} as
\[
(\alpha^2 I + (I+\tilde{B})^TS_0^t \bar{w}_0^2 S_0 (I+\tilde{B})) \ofa
= (I+\tilde{B})^TS_0^T \bar{w}_0d
\]
\[
=S_0^T d +\tilde{B}^TS_0^T d
\] 
from the definition of $\oF_0$. The left-hand side is
\[
= (\alpha^2 + S_0^T\bar{w}_0^2S_0) \ofa  + K \ofa,
\]
where
\[
K = \tilde{B}^T S_0^T\bar{w}_0^2S_0 + S_0^T\bar{w}_0^2S_0\tilde{B}
+ \tilde{B}^T S_0^T\bar{w}_0^2S_0\tilde{B}.
\]
Decompose $\ofa = \ofa^0 + \ofa^1$, where
\[
\ofa^0 = (\alpha^2 + A^2)^{-1}S_0^T d.
\]
Tracking supports, you see that
\[
S_0^T\bar{w}_0^2S_0 \ofa^0 = \S_0^TS_0\ofa^0 = A^2 \ofa^0,
\]
so 
\[
(\alpha^2 + S_0^T\bar{w}_0^2S_0) \ofa^0 =S_0^T d.
\]
Therefore, 
\[
N_{\alpha}\ofa^1 = \tilde{B}^TS_0^T \bar{w}_0d - K\ofa^0
\]
Since $\ofa^0$ is a multiple of a translate of $d$, from Lemma \ref{lem:lamlem} 
\[
\lambda_{[-\tau_{\rm max},t^{\rm max}-\tau_{\rm min}]}(\ofa^0) =
  \lambda_{[0,t^{\rm max}]}(d).
\]
Every summand in $K$ has a factor of $\tilde{B}$ or $\tilde{B}^T$ (or
both). Lemma \ref{lem:lamlem} implies that
\[
\|N_{\alpha}\ofa^1\| \le C \lambda_{[0,t^{\rm max}]}\|d\|
\] 
for suitable $C$ depending on $A, B$ hence on ${\cal U}$, $t^{\rm
  max}$. Since $N_{\alpha}-\alpha^2I \ge 0$ and $\|\ofa^0\|\le C\|d\|$,
obtain
\begin{eqnarray}
\label{eqn:f1barbd}
\|\ofa^1\| & \le & C \alpha^{-2}\lambda_{[0,t^{\rm max}]}(d))\|d\|\\
\label{eqn:fbarbd}
\|\ofa\| & \le & C(1 + \alpha^{-2}\lambda_{[0,t^{\rm max}]}(d))\|d\|
\end{eqnarray}
hence
\[
\alpha^2\|\oga\| \le C(1 + \alpha^{-2}\lambda_{[0,t^{\rm max}]})\|d\|.
\]
Finally, estimate $P \ofa = P \ofa^0  + P \ofa^1$ by applying Lemma
\ref{lem:lamlem} again to the first summand on the right, and using
the already-established bound for $\|\ofa^1\|$ to bound the second
term. Similar reasoning applied to $P^T \ofa$ and appeal to Theorem
\ref{thm:qcomm} establishes the result.
\end{proof}



\section{Transmission Caustics}

\section{Discussion}

\section{Conclusion}

\newpage

\inputdir{project}
%\plot{csqlens10}{width=\textwidth}{Lens: diameter = 0.5 km, thickness = 1.0 km, background velocity = 2 km/s, velocity at center = 1.4142  km/s.}
%\plot{shot6kmlens10conv}{width=\textwidth}{Shot gather - source at x=5100 m, z=10 m. Receivers between 4 km and 6km at depth = 3.0 km.}
%\plot{awicurve}{width=\textwidth}{RMS amplitudes of inversions for
%trace-dependent source (matched source), scaled by t (second moment
%objective). Convex combinations of homogeneous (2 km/s) and lens models.}
%\plot{./velbig_dec}{width=\textwidth}{Example: $v$ = 2 km/s for $z \le
%  0.5 km$, $=2.0 + (z-0.5)*0.5$ km/s for $z>0.5$ km. Source location
%  $x=5100$, $z=10$ m and receiver locations $x=3000 - 6000$ m,
%$z=3000$ m indicated.}
%\plot{shot6kmhom}{width=\textwidth}{Shot gather for geometry indicated
%  in Figure \ref{fig:velbig_dec_exp}, generated with homogeneous velocity:
%  $v_0=2$ km/s}
%\plot{shot6kmbig}{width=\textwidth}{Shot gather for geometry and
%  $v$ as in Figure \ref{fig:velbig_dec_exp}.}
%\plot{shot6kmbig_decon}{width=\textwidth}{Inverse image of data in 
%  Figure (\ref{shot6kmbig}) under extended modeling operator at $v_0$
%  = 2 km/s}
%\plot{shot6kmbig_bigdecon}{width=\textwidth}{Inverse image of data in 
%  Figure \ref{shot6kmbig} under extended modeling operator at target
%  $v$ (Figure \ref{fig:velbig_dec_exp}).}
%\plot{fwigrad6kmbighom}{width=\textwidth}{FWI gradient for shot gather
%defined in Figure \ref{fig:velbig_dec_exp}, at $v=v_0$ = 2 km/s. Scaled by
%10$^6$ for plotting. Note very
%small values, and sign is positive, indicating that FWI objective {\em
%increases} in the direction of the target velocity.}
%\plot{fwiscan}{width=\textwidth}{Plot of values of the FWI objective
%  along the line segment between $v_0$ and $v$, for the shot gather
%  described in Figure \ref{fig:velbig_dec_exp}. For abscissa $h$, value
%  plotted is FWI objective at $h v_0 + (1-h) v$. Note that objective
%  does in fact initially increase from its value at $v_0$, consistent
%  with the positive average value of the gradient as shown in Figure \ref{fig:fwigrad6km%bighom}. }
%\plot{ivagrad6kmbighom}{width=\textwidth}{Source-receiver Extended FWI
%  gradient, same geometry, data, and velocity model as in Figure
%  \ref{fig:fwigrad6kmbighom}. Note that overall sign is negative, indicating
% correct conclusion that $v_0$ is slow.}
%\plot{fwistack}{width=\textwidth}{FWI gradient at $v=v_0$ for line of shots 
%  extending from 2100 m to 8000 m. Scaled by
%10$^6$ for plotting. Receiver locations are windowed so 
%  that the range of offsets is the same for all shots. In effect, the 
%  gradient is a stack of translates of the gradient in Figure
%  \ref{fig:fwigrad6kmbighom}.}
%\plot{ivastack}{width=\textwidth}{Source-receiver Extended FWI gradient at $v=v_0$ for line of shots 
%  extending from 2100 m to 8000 m. Receiver locations are windowed so 
%  that the range of offsets is the same for all shots. In effect, the 
%  gradient is a stack of translates of the gradient in Figure
%  \ref{fig:ivagrad6kmbighom}.}

\bibliographystyle{seg}
\bibliography{../../bib/masterref}


% \appendix{Abstract Structure of the Gradient}
\section*{APPENDIX: ABSTRACT STRUCTURE OF THE GRADIENT}

The inverse problem studied here is far too simple to have any direct
use in applications. Its simplicity allows a rather complete account of
its properties, as the reader has seen in the preceding pages. However
this discussion has touched on features found in more complex and
prototypical problems:
\begin{itemize}
\item Nested optimization (variable projection method,
  \cite{GolubPereyra:03}) based on a model decomposition into inner
  and outer variables seems to be very important: in some cases, such
  as the simple problem presented here, the decomposition is obvious
  (inner variable = wavelet, outer variable = slowness), in others
  less so \cite[]{geoprosp:2008,Terentyev:thesis}.
\item The extended modeling operator must be surjective, or at least
  have a dense range, {\em for each value of the outer variable}, so
  that data may be fit well even for a poor initial guess of the outer
  variable. for the extended inversion approach to be successful.
\item Because of this data-fitting assumption, a straightforward
  algorithm for scaling the quadratic penalty, based on the
  Discrepancy Principle, is available - see
  \cite{FuSymes2017discrepancy,SymesChenMinkoff:21}. This
  scaling varies dynamically during iterative optimization, in effect
  changing the objective function sporadically as the iteration
  converges.
\item The derivative of the extended modeling operator with respect to
  the outer variable is well-approximated by the composition of the
  extended modeling operator itself and a pseudodifferential operator
  of order 1 \cite[]{Symes:IPTA14,tenKroode:IPTA14}.
\item The choice of the penalty operator, controlling the extended
  degrees of freedom, is critical: in order to produce an objective
  immune from cycle-skipping, this penalty operator must be
  (pseudo-) differential of order zero \cite[]{StolkSymes:03}. A smooth
  multiplier $a$, as in equation \ref{eqn:annmult}, is a special case.
\end{itemize}

The expression \ref{eqn:dexpjgen} for the derivative of the reduced
objective $\tJa$ is the result of elementary manipulations, based on the
explicit expression \ref{eqn:mod} for the modeling operator $F$. In
this appendix I give an alternative derivation that generalizes to
extended inversion formulations for much less constrained physics. I
will point out the additional reasoning necessary to reach similar
conclusions in these more complex instances of extended inversion, as
presented for example in \cite{StolkSymes:03,StolkDeHoopSymes:09,Symes:IPTA14,tenKroode:IPTA14,HuangSymes:Geo17,HuangSymes:Geo18a,HuangSymes:Geo18b,HuangNammourSymesDollizal:SEG19}.

Recall that $\aw[m;d]$ is the solution of the normal equation
\ref{eqn:norm}. The reduced objective $\tJa$ is given by
\[
  \tJa[m;d] = \Ja[m,\aw[m;d];d]
\]
\[
  = \frac{1}{2}(\|F[m]\aw[m;d]-d\|^2 + \alpha^2\|A\aw[m;d]\|^2)
\]
(equation \ref{eqn:redexp1})
\begin{equation}
  \label{eqn:redsimple}
  =\frac{1}{2}(\|d\|^2 - \langle d,F[m]\aw[m;d]\rangle),
\end{equation}
after a little algebra.

As mentioned above, $F:M \times W \rightarrow D$ is {\em not}
differentiable. Neither is $\aw: M \times D \rightarrow W$, as follows
immediately from the identity \ref{eqn:normsol}. However $F\aw: M
\times D \rightarrow D$ is differentiable, hence so is $\tJa: M \times
D \rightarrow \bR$ thanks
to the identity \ref{eqn:redsimple},
under the conditions on the multiplier $a$ identified in Theorem
\ref{thm:diffobj}.

For the model problem studied in the body of this paper, a simple
argument justifies this conclusion.  To sideline some technical
details, assume that
$a \in C^{\infty}(\bR)\cap L^{\infty}(\bR)$ and $a(t)=t$ for $|t|<\tau$ and $|a(t)| \ge
\tau$ for $|t| \ge \tau$, with $\tau$
defined in equation \ref{eqn:taudef}. Then
Proposition \ref{thm:norminvexp}, item [1], implies that
\begin{equation}
  \label{eqn:faw}
  F[m]\aw[m;d] = F[m](F[m]^TF[m] + \alpha^2 A^TA)^{-1}F[m]^T d.
\end{equation}
The normal operator and its inverse are multiplication
operators, whereas $F$ and $F^T$ are scaled shift operators, inverse
to each other except for scale. From this it is easy to see that the
operator on the RHS of equation \ref{eqn:faw} is multiplication by a
smooth function, with its arguments shifted by $mr$. Such an operator
is smooth in $m$, hence so is $F\aw$.

Theorem \ref{thm:diffobj} also gives an explicit expression for the
derivative of $\tJa$. An alternative expression
does not depend on the structure of the single-trace modeling
operator, hence applies in a much more general setting. Note that if
$w \in H^1(\bR)$, then $m \mapsto F[m]w$ is differentiable, and
\begin{equation}
\label{eqn:deriv}
(D(F[m]w)\delta m)(t) = F[m](Q[m]\delta m)w (t), 
\end{equation}
where 
\begin{equation}
\label{eqn:defq}
(Q[m]\delta m)w = -r\delta m \frac{dw}{dt}. 
\end{equation}
That is, $DF[m]\delta m$ factors into $F[m]$ following $Q[m]\delta m$,
where the latter a skew-adjoint differential operator of order 1,
depending linearly on $\delta m$.

Assume that $d \in H^1(\bR)$. From equation \ref{eqn:normsol}, it
follows that $\aw[m;d] \in H^1(\bR)$ and moreover that $m \mapsto
\aw[m;d]$ is differentiable as a map from $\bR^+$ to
$L^2(\bR)$. From equation \ref{eqn:redexp1},
\[
  \tJa[m;d] = \frac{1}{2}(\|F[m]\aw[m;d]-d\|^2 + \alpha^2 \|A\aw[m;d]\|),
\]
whence $m \mapsto \tJa[m;d]$ is differentiable. A standard calculation
invoking the normal equation \ref{eqn:norm} shows that
\[
  D\tJa[m;d]\delta m = \langle D(F[m]\aw[m;d])\delta m, F[m]\aw[m;d]-d\rangle
\]
\[
  = \langle F[m](Q[m]\delta m)\aw[m;d], F[m]\aw[m;d]-d\rangle
\]
\[
  = \langle (Q[m]\delta m)\aw[m;d], F[m]^T(F[m]\aw[m;d]-d)\rangle
\]
\[
  = -\alpha^2\langle (Q[m]\delta m)\aw[m;d],A^TA\aw[m;d] \rangle
\]
\[
  = \alpha^2 \langle \aw[m;d], (Q[m]\delta m)A^TA \aw[m;d]\rangle
\]
(using antisymmetry of $Q$)
\[
  = \alpha^2 \langle \aw[m;d],[ (Q[m]\delta m), A^TA]\aw[m;d] \rangle
  + \alpha^2 \langle (Q[m]\delta m)\aw[m;d],A^TA\aw[m;d] \rangle.
\]
Rearranging,
\begin{equation}
  \label{eqn:djq}
  D\tJa[m;d]\delta m = \frac{1}{2}\alpha^2 \langle \aw[m;d],[
  (Q[m]\delta m), A^TA]\aw[m;d] \rangle.
\end{equation}
Since $Q$ is a differential operator of order 1, and $A^TA$ an
operator of order 0, the commutator has order 0. That is, the RHS of
equation \ref{eqn:djq} defines a continuous quadratic form in $d$ with
respect to the $L^2$ norm. As shown above, the same is true of the
left hand side. Therefore their extensions by continuity to $d \in
L^2(\bR)$ are the same, and the identity \ref{eqn:djq} holds for $d
\in L^2(\bR)$.

Making the identity \ref{eqn:djq} explicit by means of equation
\ref{eqn:defq} and $A^TA = t^2$, substituting the expression
\ref{eqn:normsol} for $\aw[m;d]$, and rearranging: one obtains
precisely the expression \ref{eqn:dexpjgen}.

Similar reasoning can be applied to some of the other extended
inversion settings mentioned at the beginning of this
appendix. Provided that the outer parameter $m$ consists of (or
parametrizes) smooth coefficients in the principal part of the
governing system of wave equations, $F$ is a microlocally elliptic
Fourier Integral Operator (FIO) \cite[]{Dui:95}. The canonical
relation of $F$ takes on the role of the shift operator
$t \rightarrow t-mr$, and has the properties necessary to conclude
that the composition of $F$ and $F^T$ in both orders is
pseudodifferential, at least in an open conic subset of the cotangent
bundle. If $A^TA$ is pseudodifferential, then so is the normal
operator, and it is microlocally elliptic, so positive-definite when
restricted to a suitable subspace of the inner parameter $w$, which
appears in the hyperbolic system as components of a right-hand side or
boundary condition. The inverse in the RHS of equation \ref{eqn:faw}
must be replaced by a microlocal parametrix. Thanks to Egorov's
Theorem \cite[]{Tay:81}, a special case of the rules for composing
FIOs, the operator on the RHS of equation \ref{eqn:faw} is a
pseudodifferential operator whose symbol is algebraic in geometric
optics quantities, hence depends smoothly on the coefficients in the
system of wave equations, and therefore on $m$, as does $F\aw$.

Microlocal ellipticity of $F$ implies an analogue of relation
\ref{eqn:deriv}, in which $Q$ is an essentially skew-symmetric
pseudodifferential operator of order 1. Since the derivation 
the gradient identity \ref{eqn:djq} is algebraic, it holds
approximately in these more general cases, with lower-order (smoother)
error terms. See 
\cite{Symes:Madrid,tenKroode:IPTA14,Symes:IPTA14} for details in
several specific cases, and for a similar calculation of the Hessian.
For these examples, the principal symbol of $Q$ can be computed
explicitly, as a function of the geodesic distance (traveltime)
function. Consequently, the analog of $\tJa$ is tangent to second
order to the Hessian of an objective formulated in terms of
traveltime. This relation explains the ability of the variant of
extended inversion studied in the cited references to recover
kinematically accurate velocity fields. Of course, that is what is
shown in detail in this paper for the very simple model problem
studied here.
