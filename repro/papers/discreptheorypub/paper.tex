\title{Error Bounds for Extended Source Inversion applied to an Acoustic Transmission Inverse Problem}
\author{William W. Symes\\
  Department of Computational and Applied Mathematics, \\
  Rice University, Houston, Texas 77251, USA; {\tt symes@rice.edu}}

\lefthead{Error Bounds for Extended Source Inversion}
\righthead{Symes}

\begin{abstract}
  A simple inverse problem for the wave equation requires
  determination of both the wave velocity in a homogenous acoustic
  material and the transient waveform of an isotropic point radiator,
  given the time history of the wavefield at a remote point in
  space. The duration (support) of the source waveform and the
  source-to-receiver distance are assumed known. A least squares
  formulation of this problem exhibits the ``cycle-skipping''
  behaviour observed in field scale problems of this type, with many
  local minima differing greatly from the global minimizer. An
  extended formulation, dropping the support constraint on the source
  waveform in favor of a weighted quadratic penalty, eliminates this
  misbehaviour. With proper choice of the weight operator, the
  velocity component at {\em any} local minimizer of this extended
  objective function differs from the global
  minimizer of the least-squares formulation by less than a linear
  combination of the source waveform support radius and data
  noise-to-signal ratio.

\end{abstract}

\section{Introduction}
Inverse problems based on wave equations (acoustic, elastic,
Maxwell's...) are commonplace in geophysics, nondestructive materials
testing, medical imaging, and other areas of science and engineering
in which wave motion plays an important role. These problems may be
formulated as nonlinear least squares problems, requiring optimization
of mean square misfit between observed and predicted data. Iterative
local numerical optimization methods (relatives of Newton's method)
are feasible solution approaches for the computationally large
variants of these nonlinear least squares problems that occur in
seismology and medical imaging. However, local optimization produces
only approximate stationary points, and these exhibit a tendency to
stagnate at physically irrelevant solutions, under typical conditions
of data acquisition. This ``cycle-skipping''
pathology can sometimes be overcome through use of information other than
measured wave data, however it remains a serious obstacle to effective
solution of field-scale inverse problems in wave propagation
\cite[]{GauTarVir:86,VirieuxOperto:09,Fichtner:10,Plessix:10,Schuster:17}.

Since numerical feasibility limits practical optimization approaches
to search for stationary points, it is natural to explore alternatives
to nonlinear least squares formulations for which approximate
stationary points might perforce be acceptable solutions. Many such
alternatives have been suggested and applied in both synthetic and
field data tests, in some cases with excellent results (see
\cite{HuangNammourSymesDollizal:SEG19},
\cite{PladysBrossierLiMetivier:GEO21} for recent reviews). However
numerical tests are inevitably ``a look in the rear view mirror'',
with no guarantee that the next example will not behave quite
differently. See \cite{Symes:2020} for an example of such an unpleasant
surprise. Theoretical guarantees that stationary points are
satisfactory solutions of specific inverse problems, in some specific
sense and under specified conditions, are almost entirely absent from
the literature.

This paper provides such a theoretical guarantee for a modification of
least-squares data fitting via {\em modeling operator extension},
applied to a very simple example: recovery of the (spatially
homogeneous) wave speed of an acoustic material model, together with
the waveform of an isotropic point radiator (``wavelet'') with
specified support, from the time history (``trace'') of the resulting
pressure field at a single remote location. This is perhaps the
simplest inverse wave propagation problem that exhibits
cycle-skipping. While far too simple to have immediate
practical application, it is a special case, or subproblem, of many
field-scale inverse problems in industrial and academic seismology. It
is also routinely used to illustrate cycle-skipping (see for example
\cite{VirieuxOperto:09}, Figure 7). Modeling operator extension ideas underlie
widely used techniques in seismic data processing, and have been
applied to inversion for several decades
\cite[]{geoprosp:2008}. The particular approach used in this work
is an example of {\em source extension}.
\cite{HuangNammourSymesDollizal:SEG19} review recent source extension
contributions, some of which are closely related to the approach to
single-trace transmission inversion studied here.

Model extension is only one of several ideas proposed to remedy
cycle-skipping. For instance, measuring data misfit with versions of the Wasserstein
metric from optimal transport theory, rather than the $L^2$ norm, also
shows promise in mitigating cycle-skipping
\cite[]{Metivier:GEO18,EngquistYang:GEO18,Ramos-Martinez:SEG18,Wang:SEG19,EngquistYang:CPAM21}. \cite{MahankaliYang:21}
have established conditions under which the Wasserstein misfit is
convex in a few parameters of simple models. Aside from the present
paper, their work is one of the very few to give a complete mathematical
justification (albeit also in very
restricted circumstances) for application of local optimization to a nonlinear
least squares alternative.

The next section precisely defines the single-trace transmission
inverse problem, and the following section confirms the
cycle-skipping behaviour of the obvious nonlinear least squares
formulation, that is, that stationary points exist at arbitrarily
large distance from any global minimizer (Theorem \ref{thm:fwi}). The
modeling operator extension used here simply consists in dropping the
support constraint, thus allowing fit to any data with any wavespeed,
by suitable choice of wavelet. The link between wave speed and data is
re-established by supplementing the mean-square data misfit function
with a suitable quadratic penalty on nonzero values of the wavelet far
from the specified support. I show that any stationary point of the
resulting penalty function deviates from the global solution of the
original, constrained nonlinear least squares problem by an amount
proportional to a combination of the support diameter and the relative
data error, and this bound is sharp
(Theorems \ref{thm:rampreallygood} and \ref{thm:mnoiseres}). A
solution of the original inverse problem (with wavelet support
constraint) is recovered by truncating the wavelet obtained in the
course of minimizing the penalty function (Theorem
\ref{thm:ipnoisesuf}).

The elementary
analysis developed here leans on the special features of the problem
under study to achieve sharper results than will likely be possible in
more general contexts. The Appendix recasts some of these
constructions in a form that applies, at least formally, to more
prototypical, larger-scale problems
\cite[]{Symes:IPTA14,tenKroode:IPTA14}, and suggests a route to
achieving a similar, mathematically complete understanding of
extension-based approaches to applied wave propagation inverse
problems.

\section{An Acoustic transmission Inverse Problem}
According to linear acoustics, the causal pressure field $p(\bx,t)$ due to an
isotropic point radiator at $\bx_s \in \bR^3$ with time-varying
intensity (``wavelet'') $w(t)$, propagating in an elastic fluid with
slowness (reciprocal velocity) $m$ and density $1$ (in appropriate
units), is the solution of the the initial value problem for the wave equation \cite[]{Frie:58}:
\begin{eqnarray}
  \label{eqn:awe}
  \left(m^2\frac{\partial^2 p}{\partial t^2} - \nabla^2\right) p(\bx,t) &=&
                                                                         w(t)\delta(\bx-\bx_s), \nonumber\\
  p(\bx,t)&=&0, t\ll 0.
\end{eqnarray}
The solution is well-known, see for instance
\cite{CourHil:62}, Chapter VI, section 12, equation 47:
\begin{equation}
  \label{eqn:homsol}
  p(\bx,t) = \frac{1}{4\pi |\bx-\bx_s|}w\left(t-m|\bx-\bx_s|\right).
\end{equation}
This trace of $p$ at $\bx_r \in \bR^3$ can be viewed as the result of
applying an m-dependent linear operator $F[m]$ to the wavelet $w$:
\begin{equation}
\label{eqn:mod}
F[m]w(t)  = p(\bx_r,t) = \frac{1}{4\pi r}w\left(t-mr\right).
\end{equation}
The slowness $m$ must be positive, as follows from basic acoustics,
and in fact reside in a range characteristic of the
material model: for crustal rock, a reasonable choice would be
$m_{\rm min}=0.125, m_{\rm max}=0.6$ s/km. The pressure trace
$p(\bx_r,\cdot)$ is square-integrable if and only if the same is true
of the wavelet $w$. Since the square-integral of the pressure trace is
proportional to the accumulated energy transferred from the fluid to
the sensor \cite[]{SantosaSymes:00}, assume that $w \in L^2(\bR)$.

Natural choices for domain and
range of $F$ are thus
\begin{itemize}
\item $M=(m_{\rm min}, m_{\rm max}),\,0 < m_{\rm min} \le m_{\rm
    max}$;
\item $W = L^2(\bR)$;
\item $D=L^2([t_{\rm min},t_{\rm max}]),\, t_{\rm min}<t_{\rm max}$;
\item $F: M \times W \rightarrow D$ as specified in \ref{eqn:mod}.
\end{itemize}
It is immediately evident from these choices and from the definition
\ref{eqn:mod} that
\begin{equation}
  \label{eqn:mapprop}
  \mbox{for }m \in M, F[m] \mbox{ is bounded, and }\|F[m]\| =
  \frac{1}{4\pi r}.
\end{equation}

Since
all possible data lie in the range of $F[m]$ for any $m \in M$, some
restriction of the domain of $F$ is necessary in order that fitting
the data constrain $m$. The constraint employed in this work is the specification
of a maxium support radius $\lambda_{\rm max} >0$ (see the companion
paper \cite{SymesChenMinkoff:21} for a justification of this choice).

For
$\lambda \in (0,\lambda_{\rm max}]$, define
\begin{itemize}
\item $\lW = \{w \in W:
  \mbox{supp }w \subset [-\lambda,\lambda]\}$;
\item $\lF = F|_{M \times \lW}$.
\end{itemize}

In terms of this infrastructure, the inverse problem studied in 
this paper may be stated as

\begin{quote}
\noindent {\bf Inverse Problem:}
  given data $d \in D$, relative error level $\epsilon \in
  [0,1)$, and support radius $\lambda \in (0, \lambda_{\rm
    max}]$, find $(m,w) \in M \times \lW$ for which 
\begin{equation}
  \label{eqn:probstat0}  \|\lF[m]w-d\| \le \epsilon\|d\|,
\end{equation}
\end{quote}

Define the relative mean-square error $\lerr: M \times \lW \times D
\rightarrow \bR^+$ by
\begin{equation}
  \label{eqn:redms}
  \lerr[m,w;d]=\frac{1}{2}\|\lF[m]w-d\|^2/\|d\|^2,
\end{equation}
so that inequality \ref{eqn:probstat0} is equivalent to
\begin{equation}
  \label{eqn:probstat1}
  \lerr[m,w;d] \le \frac{1}{2}\epsilon^2.
\end{equation}
Minimization of $\lerr$ is the standard least-squares formulation of
the Inverse Problem defined above: if the global minimum value of
$\lerr$ is less than $\frac{1}{2}\epsilon^2$, then the global
minimizer is a solution of the Inverse Problem.

\noindent{\bf Remark:} The constraint $\epsilon < 1$ imposed on the
target noise level eliminates the obvious choice $(m,0)$, which
satisfies the data misfit constraint for any $m \in M$ if $\epsilon
\ge 1$. 

\noindent{\bf Remark:} I shall refer to the minimization of $\lerr$ as
``Full Waveform Inversion'' or ``FWI'', as this is the
terminology used in the seismology literature to identify this and
similar optimization problems.

The best case for data fitting
is clearly the one in which the data can be fit precisely: that is,
there exists $(m_*,w_*) \in M\times \lW$ so that
\begin{equation}
  \label{eqn:defdatanonoise}
  d=\lF[m_*]w_*.
\end{equation}
Such data $d$ is {\em noise-free}, in the range of the map $\lF$. For
such data a solution of the Problem Statement \ref{eqn:probstat0}
exists with arbitrarily small $\epsilon>0$.


\section{Full Waveform Inversion}
While $F$ is surjective, as noted above, it is
very far from injective. On the other hand, under a constraint that
will be assumed throughout, $\lF[m]$ is injective for each $m \in M$ (in fact, $4 \pi r
\lF[m]$ is an isometry):
\begin{proposition}
  \label{thm:fullrec}
  Suppose that 
  \begin{equation}
    \label{eqn:fullrec}
    [ m_{\rm min}r-\lambda_{\rm max}, m_{\rm max}r+\lambda_{\rm max}]
    \subset [t_{\rm min},t_{\rm max}].
  \end{equation}
  Then $\lF[m]$ is coercive for every $m \in M, \lambda \in
  (0,\lambda_{\rm max}]$.
\end{proposition}

\noindent{\bf Remark:} A useful consequence of the condition 
\ref{eqn:fullrec}: for every $m \in M$, 
\begin{equation}
  [-\lambda_{\rm max}, \lambda_{\rm max}] \subset[ t_{\rm min}-mr , 
  t_{\rm max}+mr]. 
  \label{eqn:zeroinc}
\end{equation}

The first main result establishes the existence of large (100 \%)
residual local minimizers for the basic FWI objective $\lerr$, even
for noise-free data.
\begin{theorem}
  \label{thm:fwi}
  Suppose that $0 <\lambda\le \lambda_{\rm max}$,  $m_* \in M, w_*
  \in \lW, d=\lF[m_*]w_*$ is noise-free data per definition \ref{eqn:defdatanonoise},
  Under assumption \ref{eqn:fullrec}, for any $m \in M$ with $|m-m_*|r>2\lambda$,
\begin{equation}
  \label{eqn:isovpm}
 \min_w \lerr[m,w;d]=\lerr[m,0;d] = \frac{1}{2},
\end{equation}
and any such $(m,0)$ is a local minimizer of $\lerr$ with relative RMS
error = 1.0.
\end{theorem}

\begin{proof} From the definition \ref{eqn:mod},
\[
 \lerr[m,w;d] =  \frac{1}{32\pi^2
    r^2\|d\|^2}\int\,dt\,\left|w\left(t-mr\right)-w_*\left(t-m_*r\right)\right|^2.
\]
Since $w_*, w$ vanish for $|t|>\lambda$,
$\lF[m_*]w_*(t)$ vanishes if $|t-m_*r|>\lambda$ and $\lF[m]w$ vanishes if $|t-mr|>\lambda$. So if $|mr-m_*r|
= |m-m_*|r > 2\lambda$, then $|t-mr|+|t-m_*r| \ge |mr-m_*r| >
2\lambda$ so either $|t-mr|>\lambda$ or $|t-m_*r|>\lambda$, that is,
either $\lF[m]w(t)=0$ or $\lF[m_*]w_*=0$. Therefore $\lF[m]w$ and
$\lF[m_*]w_*$ are orthogonal in the sense of the $L^2$ inner product
$\langle \cdot,\cdot \rangle_D$ on $D$:
\begin{equation}
  \label{eqn:ortho}
  |m- m_*|r > 2\lambda \,\, \Rightarrow \,\, \langle F[m]w,
  F[m_*]w_*\rangle_D = 0.
\end{equation}
But $d = \lF[m_*]w_*$, so this is the same as saying that $d$ is
orthogonal to $F[m]w$. So conclude after a minor manipulation that
\[
  |m- m_*|r > 2\lambda \,\, \Rightarrow \,\, \lerr[m,w;d]=\frac{1}{32\pi^2 
    r^2\|d\|^2}(\|w\|^2 + \|w_*\|^2)
\]
\begin{equation}
  \label{eqn:iso}
  = \frac{1}{2}\left(\frac{\|w\|^2}{\|w_*\|^2} + 1 \right).
\end{equation}
That is, for slowness $m$ in error by more than $2\lambda/r$ from the 
target slowness $m_*$, the means square error (FWI objective) $\lerr$ is independent of
$m$, and its minimum over $w$ is attained for $w=0$.
\end{proof}

Therefore local minimizers of $\lerr$ abound, as far as you like from the
global minimizer $(m_*,w_*)$. Local exploration of the FWI objective
$e$ gives no useful information whatever about constructive search
directions, and descent-based optimization tends to fail if the
initial estimate $m_0$ is in error by more than $2\lambda/r$. In fact the actual behaviour of FWI itererations is worse
(failure if $m_0$ is in error by ``half a wavelength''), as follows
from a more refined analysis of the cycle-skipping local behaviour of $\lerr$ near its
global minimizer.

\section{Extended Source Inversion}
The phenomenon explained in the last section can be avoided by
reformulating the inverse problem via an extended modeling operator
and a soft (penalty) constraint to replace the support
requirement. The extension simply amounts to dropping the support
constraint, and replacing $\lW$ and $\lF$ by $W$ and $F$.

The hard
support constraint implicit in the choice of $\lW$ as domain for the
modeling operator is replaced by a soft constraint in the form of a
quadratic penalty, with weight operator $A:W \rightarrow W$.
The choices for the penalty operator $A$ considered here are scalar 
multiplication operators on $W$ defined by a choice of multiplier $a \in L^{\infty}(\bR)$:
\begin{equation}
  \label{eqn:annmult}
  A w(t)= a(t)w(t), \, t\in \bR.
\end{equation}
Explicit choices
for $a$ are discuss below.

With these choices, define
\begin{eqnarray}
  \label{eqn:edef}
  e[m,w;d] & = & \frac{1}{2}\|F[m]w-d\|^2/\|d\|^2;\\
  \label{eqn:gdef}
  g[w;d] & = & \frac{1}{2}\|Aw\|^2/\|d\|^2;\\
  \label{eqn:jdef}
  \Ja[m,w;d] & = & e[m,w;d] + \alpha^2g[w;d].
\end{eqnarray}

The main theoretical device used in the proofs of our main results on 
extended inversion is reduction of the penalty objective $\Ja$
(equation \ref{eqn:jdef}) to a function $\tJa$ of 
$m$ alone, by minimization over $w$:
\begin{equation}
  \label{eqn:redexp}
  \tJa[m;d] = \inf_w \Ja[m,w;d].
\end{equation}
Reduction of this type is often termed ``Variable Projection'', after \cite{GolubPereyra:73,GolubPereyra:03}.

A minimizer $w$ on the right-hand side of definition
\ref{eqn:redexp} must solve the {\em normal equation}
\begin{equation}
  \label{eqn:norm}
  (F[m]^TF[m]+\alpha^2A^TA)w= F[m]^Td, 
\end{equation}
(here and in the following, the superscript $T$ signifies adjoint in
the sense of domain and range Hilbert structures).

With $A$ of the form \ref{eqn:annmult}, $\tilde{J}_{\alpha}$ is explicitly
computable. First observe that apart from amplitude, $F[m]$ is
unitary: for $g \in D$,
\begin{equation}
\label{eqn:tran}
F[m]^T g (t) =
\left\{
  \begin{array}{c}
    \frac{1}{4\pi r}g\left(t+mr\right), \, t \in [t_{\rm min}-mr,
    t_{\rm max}-mr],\\
    0, \mbox{ else.}
  \end{array}
\right.
\end{equation}
so
\begin{equation}
  \label{eqn:unit}
  F[m]^TF[m] = \frac{1}{(4\pi r)^2}{\bf 1}_{[t_{\rm min}-mr,  
    t_{\rm max}-mr]}
\end{equation}
in which ${\bf 1}_{S}$ denotes
multiplication by the characteristic function of a measurable 
$S \subset \bR$.

Therefore the normal equation for the minimizer on the RHS of equation \ref{eqn:redexp} is
\begin{equation}
  \label{eqn:norm1}
  \left(\frac{1}{(4\pi r)^2} {\bf 1}_{[t_{\rm min}-mr,  
      t_{\rm max}-mr]} + \alpha^2 A^TA\right)w= F[m]^Td.
\end{equation}

With these choices, the normal equation \ref{eqn:norm1} becomes
\begin{equation}
\label{eqn:norm2}
\left(\frac{1}{(4\pi r)^2}  {\bf 1}_{[t_{\rm min}-mr,  
      t_{\rm max}-mr]} + \alpha^2a^2\right)w= F[m]^Td.
\end{equation}
Evidently, $F^TF$ (the first summand in equation \ref{eqn:norm2}) is
not coercive. If $A$ is to yield small output for input localized near
$t=0$, as it must to play its desired role in replacing the hard
support constraint, then $A^TA$ must not be coercive either, that is,
its spectrum must have zero as its infimum. Therefore an additional
condition on $A$ is required to enable application of the Lax-Milgram
constrruction to the normal
equation \ref{eqn:norm}. This condition is an hypothesis of 

\begin{proposition}
  \label{thm:norminvexp}
  Assume the conditions \ref{eqn:mod}, \ref{eqn:fullrec},
  \ref{eqn:annmult}. Also assume that $\lambda \in (0,\lambda_{\rm
    max}], \alpha > 0,$ and that  $C>0$ exists so that $a \in L^{\infty}(\bR)$
  mentioned in equation \ref{eqn:annmult} satisfies the condition
  \begin{equation}
    \label{eqn:abnd} 
    a \ge 0; \, a(t) \ge C\mbox{ for }|t| \ge \lambda_{\rm max}.
  \end{equation}
%  \begin{equation}
%   \label{eqn:abnd}
%    |t| > \lambda \Rightarrow a(t) \ge C,
%  \end{equation}
%  in which $C>0$ may depend on $\lambda$.
  Then
  \begin{itemize}
  \item[1. ]the normal operator $F[m]^TF[m] + \alpha^2A^TA$ is
    invertible for any $m \in M$, $\alpha > 0$;
  \item[2. ]the solution $\aw[m;d]\in W$ of the normal equation
    \ref{eqn:norm} is given by
    \begin{equation}
      \label{eqn:normsol}
      \aw[m;d](t) = \left\{
        \begin{array}{c}
          \left(\frac{1}{(4\pi r)^2} + \alpha^2
          a^2(t)\right)^{-1}\frac{1}{4 \pi r}d(t+mr), t \in [t_{\rm
          min}-mr, t_{\rm max}-mr];\\
          0, \mbox{ else.}
        \end{array}
      \right.
    \end{equation}
  \item[3. ]if in addition $d=F[m_*]w_*, w_* \in \lW$ is noise-free, as in equation
    \ref{eqn:defdatanonoise},
    \begin{equation}
      \label{eqn:solnonoise}
      \aw[m,d](t)= \left(1+ (4\pi r)^2\alpha^2 a(t)^2\right)^{-1}w_*\left(t+(m-m_*)r\right).
    \end{equation}
  \end{itemize}
\end{proposition}

\begin{proof}
  The idea here is that $F$ fails to be coercive just where $A$ is
  coercive, and vis-versa.
  \begin{itemize}
  \item[1. ]Note that thanks to \ref{eqn:zeroinc}, if $|t|\le
    \lambda \le \lambda_{\rm max}$, then ${\bf 1}_{[t_{\rm min}-mr,  
      t_{\rm max}-mr]}(t) = 1$, whereas if $|t|>\lambda$,
    then $a(t) \ge C$, whence
    \[
      \frac{1}{(4\pi r)^2}  {\bf 1}_{[t_{\rm min}-mr,  
        t_{\rm max}-mr]} + \alpha^2a^2  \ge \min\{(4\pi r)^2,
      \alpha^2\min\{1/(4\pi r)^2,C^2\} \}> 0.
    \]
    Therefore the normal operator is invertible under the stated
    conditions.

  \item[2. ]From the identity \ref{eqn:tran}.
    \[
      \mbox{supp }F[m]^Td \subset [t_{\rm min}-mr,t_{\rm max}-mr].
    \]
    Define $w_{\rm tmp}$ to be the right-hand side of equation \ref{eqn:normsol}. Then
    from the previous observation and identity \ref{eqn:tran},
    \[
      \mbox{supp }w_{\rm tmp} \subset [t_{\rm min}-mr,t_{\rm max}-mr].
    \]
    From the identity \ref{eqn:unit}, for any $w \in W$,
    \[
      t \in [t_{\rm min}-mr,t_{\rm max}-mr] \Rightarrow F[m]^TF[m]w(t)
      = \frac{1}{(4 \pi r)^2}w(t).
    \]
    It follows from this and the previous two observations that
    $w_{\rm tmp}$ solves the normal equation \ref{eqn:norm}, and
    therefore that $\aw[m;d]=w_{\rm tmp}$.

  \item[3. ]Follows by inserting the definition
    \ref{eqn:defdatanonoise} of $d$ in \ref{eqn:normsol} and
    rearranging.
  \end{itemize}
\end{proof}

\begin{theorem}
  \label{thm:norminv}
  Assume the condition \ref{eqn:fullrec}, $C>0$, and suppose that $A$ is
  given by equation \ref{eqn:annmult} for $a \in L^{\infty}(\bR)$
  satisfying condition \ref{eqn:abnd}.  Then
  \begin{itemize}
  \item[1. ]the reduced objective $\tJa$ is given by
    \begin{equation}
      \label{eqn:redexp1}
      \tJa[m;d] = \Ja[m,\aw[m;d];d],
    \end{equation}
    in which $\aw[m;d] \in W$ is the unique solution of the normal
    equation \ref{eqn:norm}.
  \item[2. ]The following are equivalent:
    \begin{itemize}
    \item[i. ]$(m,w) \in M \times W$ is a local minimizer of
      $\Ja[\cdot,\cdot;d]$, and
    \item[ii. ]$m$ is a local minimizer of $\tJa[\cdot;d]$ and
      $w=\aw[m;d]$.
    \end{itemize}
  \end{itemize}
\end{theorem}

\begin{proof} These conclusions follow immediately from Proposition
  \ref{thm:norminvexp}.
\end{proof}

If $\Ja[\cdot,\cdot;d]$ and $\tJa[\cdot;d]$
were differentiable, then ``local minimizer'' in the conclusion of the
preceding theorem could be replaced by ``stationary point''. However,
for the problem addressed in this paper, $\Ja[\cdot,\cdot;d]$ {\em is
  not} differentiable without added smoothness constraints on $w$,
whereas $\tJa[\cdot;d]$ {\em is} differentiable for proper choice of
penalty operator $A$.
This conclusion follows from properties of the modeling operator $F$
shared with many other inverse problems in wave propagation, as
explained in the Appendix. Here, I derive differentiability from explicit
expressions for $\tJa$ and its components.

\begin{proposition}
  \label{thm:epjgen}
  Assume the hypotheses of Proposition \ref{thm:norminvexp}. Then
  \begin{equation}
  \label{eqn:residnormgen}
  e[m,\aw[m,d];d] = \frac{1}{2\|d\|^2}\int_{t_{\rm min}}^{t_{\rm max}} \,dt\,(4\pi r \alpha a(t-mr))^4(1 +
  (4\pi r \alpha a(t-mr))^2)^{-2}d(t)^2,
\end{equation}
\begin{equation}
  \label{eqn:anninormgen}
  p[m,\aw[m,d];d] = \frac{1}{2\|d\|^2}\int_{t_{\rm min}}^{t_{\rm max}} \,dt\,(4\pi r a(t-mr))^2(1 +
  (4\pi r \alpha a(t-mr))^2)^{-2}d(t)^2,
\end{equation}
and
\begin{equation}
  \label{eqn:expjgen}
\tJa[m;d] = \frac{1}{2\|d\|^2}\int_{t_{\rm min}}^{t_{\rm max}}\,dt\,(4\pi r \alpha a(t-mr))^2(1+(4\pi r \alpha 
a(t-mr))^2)^{-1}d(t)^2. 
\end{equation}
\end{proposition}

\begin{proof}
  From equation \ref{eqn:normsol},
  \[
    F[m]\aw[m;d](t) = 
    \frac{1}{4 \pi r}\left(\frac{1}{(4\pi r)^2} + \alpha^2
      a^2(t-mr)\right)^{-1}\frac{1}{4 \pi r}d(t),
  \]
  so
  \[
    (F[m]\aw[m;d]-d)(t) = (1 + (4 \pi r\alpha
    a(t-mr))^2)^{-1}-1)d(t)
  \]
  \[
    = -(4 \pi r\alpha a(t-mr))^2(1 + (4 \pi r\alpha
    a(t-mr))^2)^{-1}d(t).
  \]
  Half the integral of the square of this data residual is
  $e[m,\aw[m;d],d]$, which proves identity \ref{eqn:residnormgen}.

  To compute $p[m,\aw[m;d],d]$, note that
  \[
    A\aw[m;d](t)=a(t) \left(\frac{1}{(4\pi r)^2} + \alpha^2
      a^2(t)\right)^{-1}\frac{1}{4 \pi r}d(t+mr)
  \]
  \[
    = 4\pi r a(t) (1 + (4\pi r \alpha a(t))^2)^{-1}d(t+mr)
  \]
  for $ t \in [t_{\rm min}-mr, t_{\rm max}-mr]$, so squaring,
  integrating, and changing integration variables $t \mapsto t-mr$
  gives the result \ref{eqn:anninormgen}

  That the reduced objective $\tJa$ is given by \ref{eqn:expjgen} follows from equations \ref{eqn:redexp1}, \ref{eqn:residnormgen}, and
  \ref{eqn:anninormgen}.
\end{proof}

\begin{theorem}
  \label{thm:diffobj}
  Suppose that in addition to the hypotheses of Theorem
  \ref{thm:norminv}, $a \in W^{1,\infty}_{\rm loc}(\bR)$, then $\tJa[\cdot;d]
  \in C^1(M)$.
\end{theorem}

\begin{proof}
Suppose first that $a \in C^1(\bR)$. Differentiation under the integral sign  
  yields the expression for its derivative:
\begin{equation}
  \label{eqn:dexpjgen}
  \frac{d}{dm}\tJa[m;d] = -\frac{(4 \pi r \alpha)^2}{\|d\|^2} \int_{t_{\rm min}}^{t_{\rm max}} \,dt \, 
  \left(a\frac{da}{dt}\right)(t-mr)(1+(4\pi r \alpha 
  a(t-mr))^2)^{-2}d(t)^2. 
\end{equation}
For $a \in W^{1,\infty}_{\rm loc}(\bR)$ a limiting argument shows that the
same expression gives the derivative of $\tJa$.
\end{proof}

It will be useful to record expressions for the various components of
$\tJa$ when the data is noise-free, that is, the context of
Proposition \ref{thm:norminvexp}, item 3:
\begin{corollary}
  \label{thm:epjnonoise}
  Assume the hypotheses of Proposition \ref{thm:norminvexp}, item
  3. Then noting that $\|d\| = \|w_*\|/(4 \pi r)$
\begin{equation}
  \label{eqn:residnorm}
  e[m,\aw[m,d];d] 
= \frac{\alpha^4}{2\|w_*\|^2}\int\,dt\,a(t-(m-m_*)r)^4(1+(4\pi r)^2 \alpha^2 
    a(t-(m-m_*)r)^2)^{-2}w_*(t)^2.
\end{equation}
\begin{equation}
  \label{eqn:anninorm}
  p[m,\aw[m,d];d] = \frac{(4\pi r)^2}{2\|w_*\|^2}\int \,dt\,  
  \frac{a(t-(m-m_*)r)^2}{(1+ (4\pi r)^2\alpha^2
    a(t-(m-m_*)r)^2)^{2}}w_*(t)^2.
\end{equation}
so 
\begin{equation}
\label{eqn:expjnonoise}
\tJa[m;d] = \frac{(4\pi r \alpha)^2}{2\|w_*\|^2}\int\,dt\,a(t-(m-m_*)r)^2(1+(4\pi r)^2 \alpha^2 
  a(t-(m-m_*)r)^2)^{-1}w_*(t)^2. 
\end{equation}
Finally, if $a \in W^{1,\infty}(\bR)$, then $\tJa[\cdot;d]$ is differentiable, and 
\begin{equation}
  \label{eqn:dexpjnonoise}
  \frac{d}{dm}\tJa[m;d] = -\frac{r (4\pi r \alpha)^2}{\|w_*\|^2} \int \,dt \, 
  \frac{\left(a\frac{da}{dt}\right)(t-(m-m_*)r)}{(1+(4\pi r)^2 \alpha^2 
  a(t-(m-m_*)r)^2)^{2}}w_*(t)^2. 
\end{equation}
\end{corollary}


\section{Stationary Points}

Recall that the purpose of the penalty operator is to penalize energy away from 
$t=0$. A simple multiplier of class $W^{1,\infty}$ that accomplishes
this goal (and produces a differentiable reduced objective $\tJa$,
thanks to Theorem \ref{thm:diffobj}) is
\begin{equation}
  \label{eqn:ann}
  a(t) = \min(|t|, \tau). 
\end{equation}
for suitable $\tau>0$. In fact, 
the cutoff $\tau$ will be chosen large enough to be effectively inactive: 
hindsight suggests 
\begin{equation}
  \label{eqn:taudef}
  \tau = \max\{|t_{\rm min}-m_{\rm min}r|,|t_{\rm min}-m_{\rm max}r|, |t_{\rm max}-m_{\rm min}r|, |t_{\rm max}-m_{\rm max}r|\}. 
\end{equation}
Essentially this 
particular annihilator has been employed in earlier work on extended 
source inversion 
\cite[]{Plessix:00a,LuoSava:11,Warner:14,HuangSymes:SEG15a,Warner:16,HuangSymes:Geo17}.

\begin{proposition}
  \label{thm:rampgood}
  Suppose that
  \begin{enumerate}
  \item $m_* \in M$;
  \item $0 < \mu \le \lambda$, and $w_* \in W_{\mu}$;
  \item $d_* = F[m_*]w_*$;
  \item $a(t)=\min\{|t|,\tau\}$ in the definition \ref{eqn:annmult},
    with $\tau$ given by equation \ref{eqn:taudef}; and
  \item $\alpha > 0$.
  \end{enumerate}
  Then for any $m \in M$, 
  \begin{equation}
    | (m - m_*)r| > \lambda  \Rightarrow  \left|\frac{d}{dm}\tJa[m;d_*]\right| >  
    \frac{r(4 \pi r \alpha)^2(\lambda-\mu)}{(1+(4\pi r\alpha)^2 
      (\lambda+\mu)^2)^{2}}.
    \label{eqn:gradbndnonoise}
  \end{equation}
\end{proposition}
\begin{proof}
  As observed before, $\mbox{supp }\aw[m;d_*] \subset [t_{\rm
    min}-mr,t_{\rm max}-mr]\subset [-\tau,\tau]$, with $\tau$ defined
  in \ref{eqn:taudef}. Therefore, $a(t) = |t|$, $a a'(t) = t$ in the
  support of the integrand on the RHS of equation
  \ref{eqn:dexpjnonoise}, which therefore 
  becomes (after change of integration variable)
  %%%%%%%%%%%%%%%%%%%%%%%%%
  \begin{equation}
    \label{eqn:gradfinal}
    \frac{d}{dm}\tJa[m;d_*] = -\frac{r (4\pi r\alpha)^2}{\|w_*\|^2} \int \,dt \, 
  t(1+(4\pi r)^2 \alpha^2 
  t^2)^{-2}w_*(t+(m-m_*))^2.
  \end{equation}
  Recall that $w_*(t+(m-m_*)r)$
  vanishes if $|t+(m-m_*)r| > \lambda$. Therefore the integral on the
  RHS of equation \ref{eqn:gradfinal} can be re-written
  \[
    = -\frac{r(4 \pi r \alpha)^2}{\|w_*\|^2}\int_{-(m-m_*)r-\lambda}^{-(m-m_*)r+\lambda}
    \,dt\, t(1+(4\pi r)^2\alpha^2 t^2)^{-2}w_*\left(t+(m-m_*)r\right)^2.
  \]
  Suppose that $\mu \le \lambda$ and $w_* \in W_{\mu}$. 
  If $m > m_*+\lambda/r$, then $t+(m-m_*)r \in \mbox{supp }w_*$
  implies $-\mu - \lambda < t < \mu-\lambda<0$, so 
  \[
    t(1+(4\pi r)^2\alpha^2 t^2)^{-2} < (\mu-\lambda)(1+(4\pi r)^2\alpha^2 (\mu+\lambda)^2)^{-2}<0
  \]
  in the support of the integrand in equation
  \ref{eqn:gradfinal}. Arguing similarly for $m<m_*-\lambda/r$, obtain
  a similar inequality, implying the conclusion \ref{eqn:gradbndnonoise}.
\end{proof}

\begin{theorem}
  \label{thm:rampreallygood}
  Suppose that
  \begin{enumerate}
  \item $m_* \in M$;
  \item $0 <  \lambda$, and $w_* \in \lW$;
  \item $d_* = F[m_*]w_*$;
  \item $a(t)=\min\{|t|,\tau\}$ in the definition \ref{eqn:annmult},
    with $\tau$ given by equation \ref{eqn:taudef}; 
  \item $\alpha > 0$; and
  \item$m \in M$ is a stationary point of $\tJa[\cdot;d_*]$.
  \end{enumerate}
  Then $|m-m_*| < \lambda /r$.
\end{theorem}

\begin{proof} Follows directly from Proposition \ref{thm:rampgood} by
  taking $\mu=\lambda$.
\end{proof}

The preceding theorem established that a proper choice of annihilator
leads to a reduced penalty objective all of whose stationary points
are within $O(\lambda)$ of the target slowness $m_*$, provided that
the data are noise-free in the sense of equation
\ref{eqn:defdatanonoise}. This result leaves open two questions:
\begin{itemize}
\item how does one use this reduced penalty minimization to produce
  a solution of the inverse problem as in problem statement
  \ref{eqn:probstat0}? 
\item how does one answer the same question for noisy data?
\end{itemize}

The next result answers the first question, in the case of noise-free data:
\begin{proposition}
  \label{thm:ipnonoisesuf}
  Suppose that $a$ is given by definition \ref{eqn:ann}, $\alpha$,
  $\mu \in (0,\lambda_{\rm max}]$,
  $d$ is given by
  \ref{eqn:defdatanonoise} with $w_* \in W_{\mu}$, and  $m$ is a stationary
  point of $\tJa[\cdot;d]$. Then $(m,\aw[m;d])$ is a
  solution of the inverse problem \ref{eqn:probstat0} for any $\lambda
  \ge 2\mu$ and
  \begin{equation}
    \label{eqn:estresidnorm}
    \epsilon \ge \frac{(8\pi r \mu \alpha)^2}{1 + (8\pi r \mu\alpha)^2}.
  \end{equation}
\end{proposition}

\begin{proof}
  From the assumption $w_* \in W_{\mu}$ and Theorem
  \ref{thm:rampreallygood}, $|(m-m_*)r|\le \mu$. From the
  identity \ref{eqn:solnonoise},
  $\mbox{supp }\aw[m;d] \subset
  [(m-m_*)r-\mu,(m-m_*)r+\mu] \subset
  [-2\mu,2\mu]$. Because of the support limitation, $a(t)=|t|$ in the
  interval of integration appearing in \ref{eqn:residnorm}, so
\[
  e[m,\aw[m,d];d] 
= 8 \pi^2 r^2 \alpha^4\int^{\mu}_{-\mu}\,dt\,\frac{|t-(m-m_*)r|^4}{(1+(4\pi r)^2 \alpha^2 
|t-(m-m_*)r|^2)^{2}}w_*(t)^2
\]
\[
  \frac{1}{2} (4\pi r \alpha)^4\int^{\mu}_{-\mu}\,dt\,\frac{|t-(m-m_*)r|^4}{(1+(4\pi r)^2 \alpha^2 
|t-(m-m_*)r|^2)^{2}}d(t+m_*r)^2
\]
\[
  \le \frac{1}{2} \|d\|^2  \left(\frac{(8\pi r \alpha \mu)^4}{(1+(8\pi
      r \alpha \mu)^2)^2}\right).
  \]
\end{proof}

The inequality \ref{eqn:estresidnorm} can be interpreted as a bound 
on $\alpha$, given $\epsilon$ and $\lambda$, for a
stationary point of $\tJa$ to yield a solution of the inverse
problem: one obtains a solution, provided that $\alpha$ is
sufficiently small. On the other hand, it is clear that $\alpha$
cannot be too large if stationary points of $\tJa$ are to yield
solutions: the integrand in \ref{eqn:residnorm} is increasing in
$\alpha$ for every $t$ and $m$, and the multiplier
\[
t \mapsto (4\pi r \alpha(t-(m-m_*)r))^4(1+(4\pi r)^2 \alpha^2 
|t-(m-m_*)r|^2)^{-2}
\]
tends monotonically to $1$ as $\alpha \rightarrow \infty$, uniformly
on the complement of any open interval containing
$t=(m-m_*)r$. Therefore
\begin{equation}
  \label{eqn:elimit}
  \lim_{\alpha \rightarrow \infty}e[m,\aw[m;d];d] = \frac{1}{2}.
\end{equation}
Consequently, there exists $\alpha_{\rm max}(\epsilon,\lambda,d)$ so
that
\[
  e[m,\aw[m;d];d]  \le \frac{1}{2}\epsilon^2
  \Rightarrow \alpha \le \alpha_{\rm max}(\epsilon,\lambda,d).
\]
The existence of this limiting penalty weight has been inferred
indirectly. \cite{FuSymes2017discrepancy} describe a constructive
algorithm for its approximation, which is used in the companion paper
\cite{SymesChenMinkoff:21} to dynamically adjust $\alpha$ in the
course of optimization of $\tJa$.

I turn now to the second issue identified above, the effect of
noise. Suppose that the data trace $d$ takes the form
\begin{equation}
  \label{eqn:defdatanoisy}
  d = F[m_*]w_* + n = d_*+n,
\end{equation}
with $m_* \in M, w_* \in W_{\mu}$, $0<\mu<\lambda$, and noise trace $n \in
D$. Since no support assumptions can be made about $n$, equation
\ref{eqn:normsol} does not imply that $\aw[m;d] \in \lW$ for any values of
$\alpha$ or $\lambda$.  Therefore minimization of $\tJa$ cannot by itself yield a
solution of the inverse problem as defined in the problem statement
\ref{eqn:probstat0}. %\textcolor{red}{In this section, I explain how a solution may
%nonetheless be constructed from a stationary point of $\tJa$.
%First, examine the effect of additive noise on the estimation of the
%slowness $m$. In expressing the result, use the dimensionless
%relative data error}

%\textcolor{blue}
{To see how a solution may
  nonetheless be constructed from a stationary point of $\tJa$,
  examine the effect of additive noise on the estimation of the
slowness $m$, in terms of the dimensionless
relative data error}
\begin{equation}
  \label{eqn:defeta}
  \eta = \frac{\|n\|}{\|d_*\|}. 
\end{equation}

\begin{proposition}
  \label{thm:mnoise}
  Assume the hypotheses of Proposition \ref{thm:rampgood}, and that $d$ is
  given by definition \ref{eqn:defdatanoisy}. Suppose that $m \in M$
  is a stationary point of $\tJa[\cdot;d]$, and that
  \begin{equation}
    \label{eqn:mnoisebnd}
    \eta(1+\eta) \le \frac{16}{3\sqrt{3}}\frac{4\pi r \alpha
      (\lambda-\mu)}{(1+(4\pi r\alpha(\lambda+\mu))^2)^2}.
  \end{equation}
  Then
  \begin{equation}
    \label{eqn:mnoisebndfin}
    |m-m_*| \le \frac{\lambda}{r}.
  \end{equation}.
\end{proposition}

\begin{proof}
  From equation \ref{eqn:dexpjgen}, $d \tJa / dm$ is the
  value of a quadratic form in $d$
%  \textcolor{red}{with}
%  \textcolor{blue}
  {defined by the}
  (indefinite) symmetric operator
$B = $ multiplication by
\[
  b(t;m,\alpha)  = -\frac{(4 \pi r \alpha)^2 (t-mr)}{(1+(4\pi r \alpha (t-mr))^2)^{2}}.
\]
Therefore
\begin{equation}
  \label{eqn:gradlip}
  \left|\frac{d}{dm}\tJa[m;d]-\frac{d}{dm}\tJa[m;d_*]\right| =
  |\langle (d+d_*),B(d-d_*)\rangle| \le \max_{t \in
    \bR}|b(t;m,\alpha)|\eta(1+\eta)\|d_*\|^2
\end{equation}
A straightforward calculation shows that
\[
  \max_{t \in \bR} b(t;m,\alpha) = \frac{3\sqrt{3}}{16} 4\pi r\alpha.
\]
For a stationary point $m$ of
$\tJa[\cdot;d]$, the inequality \ref{eqn:gradlip} implies
\[
  \left|\frac{d}{dm}\tJa[m;d_*]\right| \le \frac{3\sqrt{3}}{16} 4\pi
  r\alpha \eta(1+\eta)\|d_*\|^2.
\]
On the other hand, the conclusion \ref{eqn:gradbndnonoise} of Proposition
\ref{thm:rampgood} implies that if also
\[
  \frac{3\sqrt{3}}{16} 4\pi r\alpha \eta(1+\eta)\|d_*\|^2 \le (4 \pi r
  \alpha)^2 \frac{\lambda-\mu}{(1+(4\pi r)^2\alpha^2
    (\lambda+\mu)^2)^{2}} \|d_*\|^2,
\]
then $|m-m_*|\le \lambda/r$. Rearranging, obtain the conclusion.
\end{proof}

\begin{corollary}
  \label{thm:mnoisecor}
  Asumme the hypotheses of Proposition \ref{thm:mnoise}, in particular
  that $m$ is a stationary point of $\tJa[\cdot;d]$, $d=d_*+n$. Then
  \begin{equation}
    \label{eqn:mnoisecor}
    |m-m_*| \le \frac{\mu}{r} + \frac{\eta}{\alpha} \left(\frac{3\sqrt{3}(1+\eta)}{64\pi r^2}(1+(8\pi r \alpha
      \lambda_{\rm max})^2)^2\right).
  \end{equation}
\end{corollary}

\begin{proof} Assume that $\lambda$ is chosen to obtain equality in
  the condition \ref{eqn:mnoisebnd}, substitute the bound $2
  \lambda_{\rm max}$ for $\lambda + \mu$ in the denominator, solve for
  $\lambda$ and substitute in inequality \ref{eqn:mnoisebndfin}.
\end{proof}

\noindent {\bf Remark:} This bound suggests that error in the estimate
of $m$ due to data noise is a decreasing function of $\alpha$, at
least for small $\alpha$. This result is intuitively appealing, and is
supported by numerical evidence. It may be taken as a justification
for the discrepancy-based algorithm for adjusting $\alpha$ during
inversion \cite[]{FuSymes2017discrepancy}, used in the companion paper
\cite[]{SymesChenMinkoff:21}. 

\begin{theorem}
  \label{thm:mnoiseres}
  Asumme that
  \begin{itemize}
  \item[1. ] $\alpha, \mu> 0$,
  \item[2. ] $m_* \in M, w_* \in W_{\mu}$,
  \item[3. ] $d_* = F[m_*]w_*$,
  \item[4. ] $n \in D$ and $d = d_* + n$.
  \end{itemize}
  Set $\eta = \|n\|/\|d_*\|$. Assume that $\eta$ satisfies inequality \begin{equation}
  \label{eqn:mnoisecond}    
  \eta < \frac{\sqrt{5}-1}{2},
  \end{equation}
  and that $m$ is a stationary point of $\tJa[\cdot;d]$.
  Then
  \begin{equation}
    \label{eqn:mnoisesuff}
    |m-m_*| \le \left(1+\frac{2\eta(1+\eta)}{1-\eta(1+\eta)}\right)\frac{\mu}{r}.
  \end{equation}
\end{theorem}

\begin{proof}.
  Write $\lambda = (1+\delta)\mu$, and $x=4 \pi r \alpha \mu$. Then
  the right-hand side of equation \ref{eqn:mnoisebnd} may be written as
  \begin{equation}
    \label{eqn:mnoisebndrev}
    \frac{16}{3\sqrt{3}}\frac{4\pi r \alpha
      (\lambda-\mu)}{(1+(4\pi r\alpha(\lambda+\mu))^2)^2} = D
    \frac{x}{(1+C^2 x^2)^2},
  \end{equation}
  where
  \[
    D=\frac{16}{3\sqrt{3}}\delta,\,C=2+\delta.
  \]
  The positive stationary point of the quantity on the right-hand side
  of \ref{eqn:mnoisebndrev} is a maximum, and occurs at
  $x=1/(\sqrt{3}C)$, that is
  \[
    4 \pi r \alpha \mu = \frac{1}{\sqrt{3}(2+\delta)}.
  \]
  Thus
  \[
    1+C^2x^2 = \frac{4}{3},
  \]
  hence the maximum value is
  \[
    \frac{D3\sqrt{3}}{16C} = \frac{\delta}{2+\delta}.
  \]
  This maximum value must be larger than the left hand side of inequality
  \ref{eqn:mnoisebnd}, that is,
  \[
    \eta(1+\eta) \le \frac{\delta}{2+\delta},
  \]
  in order that there be any solutions at all, but the right hand side
  is less than $1$. This observation establishes the necessity of
  hypothesis \ref{eqn:mnoisecond} of the
  theorem. Solving this above inequality for $\delta$ and unwinding
  the definitions, one finds that the right-hand side of inequality
  \ref{eqn:mnoisebnd} is bounded by \ref{eqn:mnoisesuff}, so appeal to
  Proposition \ref{thm:mnoise} finishes the proof.
\end{proof}

\noindent {\bf Remark.} That is, {\em with the choice of penalty
  multiplier $a$ given in equation \ref{eqn:ann}, and support radius
  $\mu$ of the ``noise-free'' wavelet $w_*$, $\Ja$ has no local
  minima with slownesses further than $(1+O(\eta))\mu/r)$ from the slowness used to
  generate the data}.

\noindent {\bf Remark.} The estimate $|m-m_*|r<\mu(1+O(\eta))$ for
local minima of $\Ja$ is sharp: it is possible to choose $w_* \in W_{\mu}$
so that $\mu - |m-m_*|r$ is as small as you like. In particular,
the ``exact'' or ``true'' slowness $m_*$ is not necessarily the only
slowness component of a local minimizer, or even the slowness
component of any local minimizer, and in particular
is not (necessarily) the slowness component of a global minimizer of $\Ja$.

\noindent {\bf Remark.} No similar bound could hold for much larger
noise levels than specified in condition \ref{eqn:mnoisecond}, the
right-hand side of which is a bit larger than 0.6. For example, if the noise is
the predicted data for the same wavelet $w_*$ with a substantially different
slowness $m_{\flat}$, that is, $n=F[m_{\flat}]w_*$, then a simple
symmetry argument shows that if there is a local minimizer of $\Ja[\cdot,\cdot;d_*+n]$. with
slowness near $m_*$,
there must also be a minimizer with slowness near $m_{\flat}$.
so that the difference with $m_*$ is not constrained at
all by the assumed support radius of $w_*$. So for this example with 100\% noise, no
bound of the type given by conclusion 2 could possibly hold. The
companion paper \cite[]{SymesChenMinkoff:21}
illustrates this phenomenon numerically.

\noindent{\bf Remark.} Note that $\alpha$ plays no role in the
conclusions of this theorem. It is only required that $\alpha >0$.

\noindent {\bf Remark:} I emphasize that Theorem \ref{thm:mnoiseres} states {\em sufficient} conditions for a bound on
the slowness error $|m-m_*|$ in terms of the relative data noise level $\eta$,
giving an additional ``fudge factor'' beyond the support size $\mu$
of the noise-free wavelet $w_*$ for an interval within which the slowness error is
guaranteed to lie.

Conclusion 1 in Theorem \ref{thm:mnoiseres} constrains the range of
noise level to which these results apply to a bit more than 60\%. That
is, the bound given by conclusion 2 is useful only for small noise. In
the limit as $\eta \rightarrow 0$, conclusion 2 becomes
$\lambda/\mu \ge 1 + 2\eta$, that is, the ``fudge factor'' beyond
the noise-free bound is approximatly twice the noise level.

On the other hand, stronger bounds than given by Theorem
\ref{thm:mnoiseres} are possible, given additional constraints on the
noise $n$. A natural example is uniformly distributed random noise,
filtered to have the same spectrum as the source. The expression
\ref{eqn:dexpjgen} implies that the interaction of noise $n$ and
signal $d_*$ in the derivative of $\tJa$ is local, so that the
coefficient of $\eta$ on the left-hand side of inequality
\ref{eqn:mnoisebnd} is effectively much less that 1, resulting in a
larger range of allowable $\eta$. While I will not formulate such a
result, one of the numerical examples in the companion paper
\cite[]{SymesChenMinkoff:21} suggests its feasibility.

Unless the data is noise-free, there is no reason to suppose that the
estimated wavelet $\aw[m;d]$ (Theorem \ref{thm:norminv}) will lie in
$\lW$, unless the support of the noise $n$ is restricted. In order
to construct a solution of the inverse problem \ref{eqn:probstat0}, 
project $\aw[m;d]$ onto $\lW$. For sufficiently large $\lambda,
\epsilon$, the result is a solution of the inverse problem:

\begin{theorem}
  \label{thm:ipnoisesuf}
  Assume the hypotheses of Theorem \ref{thm:mnoiseres}, and that
  inequality \ref{eqn:mnoisecond} holds, and $\mu \in
  (0,\lambda_{\rm max}]$. Then the pair
  \[
    (m,{\bf 1}_{[-\lambda,\lambda]}\aw[m,d])
  \]
  solves the inverse problem as stated in \ref{eqn:probstat0} if
  \begin{equation}
    \label{eqn:ipnoiselam}
    \left( 2+\frac{2\eta(1+\eta)}{1-\eta(1+\eta)}\right)\mu \le \lambda
    \le \lambda_{\rm max}, 
  \end{equation}
  and
  \begin{equation}
    \label{eqn:ipnoiseeps}
    \epsilon \ge \frac{(8 \pi r \alpha\lambda)^2}{1 + (8 \pi r \alpha\lambda)^2}+\eta. 
  \end{equation}    
\end{theorem}

\begin{proof}:
From Theorem \ref{thm:norminv}, 
\[
  \aw[m;d](t) = (1+ (4 \pi r \alpha t)^2)^{-1}(w_*(t+(m-m_*)) + 4\pi r 
  n(t+mr)) 
\]
\[
  = \aw[m;d_*] + (1+ (4 \pi r \alpha t)^2)^{-1}4\pi r n(t+mr).
\]
From Theorem \ref{thm:mnoiseres},
\[
  |m-m_*| \le \left(1+\frac{2\eta(1+\eta)}{1-\eta(1+\eta)}\right)\frac{\mu}{r},
\]
which from assumption \ref{eqn:ipnoiselam} is
\[
  =\left(2+\frac{2\eta(1+\eta)}{1-\eta(1+\eta)}\right)\frac{\mu}{r}
  -\frac{\mu}{r} \le \left(\frac{\lambda}{\mu}-1\right)\frac{\mu}{r}.
\]
That is,
\[
  |m-m_*|r \le \lambda-\mu.
\]
From Theorem \ref{thm:norminv},
\[
  \mbox{supp }\aw[m;d_*] \subset [-\mu-(m-m_*)r, \mu-(m-m_*)r]
  \subset [-\lambda,\lambda],
\]
so
\[
  {\bf 1}_{[-\lambda,\lambda]}\aw[m;d](t) = \aw[m;d_*](t) + {\bf 1}_{[-\lambda,\lambda]}(1+ (4 \pi r \alpha
  t)^2)^{-1}4\pi r n(t+mr).
\]
From the definition of $F[m]$, for any $t_1<t_2$, $w \in W$,  
\[
  F[m]{\bf 1}_{[t_1.t_2]}w = {\bf 1}_{[t_1+mr,t_2+mr]}F[m]w.
\]
Thus the data residual after projection is
\[
  F[m]{\bf 1}_{[-\lambda,\lambda]}\aw[m,d](t) -d(t) = F[m]\aw[m,d_*](t) -d_*(t)  
\]
\[
  + {\bf 1}_{[-\lambda+mr,\lambda+mr]}(4 \pi r \alpha (t-mr))^2 (1+ (4 \pi
  r \alpha (t-mr))^2)^{-1} n(t)
\]
\[
  -(1- {\bf 1}_{[-\lambda+mr,\lambda+mr]})n(t).
\]
From \ref{eqn:residnorm} and the bound on $m-m_*$,
\[
  \| F[m]{\bf 1}_{[-\lambda,\lambda]}\aw[m,d_*] -d_*\|^2 = (4 \pi r \alpha)^4
  \int_{-\lambda+(m-m_*)r}^{\lambda+(m-m_*)r}\,dt\, (t-(m-m_*)r)^4
\]
\[
  \times (1+(4\pi r \alpha)^2 
  (t-(m-m_*)r)^2)^{-2}d_*(t)^2
\]
\[
  \le (4 \pi r \alpha \lambda)^4\|d_*\|^2.
\]
Similarly, the norm squared of the sum of the last two terms is 
\[
  \le (4 \pi r \alpha \lambda)^2 \|{\bf 1}_{[-\lambda+mr,\lambda+mr]}
  n\|^2 + \|(1 - {\bf 1}_{[-\lambda+mr,\lambda+mr]}n\|^2.
\]
Without additional hypotheses to outlaw the accumulation of $n$ near
$t=mr$, all that can be said is that this is
\[
  \le \max \{(4 \pi r \alpha \lambda)^2, 1\} \|n\|^2.
\]
Putting this all together,
\[
  \|F[m]{\bf 1}_{[-\lambda,\lambda]}\aw[m,d]-d\| \le (4 \pi r \alpha
  \lambda)^2\|d_*\| + \max \{4\pi r \alpha \lambda, 1\}\|n\|
\]
\[
  \le ((4 \pi r \alpha \lambda)^2 +\max \{4\pi r \alpha \lambda, 1\}
  \eta)\|d_*\|.
\]
If the right-hand side is to be less than $\|d_*\|$ as required by the
definition \ref{eqn:probstat0} of the inverse problem, then
necessarily $4\pi r\alpha \lambda < 1$, so the right hand side in the
preceding inequality is bounded by the right hand side of assumption
\ref{eqn:ipnoiseeps} of the theorem. Therefore this assumption implies
that the relative residual is $\le \epsilon$.
\end{proof}

\noindent {\bf Remark:} Note that the sufficient condition \ref{eqn:ipnoiselam} for
$\lambda$ is independent of $\alpha$. It follows that for any choice
of $\lambda$ consistent with this bound, $(m,{\bf
  1}_{[-\lambda,\lambda]}\aw[m,d])$ is a solution of the inverse
problem for any $\epsilon > 0$ provided that $\alpha$ is chosen
sufficiently small ($O(\sqrt{\epsilon})$).


\bibliographystyle{seg}
\bibliography{../../bib/masterref}<

%\title{Matched Source Waveform Inversion for Transmitted Waves}
\author{Huiyi Chen, Susan E. Minkoff, and William W. Symes}

\lefthead{Symes}

\righthead{MSWI}

\maketitle
\parskip 12pt

\begin{abstract}
Matched Source Waveform Inversion applied to acoustic transmission data
produces an estimate of refractive index similar to the result of
travel time inversion, but without explicit identification of travel
times. This paper reviews the theoretical justification of this result
and its limitations, and with 2D numerical illustrations. 
\end{abstract}
\setlength{\parindent}{0cm}

\section{Introduction}
Matched Source Waveform Inversion (MSWI) is a variant of Full Waveform
Inversion (FWI) \cite[]{VirieuxOperto:09}, that sometimes overcomes
one of FWI's impediments, namely its tendency to stagnate at
suboptimal model estimates (``cycle-skip''), demonstrably far from global optima in
controlled settings and often uninformative of material structure in
field use. It is possible that such local descent algorithms are
trapped in regions around local minima, though such trapping, or even
the existence of local, non-global minima, are seldom established with
any rigor. MSWI
loosens the bond between predicted and observed data by interposing a
filter, adapted to map one to the other trace-by-trace, and penalizes
deviation of the filter from the Dirac delta. This soft penalty
effectively relaxes the FWI data-fitting problem and permits local
optimization methods to approximate data kinematics.

The aim of this paper is to review the theoretical basis of MSWI,
describe a practical computational framework for MSWI, and use it to
solve a synthetic acoustic inversion problem for which FWI fails. In
contrast, MSWI starting at the same initial estimate delivers an
improved velocity from which FWI succeeds in approaching the global
minimizer. The example is representative of a case (single arrival or
simple wavefront transmitted wave data) in which the MSWI objective
function is demonstrably close to mean square travel time error. Since
travel time inversion is generally not subject to cycle-skipping,
successful inversion is expected, and that is what we observe.

MSWI in the form described here was introduced by
\cite{HuangSymes2015SEG,HuangSymes:Geo17}. It is mathematically
equivalent to an {\em extended} formulation of the inverse problem, in
which acoustic point sources are allowed to depend on the receiving
sensors - a non-physical expansion of the simulation domain. A number
of other variants of FWI have been based on essentially the same
extension of acoustic modeling
\cite[]{Song:94c,Symes:94c,Plessix:00,LuoSava:11,LiAlkhalifah:21}. In
particular it is the basis of Adaptive Waveform Inversion (AWI) \cite[]{Warner:16,
  GuaschWarnerRavaut:GEO19,Warneretal:SEG21, Guaschetal:NPJDM20}. MSWI
is closely related to AWI, but not identical: AWI includes a
normalization of the adaptive filter, which makes the AWI objective
function an even closer approximation to travel time mean square error
than is MSWI, at least for transmitted wave data with a single
arriving wavefront \cite[]{Symes:24a}.

Transmission data with a single arrival is not just a case in which
MSWI and AWI are known to closely approximate travel time inversion:
it is the only case in which this link occurs. In particular, the link
is broken for transmission data exhibiting multiple arrivals: in the
presence of complex wavefronts, methods based on the source-receiver
extension are no less likely than FWI to cycle-skip
\cite[]{Symes:94c,HuangSymes:Geo17,Symes:24a}. However, the
source-receiver extension is not the only possible route to FWI
modification by artificially expanding the definition of energy
source. \cite{HuangNammourSymesDollizal:SEG19} overview modifications
of FWI based on various source extensions; some more recent advances
are described by
\cite{MetivierBrossier:SEG20,PladysBrossierLiMetivier:GEO21,LiAlkhalifah:21,Yongetal:GJI23,Opertoetal:GEO23}.
Numerical examples suggest that some of these extensions may avoid
cycle-skipping for transmitted wave data with complex wavefronts.

The next section describes versions of FWI and MSWI based on
acoustic wave propagation. The theoretical connection between MSWI and
travel time inversion is reviewed. The computations required to apply local
optimization to the objective functions of these inversion methods are
detailed, as is the variable projection reduction of MSWI and the use
of weighted norms in the domain of the simulation operator (wave
velocity or bulk modulus fields, in the setting developed here). The
third section presents an example, along with a detailed description
of the numerical methods for simulation and optimization methods used
in treating this example. The final section discusses some of the many
unresolved questions around MSWI and similar approaches to inverse
problems in wave propagation.

\section{Theory}
The version of MSWI discussed here uses acoustic wave propagation with
isotropic point sources and receivers.
The pressure and velocity fields $p({\bf x},t;{\bf x}_s)$, ${\bf v}({\bf x},t;{\bf x}_s)$ for the source location ${\bf x}_s$ depend on the bulk modulus $\kappa({\bf x})$, buoyancy $\beta({\bf x})$ (reciprocal of the density $\rho({\bf x})$), and wavelet $w(t;{\bf x}_s)$ through the acoustic system
\begin{eqnarray}
  \label{eqn:awe}
 \frac{\partial p}{\partial t} & = &- \kappa \nabla \cdot {\bf v} +
w(t;{\bf x}_s) \delta({\bf x}-{\bf x}_s); \nonumber \\
\frac{\partial {\bf v}}{\partial t} & = & - \beta \nabla p; \\ 
p, {\bf v} & = & 0 \mbox{ for }  t \ll 0.
\end{eqnarray}
The model vectors $m=(\kappa,\rho)$ make up the domain of the forward
map or {\em modeling operator} is $F[m]w = \{p({\bf x}_r,t;{\bf
  x}_s)\}$, for specified source and receiver positions ${\bf x}_s, {\bf x}_r$ and
recording interval $[0,t_d]$.

In the discussion that follows, the buoyancy $\beta$ will be regarded
as a fixed parameter. Thus $m$ is effectively the bulk modulus field $\kappa$.

In this context, FWI means: given source wavelet
$w(t;\bx_s)$ and data traces $d(\bx_r,\cdot;\bx_s)$, find a model $m$
so that $F[m]w \approx d$. The
simplest version of FWI concretizes this task by asking for a model
$m$ minimizing the mean square error
\begin{equation}
  \label{eqn:fwi}
  J_{\rm FWI}[m;d]= \frac{1}{2}\|F[m]w-d\|^2.
\end{equation}
Note that $w$ is assumed known and treated as a parameter in this
statement of the FWI task. In practice, it is not known (nor are the
sources and receviers necessarily isotropic), and should be estimated
along with the model $m$.

The approach to local optimization taken here (and in most work on FWI
and related topics) is based on the gradient of the objective defined
in equiation \ref{eqn:fwi}:
\begin{equation}
  \label{eqn:fwigrad}
  g = \nabla  J_{\rm FWI}[m;d] = D_m(F[m]w)^T(F[m]w-d).
\end{equation}
In this formula, $D_m(F[m]w)$ is the derivative of $F[m]w$ with
respect to $m$. 
This is the Euclidean (or $L^2$) gradient, that is, the vector $g$ for
which the Euclidean inner product
\begin{equation}
  \label{eqn:eucip}
  \langle g, \delta m\rangle = g^T\delta m
\end{equation}
with any other vector $\delta m$ gives the
rate of change in the direction of that vector of $J_{\rm FWI}$ at $m$.

For optimization within a set of slowly varying models (on the
wavelength scale),  it is appropriate to penalize oscillation of the
search vector. A convenient way to accompish this goal is the use of a
{\em weighted inner product} to define the gradient, rather than the
Euclidean inner product. A weight operator $W$ should be symmetric and
positive definite: then
\begin{equation}
  \label{eqn:wip}
  \langle g, \delta m\rangle_W = g^TW\delta m
\end{equation}
defines an alternative inner product. Comparing the definitions
\ref{eqn:fwigrad}, \ref{eqn:eucip}, and \ref{eqn:wip}, clearly the
vector $g_W$ for which $\langle g_W, \delta m \rangle_W$ gives the
rate of change of $J_{\rm FWI}$ at $m$ in the direction $\delta m$ is
\begin{equation}
  \label{eqn:fwiwgrad}
  g_W = W^{-1}\nabla  J_{\rm FWI}[m;d] =W^{-1} D_m(F[m]w)^T(F[m]w-d).
\end{equation}

If $W$ is chosen to greatly
amplify oscillatory components of the vector to which it is applied,
then those components of $g_W$ must be suppressed relative to the
corresponding components of $g$, hence $g_W$ represents a
non-oscillatory search direction. Note that only the inverse operator
$W^{-1}$ appears in the formula \ref{eqn:fwiwgrad}.

As mentioned earlier, application of local optimization methods
directly to $J_{\rm FWI}$ tends to produce unsatisfactory model
estimates. MSWI modifies the measure of distance between predicted and
observed data by inserting an adaptive filter field $u$, consisting of
one filter per trace. Since only finite time intervals of $u$ and $d$ are available in
practice, introduce the {\em truncated filter operator} $K[u]$. This
operator acts by extending its filter $u$ (given on a symmetric
interval $[-t_u,t_u]$ for each source-receiver pair) and the function to which
it is applied (given on the data interval $[0,t_d]$ for each
source-receiver pair) to be zero outside their domains of definition, convolving the
resulting functions on $\bR$, and finally restricting or truncating
the result to the time interval of the input function, that is,
$[0,t_d]$. This operator is applied to the predicted data $F[m]w$ to
produce the filtered predicted data $K[u]F[m]w$.

Note that the filtered predicted data may be viewed as the predicted
data for an {\em extended source}, with the source wavelet at location
$\bx_s$ replaced by $u(\bx_s,\bx_r,\cdot) * w(\bx_s,\cdot)$. That is,
the adaptive filter construction is equivalent to allowing the source
to depend on receiver position as well as source position - one source
wavelet for each source-receiver pair. This is an extension of
standard modeling, in that the domain (bulk modulus buoyancy, wavelet)
is larger than in the conventional formulation, and coincides with it
under the special condition that the source wavelet is independent of
the receiver location. This  {\em
  source-receiver extension} \cite[]{HuangSymes2015SEG} is a key ingredient in a number of
other papers on modifications of FWI, as mentioned in the Introduction.

It is possible to make the error between filtered predicted data and
observed data as small as one likes by choosing an appropriate filter
field $u$, so this error by itself is useless for estimating the
model. If $u(\bx_s,\bx_r,t)=\delta(t)$, on the other hand the filtered
predicted data is identical to the predicted data.
Therefore, some penalty for divergence of the filter $u$ from
$\delta(t)$ needs to supplement the filtered prediction error.. The works referenced in the Introduction mostly use the
mean-square of the filter scaled by $t$, and add it to the mean-square
of the filtered prediction error. This sum is the MSWI objective function:
\begin{equation}
  \label{eqn:filtpen}
  J_{\alpha,\sigma}[m,u;d]=\frac{1}{2}(\|K[u]F[m]w-d\|^2
  +\alpha^2\|tu\|^2 + \sigma^2\|u\|^2).
\end{equation}
As with all penalty methods, this definition involves a choice of
weight ($\alpha$). A second weight ($\sigma$) scales the norm-squared
of the filter, that is, a Tihonov regularization term, added for
technical reasons explained by \cite{Warner:16,Symes:24a}. Choice of
parameters is a critical step in the use of any penalty method. We
offer some remarks about this choice in the Discussion section.

The domain space (pairs of bulk modulus fields and adaptive filters)
is very high-dimensional, compared with the domain space for FWI (bulk
modulus fields alone, in the present context). It is possible to
optimize $J_{\alpha,\sigma}[m,u;d]$ by alternating updates of $m$ and
$u$  \cite[]{LiAlkhalifah:21}. However the {\em variable projection} \cite[]{GolubPereyra:73,GolubPereyra:03}
reduction is generally much more efficient than the altrnating, or
coordinate search, approach. In this instance, variable projection
consists in minimizing $J_{\alpha,\sigma}[m,u;d]$ over $u$ (a
quadratic optimization) to produce an optimal choice
$u_{\alpha,\sigma}[m:d]$. The reduced objective is
\begin{equation}
  \label{eqn:redfiltpen}
  \tilde{J}_{\alpha,\sigma}[m;d]=\frac{1}{2}(\|K[u_{\alpha,\sigma}[m;d]]F[m]w-d\|^2
  +\alpha^2\|tu_{\alpha,\sigma}[m;d]\|^2 + \sigma^2\|u_{\alpha,\sigma}[m;d]\|^2).
\end{equation}
Note that like the FWI objective, $\tilde{J}_{\alpha,\sigma}[m;d]$
depends only on $m$, with $d$ as a parameter.

\cite{Symes:24a} explains the relation between the reduced MSWI
objective and travel time inversion. This relation follows from the
geometric asymptotics approximation to solutions of the point radiator
problem \ref{eqn:awe} \cite[]{Friedlander:75}. Presuming that the
coefficients $\kappa, \beta$ are smooth, there are smooth functions
$\tau[m](\bx_s,\bx)$ (travel time) and $a[m](\bx_s,\bx)$ (geometric
amplitude) depending on the model $m$, so that
\begin{equation}
  \label{eqn:pwa}
  p(\bx,t;\bx_s) = a[m](\bx,\bx_s)w(t-\tau[m](\bx,\bx_s),\bx_s) + ...
\end{equation}
The elided terms are smoother than $w$, and become small with the
dominant wavelength in $w$ (\cite{Symes:24a} explains the precise
meaning of this condition). Thus we can write
\[
  F[m]w(\bx_r,t;\bx_s) \approx
  a[m](\bx_r,\bx_s)w(t-\tau[m](\bx_r,\bx_s))
\]
Supposing that the data $d$ is noise-free,
\[
  d(\bx_r,t;\bx_s) = F[m^*]w(\bx_r,t;\bx_s) \approx
  a[m^*](\bx_r,\bx_s)w(t-\tau[m^*](\bx_r,\bx_s))
\]
it follows that
\begin{equation}
  \label{eqn:tomo}
  \lim_{\alpha \rightarrow 0}\frac{1}{\alpha^2} (\tilde{J}_{\alpha,\sigma}[m;d]
  -\tilde{J}_{0,\sigma}) \approx \sum_{\bx_s,\bx_r} \frac{a[m^*]^2}{a[m]^2} (\tau[m^*]-\tau[m])^2\|g_{\frac{a[m^*]}{a[m]}\sigma}\|^2
\end{equation}
Here $g_{\sigma}$ is an approximate delta. Its Fourier transform of
is
\begin{equation}
  \label{eqn:gsig}
  \hat{g}_{\sigma} = \frac{a[m^*]^2 |\hat{w}|^2}{a[m^*]^2 |\hat{w}|^2
    + \sigma^2}
\end{equation}
whence $g_{\sigma}(\bx_s,\bx_r,t) \rightarrow \delta(t)$ in
the sense of distributions as $\sigma \rightarrow 0$.

If $\alpha$ and $\sigma$ are sufficiently small, then
$\tilde{J}_{\alpha,\sigma}$ is close to the right-hand side, which is bounded above and below by multiples of the the mean square travel time error between $m$ and
$m^*$.  This is the relation mentioned in the introduction. 
It does not
show that the only stationary points are those of the mean square
travel time error, even approximately. However if the amplitudes are
relatively insensive to changes in model, as is true if the
source-receiver distance is well away from developing multiple
arrivals, then any stationary point of $\tilde{J}_{\alpha,\sigma}$ is
either close to a tomographic stationary point, or far from $m^*$. So we
would expect minimization $\tilde{J}_{\alpha,\sigma}$ to produce a
model that closely matches the travel times inherent in the data
($\tau[m^*]$ in the notation used here).

AWI adds one more feature, namely scaling by the trace norm of the
filter. This normalization can be interpreted as a choice of weighted
norm on the space of adaptive filters. It leads to a very similar
relation to \ref{eqn:tomo} for the AWI penalty function, but without
the amplitude factors: that is, the right-hand side of the AWI
analogue of \ref{eqn:tomo}
is {\em just} the mean-square traveltime error, up to controllable
error, for single-arrival transmission data. See \cite{Symes:24a}
for details.

\cite{Symes:24a} also shows that if energetic multiple arrivals are
present in the data, for any reason, then cross-talk between travel
time branches destroys the relation between MSWI (or AWI) and any
version of the mean-square travel time error. The next sections of
this paper will illustrate the
success of MSWI for single-arrival data, and its failure for data with
multiple arrivals.

We end this section by overviewing the computations required for
minimization of $\tilde{J}_{\alpha,\sigma}$.

The reduced adaptive filter $u_{\alpha,\sigma}[m;d]$ is the solution
of the {\em normal equation}
\begin{equation}
  \label{eqn:normal}
  (S[m]^TS[m] + \alpha^2 t^2 + \sigma^2 I)u = S[m]^Td,
\end{equation}
in which $S[m]u = K[u]F[m]w$. This positive definite symmetric linear
system may be solved by various efficient numerical methods. Having
computed $u_{\alpha,\sigma}[m;d]$ hence the value of 
$\tilde{J}_{\alpha,\sigma}[m;d]$, its (Euclidean) gradient is given by
\begin{equation}
  \label{eqn:gradredfiltpen}
  \nabla \tilde{J}_{\alpha,\sigma}[m;d] =
  D_m(F[m]w)^TK[u_{\alpha,\sigma}[m;d]]^T(K[u_{\alpha,\sigma}[m;d]]F[m]w-d)
\end{equation}
Apart from the appearance of the truncated filter operator
$K[u_{\alpha,\sigma}[m;d]]$, this is almost identical to the FWI
gradient \ref{eqn:fwigrad}. In particular, the last step in the
computation on the right hand side, the application of the adjoint
$m$-derivative of F, is exactly the same.

Use of a weighted norm in the model space goes exactly as before: with
weight operator $W$, the weighted gradient is
\begin{equation}
  \label{eqn:wgradredfiltpen}
  \nabla_W \tilde{J}_{\alpha,\sigma}[m;d] =
  W^{-1}D_m(F[m]w)^TK[u_{\alpha,\sigma}[m;d]]^T(K[u_{\alpha,\sigma}[m;d]]F[m]w-d)
\end{equation}

The gradient (or weighted gradient), together with the value, are the
inputs to first-order methods such as steepest descent and
Limited-Memory Broyden-Fletcher-Goldfarb-Shanno
iterations. Methods more closely related to Newton iteration require
more involved computations (see for instance \cite{Kaufman:75}. 

\section{Numerical Illustration}

\inputdir{project}

This section presents application of MSWI to simulated acoustic data with dimensions
typical of crustal seismic exploration. We use several configurations
to illustrate the capabilities and limitations of this approach. In
all cases, we use MSWI to generate an initial bulk modulus estimate
input to FWI.

\subsection{Experimental setup}

The several experiments to be described below share a number of
features, set out in this section. 

Acoustic wave
propagation is simulated via a staggered grid finite difference method
\cite[]{vir86,lev88,Cohen:01} of order 2 in time and 8 in space. The time
step is an internal detail of the simulation, chosen to be safely
stable, given the parameter fields, difference formulae, and 
spatial sampling. The discrete pressure field is
output over the time range $0 < t < 5$ s with sample interval $0.008$
s at externally specified receiver locations
via piecewise linear interpolation in space and cubic spline
interpolation in time. Isotropic point sources are added
into the acoustic fields at each time step via the adjoints of these
interpolation operations. These and other details of the simulator are
discussed in \cite{GeoPros:11}.

Spatial fields (bulk modulus, buoyancy, gradients) occupy a
rectangular region of 8 km horizontal (``x'') extent and 4 km vertical
(``z'') extent, sampled on a 20 $\times$ 20 m grid. For all except the
third example, receivers are located on the vertical line at
horizontal coordinate $x$= 5000 m, spaced 20 m apart, starting at
depth $z$ = 200 m. Twenty isotropic point sources occupy positions lie
on the line $x=3000$ m, spaced 150 m apart, starting at depth = 500
m. For the third example, source and receiver lines are moved 2000 m
further apart, to $x=2000$ m and $x=6000$ m respectively, other
parameters remaining the same. [Note that this geometry is a 2D
cartoon of that used in cross-well seismic data acquisition.] The
source wavelet $w$ (the same for every source ) is a
$[1.0, 2.5, 7.5, 12.5]$ Hz trapezoidal bandpass filter, with a median
frequency of 5.875 Hz corresponding to a median wavelength of
$\approx$ 340 m, and shortest wavelength of 160 m. It is centered at
$t=1$ s.

The initial model $m_0$ for MSWI is homogeneous, with $\kappa = $ 4
GPa, $\beta = $ 1 cm$^3$/g. As noted earlier, we regard $\beta$ as
fixed in this series of experiments, so mention of it will be
suppressed. The corresponding data for isotropic point sources at $x=3000$ m,
point receivers at $x=5000$ m, and the bandpass filter time dependence
described earlier, is depicted in
Figure \ref{fig:chwd20}. We use both $m_0$ and the output of MSWI as
initial models for FWI..

\plot{chwd20}{width=\textwidth}{Data for homogeneous model $m_0$: 20
  sources at $x=3000$ m, 201 receivers at $x=5000$ m.}

We apply a version of weighted steepest descent optimization to the
minimization of both $J_{\rm FWI}$ and $\tilde{J}_{\alpha,\sigma}$.
The inverse weight operator ($W^{-1}$
in formula \ref{eqn:fwiwgrad}) is a
10-point moving average in both spatial directions, repeated once,
except in the fourth example as discussed below. Recall that the weight operator itself is not required. The weighted
gradient is computed via formula \ref{eqn:fwigrad}. The adjoint
derivative $DF[m]^T$ is computed via the {\em adjoint state method}
\cite[]{Chavent:74,GauTarVir:86}, with time reversal of the acoustic fields
implemented via optimal checkpointing
\cite[]{Griewank:92,Griewank:book,Symes:06a-pub}. The negative of the weighted
gradient ($g_W$, formula \ref{eqn:fwiwgrad}) is the search direction,
and the optimal step in this direction is approximated by a simple
backtracking line search algorithm \cite[]{NocedalWright}. The
iteration is terminate either when the gradient norm has fallen below
1\% of its initial value, or when a maximum number of iterations
(usually 12) is reached.

The MSWI objective $\tilde{J}_{\alpha,\sigma}$ requires choices of the
parameters $\alpha$ and $\sigma$, and solution of the normal equation
\ref{eqn:normal}. This system is far larger than is convenient to solve
by any variant of Gaussian Elimination, even for our small 2D
examples, so iterative methods are 
necessary. We choose the Conjugate Gradient (CG.) method
\cite[]{Dan:67,Steihaug:83,NocedalWright}, and stop the
iteration by monitoring the reduction in length of the normal residual
(the gradient of $J_{\alpha,\sigma}[m,u;d]$ with respect to $u$). The
reduction threshhold $\rho$ is thus another necessary input
parameter. For all calculations in this paper, we use
we choose $\rho = 0.01$ (note that unlike $\alpha$ and $\sigma$,
$\rho$ is dimensionless).

Choice of regularization weights like $\alpha$ and $\sigma$ is a
widely studied topic; we mention some methods for this task in the
discussion section. For the set of examples presented here, we take a
simpler approach. Since the purpose of $\sigma>0$ is to avoid singularity
in the system \ref{eqn:normal} even for $\alpha=0$, we choose it a bit
bigger than floating point tolerance for single precision:
$\sigma = 10^{-5}$. We choose $\alpha$ by computing the solution
$u_{\alpha,\sigma}[m_0;d]$ for the data $d$ used in the first example
below, and several powers of 10 for $\alpha$. The largest such
$\alpha$ that yields an RMS predicted data error
($\|S[m_0]u_{\alpha,\sigma}[m_0;d] - d\|$) less than $0.05 \|d\|$ is
$\alpha = 10^{-4}$. We take this relation as representative, and use
this value for all MSWI computations.

\subsection{Recovering a circular lens}

The target bulk modulus field ($\kappa$) for the first example 
is depicted in Figure \ref{fig:m}. It contains an acoustic ``lens''
positioned in the center between 1000 m and 3000 m depth
(``$z$''). The background level outside the lens is 4 GPa; at the
center it is 2.4 GPa. The
buoyancy ($\beta$)is spatially homogeneous at 1 cm$^3/g$, so the background
wave speed is 2000 m/s.

\plot{m}{width=\textwidth}{Lens model.}

Simulated data from this configuration appears as Figure
\ref{fig:cwd20}.

\plot{cwd20}{width=\textwidth}{Data for circular lens model (\ref{fig:m}): 20
  sources at $x=3000$ m, 201 receivers at $x=5000$ m.}

The value of $J_{\rm FWI}[m_0,d] \approx 4.6$, and the rate of
increase in the weighted gradient direction at $m_0$ is 4.2 $\times
10^{-6 }$. Note that the rate of increase is the same as the weighted
gradient norm. After 12 steepest descent steps, the objective value has
decreased to 2.6, and the
weighted gradient norm by more than an order of magnitude, to $\approx 2.5 \times 10^{-7}$. The
final model appears in Figure \ref{fig:cwlens20mestfwi0}, and the
corresponding data in Figure \ref{fig:cwlens20resimmestfwi0}.

%\multiplot{2}{mestfwi0,resimfwi0}{width=0.45\textwidth}{a: Bulk
%  modulus produced by 10 steepest descent steps to minimize the FWI
%  objective $J_{\rm FWI}[\cdot;d]$ with data $d$ depicted in Figure
%  \ref{fig:sim11}, starting with homogeneous model. b: Data simulated
%  from FWI inversion result.}

\plot{cwlens20mestfwi0}{width=\textwidth}{a: Bulk modulus produced by 10
  steepest descent steps to minimize the FWI objective \ref{eqn:fwi},
  using the data shown in \ref{fig:cwd20}.}

\multiplot{2}{cwlens20resimmestfwi0,cwlens20residmestfwi0}{width=0.45\textwidth}{a: Data
  corresponding to FWI inversion result shown in Figure
  \ref{fig:cwlens20mestfwi0}. b: Residual or data error - difference between
  data shown in Figure \ref{fig:cwlens20resimmestfwi0} and target data
  \ref{fig:cwd20}. Note the failure to match arrival times of the later signal in the
  central part of the display.}

While the reduction in the (weighted) gradient norm indicates progress
towards a stationary point, it is not possible to claim that the final
estimate is in the vicinity of a local minimizer. The fit to data
obtained is not at all satisfactory. More discussion of this result
will be found below.

This example is the one on which we based the choice $\alpha =
10^{-4}$. We did this by minimizing $J_{\alpha,\sigma}[m_0, u; d]$
over $u$ to identify a filter that outputs a good fit to the data
\ref{fig:cwd20} from the input \ref{fig:chwd20} (simulated from
$m_0$), and choosing $\alpha$ just small enough to obtain an RMS error
under 5\%.  The resulting adaptive filter $u_{\alpha,\sigma}[m_0;d]$
is shown in Figure \ref{fig:cwlens20uest0}. This filter exhibits a lot
of energy at non-zero times, as is necessary to move the events in
\ref{fig:chwd20} near those in \ref{fig:cwd20}. Thus the penalty term
is rather large. We will use the apparent dispersion of energy away
from $t=0$ in these adaptive filters to judge the quality of
inversions.

\plot{cwlens20uest0}{width=\textwidth}{Adaptive filter to match
  initial model data (Figure \ref{fig:chwd20}) to circular lens
  data (Figure \ref{fig:cwd20}). Note considerable energy dispersion
  away from $t=0$.}

We applied the same steepest descent algorithm to the MSWI objective
$J_{\alpha,\sigma}$. The initial bulk modulus field is once again
homogeneous at 4 GPa. The initial value of the reduced MSWI objective
is $\approx 1.49 \times 10^{-2}$, and the length of the (weighted)
gradient is $\approx 4.7 \times 10^{-10}$. After 12 iterations, the value has
decreased to $\approx 3.25 \times 10^{-3}$, or roughly a factor of 4.5.
reduction. The gradient length is
$\approx 5.6 \times 10^{-12}$, a reduction of almost two orders of
magnitude. The resulting bulk modulus is shown in Figure
\ref{fig:cwlens20mestmswi}.

\plot{cwlens20mestmswi}{width=\textwidth}{a: Bulk modulus produced by 12
  steepest descent steps to minimize the reduced MSWI objective \ref{eqn:redfiltpen},
  using the data shown in \ref{fig:cwd20}.}

The final adaptive filter is shown in Figure
\ref{fig:cwlens20uestmswi}. Note that energy in the filter has
migrated towards $t=0$, compared to the adaptive filter produced from
the homogenous bulk modulus \ref{fig:cwlens20uest0}, and the bulk of
it is within an apparent half-wavelength of $t=0$. This suggests that
the MSWI result may be an adequate initial estimate for a successful
FWI.

\plot{cwlens20uestmswi}{width=\textwidth}{Adaptive filter to match
  data for MSWI model (Figure \ref{fig:cwlens20resimmestmswi}) to circular lens
  data (Figure \ref{fig:cwd20}). Note considerable less energy dispersion
  away from $t=0$ than is evident in Figure \ref{fig:cwlens20uest0}.}

Another way to look at the improved data kinematics of the MSWI result
is via the simulate data from the model shown in
\ref{fig:cwlens20mestmswi}. Figures \ref{fig:cwlens20resimmestmswi},
\ref{fig:cwlens20residmestmswi} suggest that the
data predicted from the MSWI result is in many places within a
half-wavelength travel time shift of the data in Figure \ref{fig:cwd20}.
A half-wavelength is the conventional threshhold for FWI initial fit
error: below that it works, above that it doesn't.

\multiplot{2}{cwlens20resimmestmswi,cwlens20residmestmswi}{width=0.45\textwidth}{a:
  Data corresponding to MSWI result shown in Figure
  \ref{fig:cwlens20mestmswi}. b: Residual or data error - difference between
  data shown in Figure \ref{fig:cwlens20resimmestmswi} and target data
  \ref{fig:cwd20}. Note that the later signal in the central portion is
  still shifted, but by much less than in Figure \ref{fig:cwlens20residmestfwi0}.}

We applied 11 iterations of steepest descent to the FWI objective,
with initial estimate of $m$ equal to the final estimate from the MSWI
inversion just described (Figure \ref{fig:cwlens20mestmswi}). The initial FWI
objective value was $\approx 2.9$, and the weighted gradient norm was
$\approx 2.0 \times 10^{-4}$.  After 11 steps of steepest descent,
the prescribed reduction ($10^{-2}$) of the weighted gradient norm was achieved,
and the objective value had decreased to $\approx 9.1 \times
10^{-2}$. The resulting bulk modulus field is depicted in Figure
\ref{fig:cwlens20mestmswifwi}. The predicted data at this model, and the residual
data, are shown in Figures \ref{fig:cwlens20resimmestmswifwi} and \ref{fig:cwlens20residmestmswifwi}
respectively.

\plot{cwlens20mestmswifwi}{width=\textwidth}{a: Bulk modulus produced by 11
  steepest descent steps to minimize the FWI objective \ref{eqn:fwi},
  using the data shown in \ref{fig:cwd20}, and starting at the MSWI
  result shown in Figure \ref{fig:cwlens20mestmswi}. Reduction in
  gradient norm of 2 orders of magnitude.}

\multiplot{2}{cwlens20resimmestmswifwi,cwlens20residmestmswifwi}{width=0.45\textwidth}{a: Data
  corresponding to FWI result shown in Figure
  \ref{fig:cwlens20mestmswifwi}. b: Residual or data error - difference between
  data shown in Figure \ref{fig:cwlens20resimmestmswifwi} and target data
  \ref{fig:cwd20}.}

Yet another way to examine the kinematic fit between the target data
and the resimulated data (Figure \ref{fig:cwlens20resimmestmswifwi})
is via the adaptive filter computed with the FWI + MSWI model (Figure
\ref{fig:cwlens20mestmswifwi}).  Figure \ref{fig:cwlens20uest1}
shows how precisely the kinematics of the resimulation match those of
the target data.

\plot{cwlens20uest1}{width=\textwidth}{Adaptive filter to match
  data for FWI + MSWI model (Figure \ref{fig:cwlens20resimmestmswi}) to circular lens
  data (Figure \ref{fig:cwd20}). Dispersion away from 
  away from $t=0$ has almost disappeared, showing that the resimulated
  data is nearly a perfect kinematic match with the target data.}

The FWI iteration starting at the MSWI result has effectively achieved
the global minimum of the FWI objective function: the data match is
very close. The combination of MSWI followed by FWI has fit the data
starting from the homogeneous initial model, a result that cannot be
reached by FWI alone.

\subsection{Oblate lens: single vs. multiple arrivals}

Our next two examples use another lens model, somewhat deeper (with a
minimum bulk modulus of 2 GPa in the center), therefore more
refractive, and extended in the
horizontal direction. See Figure \ref{fig:cwm}. As is the case with
the circular lens, this oblate lens is smooth to adequate numerical
approximation.

\plot{cwm}{width=\textwidth}{Oblate lens model.}

The first example using this model has exactly the same acquisition
geometry as was used in the circular lens example: 20 sources at x =
3000 m, 201 receivers at x = 5000 m, spaced as described in the
previous section. The data is shown in Figure \ref{fig:cowd20}. 

\plot{cowd20}{width=\textwidth}{Data using model shown in Figure
  \ref{fig:cwm}, same source-receiver geometry as for circular lens
  data (Figure \ref{fig:cwd20}).}

The stronger refraction produced by this model produces a near-focus in
the center of the source-receiver line. In fact, the ray field is
triplicated, but the smearing effect of finite bandwidth obscures this
detail, and the data can be (just!) regarded as exhibiting a single
arrival on an identifiable wavefront.

In this and subsequent examples, FWI initiated from the homogeneous
model fails in the same way as shown in the  previous section. So
there is no point in showing the results, and we don't.

Application of steepest descent to the MSWI objective (with the same
parameters as in the previoius example) results in the model depicted
in Figure \ref{fig:nqcaustic20mestmswi}, and reduces the objective by
almost a factor of 7 (from 2.76 $\times 10^{-2}$ to 4.1 $\times
10^{-3}$ and the gradient norm by two full orders of magnitude. Data
simulated with this model is shown in Figure
\ref{fig:nqcaustic20resimmestmswi}, and the residual in Figure
\ref{fig:nqcaustic20residmestmswi}.

\plot{nqcaustic20mestmswi}{width=\textwidth}{Result of 8 MSWI
  iterations applied to the data in Figure \ref{fig:cowd20}, beginning
  at the homogeneous model.}

\multiplot{2}{nqcaustic20resimmestmswi,nqcaustic20residmestmswi}{width=0.45\textwidth}{a: Data
  corresponding to MSWI result shown in Figure
  \ref{fig:nqcaustic20mestmswi}. b: Residual or data error - difference between
  data shown in Figure \ref{fig:nqcaustic20resimmestmswi} and target data
  \ref{fig:cowd20}.}


The change in adaptive filters between the homogenous initial model
and the MSWI inversion is shown in Figures
\ref{fig:nqcaustic20uest0} and \ref{fig:nqcaustic20uestmswi}. The
reduction in energy dispersion is evident.

\multiplot{2}{nqcaustic20uest0,nqcaustic20uestmswi}{width=0.45\textwidth}{a:
  adaptive filter estimate for homogeneous initial model, data of
  Figure \ref{fig:cowd20}; b: adaptive filter estimate for MSWI
  inversion shown in Figure \ref{fig:nqcaustic20mestmswi}.}

These results render the MSWI inversion result a plausible initial
estimate for FWI. Twelve FWI iterations applied to the data in Figure \ref{fig:cowd20}, starting at the model of
Figure \ref{fig:nqcaustic20mestmswi}, result in the
model shown in Figure \ref{fig:nqcaustic20mestmswifwi}, at which the
FWI objective has decreased from $\approx 5.5$ to $\approx 0.13$, that
is, by more than 97\%. The gradient decreases by approximately the
same factor. This is evident in the plots of simulated data and
residual (Figures \ref{fig:nqcaustic20resimmestmswifwi} and
\ref{fig:nqcaustic20residmestmswifwi}). Finally, energy dispresion
from $t=0$ has nearly disappeared from the adaptive filter computed
from this FWI +MSWI result (Figure \ref{fig:nqcaustic20uest1}).

\plot{nqcaustic20mestmswifwi}{width=\textwidth}{a: Bulk modulus produced by 12
  steepest descent steps to minimize the FWI objective \ref{eqn:fwi},
  using the data shown in \ref{fig:cowd20}, and starting at the MSWI
  result shown in Figure \ref{fig:nqcaustic20mestmswi}.}

\multiplot{2}{nqcaustic20resimmestmswifwi,nqcaustic20residmestmswifwi}{width=0.45\textwidth}{a: Data
  corresponding to FWI + MSWI result shown in Figure
  \ref{fig:nqcaustic20mestmswifwi}. b: Residual or data error - difference between
  data shown in Figure \ref{fig:nqcaustic20resimmestmswifwi} and target data
  \ref{fig:cowd20}.}

\plot{nqcaustic20uest1}{width=\textwidth}{Adaptive filter to match
  data for FWI + MSWI model (Figure \ref{fig:nqcaustic20resimmestmswi}) to oblate lens
  data (Figure \ref{fig:cowd20}). Dispersion away from 
  away from $t=0$ has almost disappeared, showing that the resimulated
  data is nearly a perfect kinematic match with the target data.}

The second example based on the target model in Figure \ref{fig:cwm}
differs from the first {\em only} in that the sources and receivers
have been moved further away from each other: the sources lie on the
line $x=2000$, the receivers on $x=6000$. That is, the source-receiver
offset has been doubled. Otherwise, all aspects of data generation are
the same, and produce the data shown in Figure \ref{fig:fcwd20}.

\plot{fcwd20}{width=\textwidth}{Data generated using the same inputs
  as in Figure \ref{fig:cowd20}, but with sources moved to $x=2000$
  and receivers to $x=6000$.}

Notice that the additional offset has allowed distinct energetic later
arrivals to appear in the data.

As before, we compute the adaptive filter required to produce the data
of Figure \ref{fig:fcwd20} from the corresponding homogeneous medium data (Figure
\ref{fig:chwd20}). This filter, displayed in \ref{fig:caustic20uest0},
shows that for most traces, two or more energy peaks appear in the
adaptive filter, forced by the need to match multiple energy peaks in
the target data with the single energy peak in the homogeneous medium
data.

Using the same parameters (tolerance of 0.01 for the normal residual
in the CG iteration, 12 steps of steepest descent) produces the model
estimate depicted in Figure \ref{fig:caustic20mestmswi}. The overall
shape suggests decrease in bulk modulus in more or less the right
place, but the minimum bulk modulus attained (3.15 GPa) is considerably larger
than the target's, suggesting that the amount of refraction is
considerably less. That guess is verified by resimulating the data
(Figure \ref{fig:caustic20resimmestmswi}): the predicted data does not
exhibit triplication. The computed adaptive filter (Figure 
\ref{fig:caustic20uestmswi}) appears to attempt a compromise between
earlier and later arrivals. Most importantly, the final gradient norm
is approximately 5\% of the initial, suggesting that the minimizer has
not been approximated as well as was the case in the previous
examples. FWI initiated at the MSWI estimate gives an evidently
cycle-skipped model estimate (Figure \ref{fig:caustic20mestmswifwi}),
with unsatisfactory resimulation and residual (Figures
\ref{fig:caustic20resimmestmswifwi},
\ref{fig:caustic20residmestmswifwi}), and considerable energy spread
ion the estimated adaptive filter (Figure \ref{fig:caustic20uest1}.

\plot{caustic20uest0}{width=\textwidth}{Adaptive filter required to produce the data
of Figure \ref{fig:fcwd20} from corresponding homogeneous medium data (Figure
\ref{fig:chwd20}).}


\plot{caustic20mestmswi}{width=\textwidth}{Result of 12 MSWI
  iterations applied to the data in Figure \ref{fig:fcwd20}, beginning
  at the homogeneous model.}

\multiplot{2}{caustic20resimmestmswi,caustic20residmestmswi}{width=0.45\textwidth}{a: Data
  corresponding to MSWI result shown in Figure
  \ref{fig:caustic20mestmswi}. b: Residual or data error - difference between
  data shown in Figure \ref{fig:caustic20resimmestmswi} and target data
  \ref{fig:fcwd20}.}

\plot{caustic20uestmswi}{width=\textwidth}{Adaptive filter estimate for MSWI
  inversion shown in Figure \ref{fig:caustic20mestmswi}.}

\plot{caustic20mestmswifwi}{width=\textwidth}{Bulk modulus produced by 12
  steepest descent steps to minimize the FWI objective \ref{eqn:fwi},
  using the data shown in \ref{fig:fcwd20}, and starting at the MSWI
  result shown in Figure \ref{fig:caustic20mestmswi}.}

\multiplot{2}{caustic20resimmestmswifwi,caustic20residmestmswifwi}{width=0.45\textwidth}{a: Data
  corresponding to FWI + MSWI result shown in Figure
  \ref{fig:caustic20mestmswifwi}. b: Residual or data error - difference between
  data shown in Figure \ref{fig:caustic20resimmestmswifwi} and target data
  \ref{fig:fcwd20}.}

\plot{caustic20uest1}{width=\textwidth}{Adaptive filter to match
  data for FWI + MSWI model (Figure \ref{fig:caustic20resimmestmswi}) to oblate lens
  data (Figure \ref{fig:fcwd20}). Considerable energy dispersion away
  from $t=0$ remains, showing that the inverted model does not match
  data kinematics well.}

As noted above, the reduction in gradient norm achieved in 12 steepest
descent iterations for this example is not as great as was the case in
the two previous exmaples: this appears to be a more difficult
optimization. Accordingly, we perform 12 additional iterations of
steepest descent. The final gradient norm is $\approx 1.1 \times
10^{-11}$, an order of magnitude less than the final norm achieved in
the first 12 iterations and well over two orders of magnitude less
than the initial gradient norm of $\approx 5.24 \times 10^{-9}$. The
depression in the center of the estimated bulk modulus has deepened
(Figure \ref{fig:caustic20mestmswicont}) (2.53 GPa, vs 3.15 GPa at the
end of the first 12 iterations). While it is stiil insufficiently
refractive to produce triplication, the resulting resimulation has
increased in magnitude for the central tracess, indicating approach to a focus, and delayed by
roughly a half-wavelength compared to the corresponding result at the
end of the first MSWI iteration. As the remaining figures show, FWI
starting from this result is much more satisfactory. The latest
arrival for the central traces still does not have quite correct
amplitude, but all data phases are correct.



\plot{caustic20mestmswicont}{width=\textwidth}{Result of 24 MSWI
  iterations applied to the data in Figure \ref{fig:fcwd20}, beginning
  at the homogeneous model.}

\multiplot{2}{caustic20resimmestmswicont,caustic20residmestmswicont}{width=0.45\textwidth}{a: Data
  corresponding to MSWI result shown in Figure
  \ref{fig:caustic20mestmswicont}. b: Residual or data error - difference between
  data shown in Figure \ref{fig:caustic20resimmestmswicont} and target data
  \ref{fig:fcwd20}.}

\plot{caustic20uestmswicont}{width=\textwidth}{Adaptive filter estimate for MSWI
  inversion shown in Figure \ref{fig:caustic20mestmswicont}.}

\plot{caustic20mestmswifwicont}{width=\textwidth}{Bulk modulus produced by 25
  steepest descent steps to minimize the FWI objective \ref{eqn:fwi},
  using the data shown in \ref{fig:fcwd20}, and starting at the MSWI
  result shown in Figure \ref{fig:caustic20mestmswicont}.}

\multiplot{2}{caustic20resimmestmswifwicont,caustic20residmestmswifwicont}{width=0.45\textwidth}{a: Data
  corresponding to FWI + MSWI result shown in Figure
  \ref{fig:caustic20mestmswifwicont}. b: Residual or data error - difference between
  data shown in Figure \ref{fig:caustic20resimmestmswifwicont} and target data
  \ref{fig:fcwd20}.}

\plot{caustic20uest2}{width=\textwidth}{Adaptive filter to match
  data for FWI + MSWI model (Figure \ref{fig:caustic20resimmestmswicont}) to oblate lens
  data (Figure \ref{fig:fcwd20}). Dispersion away from 
  away from $t=0$ has almost disappeared, showing that the resimulated
  data is nearly a perfect kinematic match with the target data.}

\subsection{Non-smooth models: the Camembert}

\subsection{Noise}



\section{Discussion}
Several matters deserve attention: character of the FWI result; choice of parameters in the
definition of the MSWI objective; and application to inverse problems
defined by other types of wave physics.

\subsection{FWI failure - local mins?}
The result shown in Figure \ref{fig:cwlens20mestfwi0} is certainly
unsatisfactory, as is evident from the residual plot (Figure
\ref{fig:cwlens20residmestfwi0}). However one cannot conclude {\em
  from this result alone} that further
iterations of steepest descent, or of a more sophisticated local
optimization algorithm, starting at the same homogeneous initial
estimate, would not eventually meander to a global
minimizer, fitting the data - most importantly, the arrival times -
with the same sort of accuracy that FWI achieves when started at the
MSWI result (Figure \ref{fig:cwlens20mestmswi}). Actually, further iterations don't help
in this instance - at least, not O(10) or O(100) additional
iterations (we tried!). On the other hand, if O($10^6$)
additional iterations {\em would} yield a good invertion, it doesn't matter: such vast amount of computational
work is simply not practical. Nor is it sensible, since alternatives
exist (FWI + MSWI) that accomplish the inversion goal with far less
work.

So: we don't know whether Figure \ref{fig:cwlens20mestfwi0} represents an
approximation to a non-global local minimizer, or not - but it doesn't matter.

\subsection{Parameters}
The roles of parameters $\sigma$ and $\alpha$ are quite
different. Regularization, that is, avoidance of singularity in the
normal equation \ref{eqn:normal}, is the purpose of setting $\sigma >
0$. As formulated here, $\sigma$ is a dimensional parameter. It is
simple to make it dimensionless, and that should be done, so that a
``small'' choice is meaningful.

The penalty parameter $\alpha$ however plays a very different role. In
principle, $\alpha \rightarrow \infty$ should force the adaptive
filter to become ``physical'', which in this case means the identity
operator, i.e. with kernel $\delta(t)$ regardless of source and
receiver, so that the penalty problem becomes identical to FWI.

There are a variety of rules for setting penalty parameters like
$\alpha$, of varying degrees of rigor. The discrepancy concept is one
of these: it is simple to implement, auto-starting for problems like
MSWI, and steers the penalty problem towards its limit. No magic
however: this approach requires the assertion of data error. $\alpha$
is adjusted so that the data error term in \ref{eqn:redfiltpen} or
similar function is close to the posited size. Doing so is quite
simple - see \cite{Fu:Geo17b,SymesMinkoffChen:IP22}. It might be objected that
then one opaque parameter ($\alpha$) has been exchanged for another
unknown one (data error), however it is even possible to turn this
construction around to estimate data error
\cite[]{ChenSymesMinkoff:IMAGE22}. In any case, data error has an
obvious meaning, whether it is a known quantity or not, whereas
$\alpha$ is merely a Lagrange multiplier.

\subsection{Other Physics, Other Data}
Acoustics is seldom exactly the right choice of physics for mechanical
vibration modeling; in many settings, some variant of elastodynamics,
perhaps coupled with acoustics in some regions, is the right
choice. Elastic waves travel at several speeds, intrinsically
providing multiple wavefronts. Extension of MSWI and similar
approahces to elastodynamic inverse problems, either by data mode
separation or some more intrinsic construction, would appear to be a
very important next step, as would application to electromagnetic imaging.

Sensors (and energy sources) are seldom punctual and/or
isotropic. Actual source encountered in lab and field can exhibit
non-trivial radiation patterns or even physical extent non-negligible
on the wavelength scale. Inclusion of more realistic source modeling
also seems a very important next step, even for those problems
admitting a successful approach via source-receiver extension.

As has been emphasized several times, wavefront complexity can prevent
successful application of inversion methods based on source-receiver
extension. A variety of other source extension approaches have been
explored, some with positive numerical results
\cite[]{HuangNammourSymesDollizal:SEG19}. For many of these
approaches, the missing piece is a theoretical justification like that
summarized in equation \ref{eqn:tomo} for MSWI.

Finally, the assumption of smooth material parameter variation is
unlikely to be justified in actual physical media. Sedimentary rocks,
human tissue, manufactured objects all exhibit juxtaposition of
structure at all scales, from long scales modeled by smooth variation,
to sub-wavelength oscillations. Most likely, settings in which
ballistic transmitted waves dominate data energetics will permit more
or less unmodified application of techniques like MSWI, at least to
gain a low-resolution image of wave velocity, bulk modulus,.... However
theoretical results in this direction are absent, as are results on
the related question of applicability of algorithms like MSWI to
predominantly reflected wave data.



\bibliographystyle{seg}
\bibliography{../../bib/masterref}


% \appendix{Abstract Structure of the Gradient}
\section*{APPENDIX: ABSTRACT STRUCTURE OF THE GRADIENT}

The inverse problem studied here is far too simple to have any direct
use in applications. Its simplicity allows a rather complete account of
its properties, as the reader has seen in the preceding pages. However
this discussion has touched on features found in more complex and
prototypical problems:
\begin{itemize}
\item Nested optimization (variable projection method,
  \cite{GolubPereyra:03}) based on a model decomposition into inner
  and outer variables seems to be very important: in some cases, such
  as the simple problem presented here, the decomposition is obvious
  (inner variable = wavelet, outer variable = slowness), in others
  less so \cite[]{geoprosp:2008,Terentyev:thesis}.
\item The extended modeling operator must be surjective, or at least
  have a dense range, {\em for each value of the outer variable}, so
  that data may be fit well even for a poor initial guess of the outer
  variable. for the extended inversion approach to be successful.
\item Because of this data-fitting assumption, a straightforward
  algorithm for scaling the quadratic penalty, based on the
  Discrepancy Principle, is available - see
  \cite{FuSymes2017discrepancy,SymesChenMinkoff:21}. This
  scaling varies dynamically during iterative optimization, in effect
  changing the objective function sporadically as the iteration
  converges.
\item The derivative of the extended modeling operator with respect to
  the outer variable is well-approximated by the composition of the
  extended modeling operator itself and a pseudodifferential operator
  of order 1 \cite[]{Symes:IPTA14,tenKroode:IPTA14}.
\item The choice of the penalty operator, controlling the extended
  degrees of freedom, is critical: in order to produce an objective
  immune from cycle-skipping, this penalty operator must be
  (pseudo-) differential of order zero \cite[]{StolkSymes:03}. A smooth
  multiplier $a$, as in equation \ref{eqn:annmult}, is a special case.
\end{itemize}

The expression \ref{eqn:dexpjgen} for the derivative of the reduced
objective $\tJa$ is the result of elementary manipulations, based on the
explicit expression \ref{eqn:mod} for the modeling operator $F$. In
this appendix I give an alternative derivation that generalizes to
extended inversion formulations for much less constrained physics. I
will point out the additional reasoning necessary to reach similar
conclusions in these more complex instances of extended inversion, as
presented for example in \cite{StolkSymes:03,StolkDeHoopSymes:09,Symes:IPTA14,tenKroode:IPTA14,HuangSymes:Geo17,HuangSymes:Geo18a,HuangSymes:Geo18b,HuangNammourSymesDollizal:SEG19}.

Recall that $\aw[m;d]$ is the solution of the normal equation
\ref{eqn:norm}. The reduced objective $\tJa$ is given by
\[
  \tJa[m;d] = \Ja[m,\aw[m;d];d]
\]
\[
  = \frac{1}{2}(\|F[m]\aw[m;d]-d\|^2 + \alpha^2\|A\aw[m;d]\|^2)
\]
(equation \ref{eqn:redexp1})
\begin{equation}
  \label{eqn:redsimple}
  =\frac{1}{2}(\|d\|^2 - \langle d,F[m]\aw[m;d]\rangle),
\end{equation}
after a little algebra.

As mentioned above, $F:M \times W \rightarrow D$ is {\em not}
differentiable. Neither is $\aw: M \times D \rightarrow W$, as follows
immediately from the identity \ref{eqn:normsol}. However $F\aw: M
\times D \rightarrow D$ is differentiable, hence so is $\tJa: M \times
D \rightarrow \bR$ thanks
to the identity \ref{eqn:redsimple},
under the conditions on the multiplier $a$ identified in Theorem
\ref{thm:diffobj}.

For the model problem studied in the body of this paper, a simple
argument justifies this conclusion.  To sideline some technical
details, assume that
$a \in C^{\infty}(\bR)\cap L^{\infty}(\bR)$ and $a(t)=t$ for $|t|<\tau$ and $|a(t)| \ge
\tau$ for $|t| \ge \tau$, with $\tau$
defined in equation \ref{eqn:taudef}. Then
Proposition \ref{thm:norminvexp}, item [1], implies that
\begin{equation}
  \label{eqn:faw}
  F[m]\aw[m;d] = F[m](F[m]^TF[m] + \alpha^2 A^TA)^{-1}F[m]^T d.
\end{equation}
The normal operator and its inverse are multiplication
operators, whereas $F$ and $F^T$ are scaled shift operators, inverse
to each other except for scale. From this it is easy to see that the
operator on the RHS of equation \ref{eqn:faw} is multiplication by a
smooth function, with its arguments shifted by $mr$. Such an operator
is smooth in $m$, hence so is $F\aw$.

Theorem \ref{thm:diffobj} also gives an explicit expression for the
derivative of $\tJa$. An alternative expression
does not depend on the structure of the single-trace modeling
operator, hence applies in a much more general setting. Note that if
$w \in H^1(\bR)$, then $m \mapsto F[m]w$ is differentiable, and
\begin{equation}
\label{eqn:deriv}
(D(F[m]w)\delta m)(t) = F[m](Q[m]\delta m)w (t), 
\end{equation}
where 
\begin{equation}
\label{eqn:defq}
(Q[m]\delta m)w = -r\delta m \frac{dw}{dt}. 
\end{equation}
That is, $DF[m]\delta m$ factors into $F[m]$ following $Q[m]\delta m$,
where the latter a skew-adjoint differential operator of order 1,
depending linearly on $\delta m$.

Assume that $d \in H^1(\bR)$. From equation \ref{eqn:normsol}, it
follows that $\aw[m;d] \in H^1(\bR)$ and moreover that $m \mapsto
\aw[m;d]$ is differentiable as a map from $\bR^+$ to
$L^2(\bR)$. From equation \ref{eqn:redexp1},
\[
  \tJa[m;d] = \frac{1}{2}(\|F[m]\aw[m;d]-d\|^2 + \alpha^2 \|A\aw[m;d]\|),
\]
whence $m \mapsto \tJa[m;d]$ is differentiable. A standard calculation
invoking the normal equation \ref{eqn:norm} shows that
\[
  D\tJa[m;d]\delta m = \langle D(F[m]\aw[m;d])\delta m, F[m]\aw[m;d]-d\rangle
\]
\[
  = \langle F[m](Q[m]\delta m)\aw[m;d], F[m]\aw[m;d]-d\rangle
\]
\[
  = \langle (Q[m]\delta m)\aw[m;d], F[m]^T(F[m]\aw[m;d]-d)\rangle
\]
\[
  = -\alpha^2\langle (Q[m]\delta m)\aw[m;d],A^TA\aw[m;d] \rangle
\]
\[
  = \alpha^2 \langle \aw[m;d], (Q[m]\delta m)A^TA \aw[m;d]\rangle
\]
(using antisymmetry of $Q$)
\[
  = \alpha^2 \langle \aw[m;d],[ (Q[m]\delta m), A^TA]\aw[m;d] \rangle
  + \alpha^2 \langle (Q[m]\delta m)\aw[m;d],A^TA\aw[m;d] \rangle.
\]
Rearranging,
\begin{equation}
  \label{eqn:djq}
  D\tJa[m;d]\delta m = \frac{1}{2}\alpha^2 \langle \aw[m;d],[
  (Q[m]\delta m), A^TA]\aw[m;d] \rangle.
\end{equation}
Since $Q$ is a differential operator of order 1, and $A^TA$ an
operator of order 0, the commutator has order 0. That is, the RHS of
equation \ref{eqn:djq} defines a continuous quadratic form in $d$ with
respect to the $L^2$ norm. As shown above, the same is true of the
left hand side. Therefore their extensions by continuity to $d \in
L^2(\bR)$ are the same, and the identity \ref{eqn:djq} holds for $d
\in L^2(\bR)$.

Making the identity \ref{eqn:djq} explicit by means of equation
\ref{eqn:defq} and $A^TA = t^2$, substituting the expression
\ref{eqn:normsol} for $\aw[m;d]$, and rearranging: one obtains
precisely the expression \ref{eqn:dexpjgen}.

Similar reasoning can be applied to some of the other extended
inversion settings mentioned at the beginning of this
appendix. Provided that the outer parameter $m$ consists of (or
parametrizes) smooth coefficients in the principal part of the
governing system of wave equations, $F$ is a microlocally elliptic
Fourier Integral Operator (FIO) \cite[]{Dui:95}. The canonical
relation of $F$ takes on the role of the shift operator
$t \rightarrow t-mr$, and has the properties necessary to conclude
that the composition of $F$ and $F^T$ in both orders is
pseudodifferential, at least in an open conic subset of the cotangent
bundle. If $A^TA$ is pseudodifferential, then so is the normal
operator, and it is microlocally elliptic, so positive-definite when
restricted to a suitable subspace of the inner parameter $w$, which
appears in the hyperbolic system as components of a right-hand side or
boundary condition. The inverse in the RHS of equation \ref{eqn:faw}
must be replaced by a microlocal parametrix. Thanks to Egorov's
Theorem \cite[]{Tay:81}, a special case of the rules for composing
FIOs, the operator on the RHS of equation \ref{eqn:faw} is a
pseudodifferential operator whose symbol is algebraic in geometric
optics quantities, hence depends smoothly on the coefficients in the
system of wave equations, and therefore on $m$, as does $F\aw$.

Microlocal ellipticity of $F$ implies an analogue of relation
\ref{eqn:deriv}, in which $Q$ is an essentially skew-symmetric
pseudodifferential operator of order 1. Since the derivation 
the gradient identity \ref{eqn:djq} is algebraic, it holds
approximately in these more general cases, with lower-order (smoother)
error terms. See 
\cite{Symes:Madrid,tenKroode:IPTA14,Symes:IPTA14} for details in
several specific cases, and for a similar calculation of the Hessian.
For these examples, the principal symbol of $Q$ can be computed
explicitly, as a function of the geodesic distance (traveltime)
function. Consequently, the analog of $\tJa$ is tangent to second
order to the Hessian of an objective formulated in terms of
traveltime. This relation explains the ability of the variant of
extended inversion studied in the cited references to recover
kinematically accurate velocity fields. Of course, that is what is
shown in detail in this paper for the very simple model problem
studied here.
