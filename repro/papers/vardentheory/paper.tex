\title{Asymptotic Inversion of the Variable Density Acoustic
  Model\\Part 1: Theory}

\author{
Raanan Dafni\thanks{Department of Computational and Applied Mathematics, Rice University, 
Houston, TX, 77005, USA, 
{\tt symes@rice.edu}},
William Symes\thanks{Department of Computational and Applied Mathematics, Rice University, 
Houston, TX, 77005, USA, 
{\tt symes@rice.edu}}
}

\lefthead{Dafni Symes}

\righthead{Asymptotic Acoustic Inversion}

\maketitle
\parskip 12pt

\begin{abstract}
Techniques developed for asymptotic wave equation based inversion of
subsurface offset extended constant density acoustics also apply to variable density
acoustics. The data may be matched by proper choice of an extended
relative bulk modulus perturbation, with zero density
perturbation. This bulk modulus perturbation may be treated as an
offset-dependent reflectivity volume, and AVA-like techniques applied
to extract asymptotically correct estimates of velocity and density
perturbations, assuming that the background velocity is kinematically
consistent with the data. This first part of the paper presents a 
theoretical justification of the inversion algorithm. The second part
describes implementation details and shows numerical examples.
\end{abstract}
\setlength{\parindent}{0cm}
\section{Introduction}

\section{Extended Variable Density Acoustics}
The dynamical laws of linear acoustics may be written
\begin{eqnarray}
\label{eqn:awe}
\frac{1}{\kappa}\frac{\partial p}{\partial t}& =& -\nabla \cdot \bv +
g,\nonumber \\
\rho \frac{\partial \bv}{\partial t} & = & -\nabla p.
\end{eqnarray}
In this system, $p$ is pressure, $\bv$ is particle velocity, $\kappa$
is bulk modulus, $\rho$ is material density, and $g$ is the defect
representation of an energy source. The first equation is the
constitutive law relating stress (pressure) rate to strain rate (actually to its
trace, which is proportional to the velocity divergence). The second
expresses momentum balance. When completed by suitable initial and
boundary conditions, the system (\ref{eqn:awe}) has a unique finite
energy solution $(p,\bv)$ under very mild constraints on $\kappa,
\rho$ \cite[]{BlazekStolkSymes:13}.

Define the acoustic forward map as
\begin{equation}
\label{eqn:afm}
{\cal F}[\kappa,\rho] = {\cal S}[p,\bv],
\end{equation}
in which ${\cal S}$ is a (linear) sampling operator, extracting traces of pressure
and/or velocity at receiver points $\bx_r$ for $t \in [0,T]$. Note that
the source function $g$ also depends on an experimental
parameter. While other possibilities are interesting, for present
purposes assume that $g$ depends on a location $\bx_s$, so the value of
$\cal{F}$ does as well. 

The subsurface offset extension of this system generalizes $\kappa$
and $\rho$ to be operators. That is, the relations between stress
and strain and between force and acceleration may be non-local.
Of course one
can think of the ordinary stress-strain relation as being mediated by
an operator, namely  the operator of multiplication by
$\kappa$. Multiplication by $1/\kappa$ is just the inverse of this
operator. Extended model replaces multiplication by $\kappa$ by the
action of a (possibly non-local) operator $\bar{\kappa}$, and multiplication by
$\rho$ by the action of a (possibly non-local) operator $\bar{\rho}$. Just as $\kappa$ is
positive and multiplication by (real) $\kappa$ is a symmetric
operator, require $\bar{\kappa}$ to be symmetric positive
definite and bounded on $L^2$, and similar for $\bar{\rho}$.

As it happens, the extension of both parameters $\kappa$ and $\rho$,
as opposed to just $\kappa$, is not
terribly useful. Also the quasi-physical justification for extending just
$\kappa$ is clearer. However it is no more difficult to work through the
consequences of extending both, so we shall do so, and specialize at
the end to draw conclusions for $\kappa$-only extended inversion, and
for ordinary non-extended inversion for
$\kappa$ and $\rho$.

The extended acoustics system for extended pressure and velocity
fields $\bar{p}$ and $\bar{\bv}$ is
\begin{eqnarray}
\label{eqn:eawe}
\bar{\kappa}^{-1}\frac{\partial \bar{p}}{\partial t}& =& -\nabla \cdot
\bar{\bv} + g,\nonumber \\
\bar{\rho} \frac{\partial \bar{\bv}}{\partial t} & = & -\nabla \bar{p}.
\end{eqnarray}
\cite{BlazekStolkSymes:13} show that $\bar{p}$ and $\bar{\bv}$ are
well-defined, as is the extended forward map, defined as 
\begin{equation}
\label{eqn:eafm}
\bar{{\cal F}}[\bar{\kappa},\bar{\rho}] = {\cal S}(\bar{p},\bar{\bv}).
\end{equation}

\section{Linearization}
Using the symbol $\delta$ as shorthand for perturbation, obtain for the
perturbation $\delta \bar{p}, \delta \bar{\bv}$ in the solution of
(\ref{eqn:eawe}) resulting from $\delta \bar{\kappa}, \delta{\rho}$ the
system
\begin{eqnarray}
\label{eqn:leawe}
\bar{\kappa}^{-1}\frac{\partial \delta \bar{p}}{\partial t}& =& -\nabla \cdot
\delta \bar{\bv} +
\bar{\kappa}^{-1}\delta\bar{\kappa}\bar{\kappa}^{-1}\frac{\partial \bar{p}}{\partial t},\nonumber\\
\bar{\rho} \frac{\partial \delta \bar{\bv}}{\partial t} & = & -\nabla \delta
\bar{p} - \delta \bar{\rho} \frac{\partial \bar{\bv}}{\partial t}.
\end{eqnarray}
The linearized forward map is defined in terms of the solution of
(\ref{eqn:leawe}) as 
\begin{equation}
\label{eqn:leafm}
D\bar{{\cal F}}[\bar{\kappa},\bar{\rho}][\delta \bar{\kappa},\delta \bar{\rho}] =
{\cal S}(\delta \bar{p},\delta \bar{\bv}),
\end{equation}
thanks to the linearity of ${\cal S}$.

Physically, of course, the extended system is nonsense, in that it
violates the continuum mechanical axiom of
no-action-at-a-distance. 

\section{Expression in terms of Green's Functions}
From here on, assume that the reference fields are physical.

Differentiate the first equation in (\ref{eqn:leawe}) and
(\ref{eqn:awe}) with respect to $t$,
and use the second equation to eliminate $\delta \bar{\bv}$ and $\bv$ to arrive at
the second order wave equations for $p,\delta \bar{p}$:
\begin{eqnarray}
\label{eqn:awe2}
\frac{1}{\kappa}\frac{\partial^2 p}{\partial t^2} &=&
\nabla \cdot \frac{1}{\rho} \nabla p + \frac{\partial g}{\partial t}\\
\label{eqn:leawe2}
\frac{1}{\kappa}\frac{\partial^2 \delta \bar{p}}{\partial t^2} &=&
\nabla \cdot \frac{1}{\rho} \nabla \delta \bar{p} + \frac{1}{\kappa}
\delta \bar{\kappa} \frac{1}{\kappa}\frac{\partial^2 p}{\partial t^2} -
  \nabla \cdot \frac{1}{\rho}\delta \bar{\rho}\frac{1}{\rho} \nabla p. 
\end{eqnarray}
Note that the two copies of $1/\kappa$ must be kept separate, on
either side of $\delta \bar{\kappa}$, as the latter is an operator and
does not commute with multiplication by $1/\kappa$ unless $\delta
\bar{\kappa}$ is itself physical, that is, multiplication by a
function. The same applies to $1/\rho$.

To express the relation between $\delta \bar{\kappa}$, $\delta \rho$
and $D\bar{{\cal F}}[\bar{\kappa},\rho][\delta \bar{\kappa},\delta
\rho]$ in integral form, assume that (at least formally) $\delta
\bar{\kappa}$, $\delta \bar{\rho}$ are integral operators:
\begin{eqnarray}
\label{eqn:dkappa}
(\delta \bar{\kappa} u)(\bx) &=& \int d\by \delta 
  \bar{\kappa}(\bx,\by)u(\by),\\
(\delta \bar{\rho} u)(\bx) &=& \int d\by \delta 
  \bar{\rho}(\bx,\by)u(\by),\\
\end{eqnarray}
and introduce the Green's function $G(\bx,t;\by)$, the causal solution
of
\begin{equation}
\label{eqn:green}
\left(\frac{1}{\kappa}\frac{\partial^2 G}{\partial t^2} - \nabla_{\bx} \cdot
  \frac{1}{\rho} \cdot \nabla_{\bx}G\right)(\bx,t;\by) =
\delta(\bx-\by)\delta(t), \,\, G=0 \mbox{ for } t < 0.
\end{equation}
Then, by superposition,
\[
\delta \bar{p}(\bx,t)  = \int d\by \int d\tau G(\bx,t-\tau;\by)
\frac{1}{\kappa(\by)}  \int d\bz \delta \bar{\kappa}(\by,\bz) \frac{1}{\kappa(\bz)}
 \frac{\partial^2 p}{\partial t^2}(\bz,\tau) 
\]
\begin{equation}
\label{eqn:intrep1}
- \int d\by \int d\tau G(\bx,t-\tau;\by)\nabla_{\by} \cdot \left(
\frac{1}{\rho(\by)}\int d\bz \delta \bar{\rho}(\by,\bz)\frac{1}{\rho(\bz)}) \nabla_{\bz} p(\bz,\tau)\right). 
\end{equation}
It is conventional to introduce alternate integration variables
\begin{eqnarray}
\label{eqn:jon}
\bx &=& \frac{1}{2}(\by + \bz) \nonumber\\
\bh &=& \frac{1}{2}(\by - \bz).
\end{eqnarray}
These play the roles of Claerbout's sunken midpoint and half-offset
vectors (REF). Re-parametrize $\delta\bar{\kappa}(\by,\bz) \rightarrow
\delta\bar{\kappa}(\bx,\bh)$, and similarly for $\delta \bar{\rho}$. Then (\ref{eqn:intrep1}) becomes
\[
\delta \bar{p}(\bx,t)  = \int d\bx\int d\bh \int d\tau
\left( G(\bx_r,t-\tau;\bx+\bh) 
\frac{1}{\kappa(\bx+\bh)}  \delta \bar{\kappa}(\bx,\bh) \frac{1}{\kappa(\bx-\bh)}
 \frac{\partial^2 p}{\partial t^2}(\bx-\bh,\tau)
\right.
\]
\begin{equation}
\label{eqn:intrep2}
-\left.  G(\bx_r,t-\tau;\bx+\bh) \nabla \cdot \left[
\frac{1}{\rho(\bx+\bh)}  \delta \bar{\rho}(\bx,\bh) \frac{1}{\rho(\bx-\bh)}\nabla
 p(\bx-\bh,\tau)\right]\right)
\end{equation}

\section{The Impulse Response}
We make two restrictions on the source term $g$ in the systems
\ref{eqn:awe} and \ref{eqn:eawe}:
\begin{itemize}
\item For each source position $\bx_s$, $g$ represents an isotropic
  point radiator located at $\bx_s$;
\item The source spectrum is flat, at least for frequencies in the
  passband of significant S/N.
\end{itemize}
Neither assumption reflects the reality of sources recorded in the
field. While the localization assumption may be acceptable, many field
sources have significant radiation patterns, and certainly non-flat
spectra. We are in effect ignoring directional variation in radiation,
and assuming that (in principle perfect) source signature
deconvolution has been performed.

Introduction of a multipole source
expansion would allow us to carry out an analysis similar to that to
follow, while accounting for significantly directional
sources, however we reserve that opportunity for another
work. Similarly, while source estimation could be folded into the
estimation of the material parameters, but that is for another time.

Since the time dependence is equivalent to an impulse over the data
bandwidth, we go ahead and treat it as an impulse with infinite
bandwidth. This assumption is standard in work on asymptotics since
its outset \cite[]{CohBlei:77}, and has some negative consequences, for example
inaccuracy at passband edges. However we will stick with it for
convenience.

The upshot is 
\begin{equation}
\label{eqn:isopt}
g(\bx,t) = \delta(t)\delta(\bx-\bx_s).
\end{equation}
Then $p(\bx,t) = \frac{\partial G}{\partial t}(\bx,\cdot;\bx_s)(t)$ depends on $\bx_s$ as
does its perturbation. 

Further assume that the sampling operator simply sampling pressure at receiver
points $\bx_r$. Then
\[
D\bar{{\cal F}}[\kappa,\rho][\delta 
\bar{\kappa},\delta\bar{\rho}](\bx_r,t;\bx_s) 
\]
\[
= \int d\bx\int d\bh \int d\tau \left(G(\bx_r,t-\tau;\bx+\bh) 
\frac{1}{\kappa(\bx+\bh)}  \delta \bar{\kappa}(\bx,\bh) \frac{1}{\kappa(\bx-\bh)}
 \frac{\partial^3 G}{\partial t^3}(\bx-\bh,\tau;\bx_s) 
\right. 
\]
\[
-\left.  G(\bx_r,t-\tau;\bx+\bh) \nabla \cdot \left[
\frac{1}{\rho(\bx+\bh)}  \delta \bar{\rho}(\bx,\bh) \frac{1}{\rho(\bx-\bh)}\nabla 
 \frac{\partial G}{\partial t}(\bx-\bh,\tau;\bx_s)\right]\right)
\]
\[
= \int d\bx\int d\bh \int d\tau \left(G(\bx_r,t-\tau;\bx+\bh) 
\frac{1}{\kappa(\bx+\bh)}  \delta \bar{\kappa}(\bx,\bh) \frac{1}{\kappa(\bx-\bh)}
 \frac{\partial^3 G}{\partial t^3}(\bx-\bh,\tau;\bx_s) 
\right.
\]
\begin{equation}
\label{eqn:intrep3}
+\left. \nabla G(\bx_r,t-\tau;\bx+\bh) \cdot \left[
\frac{1}{\rho(\bx+\bh)}  \delta \bar{\rho}(\bx,\bh) \frac{1}{\rho(\bx-\bh)}\nabla 
\frac{ \partial G}{\partial t}(\bx-\bh,\tau;\bx_s)\right]\right)
\end{equation}
after integration by parts.

For future reference, we include the integral expression for the
adjoint of $D\bar{{\cal F}}[\kappa,\rho]$:
\[
D\bar{{\cal F}}[\kappa,\rho]^Td(\bx,\bh) =
\]
\begin{equation}
\label{intrep3adj}
\left[
\begin{array}{c}
\frac{1}{\kappa(\bx+\bh)}\frac{1}{\kappa(\bx-\bh)}\int d\bx_r d\bx_s dt d\tau G(\bx_r,t-\tau;\bx+\bh) 
 \frac{\partial^3 G}{\partial t^3}(\bx-\bh,\tau;\bx_s)
  d(\bx_r,t;\bx_s)  \\
\frac{1}{\rho(\bx+\bh)}\frac{1}{\rho(\bx-\bh)}\int d\bx_r d\bx_s dt
  d\tau \nabla G(\bx_r,t-\tau;\bx+\bh) \cdot \nabla \frac{ \partial
  G}{\partial t}(\bx-\bh,\tau;\bx_s) d(\bx_r,t;\bx_s)
\end{array}
\right].
\end{equation}

It is convenient for analysis (and for QC!) to make the perturbations
$\delta \bar{\kappa}$ and $\delta \bar{\rho}$ relative to the
reference quantities. Define the {\em extended reflectivities} $\bar{r}_{\kappa}$
and $\bar{r}_{\rho}$ as 
\begin{eqnarray}
\label{eqn:relfs}
\bar{r}_{\kappa}(\bx,\bh)& = & \frac{\delta
                               \bar{\kappa}(\bx,\bh)}{\sqrt{\kappa(\bx+\bh)\kappa(\bx-\bh)}}
                               \nonumber \\
\bar{r}_{\rho}(\bx,\bh)& = & \frac{\delta \bar{\rho}(\bx,\bh)}{\sqrt{\rho(\bx+\bh)\rho(\bx-\bh)}}
\end{eqnarray}
Define the relative linearized forward operator $\bar{F}[\kappa,\rho]$ by
\begin{equation}
\label{eqn:nondim}
\bar{F}[\kappa,\rho][\bar{r}_{\kappa},\bar{r}_{\rho}] =
D\bar{{\cal F}}[\kappa,\rho][\delta 
\bar{\kappa},\delta\bar{\rho}](\bx_r,t;\bx_s) 
\end{equation}
where $\bar{r}_{\kappa},\bar{r}_{\rho}$ are related to $\delta 
\bar{\kappa},\delta\bar{\rho}$ by equation \ref{eqn:relfs}. Then
\[
\bar{F}[\kappa,\rho][\bar{r}_{\kappa},\bar{r}_{\rho}](\bx_r,t;\bx_s) 
=\int d\bx\int d\bh \int d\tau 
\]
\[
\left(G(\bx_r,t-\tau;\bx+\bh) 
\frac{1}{\sqrt{\kappa(\bx+\bh)}}  \bar{r}_{\kappa}(\bx,\bh) \frac{1}{\sqrt{\kappa(\bx-\bh)}}
 \frac{\partial^3 G}{\partial t^3}(\bx-\bh,\tau;\bx_s) 
\right.
\]
\begin{equation}
\label{eqn:intrep3nd}
+\left. \nabla G(\bx_r,t-\tau;\bx+\bh) \cdot \left[
\frac{1}{\sqrt{\rho(\bx+\bh)}} \bar{r}_{\rho}(\bx,\bh) \frac{1}{\sqrt{\rho(\bx-\bh)}}\nabla 
 \frac{\partial G}{\partial t}(\bx-\bh,\tau;\bx_s)\right]\right)
\end{equation}


\section{Simple Ray Geometry}

Suppose that unique rays connect sources and receivers to scattering
points, as is the case for smooth $\kappa, \rho$ when sources,
receivers, and scatterers are sufficiently close together. Then in 2D,
\begin{equation}
\label{eqn:simp2d}
G(\bx,t;\by) = a(\bx,\by)t_+^{-1/2}(t-T(\bx,\by)) 
\end{equation}
whereas in 3D,
\begin{equation}
\label{eqn:simp3d}
G(\bx,t;\by) = a(\bx,\by)\delta(t-T(\bx,\by)) 
\end{equation}
in which $T$ is traveltime and $a$ is geometric amplitude. 

For simplicity assume dimension = 3 for the remainder of this
section. The following section will present analgous computations for 2D.

Up to a lower order error,
\[
\nabla G \approx -a \delta'(t-T)\nabla T.
\]
Substituting this approximation and using $\delta'*\delta''=\delta'''$, \ref{eqn:intrep3} becomes
\[
\bar{F}[\kappa,\rho][\bar{r}_{\kappa},\bar{r}_{\rho}](\bx_r,t;\bx_s) 
\]
\[
\approx \int d\bx\int d\bh a(\bx_r,\bx+\bh) a(\bx_s,\bx-\bh) 
\delta'''(t-(T(\bx_r,\bx+\bh)+T(\bx_s,\bx-\bh))) 
\]
\[
\times \left(
\frac{1}{\sqrt{\kappa(\bx+\bh)}}  \bar{r}_{\kappa}(\bx,\bh) \frac{1}{\sqrt{\kappa(\bx-\bh)}}
%\frac{1}{\kappa(\bx+\bh)}  \delta \bar{\kappa}(\bx,\bh) \frac{1}{\kappa(\bx-\bh)}
\right.
\]
\begin{equation}
\label{eqn:intrep4}
+\left. 
   \nabla  T(\bx_r,\bx+\bh) \cdot \nabla T(\bx_s,\bx-\bh) 
\frac{1}{\sqrt{\rho(\bx+\bh)}} \bar{r}_{\rho}(\bx,\bh) \frac{1}{\sqrt{\rho(\bx-\bh)}}
%\frac{1}{\rho(\bx+\bh)}  \delta \bar{\rho}(\bx,\bh) \frac{1}{\rho(\bx-\bh)}
\right)
\end{equation}
We remove the third 
time derivative by applying the causal time antiderivative by $I_t$,
\[
I_t^3 \bar{F}[\kappa,\rho][\bar{r}_{\kappa},\bar{r}_{\rho}](\bx_r,t;\bx_s) 
\]
\[
= \int d\bx\int d\bh a(\bx_r,\bx+\bh) a(\bx_s,\bx-\bh) 
\delta(t-(T(\bx_r,\bx+\bh)+T(\bx_s,\bx-\bh)) ) 
\]
\[
\times \left(
\frac{1}{\sqrt{\kappa(\bx+\bh)}}  \bar{r}_{\kappa}(\bx,\bh) \frac{1}{\sqrt{\kappa(\bx-\bh)}}
%\frac{1}{\kappa(\bx+\bh)}  \delta \bar{\kappa}(\bx,\bh) \frac{1}{\kappa(\bx-\bh)}
\right. 
\]
\begin{equation}
\label{eqn:intrep5}
+\left. 
   \nabla  T(\bx_r,\bx+\bh) \cdot \nabla T(\bx_s,\bx-\bh) 
%\frac{1}{\rho(\bx+\bh)}  \delta \bar{\rho}(\bx,\bh)
%\frac{1}{\rho(\bx-\bh)}
\frac{1}{\sqrt{\rho(\bx+\bh)}} \bar{r}_{\rho}(\bx,\bh) \frac{1}{\sqrt{\rho(\bx-\bh)}}
\right)
\end{equation}
%The normal operator of $I_t^2D{\cal F}$ is a operator matrix of the
%form
%\begin{equation}
%\label{eqn:normalop}
%(I_t^2D{\cal F}[\kappa,\rho])^*(I_t^2D{\cal F}[\kappa,\rho])
%=\left(\begin{array}{cc}
%M_{\kappa \kappa} & M_{\kappa \rho}\\
%M_{\kappa \rho}^* & M_{\rho \rho}
%\end{array}
%\right)
%\end{equation}
%in which $M_{i,j}, i,j \in \{\kappa, \rho\}$ take the form of the GRT
%operator defined by a 3D form of \cite{HouSymes:17}, equation 4, 
%\begin{equation}
%\begin{aligned}
%\label{eqn:grt3d}
%M u(\bx,\bh) =& \int d\bx_r d\bx_s d\bx' d\bh' A(\bx_r,\bx_s,\bx,\bh,\bx',\bh')\\
%&\times \delta(\phi(\bx_r,\bx_s,\bx,\bh) - \phi(\bx_r,\bx_s,\bx',\bh')) u(\bx',\bh'),
%\end{aligned}
%\end{equation}

\section{Two Dimensions and Horizontal Offset}
Since the 2D case is a bit simpler, we now turn to it, and also
introduce the horizontal offset restriction. Instead of $\bx$ we shall
write $(x,z)$, also $x_r,x_s$ rather than $\bx_r,\bx_s$, suppressing
the $z$ coordinates of the sources and receiver, idealized to
constants. [So the following derivation would have to be modified to
model so-called broadband acquisition with variable streamer depth.] 
Since offset is now one-dimensional (and horizontal), we
will write $(x,z,h)$ instead of $(\bx,\bh)$, and $(x\pm h,z)$ instead
of $\bx\pm \bh$.

For 2D, the $t^{-1/2}_+$ singularity of the Green's function removes
one of the time derivatives appearing in \ref{eqn:intrep5}, due to the
identity
\begin{equation}
\label{eqn:convsq}
t_+^{-1/2}*t_+^{-1/2} = \pi H(t)
\end{equation}
(see \cite{GelShil:58}, p. 116, formula (3')), so
\[
I_t^2\bar{F}[\kappa,\rho][\bar{r}_{\kappa},\bar{r}_{\rho}]
%\bar{{\cal F}}[\kappa,\rho][\delta 
%\bar{\kappa},\delta\bar{\rho}](x_r,t;x_s) 
\]
\[
= \pi\int dx \int dz \int dh a(x_r,x+h,z) a(x_s,x-h,z) 
\delta(t-(T(x_r,x+h,z)+T(x_s,x-h,z))) 
\]
\[
\times \left(\frac{1}{\sqrt{\kappa(x+h,z)}} \bar{r}_{\kappa}(x,z,h) \frac{1}{\sqrt{\kappa(x-h,z)}}
\right. 
\]
\begin{equation}
\label{eqn:intrep2d}
+\left. \left. 
   \nabla_{x,z}  T(x_r,x+h,z) \cdot \nabla_{x,z} T(x_s,x-h,z) 
\frac{1}{\sqrt{\rho(x+h,z)}}  \bar{r}_{\rho}(x,z,h) \frac{1}{\sqrt{\rho(x-h,z)}}
\right)\right\}
\end{equation}

Note the following relation btween
partial derivatives of $T$ and the ray takeoff angles {\em at the
  source and receiver surfaces}
$\theta_r,\theta_s$ of the rays from $(x_r,z_r)$ respectively
$(x_s,z_s)$ to the scattering points $(x+h,z)$ respectively $(x-h,z)$:
\begin{eqnarray}
\frac{\partial T}{\partial z_r}(x_r,x+h,z)  &=& -\frac{\cos 
  \theta_r}{v(x_r,z_r)} \nonumber \\
\label{eqn:dtdxr}
\frac{\partial T}{\partial z_s}(x_s,x-h,z)  &=& -\frac{\cos 
  \theta_s}{v(x_s,z_s)}\\
\end{eqnarray}
Thus
\[
I_t^2\frac{\partial^2}{\partial z_s \partial z_r}
I_t^2\bar{F}[\kappa,\rho][\bar{r}_{\kappa},\bar{r}_{\rho}]
\]
\[
=\pi\int dx \int dz \int dh \frac{\cos \theta_r(x_r,x+h,z)}{v(x_r,z_r)}
\frac{\cos \theta_s(x_s,x-h,z)}{v(x_s,z_s)}
\]
\[
\times a(x_r,x+h,z) a(x_s,x-h,z) 
\delta(t-(T(x_r,x+h,z)+T(x_s,x-h,z)))  
\]
\[
\times \left(\frac{1}{\sqrt{\kappa(x+h,z)}} \bar{r}_{\kappa}(x,z,h) \frac{1}{\sqrt{\kappa(x-h,z)}}
\right. 
\]
\begin{equation}
\label{eqn:intrep2dsr}
+\left. \left. 
   \nabla_{x,z}  T(x_r,x+h,z) \cdot \nabla_{x,z} T(x_s,x-h,z) 
\frac{1}{\sqrt{\rho(x+h,z)}}  \bar{r}_{\rho}(x,z,h) \frac{1}{\sqrt{\rho(x-h,z)}}
\right)\right\}
\end{equation}
The operator matrix 
\begin{equation}
\label{eqn:dwnop}
{\bf M} = (I_t^2\bar{F}[\kappa,\rho])^T
I_t^2\frac{\partial^2}{\partial z_s \partial z_r}I_t^2\bar{F}[\kappa,\rho]
\end{equation}
thus has components of the form given by
\cite{HouSymes:17}, equation 4,
% which we repeat 
%here for the readers convenience:
\begin{equation}
\begin{aligned}
\label{eqn:grt2d}
M u(x,z,h) =& \int dx_r dx_s dx' dz' dh' \tilde{A}(x_r,x_s,x,z,h,x',z',h')\\
&\times \delta(\phi(x_r,x_s,x,z,h) - \phi(x_r,x_s,x',z',h')) u(x',z',h'),
\end{aligned}
\end{equation}
in which $\phi(x_r,x_s,x,z,h) = T(x_r,x+h,z) + T(x_s,x-h,z)$. The
coefficients $\tilde{A} = \tilde{A}_{i,j}, i,j \in \{\kappa, \rho\}$ are
\[
\tilde{A}_0(x_r,x_s,x,z,h,x',z',h')=
\]
\[
\pi^2 a(x_r,x+h,z) 
a(x_s,x-h,z) \frac{\cos \theta_r(x_r,x'+h',z')}{v(x_r,z_r)}
\]
\begin{equation}
\label{eqn:grt2dcoeff1}
\times \, \frac{\cos \theta_s(x_s,x'-h',z')}{v(x_s,z_s)} a(x_r,x'+h',z')a(x_s,x'-h',z')
\end{equation}
\begin{equation}
\label{eqn:grt2dcoeff2}
\tilde{A}_{\kappa,\kappa}(x_r,x_s,x,z,h,x',z',h') = 
\frac{\tilde{A}_0(x_r,x_s,x,z,h,x',z',h')}{\sqrt{\kappa(x+h,z)}\sqrt{\kappa(x-h,z)}\sqrt{\kappa(x'+h',z')}\sqrt{\kappa(x'-h',z')}}
\end{equation}
\[
\tilde{A}_{\rho,\kappa}(x_r,x_s,x,z,h,x',z',h') =
\]
\begin{equation}
\label{eqn:grt2dcoeff3}
\frac{\tilde{A}_0(x_r,x_s,x,z,h,x',z',h')\nabla T(x_r,x+h,z)
\cdot\nabla T(x_s,x-h,z)}
{\sqrt{\rho(x+h,z)}\sqrt{\rho(x-h,z)}\sqrt{\kappa(x'+h',z')}\sqrt{\kappa(x'-h',z')}}
\end{equation}
\begin{equation}
\label{eqn:grt2dcoeff4}
\tilde{A}_{\kappa,\rho}(x_r,x_s,x,z,h,x',z',h') =
\tilde{A}_{\rho,\kappa}(x_r,x_s,x',z',h',x,z,h)
\end{equation}
\[
\tilde{A}_{\rho,\rho}(x_r,x_s,x,z,h,x',z',h') =\tilde{A}_0(x_r,x_s,x,z,h,x',z',h') \times
\]
\begin{equation}
\label{eqn:grt2dcoeff5}
\frac{ \nabla_{x,z}  T(x_r,x+h,z)\cdot\nabla_{x,z}  T(x_s,x-h,z)
\nabla_{x',z'} T(x_r,x'+h',z')\cdot \nabla_{x',z'}T(x_s,x'-h',z')}
{\sqrt{\rho(x+h,z)}\sqrt{\rho(x-h,z)}\sqrt{\rho(x'+h',z')}\sqrt{\rho(x'-h',z')}}
\end{equation}

Asymptotic evaluation of the operator matrix $\bf{M}$ leads to one of
our two major results. The other follows by asymptotic evaluation of a
related integral. Note that

\[
\frac{\partial}{\partial t}I_t^2\frac{\partial^2}{\partial z_s \partial z_r}
I_t^2\bar{F}[\kappa,\rho][\bar{r}_{\kappa},\bar{r}_{\rho}]
\]
\[
=\pi\int dx \int dz \int dh \frac{\cos \theta_r(x_r,x+h,z)}{v(x_r,z_r)}
\frac{\cos \theta_s(x_s,x-h,z)}{v(x_s,z_s)}
\]
\[
\times a(x_r,x+h,z) a(x_s,x-h,z) 
\delta'(t-(T(x_r,x+h,z)+T(x_s,x-h,z))) 
\]
\[
\times \left(\frac{1}{\sqrt{\kappa(x+h,z)}} \bar{r}_{\kappa}(x,z,h) \frac{1}{\sqrt{\kappa(x-h,z)}}
\right. 
\]
\[
+\left. \left. 
   \nabla_{x,z}  T(x_r,x+h,z) \cdot \nabla_{x,z} T(x_s,x-h,z) 
\frac{1}{\sqrt{\rho(x+h,z)}}  \bar{r}_{\rho}(x,z,h) \frac{1}{\sqrt{\rho(x-h,z)}}
\right)\right\}
\]
\[
=\pi\int dx \int dz \int dh \frac{\cos \theta_r(x_r,x+h,z)}{v(x_r,z_r)}
\frac{\cos \theta_s(x_s,x-h,z)}{v(x_s,z_s)}
\]
\[
\times a(x_r,x+h,z) a(x_s,x-h,z) 
\frac{\partial}{\partial z}\delta(t-(T(x_r,x+h,z)+T(x_s,x-h,z))) 
\]
\[
  \times \left[-\left(\frac{\partial T}{\partial 
  z}(x_r,x+h,z) + \frac{\partial T}{\partial z}(x_s,x-h,z)\right)^{-1}\right] 
\]
\[
\times \left(\frac{1}{\sqrt{\kappa(x+h,z)}} \bar{r}_{\kappa}(x,z,h) \frac{1}{\sqrt{\kappa(x-h,z)}}
\right. 
\]
\[
+\left. \left. 
   \nabla_{x,z}  T(x_r,x+h,z) \cdot \nabla_{x,z} T(x_s,x-h,z) 
\frac{1}{\sqrt{\rho(x+h,z)}}  \bar{r}_{\rho}(x,z,h) \frac{1}{\sqrt{\rho(x-h,z)}}
\right)\right\}
\]
\[
\approx \pi\int dx \int dz \int dh \frac{\cos \theta_r(x_r,x+h,z)}{v(x_r,z_r)}
\frac{\cos \theta_s(x_s,x-h,z)}{v(x_s,z_s)}
\]
\[
\times a(x_r,x+h,z) a(x_s,x-h,z) 
\delta(t-(T(x_r,x+h,z)+T(x_s,x-h,z))) 
\]
\[
  \times \left(\frac{\partial T}{\partial 
  z}(x_r,x+h,z) + \frac{\partial T}{\partial z}(x_s,x-h,z)\right)^{-1}
\]
\[
\times \left(\frac{1}{\sqrt{\kappa(x+h,z)}} \frac{\partial}{\partial z}\bar{r}_{\kappa}(x,z,h) \frac{1}{\sqrt{\kappa(x-h,z)}}
\right. 
\]
\begin{equation}
\label{eqn:intrep2dsrbis}
+\left. \left. 
   \nabla_{x,z}  T(x_r,x+h,z) \cdot \nabla_{x,z} T(x_s,x-h,z) 
\frac{1}{\sqrt{\rho(x+h,z)}}  \frac{\partial}{\partial z} \bar{r}_{\rho}(x,z,h) \frac{1}{\sqrt{\rho(x-h,z)}}
\right)\right\}
\end{equation}
after integration by parts in $z$. The approximation in the last step
consists of leaving out all terms except those involving the $z$
derivative of $\bar{r}_{\kappa}, \bar{r}_{\rho}$, as is consistent
with the asymptotic evaluation to come: the reflectivities will be
presumed oscillatory, while all other quantities appearing in the
integral \ref{eqn:intrep2dsrbis} are smooth. In the limit of infinite
spatial frequency components dominating $\bar{r}_{\kappa},
\bar{r}_{\rho}$, that is, for singular reflectivities, the terms
retained on the RHS of \ref{eqn:intrep2dsrbis} dominate. 

Thus the operator matrix 
\[
{\bf M_1} = (I_t^2\bar{F}[\kappa,\rho])^{*}
\frac{\partial}{\partial t}I_t^2\frac{\partial^2}{\partial z_s \partial z_r}I_t^2\bar{F}[\kappa,\rho]
\]
\begin{equation}
\label{eqn:dwnopbis}
 = (I_t^2\bar{F}[\kappa,\rho])^{*}
I_t\frac{\partial^2}{\partial z_s \partial z_r}I_t^2\bar{F}[\kappa,\rho]
\end{equation}
also has components of the integral form \ref{eqn:grt2d}  
(\cite{HouSymes:17}, equation 4), in fact precisely the same as in
\ref{eqn:grt2dcoeff1}-\ref{eqn:grt2dcoeff5}, except that in the integrand,
$\partial \bar{r}_{\kappa}/\partial z$ and $\partial
\bar{r}_{\rho}/\partial z$ replace $\bar{r}_{\kappa},\bar{r}_{\rho}$ on the RHS and there is an extra factor of $(\partial
\phi/\partial z)^{-1}$.

\cite{HouSymes:17} develop an asymptotic evaluation
of the operator $M$ in definition \ref{eqn:grt2d}. 
This result involves the
rate of change of the takeoff angles {\em at the scattering points}
$\alpha_r,\alpha_s$ with respect to the source and receiver
coordinates, and also the mapping from the extended phase space point
$(x,z,h,k_x,k_z,k_h)$ to the source and receiver coordinates
$x_r,x_s$. \cite{tenKroode:12} intoduced this map and explained its
central importance: it defines the scattering (``canonical'') relation of
$D{\cal F}$. \cite{HouSymes:15} present an alternate description,
which we will follow here: the extended phase space
point determines the ray pair, by determining the traveltime gradients
which are also the ray takeoff vectors at the extended scattering
point $(x,z,h)$. The relation is homogeneous in the wavevector
$(k_x,k_h,k_z)$, so can be expressed in terms of projective
coordinates $(k_x/k_z,k_h/k_z)$ under the assumption (implicit in the
use of the horizontal offset extension) that $k_z > 0$. Accordingly we
will write $x_{r,s} = x_{r,s}(x,z,h,k_x/k_z,k_h/k_z)$ for the
emergence points of the receiver, respectively source, rays. In terms
of these quantities, the asymptotic evaluation of $M$ is given by
equation A-4 in \cite[]{HouSymes:17}:
\[
Mu(x,z,h) \approx \int 
dk_{x}dk_{z}dk_{h}\frac{e^{i(xk_{x}+zk_{z}+hk_{h})}\widehat{u}(k_{x},k_{z},k_{h})}{2\pi 
  k^2_z}
\]
\[
\times \, A(x_r(x,z,h,k_x/k_z,k_h/k_z),x_s(x,z,h,k_x/k_z,k_h/k_z),x,z,h) 
\]
\[
\times \left(\frac{\partial T}{\partial 
    z}(x_r(x,z,h,k_x/k_z,k_h/k_z),x+h,z) +
\frac{\partial T}{\partial 
    z}(x_s(x,z,h,k_x/k_z,k_h/k_z),x-h,z) \right)^{-1}
\]
\begin{equation}
\label{eqn:grt2deval}
\times |\det \mbox{
  Hess}(x_r(x,z,h,k_x/k_z,k_h/k_z),x_s(x,z,h,k_x/k_z,k_h/k_z),x,z,h)|^{-1/2} . 
\end{equation}
The coefficient $A$ is the diagonal value of the coefficient
$\tilde{A}$ in equation \ref{eqn:grt2d}, for $x'=x,z'=z,h'=h$:
\begin{equation}
\label{eqn:diag}
A(x_r,x_s,x,z,h) = \tilde{A}(x_r,x_s,x,z,h,x,z,h).
\end{equation}
The restriction $x'=x,z'=z,h'=h$ is one of the stationary phase
conditions that results from 
asymptotic evaluation of the integral in \ref{eqn:grt2d}. This
calculation also leads to the
conclusion that the vectors $(k_x,k_z)$ and $(k_h,k_z)$ are parallel
to $\nabla_{x,z}T(x_r,x+h,z)+\nabla_{x,z}T(x_s,x-h,z)$ and
$\nabla_{h,z}T(x_r,x+h,z)+\nabla_{h,z}T(x_s,x-h,z)$ respectively. This
condition in turn leads to the determination of
$\nabla_{x,z}T(x_r,x+h,z)$ and $\nabla_{x,z}T(x_s,x-h,z)$
individually, hence the rays connecting $(x\pm h,z)$ with $(x_r,z_s)$
and $(x_s,z_s)$ respectively, from the phase space vector
$(x,z,h,k_x,k_z,k_h)$, as
explained by \cite{HouSymes:15}. The determination of the rays in turn
determines $x_r,x_s$ as functions of  $(x,z,h,k_x,k_z,k_h)$, as
mentioned above.

The Hessian factor $|\det \mbox{ Hess}|^{-1/2}$ is most
conveniently expressed in terms of the ray slownesses, velocity $v
= \sqrt{\kappa/\rho}$, and slowness $s=1/v$, for which we
introduce abbreviations, regarding the arguments as understood and
dropping them from the notation:
\begin{eqnarray}
p_r & = & \frac{\partial T}{\partial x}(x_r,x+h,z) \nonumber \\  
p_s & = & \frac{\partial T}{\partial x}(x_s,x-h,z) \nonumber \\  
q_r & = & \frac{\partial T}{\partial z}(x_r,x+h,z) \nonumber \\
q_s & = & \frac{\partial T}{\partial z}(x_s,x-h,z) \nonumber \\
v_{\pm} & = & v(x\pm h,z) \nonumber \\
\label{eqn:rayslow}   
s_{\pm} & = & 1/v_{\pm}
\end{eqnarray}
Then
\begin{equation}
\label{eqn:hess}
|\det \mbox{ Hess}|^{-1/2} =\frac{1}{2} (q_r+q_s)^4\left(\frac{\partial 
    \alpha_r}{\partial x_r}\frac{\partial \alpha_s}{\partial 
    x_s}\right)^{-1}[s_-^2q_r^2 + s_+^2q_s^2 + (s_+^2+s_-^2)q_rq_s]^{-1}
\end{equation}
%Note that the right-hand sides in the system \ref{eqn:rayslow} are all
%functions of $(x,z,h,k_x,k_z,k_h)$, because of the stationary phase
%conditions, as already explained. Thus $Q$ is also a function of
%$(x,z,h,k_x,k_z,k_h)$. Since $Q$ is homogeneous of degree zero in
%$(k_x,k_z,k_h)$, it can be expressed as a function of $(k_x/k_z,
%k_h,k_z)$, as indicated in equation \ref{eqn:grt2deval}. 
(\cite{HouSymes:15}, equation A-19).

Because of the stationarity conditions $x'=x, z'=z, h'=h$, the
coefficients appearing in the asymptotic evalution of the integral
defining the matrix operator $\bf M$ simplify. It is convenient to
introduce the generalized scattering angle $\theta =
\theta(x,z,h,k_x/k_z,k_h/k_z)$, usually defined as half the angle
between source and receiver rays at the scattering point:
\[
\nabla_{x,z}T(x_r(x,z,h,k_x/k_z,k_h/k_z),x+h,z)\cdot 
\nabla_{x,z}T(x_s(x,z,h,k_x/k_z,k_h/k_z),x-h,z) 
\]
\begin{equation}
\label{eqn:thetadef}
= \frac{\cos 2\theta(x,z,h,k_x/k_z,k_h/k_z)}{v(x+h,z) v(x-h,z)} =
\frac{(1-2\sin^2\theta(x,z,h,k_x/k_z,k_h/k_z))}{v(x+h,z) v(x-h,z)}
\end{equation}
Then the 
coefficients $A = A_{i,j}, i,j \in \{\kappa, \rho\}$ appearing in the
asymptotic evaluation \ref{eqn:grt2deval} of $\bf M$ are 
\[
A_0(x_r,x_s,x,z,h)=
\]
\[
\pi^2 a(x_r,x+h,z)^2  
a(x_s,x-h,z)^2 
\]
\begin{equation}
\label{eqn:grt2dcoeff1red}
\times \frac{\cos \theta_r(x_r,x+h,z)}{v(x_r,z_r)} \frac{\cos\theta_s(x_s,x-h,z)}{v(x_s,z_s)} 
\end{equation}
\begin{equation}
\label{eqn:grt2dcoeff2red}
A_{\kappa,\kappa}(x_r,x_s,x,z,h) = 
\frac{A_0(x_r,x_s,x,z,h)}{\kappa(x+h,z)\kappa(x-h,z)}
\end{equation}
\[
A_{\rho,\kappa}(x_r,x_s,x,z,h) 
\]
\[
=\frac{A_0(x_r,x_s,x,z,h) \cos 2 \theta(x,z,h,k_x/k_z,k_h/k_z)}
{v(x+h,z)
  v(x-h,z)\sqrt{\rho(x+h,z)}\sqrt{\rho(x-h,z)}\sqrt{\kappa(x+h,z)}\sqrt{\kappa(x-h,z)}}
\]
\begin{equation}
\label{eqn:grt2dcoeff3red}
=\frac{A_0(x_r,x_s,x,z,h) \cos 2 \theta(x,z,h,k_x/k_z,k_h/k_z)}
{\kappa(x+h,z)\kappa(x-h,z)}
\end{equation}
\begin{equation}
\label{eqn:grt2dcoeff4red}
A_{\kappa,\rho}(x_r,x_s,x,z,h) =
A_{\rho,\kappa}(x_r,x_s,x,z,h)
\end{equation}
\[
A_{\rho,\rho}(x_r,x_s,x,z,h) =
\]
\[
=\frac{A_0(x_r,x_s,x,z,h) \cos^2 2 \theta(x,z,h,k_x/k_z,k_h/k_z)}
{v(x+h,z)^2v(x-h,z)^2\rho(x+h,z)\rho(x-h,z)}
\]
\begin{equation}
\label{eqn:grt2dcoeff5red}
=\frac{A_0(x_r,x_s,x,z,h) \cos^2 2 \theta(x,z,h,k_x/k_z,k_h/k_z)}
{\kappa(x+h,z)\kappa(x-h,z)}
\end{equation}
whence
\[
{\bf M} 
\left(\begin{array}{c}
\bar{r}_{\kappa} \\
\bar{r}_{\rho}
\end{array}
\right) = 
\]
\[
= \int dk_x dk_z dk_h \frac{A_0 e^{i(xk_{x}+zk_{z}+hk_{h})}}{2\pi 
  k^2_z(q_r + q_s)\kappa_+\kappa_-} |\mbox{det Hess}|^{-1/2} 
\]
\begin{equation}
\label{eqn:rk2}
\times 
\left(
\begin{array}{c}
\hat{\bar{r}}_{\kappa} + \hat{\bar{r}}_{\rho} \cos 2 \theta \\ 
(...) 
\end{array}
\right)
\end{equation}
Following the remark after equation \ref{eqn:dwnopbis}, $\bf M_1$ has
a similar expression:
\[
{\bf M_1} 
\left(\begin{array}{c}
\bar{r}_{\kappa} \\
\bar{r}_{\rho}
\end{array}
\right) = 
\]
\[
= \int dk_x dk_z dk_h \frac{A_0 e^{i(xk_{x}+zk_{z}+hk_{h})}}{2\pi 
  (-ik_z)(q_r + q_s)^2\kappa_+\kappa_-} |\mbox{det Hess}|^{-1/2} 
\]
\begin{equation}
\label{eqn:rk2bis}
\times 
\left(
\begin{array}{c}
\hat{\bar{r}}_{\kappa} + \hat{\bar{r}}_{\rho} \cos 2 \theta \\ 
(...) 
\end{array}
\right)
\end{equation}

%Define
%\begin{equation}
%\label{eqn:rsig}
%\bar{r}_{\sigma} \equiv \frac{1}{2}(\bar{r}_{\kappa} + \bar{r}_{\rho})
%\end{equation}
%Note that for $h=0$, or alternatively for physical perturbations
%$\delta \bar{\kappa}(x,z,h) = \delta \kappa(x,z)\delta(h),
%\delta \bar{\rho}(x,z,h) = \delta \rho(x,z)\delta(h)$, 
%\[
%\bar{r}_{\kappa} = \frac{\delta \kappa}{\kappa}\delta(h),
%\bar{r}_{\rho} = \frac{\delta \rho}{\rho}\delta(h),
%\]
%whence
%\[
%\bar{r}_{\sigma} = \frac{\delta \sigma}{\sigma} \delta(h), \,\sigma =
%\rho v = \sqrt{\kappa \rho}.
%\]
%That is, in the physical case, $\bar{r}_{\sigma}$ is the relative
%perturbation in acoustic impedance. In general, $\bar{r}_{\sigma}$
%does not have a simple physical meaning.

%In terms of $\bar{r}_{\sigma}$, the $\kappa$ component in equation
%\ref{eqn:rk2} may be expressed as
%\[
%{\bf M} 
%\left(\begin{array}{c}
%2\bar{r}_{\sigma}-\bar{r}_{\rho} \\
%\bar{r}_{\rho}
%\end{array}
%\right) = 
%\]
%\[
%= \int dk_x dk_z dk_h \frac{A_0 e^{i(xk_{x}+zk_{z}+hk_{h})}}{2\pi 
%  k^2_z(q_r + q_s)\kappa_+\kappa_-} |\mbox{det Hess}|^{-1/2} 
%\]
%\begin{equation}
%\label{eqn:rk2sigma}
%\times 
%\left(
%\begin{array}{c}
%2(\hat{\bar{r}}_{\sigma} - \hat{\bar{r}}_{\rho} \sin^2 \theta) \\ 
%(...) 
%\end{array}
%\right)
%\end{equation}

Equations \ref{eqn:hess} and
\ref{eqn:grt2dcoeff1red} combine to show that the common scalar factor in
the integral \ref{eqn:rk2} takes the form
\[
\frac{A_0 e^{i(xk_{x}+zk_{z}+hk_{h})}}{2\pi 
  k^2_z(q_r + q_s)\kappa_+\kappa_-} |\mbox{det Hess}|^{-1/2} 
\]
\[
=\pi^2 a(x_r,x+h,z)^2  a(x_s,x-h,z)^2 
 \frac{\cos \theta_r(x_r,x+h,z)}{v(x_r,z_r)}
\frac{\cos\theta_s(x_s,x-h,z)}{v(x_s,z_s)} 
\]
\[
\times \frac{1}{2} (q_r+q_s)^4\left(\frac{\partial 
    \alpha_r}{\partial x_r}\frac{\partial \alpha_s}{\partial 
    x_s}\right)^{-1}[s_-^2q_r^2 + s_+^2q_s^2 +
(s_+^2+s_-^2)q_rq_s]^{-1}
\]
\begin{equation}
\label{eqn:rk3}
\times \frac{e^{i(xk_{x}+zk_{z}+hk_{h})}}{2\pi 
  k^2_z(q_r + q_s)\kappa_+\kappa_-} 
\end{equation}
Similarly,  the common scalar factor in
the integral \ref{eqn:rk2bis} takes the form
\[
\frac{A_0 e^{i(xk_{x}+zk_{z}+hk_{h})}}{2\pi 
  (-ik_z)(q_r + q_s)^2\kappa_+\kappa_-} |\mbox{det Hess}|^{-1/2} 
\]
\[
=\pi^2 a(x_r,x+h,z)^2  a(x_s,x-h,z)^2 
 \frac{\cos \theta_r(x_r,x+h,z)}{v(x_r,z_r)}
\frac{\cos\theta_s(x_s,x-h,z)}{v(x_s,z_s)} 
\]
\[
\times \frac{1}{2} (q_r+q_s)^4\left(\frac{\partial 
    \alpha_r}{\partial x_r}\frac{\partial \alpha_s}{\partial 
    x_s}\right)^{-1}[s_-^2q_r^2 + s_+^2q_s^2 +
(s_+^2+s_-^2)q_rq_s]^{-1}
\]
\begin{equation}
\label{eqn:rk3bis}
\times \frac{e^{i(xk_{x}+zk_{z}+hk_{h})}}{2\pi 
  (-ik_z)(q_r + q_s)^2\kappa_+\kappa_-} 
\end{equation}

Invoke the crucial observation of Norm Bleistein, Yu
Zhang, and others, that the geometric amplitude $a(...)$ may be
expressed in terms of the rate of change of arrival time and takeoff
angles of rays reaching the source and receiver surfaces with respect
to the source or receiver coordinate
\cite[]{ZhangBleistein:03,ZhangBleistein:05,ZhangSunGray:07,ZhangYuSun:08}.
Specifically,
\[
\frac{a(x_r,x+h,z)^2}{\rho(x+h,z)}\frac{\cos
  \theta_r(x_r,x+h,z)}{v_r(x_r,z_r)}\left(\frac{\partial
    \alpha_r}{\partial x_r}(x_r,x+h,z)\right)^{-1} 
\]
\[
 = \frac{a(x_s,x-h,z)^2}{\rho(x-h,z)}\frac{\cos
   \theta_s(x_s,x-h,z)}{v_s(x_s,z_s)}\left(\frac{\partial
     \alpha_s}{\partial x_s}(x_s,x-h,z)\right)^{-1}
\]
\begin{equation}
\label{eqn:raytube}
= \frac{1}{8\pi^2}. 
\end{equation}
(\cite[]{HouSymes:15}, equations B-6, B-7 - note that for variable
density acoustics, the transport equation reads $\nabla \cdot
(a^2/\rho) \nabla T=0$). Thus writing $\rho(x\pm h,z)=\rho_{\pm}$, the
right-hand side of equation \ref{eqn:rk3} becomes
\[
\frac{A_0 e^{i(xk_{x}+zk_{z}+hk_{h})}\rho_+\rho_-}{2\pi 
  k^2_z(q_r + q_s)\kappa_+\kappa_-} |\mbox{det Hess}|^{-1/2} 
\]
\[
=\pi^2 \left(\frac{1}{8\pi^2}\right)^2 
 \frac{1}{2} (q_r+q_s)^4[s_-^2q_r^2 + s_+^2q_s^2 +
(s_+^2+s_-^2)q_rq_s]^{-1}
\]
\begin{equation}
\label{eqn:rk4}
\times \frac{e^{i(xk_{x}+zk_{z}+hk_{h})}}{2\pi 
  k^2_zv_+^2v_-^2(q_s+q_r)} 
\end{equation}
and the right-hand side of equation \ref{eqn:rk3bis} becomes
\[
\frac{A_0 e^{i(xk_{x}+zk_{z}+hk_{h})}\rho_+\rho_-}{2\pi 
  (-ik_z)(q_r + q_s)^2\kappa_+\kappa_-} |\mbox{det Hess}|^{-1/2} 
\]
\[
=\pi^2 \left(\frac{1}{8\pi^2}\right)^2 
 \frac{1}{2} (q_r+q_s)^4[s_-^2q_r^2 + s_+^2q_s^2 +
(s_+^2+s_-^2)q_rq_s]^{-1}
\]
\begin{equation}
\label{eqn:rk4bis}
\times \frac{e^{i(xk_{x}+zk_{z}+hk_{h})}}{2\pi 
  (-ik_z)v_+^2v_-^2(q_s+q_r)^2} 
\end{equation}

Combining equations \ref{eqn:rk4}, respectively \ref{eqn:rk4bis}, and \ref{eqn:grt2dcoeff1red}-\ref{eqn:grt2dcoeff5red}, obtain
\[
{\bf M} 
\left(\begin{array}{c}
\bar{r}_{\kappa} \\
\bar{r}_{\rho}
\end{array}
\right) = 
\]
\[
= \int dk_x dk_z dk_h \frac{e^{i(xk_{x}+zk_{z}+hk_{h})}(q_r+q_s)^3}{256\pi^3 
  k^2_z(s_-^2q_r^2 + s_+^2q_s^2 +
(s_+^2+s_-^2)q_rq_s)v^2_+v^2_-} 
\]
\begin{equation}
\label{eqn:rk3sigma}
\times 
\left(
\begin{array}{c}
\hat{\bar{r}}_{\kappa} + \hat{\bar{r}}_{\rho} \cos 2 \theta) \\ 
(...) 
\end{array}
\right)
\end{equation}
and
\[
{\bf M_1} 
\left(\begin{array}{c}
\bar{r}_{\kappa} \\
\bar{r}_{\rho}
\end{array}
\right) = 
\]
\[
= \int dk_x dk_z dk_h \frac{e^{i(xk_{x}+zk_{z}+hk_{h})}(q_r+q_s)^2}{256\pi^3 
  (-ik_z)(s_-^2q_r^2 + s_+^2q_s^2 +
(s_+^2+s_-^2)q_rq_s)v^2_+v^2_-} 
\]
\begin{equation}
\label{eqn:rk3sigmabis}
\times 
\left(
\begin{array}{c}
\hat{\bar{r}}_{\kappa} + \hat{\bar{r}}_{\rho} \cos 2 \theta) \\ 
(...) 
\end{array}
\right)
\end{equation}

The computations done so far are formal in nature, that is,
restrictions necessary to ensure that rays with appropriate slowness
vectors actually exist have not been imposed, nor have the
reflectivities $\bar{r}$ or the data $d$ been constrained so that the
expressions \ref{eqn:rk3sigma},  \ref{eqn:rk3sigmabis} and related equations are correct to
leading order in frequency. To make such constraints manifest and
recover an asymptotically correct result from \ref{eqn:rk3sigma}, \ref{eqn:rk3sigmabis}
requires a closer study of the ray geometry of extended acoustics.

\section{The Canonical Relation}
To make sense of the integrand in equation \ref{eqn:rk3sigma}, it is
necessary to
understand how to compute the generalized scattering angle $\theta$ in
terms of the extended phase space coordinates $(x,z,h,k_x/k_z,
k_h/k_z)$. This in turn requires computation of the ray slowness
vectors $(p_r,q_r)$ and $(p_s,q_s)$ in terms of the same
quantities. 

The critical hypothesis for horizontal subsurface offset extended
scattering is that rays carrying significant energy do not turn
horizontal, that is, $q_r, q_s > 0$ at all points on such
rays. \cite{deHoopStolk:05,deHoopStolk:06} call this assumption the
``DSR'' (double square root) hypothesis, as it is also the assumption
that permits to the wave equation to be replaced by an asymptotically
equivalent evolution equation in depth known as the double square root equation.

The stationary phase conditions include 
\begin{eqnarray}
p_r & = & \frac{1}{2}\frac{k_x+k_h}{k_z}(q_r+q_s) \nonumber \\
\label{eqn:stat}
p_s & = & \frac{1}{2}\frac{k_x-k_h}{k_z}(q_r+q_s) 
\end{eqnarray}
(\cite[]{HouSymes:15}, equation A-20). Introduce projective
coordinates $p_x=k_x/k_z,
p_h=k_h/k_z$. Then an equivalent system is
\begin{eqnarray}
p_r & = & \frac{1}{2}(p_x+p_h)(q_r+q_s) \nonumber \\
\label{eqn:stat1}
p_s & = & \frac{1}{2}(p_x-p_h)(q_r+q_s) 
\end{eqnarray}
Rearranging,
\begin{eqnarray}
p_x & = &\frac{p_r+p_s}{q_r+q_s} \nonumber \\
\label{eqn:stat2}
p_h & = &\frac{p_r-p_s}{q_r+q_s} 
\end{eqnarray}
Since $(p_r,q_r)$ and $(p_s,q_s)$ traverse the product of hemispheres
$q_r,q_s > 0$ of
radius $s_+$ and $s_-$ respectively, $(p_x,p_h)$ must lie in the image
of this product of hemispheres under the map defined in equation
\ref{eqn:stat2}. This set is not so easy to describe explicitly in
general, however for the case $h=0$ (or, more precisely, $s_+=s_-=s$),
the description is simple: in that case, 
\[
p_xp_h = \frac{p_r^2-p_s^2}{(q_r+q_s)^2} = - \frac{q_r-q_s}{q_r+q_s}
\]
\begin{equation}
\label{eqn:stat3}
=-\frac{\frac{q_r}{q_s}-1}{\frac{q_r}{q_s}+1}
\end{equation}
The map $q_r/q_s \mapsto p_xp_h$ so defined is a bijection from
$\bR^+$ to $(-1,1)$. That is, for $h=0$ necessarily
\begin{equation}
\label{eqn:stat4}
|p_xp_h| < 1.
\end{equation}
See \cite[]{tenKroode:12}, definition of the set $\Omega_X$ for a related
construction. As in that reference, we shall see that for $(p_x,p_h)$
satisfying equation \ref{eqn:stat4}, there is a unique DSR ray pair
with slowness vectors $(p_r,q_r)$ and $(p_s,q_s)$ satisfying equation 
\ref{eqn:stat1}.

To begin the construction of the inverse map from
$p_x=k_x/k_z,p_h=k_h/k_z$ to $(q_s,q_r)$ and hence the rays connecting
scattering points with sources and receivers, note that the eikonal
equation together with equation \ref{eqn:stat} implies that
\begin{eqnarray}
q_r & = & \left(s_+^2 -
          \frac{1}{4}\frac{(k_x+k_h)^2}{k_z^2}(q_s+q_r)^2\right)^{1/2} \nonumber \\
\label{eqn:eik}
q_s & = & \left(s_-^2 -
          \frac{1}{4}\frac{(k_x-k_h)^2}{k_z^2}(q_s+q_r)^2\right)^{1/2}
\end{eqnarray}
whence
\[
s_+s_-\cos 2 \theta = p_rp_s + q_r q_s 
= \frac{1}{4}\frac{k_x^2-k_h^2}{k_z^2}(q_r+q_s)^2
\]
\begin{equation}
\label{eqn:theta1}
+  \left(s_+^2 -
          \frac{1}{4}\frac{(k_x+k_h)^2}{k_z^2}(q_s+q_r)^2\right)^{1/2}
\left(s_-^2 -
          \frac{1}{4}\frac{(k_x-k_h)^2}{k_z^2}(q_s+q_r)^2\right)^{1/2}
\end{equation}

Thus the vertical ray slownesses $q_s$ and $q_r$ (equation
\ref{eqn:eik}), the phase Hessian factor in the evaluation $\bf M$
(equation \ref{eqn:hess}), and the scattering angle (equation
\ref{eqn:theta1}) may all be expressed in terms of the vertical
gradient of the two-way time at the stationary point, $q_r+q_s$. This
quantity is in turn a function of the generalized phase
coordinates: it follows from equations \ref{eqn:stat} and
\ref{eqn:eik} that its square solves a quadratic with coefficients
that are algebraic functions of $s_+,s_-$ and $k_x/k_z, k_h/k_z$
(\cite[]{tenKroode:12}, also \cite[]{HouSymes:15}, equations A-20
through A-28). Explicitly, 
\begin{equation}
\label{eqn:dphidz}
(q_r+q_s)^2 = \frac{-b+\sqrt{b^2-4ac}}{2a}
\end{equation}
where
\begin{eqnarray}
a & = & \frac{(k_x^2+k_z^2)(k_h^2+k_z^2)}{k_z^4}\nonumber \\
b & = & -2 \left[(s_+^2-s_-^2)\frac{k_xk_h}{k_z^2} +
        (s_+^2+s_-^2)\right] \nonumber \\
\label{eqn:abc}
c & = & (s_+^2-s_-^2)^2.
\end{eqnarray}
 
While explicit, the expression \ref{eqn:dphidz} is somewhat
opaque. Fortunately it simplifies considerably in the physical case,
that is, $h=0$: then $s_+=s_-$, $b=-4 s^2$, $c=0$, and 
\begin{equation}
\label{eqn:dphidzh0}
(q_r+q_s)^2 = 4\frac{k_z^4s^2}{(k_x^2+k_z^2)(k_h^2+k_z^2)}=\frac{4s^2}{(p_x^2+1)(p_h^2+1)}
\end{equation}
whence
\begin{eqnarray}
q_r & = & s\left(1 -
          \frac{(p_x+p_h)^2}{(p_x^2+1)(p_h^2+1)}\right)^{1/2} \nonumber \\
\label{eqn:eikh0}
q_s & = & s\left( 1 -
          \frac{(p_x-p_h)^2}{(p_x^2+1)(p_h^2+1)}\right)^{1/2}.
\end{eqnarray}
Since $s_+=s_-=s$ when $h=0$, all factors of $s$ in equation
\ref{eqn:theta1} are common and can be eliminated, leading to
\[
\cos 2 \theta
= \frac{(p_x^2-p_h^2)}{(p_x^2+1)(p_h^2+1)}
\]
\begin{equation}
\label{eqn:theta2}
+  \left(1 -
          \frac{(p_x+p_h)^2}{(p_x^2+1)(p_h^2+1)}\right)^{1/2}
\left(1 -
          \frac{(p_x-p_h)^2}{(p_x^2+1)(p_h^2+1)}\right)^{1/2}
\end{equation}
A bit of algebra, which we leave
to the reader, togerther with the condition that (equation
\ref{eqn:stat4}) $|p_xp_h|<1$ for $h=0$  yields
\begin{equation}
\label{eqn:theta3}
\cos 2 \theta = 
\frac{1-p_h^2}{1+p_h^2}, \, \sin \theta = \frac{p_h}{\sqrt{1+p_h^2}},
  \, \tan \theta = p_h.
\end{equation}
(Again, these identities hold only for $h=0$). It is easy to check
that $q_r,q_s$ given by \ref{eqn:eikh0} are positive, and finally set
$p_r,p_s$ by \ref{eqn:stat1}. That is, we have described a homogeneous
bijective map from $k_x,k_z,k_h$ to rays satisfying the DSR condition
in the special case of coincident scattering points $h=0$. The
explicit description of the domain of this map description for
$h \ne 0$ is less straightforward than condition \ref{eqn:stat4}.

Following \cite{tenKroode:12}, we denote by $\Omega_X$ the set of
points $(x,z,h,k_x,k_z,k_z)$ in phase space, satsifying the scattering
(stationary phase) conditions for a pair of downgoing rays. In view of
equation \ref{eqn:stat4}, 
\begin{equation}
\label{eqn:omegaxh0}
(x,z,0,k_x,k_z,k_z) \in \Omega_X \Leftrightarrow
\left|\frac{k_xk_h}{k_z^2}\right| < 1
\end{equation}
Let $B$ be
the symbol of a pseudodifferential operator essentially supported on this (conic)
set, a smoothing of the characteristic function of $\Omega_X$. When
convenient we will treat $B$ as if it were the characteristic function
of $\Omega_X$. Under the assumption that only downgoing rays carry
significant energy (technically, singularities), $D{\cal F}$ produces
asymptotically (up to a smooth error)
the same data from $(\delta \kappa,\delta \rho)$ and from 
$(Op(B)\delta \kappa, Op(B)\delta \rho)$. Thus $B$ may be regarded as
being part of the symbol of $\bf M$, without loss of generality.

\section{Near-focus Approximation}
The asymptotic evaluation in equation \ref{eqn:rk3sigma} implies that
\[
({\bf M}(\bar{r}_{\kappa},\bar{r}_{\rho})^T)_{\kappa} = \left((I_t^2\bar{F}[\kappa,\rho])^{*}
I_t^2\frac{\partial^2}{\partial z_s \partial z_r}I_t^2\,
\bar{F}[\kappa,\rho][\bar{r}_{\kappa},\bar{r}_{\rho}]\right)_{\kappa}
\]
\[
 \approx \int dk_x dk_z dk_h \frac{e^{i(xk_{x}+zk_{z}+hk_{h})}(q_r+q_s)^3}{256\pi^3 
  k^2_z(s_-^2q_r^2 + s_+^2q_s^2 +
(s_+^2+s_-^2)q_rq_s)v^2_+v^2_-} 
\]
\begin{equation}
\label{eqn:rk3mig}
\times 
(\hat{\bar{r}}_{\kappa} + \hat{\bar{r}}_{\rho} \cos 2 \theta)
\end{equation}
Similarly, \ref{eqn:rk3sigmabis} implies that
\[
({\bf M_1}(\bar{r}_{\kappa},\bar{r}_{\rho})^T)_{\kappa} = \left((I_t^2\bar{F}[\kappa,\rho])^{*}
I_t\frac{\partial^2}{\partial z_s \partial z_r}I_t^2\,
\bar{F}[\kappa,\rho][\bar{r}_{\kappa},\bar{r}_{\rho}]\right)_{\kappa}
\]
\[
 \approx \int dk_x dk_z dk_h \frac{e^{i(xk_{x}+zk_{z}+hk_{h})}(q_r+q_s)^2}{256\pi^3 
  (-ik_z)(s_-^2q_r^2 + s_+^2q_s^2 +
(s_+^2+s_-^2)q_rq_s)v^2_+v^2_-} 
\]
\begin{equation}
\label{eqn:rk3migbis}
\times 
(\hat{\bar{r}}_{\kappa} + \hat{\bar{r}}_{\rho} \cos 2 \theta)
\end{equation}
These relation
significantly simplify using the evaluation of slownesses $q_s,q_r$ for $h=0$,
explained in the last section. The error resulting from these
approximations amounts to inserting a 
pseudodifferential operator $P$ of order zero, whose symbol is $\equiv
1$ for $h=0$.
  
To begin with, \ref{eqn:dphidzh0} implies that
\[
\frac{(q_r+q_s)^3}{128\pi^3 
  k^2_z(s_-^2q_r^2 + s_+^2q_s^2 +
(s_+^2+s_-^2)q_rq_s)v^2_+v^2_-} 
\]
\[
\approx  
\frac{2sk_z^2}{(k_x^2+k_z^2)^{1/2}(k_h^2+k_z^2)^{1/2}}\frac{1}{256  
  \pi^3 s^2k_z^2v^4}
\] 
\begin{equation}
=\frac{1}{128\pi^3 v^3 \sqrt{k_x^2+k_z^2}\sqrt{k_h^2+k_z^2}}
\label{eqn:kksymbh0}
\end{equation} 
and 
\[
\frac{(q_r+q_s)^2}{256\pi^3 
  (-ik_z)(s_-^2q_r^2 + s_+^2q_s^2 +
(s_+^2+s_-^2)q_rq_s)v^2_+v^2_-} 
\]
\begin{equation}
\label{eqn:kksymbh0bis}
\approx 
\frac{1}{256 \pi^3 (-ik_z)v^2}
\end{equation}



``Approximately'' here means ``up to multiplication by a symbol that
is $\equiv 1$ for $h=0$''. So 
\[
({\bf M}(\bar{r}_{\kappa},\bar{r}_{\rho})^T)_{\kappa} 
\approx \int dk_x dk_z dk_h \frac{e^{i(xk_{x}+zk_{z}+hk_{h})} }{128 \pi^3 v^3 \sqrt{k_x^2+k_z^2}\sqrt{k_h^2+k_z^2}}PB 
\]
\begin{equation}
\label{eqn:rk4sigma}
\times 
(\hat{\bar{r}}_{\kappa} + \hat{\bar{r}}_{\rho} \cos 2 \theta ) 
\end{equation}
in which $P=P(x,z,h,k_x,k_z,k_h)$ is a symbol of order 0 satisfying
$P=1$ for $h=0$, and $B$ is the symbol mentioned at the end of the
last section. We claim that these symbols are actually of class
$S^0_{1,0}$, that is, classical symbols of order zero, and that the
RHS of equation \ref{eqn:rk4sigma} defines a classical
pseudodifferential operator of order zero. We defer the proof.

Similarly,
\[
({\bf M_1}(\bar{r}_{\kappa},\bar{r}_{\rho})^T)_{\kappa} 
= \int dk_x dk_z dk_h \frac{e^{i(xk_{x}+zk_{z}+hk_{h})} }{256 \pi^3 (-ik_z)v^2}P_1B 
\]
\begin{equation}
\label{eqn:rk4sigmabis}
\times 
(\hat{\bar{r}}_{\kappa} + \hat{\bar{r}}_{\rho} \cos 2 \theta ) 
\end{equation}
in which $P_1$ has the same properties as $P$, that is, it is also a
classical symbol of order 0. 

\begin{theorem}
Define
\begin{equation}
\label{eqn:defL}
L=(-\nabla^2_{x,z})^{1/2}(-\nabla^2_{h,z})^{1/2},
\end{equation}
\begin{eqnarray}
\label{eqn:defWm}
W_m & = & \frac{1}{16}v^{-\frac{3}{2}}L^{-1}v^{-\frac{3}{2}},\\
\label{eqn:defWminv}
W_m^{-1} & = & 16 v^{\frac{3}{2}}Lv^{\frac{3}{2}},\\
\label{eqn:defWd}
W_d &=& I_t^2 I_t\frac{\partial}{\partial z_s}I_t\frac{\partial}{\partial z_r}I_t^2
\end{eqnarray}
and 
\begin{equation}
\label{eqn:rk5sigma} 
\bar{F}[\kappa,\rho]^{\dagger}  = W_m^{-1} (\bar{F}^T W_d)_{\kappa} 
\end{equation}
Then
\begin{equation}
\label{eqn:rk6sigma}
\bar{F}[\kappa,\rho]^{\dagger}\bar{F}[\kappa,\rho][\bar{r}_{\kappa},\bar{r}_{\rho}]
\approx  
\frac{1}{8 \pi^3}\int dk_x dk_z dk_h e^{i(xk_{x}+zk_{z}+hk_{h})}B P  
 (\hat{\bar{r}}_{\kappa} + \hat{\bar{r}}_{\rho} \cos 2 \theta )  
\end{equation} 
\end{theorem}
\begin{proof}
The concluion \ref{eqn:rk6sigma} follows from \ref{eqn:rk4sigma} and the calculus of 
pseudodifferential operators.
\end{proof}

The definition \ref{eqn:rk5sigma} is equivalent to equation 21 in \cite[]{HouSymes:15}, or
equation 10 in \cite[]{HouSymes:17}, once the translation from
$\kappa$ to $v$ (with $\rho=1$) is accounted for, for the focused case
where most energy in $R$ is concentrated at $h=0$. This equivalence
will be explained in detail in the next section. 

Recall from \cite[]{HouSymes:15} that $W_d$ (defined in
\ref{eqn:defWd}) is acts as a pseudodifferential operator with a
positive real symbol of order -2, when restricted to a conic set of
covectors corresponding to downgoing source and upcoming receiver
wavefields. Therefore, for data satisfying the downgoing/upcoming
constraint, $W_d$ defines a weighted norm equivalent to the Sobolev
$-2$ norm. Since $\bar{F}$ and $\bar{F}^T$ are of order zero, in the
sense that they preserve the order of Sobolev classes, and $W_m$ is
plainly of order 2, $F^{\dagger}$ is also of order zero. Furthermore,
since $W_m$ is also positive (semi-)definite symmetric, it also
defines a norm on the a subspace of relative model perturbations equivalent
to the Sobolev 2-norm. We conclude that when restricted to relative
perturbations of the form $[\bar{r}_{\kappa},0]$,  $\bar{F}^{\dagger}$ is the
adjoint of $\bar{F}$ with respect to the norms in model and data
spaces defined by $W_m$ and $W_d$. This observation was used to
accelerate the solution of the least squares inverse problem in
\cite[]{HouSymes:16}.

A similar result is possible based on the second approximationt
developed above. Define 
\begin{eqnarray}
\label{eqn:defVminv}
V_m & = & 32 v^2 \frac{\partial}{\partial z} \\
\label{eqn:defVd}
V_d &=& -\frac{\partial}{\partial t}I_t^2 I_t\frac{\partial}{\partial z_s}I_t\frac{\partial}{\partial z_r}I_t^2
\end{eqnarray}
and 
\begin{equation}
\label{eqn:rk5sigmabis} 
\bar{F}[\kappa,\rho]^{\ddagger}  = V_m (\bar{F}^T V_d)_{\kappa} 
\end{equation}
Using the calculus again, together with equation \ref{eqn:rk4sigmabis}, obtain
\begin{equation}
\label{eqn:rk6sigmabis}
\bar{F}[\kappa,\rho]^{\ddagger}\bar{F}[\kappa,\rho][\bar{r}_{\kappa},\bar{r}_{\rho}]
\approx  
\frac{1}{8 \pi^3}\int dk_x dk_z dk_h e^{i(xk_{x}+zk_{z}+hk_{h})}B P_1
 (\hat{\bar{r}}_{\kappa} + \hat{\bar{r}}_{\rho} \cos 2 \theta ) 
\end{equation}
This result is equivalent to conclusion of \cite{HouSymes:17}, as will
be explained below.

One key difference between the representations \ref{eqn:rk5sigma} and
\ref{eqn:rk5sigmabis} is that the operators figuring in the latter,
$V_m$ and $V_d$, are not even symmetric, let alone positive definite,
so do not define alternate norms. Thus \ref{eqn:rk5sigmabis} does not
appear to exhibit $\bar{F}^{\dagger}$ as an adjoint. This appearance
is however deceptive. Note that $ik_z = |k_z|i{\rm sgn}\,k_z$. The
first factor is positive real, and the rest (the
Fourier multiplier defining the Hilbert transform)  is skew symmetric;
both can be made into symbols of order 1 and 0 respectively by
modification around $k_z=0$.  Define
\begin{eqnarray}
\label{eqn:kzkt1} 
K_zu(x,z,h) &=& \int dk_z e^{i(k_xx+k_z z+k_hh)}
                (1-\phi(k_z))|k_z|\hat{u}(k_x,k_z,k_h)\\
\label{eqn:kzkt2} 
H_zu(x,z,h) &=& \int dk_z e^{i(k_xx+k_z z+k_hh)}
                (1-\phi(k_z))i\rm{sgn}\,k_z\,\hat{u}(k_x,k_z,k_h)\\
\label{eqn:kzkt3} 
K_tu(x_r,t;x_s) &=& \int d\omega e^{i(k_{x_r} x_r+k_{x_s} x_s+\omega
                    t)}
                    (1-\phi(\omega))|\omega|\hat{u}(k_{x_r},\omega;k_{x_s}) \\
\label{eqn:kzkt4} 
H_tu(x_r,t;x_s) &=& \int d\omega e^{i(k_{x_r} x_r+k_{x_s} x_s+\omega
                    t)}
                    (1-\phi(\omega))i\rm{sgn}\,\omega\,\hat{u}(k_{x_r},\omega;k_{x_s}) 
\end{eqnarray}
in which $\phi \in C^{\infty}_0(\bR)$ is $=1$ in a neigborhood of
$0$. 
\begin{theorem} The operators defined in
  \ref{eqn:kzkt1}-\ref{eqn:kzkt4}  satisfy
\begin{eqnarray}
\frac{\partial}{\partial z} & \approx& K_z H_z = H_z K_z, \nonumber \\
\frac{\partial}{\partial t} &\approx& K_t H_t = H_t K_t, \nonumber\\
H_z \bar{F}^T &\approx& \bar{F}^T H_t.
\label{eqn:kzkt5}
\end{eqnarray}
Define 
\begin{eqnarray}
\label{eqn:defbarWm}
\bar{W}_m^{-1} &=& 32 v^2 K_z \\
\label{eqn:defbarWd}
\bar{W}_d & = & K_t I_t^2 I_t\frac{\partial}{\partial 
  z_s}I_t\frac{\partial}{\partial z_r}I_t^2\\
\label{eqn:defdd}
\bar{F}^{\ddagger} &=& \bar{W}_m^{-1}
(\bar{F}^T\bar{W}_d)_{\kappa}.
\end{eqnarray}
Then
\begin{equation}
\label{eqn:rk6sigmaterce}
\bar{F}[\kappa,\rho]^{\ddagger}\bar{F}[\kappa,\rho][\bar{r}_{\kappa},\bar{r}_{\rho}]
\approx  
\frac{1}{8 \pi^3}\int dk_x dk_z dk_h e^{i(xk_{x}+zk_{z}+hk_{h})}B P_1
 (\hat{\bar{r}}_{\kappa} + \hat{\bar{r}}_{\rho} \cos 2 \theta ) .
\end{equation}
\end{theorem}
\begin{proof}
The first two approximations in \ref{eqn:kzkt5} follow directly from
the definitions, the third from Egorov's Theorem \cite[]{Tay:81}.
Since $H_t^2 \approx -I$,  
\[ 
V_m \bar{F}^T V_d =- 32 v^2 \frac{\partial}{\partial z} \bar{F}^T  
\frac{\partial}{\partial t}I_t^2 I_t\frac{\partial}{\partial 
  z_s}I_t\frac{\partial}{\partial z_r}I_t^2  
\]
\begin{equation}
\label{eqn:getspd}
=32 v^2 K_z \bar{F}^T K_t I_t^2 I_t\frac{\partial}{\partial 
  z_s}I_t\frac{\partial}{\partial z_r}I_t^2  
\end{equation} 
and  equations \ref{eqn:getspd} and \ref{eqn:rk5sigmabis} combine to 
yield the results \ref{eqn:rk6sigmaterce}.
\end{proof}

Since $\bar{W}_m$ and $\bar{W}_d$ are positive (semi-)definite
symmetric under the same circumstances as $W_m$ and $W_d$,
$\bar{F}^{\ddagger}$ can actually be viewed as an adjoint, hence
the representation \ref{eqn:rk6sigmaterce} gives another way of seeing
$\bar{F}$ as almost-unitary.

\section{Focused Inversion}
In this section, assume that
$\bar{r}_{\kappa}$ and $\bar{r}_\rho$ are physical,
  that is,
\[
\bar{r}_{\kappa} = r_{\kappa}\delta(h), r_{\kappa} =\frac{\delta
  \kappa}{\kappa}; \,\,
\bar{r}_{\rho} = r_{\rho}\delta(h), r_{\rho}=\frac{\delta \rho}{\rho} 
\]
Also, as noted before,
\[
r_{\kappa} = r_{\rho} + 2 r_{v}, \,\,r_{\sigma} = r_{\rho}+r_{v} =
\frac{1}{2}(r_{\kappa}+r_{\rho})
\]

The first task of this section is to establish the equivalence of the
inversion formulae \ref{eqn:rk6sigma} and \ref{eqn:rk6sigmabis} with the
results presented by \cite{HouSymes:15,HouSymes:17}. The remainder
will be devoted to estimation of $r_{\kappa}$ and $r_{\rho}$ (or the
equivalent) separately.

The papers \cite[]{HouSymes:15,HouSymes:17} analyzed the mapping
$\delta v \rightarrow \delta d$, assuming that $\rho = 1$, $\delta \rho =
0$. Denote this map by $F_v$, and its subsurface offset extension by
$\bar{F}_v$. Since $\delta \rho = 0$ implies that $r_{\kappa} = 2 r_v
= 2 \delta v / v$, $F_v$ is related to $F$ by 
\[
Fr_{\kappa} = F_v \delta v = F_v (v r_{\kappa}/2
\]
whence 
\[ 
F = \frac{1}{2}F_v v
\]
For nearly focused extended models $\bar{r}_{\kappa}$, supported near
$h=0$, this implies that 
\[
\bar{F} \approx \bar{F}_v \frac{v}{2}
\]
so in that case
\begin{equation}
\label{eqn:vkap}
\bar{F}^{\dagger} \bar{F} = W_m^{-1} \bar{F}^T W_d \bar{F}  \approx
W_m^{-1} \frac{v}{2} \bar{F}_v^T W_d \bar{F}_v \frac{v}{2} \approx
W_m^{-1}\frac{v^2}{4} \bar{F}_v^T W_d \bar{F}_v
\end{equation}
From equations \ref{eqn:rk6sigma} and \ref{eqn:defWminv}, it follows
that for focused extended models, 
\begin{equation}
\label{eqn:hou1}
\bar{F}_v^{\dagger} \bar{F}_v \approx P_v
\end{equation}
where $P_v$ is a pseudodifferential operator that acts as an
approximate identity on focused extended models, and 
\begin{eqnarray}
\label{eqn:hou2}
\bar{F}_v^{\dagger} &=& W_v^{-1}\bar{F}_v^TW_d\\
\label{eqn:hou3}
W_v^{-1} &=& 4 v^5 L.
\end{eqnarray}
Equations \ref{eqn:hou1}, \ref{eqn:hou2}, and \ref{eqn:hou3} precisely
recapitulate the main result of \cite{HouSymes:15}.


The factor $B$ did not appear in these earlier analyses - for the constant density
case, it plays little role, since the method for extracting an
estimate of $r_v$ is simply stacking over $h$. The stack weight in
principle can be any function of $h$ that $=1$ at $h=0$. The choice
made in \cite[]{HouSymes:15} is to use the full range (in principle,
infinite) of $h$, which is equivalent to sampling at $k_h=0$, or $p_h=0$.
As noted in the last section, for $h=0$, $B=1$ for $|p_xp_h| < 1$, and
therefore for the purposes of extracting $r_v$ the factor $B$ can also
be ignored.

However to extract both $r_{\sigma}$ and $r_{\rho}$, it is no longer
possible to sample at $k_h=0$, since that also implies $\theta=0$,
i.e. normal incidence reflection, at which $r_{\rho}$ has no influence.
According to the identities \ref{eqn:theta3}, valid at $h=0$, the
non-oscillatory part of the integrand in \ref{eqn:rk5sigma} is
\[
B (\hat{r}_{\sigma} - \hat{r}_{\rho} \sin^2 \theta )
 = B \left(\hat{r}_{\sigma} - \hat{r}_{\rho} \frac{k_h^2}{k_z^2+k_h^2}\right)
\]
To isolate $r_{\rho}$, note that
\[
\left.-\frac{1}{2}\frac{\partial^2}{\partial k_h^2}
  k_z^2\left(\hat{r}_{\sigma} -
  \frac{k_h^2}{k_z^2+k_h^2}\hat{r}_{\rho}(k_x,k_z)\right)\right|_{k_h=0} = \hat{r}_{\rho}(k_x,k_z)
\]
The second $k_h$ derivative corresponds to multiplication by $-h^2$,
and multiplication by $k_z^2$ to application of
$-(\partial_z^2+\partial_h^2)$. Finally, restriction to $k_h=0$ is
equivalent to (unweighted) stacking over $h$. Note once again that
stacking in effect makes $B=1$, to leading order in frequency. 

The upshot is that in the focused case, the relative impedance and
density perturbations  can be extracted from the modified migrated
image volume $\bar{R}$:
\begin{eqnarray}
r_{\sigma}(x,z) &=& \int \, dh \, \bar{R}(x,z,h) \nonumber \\
r_{\rho}(x,z) &=& -\frac{1}{2}\int \, dh \,h^2
                  \frac{\partial^2}{\partial z^2}\bar{R}(x,z,h) 
\label{eqn:vdinv}
\end{eqnarray}

Note that if $\bar{R}$ is defined by \ref{eqn:rk5sigma}, then $d \approx
\bar{F}[\kappa,\rho](\bar{R},0)$. That is, the $\kappa$ component
defines an asymptotic right inverse. A precise statement of this type requires microlocal
constraints on $d$, following ten Kroode. Modulo these constraints,
this observation may be paraphrased, ``the extended
model fits everything'', in which by ``extended model'' is understood
the offset-dependent relative bulk modulus perturbation constructed
above, constructed with possibly variable background density
but zero density perturbation.

The estimates \ref{eqn:vdinv} are asymptotically correct hence can be
used to precondition the least squares estimation of the acoustic parameters.


\bibliographystyle{seg}
\bibliography{../../bib/masterref}
