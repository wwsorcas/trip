\section{Plots for DL Lecture}

\plot{fine1unbornr90}{width=\textwidth}{Simulation of shot 90
  (nonlinear)}
\plot{fine1bgunbornr90}{width=\textwidth}{Simulation of shot 90 in
  background model (nonlinear)}
\plot{fine1diffr90}{width=\textwidth}{Difference between target,
  background simulations of shot 90 (nonlinear)}
\plot{fine1linerrr90}{width=\textwidth}{Linearization error}



Regularization by DSO annihilator: minimization of 
\begin{equation}
\label{eqn:regobj}
J_{\alpha}[\bar{\kappa}] =\frac{1}{2}( \|\doF \delta \bar{\kappa}
-\delta d\|_d^2 +\alpha^2
\|A\delta \bar{\kappa}\|^2_m)
\end{equation}

\begin{equation}
\label{eqn:regnormal}
W_m^{-1}(\doF^T W_d \doF + \alpha^2 A^TA)\delta \bar{\kappa} =
W_m^{-1}\doF^T W_d \delta d
\end{equation}

Define the augmented modeling operator
\begin{equation}
\label{eqn:columnop}
L = \left(\begin{array}{c}
\doF\\
A
\end{array}
\right) 
\end{equation}
View the range of this operator as the cartesian product of data
space and model space, with 
