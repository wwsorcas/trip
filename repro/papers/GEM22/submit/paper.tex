
\documentclass[georeport,12pt]{geophysics}

\usepackage{amsmath}
\usepackage{amssymb}
\usepackage{amsthm}
\usepackage{amscd}
\usepackage{setspace}
\usepackage{wasysym}
\usepackage{algorithm}
\usepackage{algpseudocode}


\newcommand{\mb}{\mathbf}
\newcommand{\ba}{\mathbf{a}}
\newcommand{\bx}{\mathbf{x}}
\newcommand{\bxi}{\mathbf{\xi}}
\newcommand{\bk}{\mathbf{k}}
\newcommand{\bv}{\mathbf{v}}
\newcommand{\bff}{\mathbf{f}}
\newcommand{\bu}{\mathbf{u}}
\newcommand{\by}{\mathbf{y}}
\newcommand{\bz}{\mathbf{z}}
\newcommand{\bh}{\mathbf{h}}
\newcommand{\bR}{\mathbf{R}}

\newtheorem{lemma}{Lemma}
\newtheorem{theorem}{Theorem}
\newtheorem{alg}{Algorithm}
\newtheorem{remark}{Remark}
\newtheorem{cor}{Corollary}
\newtheorem{example}{Example}
\newtheorem{definition}{Defenition}



\setfigdir{.}

\begin{document}
\title{Efficient Computation of Extended Surface Sources}
\author{William W. Symes\\
 %\thanks{The Rice Inversion Project,
  Department of Computational and Applied Mathematics,\\
  Rice University,Houston TX 77251-1892 USA,\\
  email {\tt symes@rice.edu},\\
ORCID 0000-0001-6213-4272}

\lefthead{Symes}

\righthead{Approximate Source Inversion}

\maketitle
\begin{abstract}
Source extension is a reformulation of inverse problems in wave propagation, that at least in some cases leads to computationally tractable iterative solution methods. The core subproblem in all source extension methods is the solution of a linear inverse problem for a source (right hand side in a system of wave equations) through minimization of data error in the least squares sense with soft imposition of physical constraints on the source via an additive quadratic penalty. A variant of the time reversal method from photoacoustic tomography provides an approximate solution that can be used to precondition Krylov space iteration for rapid convergence to the solution of this subproblem. An acoustic 2D example for sources supported on a surface, with a soft contraint enforcing point support, illustrates the effectiveness of this preconditioner.
\end{abstract}

\noindent {\bf Keywords:} inverse problems, wave propagation, time
reversal, Krylov subspace methods, preconditioning

% \inputdir{../sseprecond/project}
\inputdir{.}

\section{Introduction}
Full Waveform Inversion (FWI) can be described in terms of 
\begin{enumerate}
\item a linear wave operator $L[{\bf c}]$, depending on a vector of
  space-dependent coefficients ${\bf c}$ and acting on causal vector wavefields $\bu$ vanishing in negative time:
\begin{equation}
\label{eqn:init}
\bu \equiv 0, t \ll 0; 
\end{equation}
\item a trace sampling operator $P$ acting on wavefields and producing data traces;
\item and a (vector) source function (of space and time) $\bff$ representing energy input to the system. 
\end{enumerate}
The basic FWI problem is: given data $d$, find ${\bf c}$ so that 
\begin{equation}
\label{eqn:fwi}
P\bu \approx d \mbox{ and } L[\bf{c}]\bu = \bff.
\end{equation}
In this formulation, the source function $\bff$ may be given, or
to be determined subject to some constraints.

%The energy source $\bff$ may also be largely undetermined, apart from some known characteristics such as localization in space and/or time. In fact, additional source degrees of freedom, beyond those needed to describe physically realized sources, may be useful in rendering the FWI problem \ref{eqn:fwi} more amenable to numerical solution, via so-called extended modeling (see \cite{geoprosp:2008}, \cite{LeeuwenHerrmannWRI:13}, \cite{HuangNammourSymesDollizal:SEG19}, and many references cited there). Therefore it is natural to view $\bff$ as also an unknown in formulating the problem \ref{eqn:fwi} via nonlinear least squares:
A simple nonlinear least squares formulation is:
\begin{equation}
\label{eqn:ols}
\mbox{choose } {\bf c} \mbox{ to minimize } \|PL[{\bf c}]^{-1}\bff -d \|^2.
\end{equation}
Practical optimization formulations typically augment the objective in
\ref{eqn:ols} by additive penalties or other constraints.

As is well-known, local optimization methods are the only feasible
approach given the dimensions of a typical instance of \ref{eqn:fwi},
and those have a tendency to stall due to ``cycle-skipping''. See for
example \cite{VirieuxOperto:09} and many references cited there. Source
extension is one approach to avoiding this problem. It consists in
imposing the wave equation as a soft as opposed to hard constraint, by
allowing the source field $\bff$ to have more degrees of freedom than
is permitted by a faithful model model of the seismic experiment, and
constraining these additional degrees of freedom by means of an
additive quadratic penalty modifying the problem \ref{eqn:ols}:
\begin{equation}
\label{eqn:esi}
\mbox{choose } {\bf c}, \bff \mbox{ to minimize } \|PL[{\bf c}]^{-1}\bff -d \|^2 + \alpha^2 \|A\bff\|^2 
\end{equation}
The operator $A$ penalizes deviation from known (or assumed)
characteristics of the source function - its null space consists of
feasible (or ``physical'') source models.

\cite{HuangNammourSymesDollizal:SEG19} present an overview of the
literature on source extension methods, describing a variety of
methods to add degrees of freedom to physical source model. The present paper
concerns {\em surface source extension}: physical sources are
presumed to be concentrated at points $\bx_s$ in space, whereas their extended
counterparts are permitted to spread energy over surfaces containing
the physical source locations. A simple choice for the penalty
operator $A$ is then multiplication by the distance $|\bx-\bx_s|$ to the physical
source location:
\begin{equation}
  \label{eqn:penop}
  (A\bff)(\bx,t) = |\bx-\bx_s|\bff(\bx,t)
\end{equation}
I shall use this choice of penalty operator whenever a specific choice
is necessary in the development of the theory below.

This paper presents a numerically efficient approach to solving the
{\em source subproblem} of problem \ref{eqn:esi}:
\begin{equation}
\label{eqn:esis}
\mbox{given } {\bf c}, \mbox{ choose } \bff \mbox{ to minimize }
\|PL[{\bf c}]^{-1}\bff -d \|^2 + \alpha \|A\bff\|^2 
\end{equation}
Solution of this subproblem is an essential component of {\em variable
  projection} algorithms for solution of the nonlinear inverse problem
\ref{eqn:esi}. Variable projection is not merely a convenient choice
of algorithm for this purpose: it is in some sense essential, see for
example \cite{Symes:SEG20}. It replaces the nonlinear
least squares problem \ref{eqn:esi} with a {\em reduced} problem, to
be solved iteratively. Each iteration involves solution of the
subproblem \ref{eqn:esis}. Therefore efficient solution of the
subproblem is essential to efficient solution of the nonlinear problem
via variable projection.

The penalty operator $A$ defined in \ref{eqn:penop} is linear, so the source
subproblem is a linear least squares problem. Under some additional
assumptions to be described below, I shall show how to construct an
accurate approximate solution operator for problem
\ref{eqn:esis}. This approximate solution operator may be used to
accelerate Krylov space methods for the solution of the surface source
subproblem \ref{eqn:esis}. Numerical examples suggest the
effectiveness of this acceleration.

I will fully describe a preconditioner for a special
case of the source subproblem \ref{eqn:esis}, in which ${\bf u}$ is an
acoustic field, $L[{\bf c}]$ is the wave operator of linear
acoustodynamics, the spatial positions of traces extracted by $P$ lie
on a depth plane $z=z_r$, and the positions at which the extended
source $\bff$ is nonzero lie on another, parallel, depth plane $z=z_s$. 
%That is, sources take the form of a combination of constitutive law
%defect and normal force,
%\begin{equation}
%  \label{eqn:surfsrc}
%  \bff(x,y,z,t) = h_s(x,y,t)\delta(z-z_s)(1,0,0,0)^T + f_s(x,y,z,t)(0,0,0,1)^T
%\end{equation}
%and the sampling operator $P$ is defined by
%\begin{equation}
%  \label{eqn:surfsam}
%  P\bu(x,y,t) =
%\end{equation}
%supported on $z=z_s$, and similar
%surface vertical loads.
This
``crosswell'' configuration simplifies the analysis underlying the
construction of approximate solutions for the source subproblem
\ref{eqn:esis}. It is only one of many transmission configurations for
which similar developments are possible. Perhaps the most important
alternative example is the diving wave configuration, which plays a
central role in contemporary FWI. 

The preconditioner construction is very similar to the time-reversal
method in photoacoustic tomography
\cite[]{StefanovUhlmannIP:09}. Preconditioning amounts to a change of
norm in the domain and range spaces of the modeling operator. In this
case, the modfied norms are weighted $L^2$, and the weight operators
map pressure to corresponding surface source on the source and
receiver planes. This pressure-to-source map is closely related to the ``hyperbolic
Dirichlet-to-Neumann'' operator that plays a prominent role in
photoacoustic tomography and other wave inverse problems
\cite[]{Rachele:00,StefUhl:05}. \cite{HouSymes:EAGE16} demonstrated a
very similar preconditioner for Least Squares Migration, also for its
subsurface offset extension \cite[]{HouSymes:16}, motivated by
\cite{tenKroode:12}. These constructions also involve the
Dirichlet-to-Neumann operator. This concept also turns up in hidden
form in the work of Yu Zhang and collaborators on true amplitude
migration
\cite[]{YuZhang:14,TangXuZhang:13,XuWang:2012,XuZhangTang:11,Zhang:SEG09}.

The obvious computation of the pressure-to-source map - prescribe the
pressure, solve the wave equation with this boundary condition, read
off the equivalent source - suffers from intrinsic numerical
inaccuracy. I suggest an alternative computationally feasible
approach, via economical short-distance wave propagation. Since the
map is symmetric only in an approximate, asymptotic sense, it must be
symmetrized for use as a Krylov preconditioner. I describe a
symmetrization procedure that requires no further wave computations
beyond those necessary to compute the action of the operator itself.

The discussion in this paper is formal and incomplete, in the sense
that some important mathematical underpinnings are only
sketched. I will treat the modeling operator $PL[{\bf c}]^{-1}$ as if
it mapped square integrable surface sources to square integrable sampled
data. This is not true in full generality: while the surface source
problem has distribution solutions, they are not generally square
integrable (finite acoustic field energy). Even if the solutions have
finite energy, they do not in general have well-defined
restrictions to lower-dimensional sets. In other words, the action of
the sampling
operator $P$ on the space-time plane $z=z_r$ is not well-defined for arbitrary
finite-energy acoustic fields. Thus the modeling operator envisioned above
may not be well-defined.

This
phenomenon is related to the ill-posedness of wave equations as
evolution equations in spatial variables, an observation attributed to
Hadamard (see \cite{CourHil:62}, Chapter 6, section 17). Some constraint on the acoustic field,
beyond finite energy, is mandatory in any precise mathematical formulation the inverse problems
\ref{eqn:esi} and \ref{eqn:esis}. In fact, the natural constraint in
the ``crosswell'' geometry of this paper is that high-frequency energy
travel {\em only} along rays crossing the
surfaces $z=z_s, z=z_r$ transversally. That is, source functions on
$z=z_s$ generate waves with energy traveling along rays leaving the surface at a non-zero angle, and
energy arrives at the recording surface $z=z_r$ along rays making
non-zero angle with it. I will call sources, sampled data, and
acoustic fields with this property {\em
  downgoing} (even though the concept also encompasses {\em upcoming}
propagation). Note that the downgoing property restricts the behaviour of
acoustic fields near the source and receiver surfaces - what the
fields do elsewhere is their own business.

Several works have explored the mathematics of the
downgoing condition and its consequences in the context of the scalar second order wave equation, see for instance
\cite{Payn:75,Symes:83,Lasi:86,LasLionsTrig:86,Lasi:87, BaoSy:91b}. 
Elaboration of these mathematical details is beyond the scope of this
paper, which aims instead to explore the algorithmic
consequences of the mathematical structure implied by the downgoing
condition.

The next section defines the modeling operator $PL[{\bf c}]$, its
adjoint, and important specializations (pressure vs. normal velocity
sources and data). The sections to follow define the
source-to-pressure operator, construct an approximate inverse of the
modeling operator by time reversal (as suggested by work in
photoacoustic tomography), use the source-to-pressure operator to
express the approximate inverse as the modeling operator adjoint in
weighted norms (thus establishing that the modeling operator is {\em approximately
  unitary} in the sense of these norms), explain how to use this
construction to precondition Conjugate Gradient iteration, and
organize the preconditioning computation so as to involve only one
extra and relatively inexpensive
wave propagation calculation. The penultimate section displays simple
2D numerical examples of all of the key steps, culminating in a
comparison of straight vs. preconditioned CG iteration. The paper ends
with a brief discussion-and-conclusion section, reviewing what has
been accomplished and listing a few of the many questions left open.

\section{Operators}

For acoustic wave physics, the coefficient vector is
$\bf{c}=(\kappa,\rho)^T$, with components bulk modulus $\kappa$ and
density $\rho$, and the state vector $\bu=(p,\bv)^T$ consists of
pressure $p$ (a scalar space-time field) and particle velocity $\bv$
(a vector space-time field). The wave operator $L[\bf{c}]$ is:
\begin{equation}
\label{eqn:aweop}
L[\bf{c}]\bf{u} = 
\left(
\begin{array}{c}
\frac{1}{\kappa}\frac{\partial p}{\partial t}  + \nabla \cdot \bv, \\
\rho\frac{\partial \bv}{\partial t} + \nabla p.
\end{array}
\right) 
\end{equation}
That is,
\begin{equation}
  \label{eqn:awemat}
  L[{\bf c}] = \left(
    \begin{array}{cc}
      \frac{1}{\kappa}\frac{\partial}{\partial t} & \nabla \cdot \\
      \nabla & \rho \frac{\partial}{\partial t}
    \end{array}
  \right)
\end{equation}
$L[{\bf c}]$ has a well-defined inverse in the sense of distributions
if it is restricted to either causal or anti-causal vector wavefields.

Most of what follows is valid for any space dimension $n >0$. The
coefficient vector $\bf{c}=(\kappa,\rho)$ is defined throughout space
$\bR^n$, the state vector $\bu$ throughout space-time
$\bR^{n+1}$. Whenever convenient for mathematical manipulations,
$n=3$: for instance, I will write $\bx=(x,y,z)^T$ for the spatial
coordinate vector, and refer to the third (vertical) coordinate of
particle velocity as $v_z$. Examples
later in this paper will use $n=2$ for computational convenience.

Since all of the operators in the discussion that follows depend on
the coefficient vector 
$\bf{c}$, I will suppress it from the notation, for example, $L=L[\bf{c}]$.

The surface source extension replaces point sources on or near a
surface in $\bR^3$ with source functions confined to the same
surface. The simplest example of this extended geometry specifies a
plane $\{(x,y,z,t): z=z_s\}$ at source depth $z_s$ as the surface. For
acoustic modeling, surface sources are combinations of constitutive law
defects and loads normal to the surface, localized on $z=z_s$. That
is, right-hand sides in the system $L\bu=\bff$ take the form
$\bff(\bx,t) = (h_s(x,y,t)\delta(z-z_s),
f_s(x,y,t)\bf{e}_z\delta(z-z_s))^T$ for scalar defect $h_s$ and normal
force $f_s$ ($\bf{e}_z=(0,0,1)$). With the choice $L$ given in
\ref{eqn:awemat}, the causal/anti-causal wave system $L\bu^{\pm}=\bff$
takes the form
\begin{eqnarray}
\label{eqn:awepm}
\frac{1}{\kappa}\frac{\partial p^{\pm}}{\partial t} & = & - \nabla \cdot \bv^{\pm} +
h_s \delta(z-z_s), \nonumber \\
\rho\frac{\partial \bv^{\pm}}{\partial t} & = & - \nabla p^{\pm} +
                                                f_s{\bf e} \delta(z-z_s),\nonumber \\
p^{\pm} & =& 0 \mbox{ for } \pm t \ll 0,\nonumber\\ 
\bv^{\pm} & = & 0 \mbox{ for } \pm t \ll 0.
\end{eqnarray}

\noindent {\bf Remark:} In system \ref{eqn:awepm} and many similar
systems to follow, I will use the shorthand
\[
  p^+ = 0 \mbox{ for } t \ll 0 
\]
to mean that $p^+$ is {\em causal}, that is,
\[
  \mbox{For some } T \in \bR, p^+(\cdot,t) = 0 \mbox{ for all } t <
  T.
\]
Similarly,
\[
  p^- = 0 \mbox{ for } t \gg 0 
\]
signifies that $p^-$ is anti-causal.

Extended forward modeling consists in solving \ref{eqn:awepm} and
sampling the solution components at receiver locations. For
simplicity, throughout this paper I will assume that the receivers are
located on another spatial hyperplane $\{(x,y,z,t): z=z_r\}$ at
receiver depth $z_r>z_s$. The constructions to follow involve interchange
of the roles of $z_s$ and $z_r$ (that is, locating sources on $z=z_r$
and receivers at $z=z_s$), so rather than the sampling operator $P$ of
the introduction, I will denote by $P_s,P_r$ the sampling 
operators on $z=z_s$, $z=z_r$ respectively. In practice, sampling
occurs at a discrete array of points (trace locations) on these
surfaces, and over a zone of finite extent. In this theoretical
discussion, I will neglect both finite sample rate and extent, and
regard the data, for example $P_rp^+$, as continuously sampled and
extending over the entire plane $z=z_r$.

As explained in the Introduction, the downgoing constraint on the square-integrable
source functions $h_s, f_s$ is
essential both for finite energy solutions of the system
\ref{eqn:awepm} to exist, and for these solutions to have well-defined
traces on the receiver surface $z=z_r$. This constraint will be
assumed throughout, often tacitly.
For downgoing solutions of system \ref{eqn:awepm}, the key
components ($p^{\pm}$ and $v^{\pm}_z$) are continuous functions of $z$
in the open slab $z_s<z<z_r$ with well-defined limits at the boundary
planes, but may be discontinuous at the source plane
$z=z_s$. Similarly, the roles of $z_s$ and $z_r$ will be interchanged
in some of the constructions to come, and the corresponding solutions
may be discontinuous at $z=z_r$. Accordingly, interpret $P_s$, $P_r$
as the limit from right and left respectively: for $u=p^{\pm}$ or
$v^{\pm}_z$,
\begin{eqnarray}
  \label{eqn:defsamp}
  P_su(x,y,t) &=& \lim_{z \rightarrow z_s^+} u(x,y,z,t),\nonumber \\
  P_ru(x,y,t) &=& \lim_{z \rightarrow z_r^-} u(x,y,z,t).                  
\end{eqnarray}


The causal/anti-causal vector
modeling operators ${\cal S}^{\pm}_{z_s,z_r}$ are defined in terms of
the solutions $(p^{\pm},\bv^{\pm})$ of the systems \ref{eqn:awepm} by
\begin{equation}
  {\cal S}^{\pm}_{z_s,z_r}(h_s,f_s)^T  = (P_rp^{\pm},P_r v_z^{\pm})^T,
  \label{eqn:fwd}
\end{equation}
The subscript signifies that sources are located on $z=z_s$, the
receivers on $z=z_r$. It is necessary to include this information in
the notation, as versions of ${\cal S}^{\pm}$ with sources and receivers in
several locations will be needed in the discussion below.

\noindent {\bf Remark:} To connect with the formulation presented in
the introduction, note that for continuous $u$,
$P_su(x,y,t)=u(x,y,z_s,t)$, and therefore the adjoint of $P_s$ (in the
sense of distributions) is $P_s^Th(x,y,z,t) =
h(x,y,t)\delta(z-z_s)$. Write ${\cal P}_s = \mbox{diag }(P_s,P_s)$ and
similarly for ${\cal P}_r$. Then
\[
  {\cal S}^{+}_{z_s,z_r} = {\cal P}_r L^{-1}({\cal P}_s)^T,
\]
in which $L^{-1}$ is interpreted in the causal sense, and similarly
for ${\cal S}^{-}$. Sources confined to $z=z_s$ are precisely those
functions (distributions, really) output by ${\cal P}_s^T$, so the
problem statements \ref{eqn:esi} and \ref{eqn:esis} can be rewritten
in terms of ${\cal S}^+_{z_s,z_r}$, with $P$ identified with ${\cal P}_r$.

${\cal S}^{\pm}$ is not stably invertible: its columns are
approximately linearly dependent, as will be verified below. The
diagonal components of ${\cal S}^{\pm}$ thus carry essentially all of
its information, and it is in terms of these that a sensible inverse problem
is defined.

Denote by $\Pi_i, i=0,1$ the projection on the first,
respectively second, component of a vector in $\bR^2$. The 
forward modeling operator from pressure source to pressure trace is
\begin{equation}
  \label{eqn:sdef}
  S^{\pm}_{z_s,z_r} = \Pi_0 {\cal S}^{\pm}_{z_s,z_r} \Pi_0^T 
\end{equation}
and the forward modeling operator from velocity source (normal force)
to velocity trace is
\begin{equation}
  \label{eqn:vdef}
  V^{\pm}_{z_s,z_r} = \Pi_1 {\cal S}^{\pm}_{z_s,z_r} \Pi_1^T 
\end{equation}

With these conventions, we can write the version of the source
subproblem \ref{eqn:esis} studied in this paper as
\begin{equation}
  \label{eqn:esisp}
  \mbox{find }h_s\mbox{ to minimize }\|S^{+}_{z_s,z_r}h_s- d\|^2 +
  \alpha^2\|Ah_s\|^2.
\end{equation}


It follows from the adjoint state method (see Appendix A for details) that
\begin{equation}
  \label{eqn:sadj1}
  ({\cal S}^{\pm}_{z_s, z_r})^T = -{\cal S}^{\mp}_{z_r,z_s}
\end{equation}

Define $R$ to be the {\em time-reversal operator} on functions of
space-time, $Rf(\bx,t) = f(\bx,-t)$, and ${\cal R}$ to be the {\em
  acoustic field time-reversal operator}
\begin{equation}
  \label{eqn:trdef}
  {\cal R} \left(
    \begin{array}{c}
      p\\
      \bv
    \end{array}
  \right) =
  \left(
    \begin{array}{c}
      Rp\\
      -R\bv
    \end{array}
  \right)
\end{equation}
Then 
\begin{equation}
  \label{eqn:trsadj}
  {\cal R}{\cal S}^{\mp} = -{\cal S}^{\pm}_{z_r,z_s}{\cal R}
\end{equation}
Since $R^2 = I$ and ${\cal R}^2 = I$, the identities \ref{eqn:sadj1} and \ref{eqn:trsadj} imply that
\begin{equation} 
  \label{eqn:trtr}
 ({\cal S}^{\pm}_{z_s,z_r})^T = {\cal R}{\cal S}_{z_r,z_s}^{\pm}{\cal R}=
 -{\cal S}^{\mp}_{z_r,z_s}.
\end{equation}
The relation \ref{eqn:trtr} implies that
\begin{eqnarray}
  (S^{\pm}_{z_s,z_r})^T &=& -S^{\mp}_{z_r,z_s} \nonumber\\
                        &=& R S^{\pm}_{z_r,z_s}R, \nonumber\\
    (V^{\pm}_{z_s,z_r})^T &=& -V^{\mp}_{z_r,z_s} \nonumber\\
                        &=& R V^{\pm}_{z_r,z_s}R.
                            \label{eqn:trtrcomp}
\end{eqnarray}




\section{Pressure-to-Source}

Since the system \ref{eqn:awepm} has a unique solution by standard
theory \cite[]{Lax:PDENotes}, the source vector field $(h_s,f_s)$
determines the acoustic field $(p^{\pm},\bv^{\pm})$ in space time, and
in particular the limits from the right at $z=z_s$, $P_sp^{\pm}$ and
$P_sv_z^{\pm}$. This relation is not invertible: it is not possible to
prescribe both pressure and normal velocity on a surface such as
$z=z_s$. So the columns of the matrix operator
${\cal S}^{\pm}_{z_s,z_r}$ must satisfy a linear relation. In this
section I will explain this relation; it involves the {\em
  pressure-to-source} map. This operator also turns out to be the
principal component of a preconditioning strategy for iterative
solution of the optimization problem \ref{eqn:esis}, so I will devote
some effort to its proper definition. It is closely related to the
Dirichlet-to-Neumann operator mentioned in the introduction.

While it is not possible to prescribe both pressure and velocity on
$z=z_s$ in solutions of \ref{eqn:awepm}, it is possible to
prescribe pressure only, for instance: if the function $\phi$ on
the surface $z=z_s$ satisfies suitable conditions, for
example the downgoing constraint mentioned earlier, a unique solution
exists for the acoustic system in both half-spaces $\pm z > z_s$:
\begin{eqnarray}
\label{eqn:awe0}
  \frac{1}{\kappa}\frac{\partial p_{\pm}}{\partial t} & = & - \nabla \cdot \bv_{\pm}, \nonumber \\
  \rho\frac{\partial \bv_{\pm}}{\partial t} & = & - \nabla
                                                    p_{\pm}, \nonumber \\
  p_{\pm} & =& 0,  \mbox{ for } t \ll 0, \nonumber\\ 
  \bv_{\pm} & = & 0 \mbox{ for } t \ll 0, \nonumber\\
  \lim_{z \rightarrow z_s^{\pm}}p_{\pm}(x,y,t,z)& =& \phi(x,y,t).
\end{eqnarray}
Note that the subscript $\pm$ here refers to the sign of $z-z_s$, as opposed
to the superscript ${\pm}$, which refers to the sign of $t$ throughout
this paper.

From the boundary condition (last equation in \ref{eqn:awe0}), one
sees that the pressures $p_{\pm}$ in the two half-spaces have the same
limit at the boundary $z=z_s$. Stick the two half-space
solutions together to form an acoustic field $(p^+,\bv^+)$ in all of
space-time, that is,
\begin{equation}
  \label{eqn:awealt}
  p^+(x,y,z,t) =
  \left\{
    \begin{array}{c}
      p_+(x,y,z,t) \mbox{ if } z>0,\\
      p_-(x,y,z,t) \mbox{ if } z<0,
    \end{array}
  \right.
\end{equation}
and a similar definition for $\bv^+$. Then $p^+$ is continuous across
$z=z_s$, and the boundary condition in system \ref{eqn:awe0} may be
written as $P_sp^+=\phi$.

The same construction can be carried out in the anti-causal sense,
with anti-causal half-space solutions glued together to form a
full-space distribution solution $(p^-,\bv^-)$, with the property that
$p^-$ is continuous across $z=z_s$ and $P_sp^-=\phi$.

The reader may object that the notation $(p^\pm,\bv^\pm)$ is already in
use, for the solution of \ref{eqn:awepm}. This objection is
valid. However, {\em in the sense
  of distributions}, $(p^{\pm},\bv^{\pm})$ as defined in display
\ref{eqn:awealt}, is {\em exactly} the causal solution of \ref{eqn:awepm}
for the choice $h_s = -[v^{\pm}_{z}]|_{z=z_s}, f_s=0$, as follows from a
simple integration-by-parts calculation. So the notation is consistent!

The negative jump $-[v^{\pm}_{z}]|_{z=z_s}$ is thus a function of $\phi$. Define
the {\em pressure-to-source} operator $\Lambda^{\pm}_{z_s}$ by
\begin{equation}
  \label{eqn:deflam}
  \Lambda^{\pm}_{z_s}\phi = -[v^{\pm}_{z}]|_{z=z_s}
\end{equation}
The conclusion: if $h_s = \Lambda^{\pm}_{z_s}\phi$ and $f_s=0$ in the
system \ref{eqn:awepm}, then $\phi=P_sp^{\pm}$.

Otherwise put, $S^{\pm}_{z_s,z_s}\Lambda^{\pm}_{z_s} \phi = \phi$, so
$\Lambda^{\pm}_{z_s}$ is inverse to $S^{\pm}_{z_s,z_s}$. The relation
\ref{eqn:trtrcomp} implies in turn that
\begin{equation}
  \label{eqn:lamadj}
  (\Lambda^{\pm}_{z_s})^T = - \Lambda^{\mp}_{z_s}
\end{equation}

There is also a {\em velocity-to-source} operator. For the solution
$(p^{\pm},\bv^{\pm})$ of system \ref{eqn:awepm} with $h_s=0$, the
normal component of velocity, $v^{\pm}_z$, is continuous across
$z=z_s$, and the velocity source (vertical load)
$f_s=-[p^{\pm}]_{z=z_s}$. I will not name the velocity-to-source
operator, as it does not appear explicitly in the developments to
follow. As will be seen, it is essentially the inverse of the
pressure-to-source operator.

The quadratic form defined by $\Lambda^{\pm}_{z_s}$ has fundamental
physical significance. Define the total acoustic energy $E^{\pm}(t)$ of the
field $(p^{\pm},\bv^{\pm})$, at time $t$ by
\begin{equation}
  \label{eqn:defae0}
  E^{\pm}(t) = \frac{1}{2} \int \,d\bx \, \left(\frac{(p^{\pm})^2}{\kappa} + \rho |\bv^{\pm}|^2\right)(\bx,t).
\end{equation}
Then
\begin{equation}
  \label{eqn:elim}
  \pm \lim_{\pm t \rightarrow \infty} E^{\pm}(t) =  \langle P_sp^{\pm},
  (\Lambda^{\pm}_{z_s} P_sp^{\pm}) \rangle_{L^2(z=z_s)}.
\end{equation}
That is, the value of the quadratic form defined by
$\Lambda^{\pm}_{z_s}$, evaluated at the pressure trace on $z=z_s$,
gives the total energy transferred from the source to the
acoustic field over time. Since $E$ is itself a positive definite
quadratic form in the acoustic field, it follows that $\pm
\Lambda^{\pm}_{z_s}$ is positive semi-definite. 

While $\Lambda^{\pm}_{z_s}$ is positive semi-definite, it is not
symmetric. However, it is {\em approximately symmetric} in the
high-frequency sense. This fact follows from a
geometric optics analysis of the half-space solution. This leads to
the identification of $\Lambda^{\pm}_{z_s}$ as a {\em
  pseudodifferential operator} of order zero on $z=z_s$, with principal symbol
\begin{equation}
  \label{eqn:lamsym}
  \sigma_0(\Lambda^{\pm}_{z_s}) = \pm 2(\kappa(\bx)
  \rho(\bx))^{1/2}\left(1-\frac{\kappa(\bx)(\xi^2+\eta^2)}{\rho(\bx)\omega^2}\right)^{-1/2}.
\end{equation}
Here $\xi$, $\eta$, and $\omega$ are the dual Fourier variables to
$x$, $y$, and $t$ respectively. The downgoing assumptions means that
for local planewave components of $P_sp$, the quantity inside the
square root is positive. Thus $\Lambda^{\pm}_{z_s}$ has real principal
symbol (in fact, the entire symbol is real) hence defines an
asymptotically symmetric operator:
\begin{equation}
  \label{eqn:lamappsim}
  (\Lambda^{\pm}_{z_s})^T \approx \Lambda^{\pm}_{z_s}.
\end{equation}
(For more on this, see \cite{StefUhl:05}.)
The analysis also reveals that the solution components not continuous
at $z=z_s$ are odd there:
\begin{equation}
  \label{eqn:odd1}
  \lim_{z\rightarrow z_s^+} v^{\pm}_{z} \approx - \lim_{z\rightarrow z_s^-}
  v^{\pm}_{z}
\end{equation}
for the solution of \ref{eqn:awepm} with $f_s=0$.
Similarly, 
\begin{equation}
  \label{eqn:odd2}
  \lim_{z\rightarrow z_s^+} p^{\pm}\approx - \lim_{z\rightarrow z_s^-}
  p^{\pm}
\end{equation}
for the solution of \ref{eqn:awepm} with $h_s=0$. Here ``$\approx$''
means in the sense of high frequency asymptotics, that is, that the
difference between the two sides is relatively smooth, hence small if
the data is highly oscillatory. Therefore if $f_s=0$ in system \ref{eqn:awepm},
\begin{equation}
  h_s = \Lambda^{\pm}_{z_s}P_sp^{\pm} = -[v^{\pm}_{z}]|_{z=z_s} \approx -2
  P_sv^{\pm}_{z}
  \label{eqn:tracejump10}
\end{equation}
Similarly, if $h_s=0$ in system \ref{eqn:awepm}, then
\begin{equation}
  \label{eqn:tracejump20}
  f_s = -[p^{\pm}]|_{z=z_s} \approx -2 P_s p^{\pm}.
\end{equation}
Thus $f_s$ determines approximately the boundary value of $p^{\pm}$,
as a solution of the acoustic wave system in the half-space
$z>z_s$. However, as repeated in equation \ref{eqn:tracejump10}, a
solution with this boundary value is also the restriction to $z>z_s$
of a solution to \ref{eqn:awepm} with $f_s=0$ and $h_s=
\Lambda^{\pm}_{z_s}P_sp^{\pm}$. Therefore if
\begin{equation}
  \label{eqn:hfcondn}
  h_s =-\frac{1}{2}\Lambda^{\pm}_{z_s}f_s,
\end{equation}
then the pressure boundary value $P_sp^{\pm}$ is the
same for the solutions of \ref{eqn:awepm} for source vectors $(h_s,0)$
and $(0,f_s)$. Since the pressure boundary values are the same, the solutions
in $z>z_s$ are the same. In particular, since $z_r>z_s$ and ${\cal
  S}^{\pm}_{z_s,z_r}(h_s,f_s)^T = (P_rp^{\pm},P_rv^{\pm}_z)^T$, it follows
that
\begin{equation}
  \label{eqn:snull}
  {\cal S}^{\pm}_{z_s,z_r}\left(\frac{1}{2}\Lambda^{\pm}_{z_s}f_s,f_s\right)^T \approx 0.
\end{equation}

Equation \ref{eqn:snull} states the relation between the columns of $
{\cal S}^{\pm}_{z_s,z_r}$ mentioned in the introduction to this
section.


\section{Time Reversal}

Recall that the source vector $(h_s,f_s)$ is assumed to produce a
downgoing field $(p^+,\bv^+)$, that is, emanates high-frequency energy only along
rays that make an angle with the vertical bounded below by a common
minimum angle. Such rays leave $\Omega$ within a common maximum
time. Consequently (Appendix B), in the
slab $z_s<z<z_r$, the field $(p^+,\bv^+)$ approximates the solution of an
anti-causal evolution equation. Choose $\chi(t)$ to be a smooth function
that is $= 0$ for $t \gg 0$ and $=1$ at times when near rays carrying
high-frequency energy in $(p^+,\bv^+)$ cross $z=z_r$. Define 
$(\tilde{p}^-,\tilde{\bv}^-)$ to be the solution in the half-space
$\Omega \times \bR$ of
\begin{eqnarray}
\label{eqn:revawe}
  \frac{1}{\kappa}\frac{\partial \tilde{p}^-}{\partial t} & = & - \nabla \cdot \tilde{\bv}^-, \nonumber \\
  \rho\frac{\partial \tilde{\bv}^-}{\partial t} & = & - \nabla \tilde{p}^-,\nonumber \\
  \tilde{p}^- & =& 0,  \mbox{ for } t \gg 0\\ 
  \tilde{\bv}^- & = & 0 \mbox{ for } t \gg 0\\
  P_r\tilde{p}^- &=& \chi P_rp^+ . 
\end{eqnarray}
That is, $\tilde{p}^-$ has the same boundary value on $z=z_r$ as
$p^+$, except for low-frequency residue that is muted by
$\chi$. Therefore
$p^+ \approx \tilde{p}^-, \bv^+ \approx \tilde{\bv}^-$ near
$z=z_r$. Since the right-hand sides in the system \ref{eqn:awepm} are
singular only on $z=z_s$, and the high-frequency components of
$(p^+,\bv^+)$ are carried by downgoing rays, these differ negligibly
from the the high-frequency components of
$(\tilde{p}^-,\tilde{\bv}^-)$ in the space-time slab $z_s<z<z_r$, and
the approximation holds throughout this region. In particular
$P_sv^+_z \approx P_s \tilde{v}^-_z$. In view of the relation
\ref{eqn:tracejump10},
\begin{equation}
  \label{eqn:tildevtohsubs}
  -2P_s\tilde{v}^-_z \approx h_s,
\end{equation}
so solution
of the system \ref{eqn:revawe} followed by restriction to $z=z_s$ and
multiplication by $-2$ 
approximately inverts the map $S^+_{z_s,z_r}: h_s \mapsto P_rp^+$.

Next observe that in view of the relation \ref{eqn:tracejump20}, and
the downgoing nature of the ray system carrying the high frequency
energy in $(p^+,\bv^+)$, the field $(\tilde{p}^-,\tilde{\bv}^-)$ is
actually the restriction to $z<z_r$ of the anti-causal solution of \ref{eqn:awepm}
with $z_s$ replaced by $z_r$, zero constitutive defect, and vertical
load given by the jump in pressure at $z=z_r$ - for this field, use
the same notation. Continuity of vertical
velocity $\tilde{v}^-_z$ at $z=z_r$ implies that the vertical load is
\[
  f_r = -[\tilde{p}^-]|_{z=z_r} =-(\lim_{z\rightarrow
    z_r^+}\tilde{p}^- - \lim_{z\rightarrow
    z_r^-}\tilde{p}^-)
\]
\[
  \approx 2 P_r \tilde{p}^- = 2 P_r p^+
\]
(from the definition \ref{eqn:defsamp}, $P_r$ is the limit from the
left). Thus
\[
  P_s \tilde{v^-_z} \approx V^-_{z_r,z_s}(2 P_rp^+) \approx
  2V^-_{z_r,z_s}S^+_{z_r,z_s}h_s.
\]
so
\[
  h_s \approx -2 P_s v^+_z \approx -2 P_s \tilde{v}^-_z \approx
  -4V^-_{z_r,z_s}S^+_{z_r,z_s}h_s
\]
Combine this observation with \ref{eqn:tildevtohsubs} to obtain
\[
 -4  V^-_{z_r,z_s} S^+_{z_s,z_r}  \approx  I,
\]
This relation combines with the identity \ref{eqn:trtrcomp} to
yield the first main result of this section:
\begin{eqnarray}
  \label{eqn:approxinv}
  (V^+_{z_s,z_r})^T S^+_{z_s,z_r} & \approx & \frac{1}{4}I, \nonumber\\
  (S^+_{z_s,z_r})^T V^+_{z_s,z_r} & \approx & \frac{1}{4}I, \nonumber\\
  V^+_{z_s,z_r} (S^+_{z_s,z_r})^T & \approx & \frac{1}{4}I, \nonumber\\
  S^+_{z_s,z_r} (V^+_{z_s,z_r})^T & \approx & \frac{1}{4}I.
\end{eqnarray}.
The second equation is simply the transpose of the first, and the
last two follow by by an exactly analogous argument using time
reversal and interchange of the roles of $z_s$ and$z_r$.

The conclusion is significant enough to merit restating in English:
provided that high-frequency energy in the various fields is carried
along downgoing ray fields, the transpose of $V^+$ is an approximate
inverse to $S^+$, modulo a factor of 4. To recover the pressure source
$h_s$ generating a pressure gather $P_rp$ at $z=z_r$, multiply the
latter by -2, then apply the transpose of $V^+_{z_s,z_r}$ to this
gather, reading out a vertical velocity field at $z=z_s$. Multiply
again by -2 and you have a high-frequency approximation to $h_s$.


\section{Unitarity}

The next chapter in this story recognizes the relations in display
\ref{eqn:approxinv} as asserting the approximate unitarity of
$S^+_{z_s,z_r}$.

The matrix identity \ref{eqn:snull} implies a relation between $S, V,$
and $\Lambda$ of some interest in itself. After minor re-arrangement, the second row of reads
\begin{equation}
  \label{eqn:snull2}
-\frac{1}{2}\Pi_1{\cal S}^{\pm}_{z_s,z_r}\Pi_0^T\Lambda^{\pm}_{z_s}  \approx
V^{\pm}_{z_s,z_r}.
\end{equation}
In these relations, the projection on the left picks out the vertical velocity component
of a downgoing wavefield at $z=z_r$: that is,
\[
-\frac{1}{2}\Pi_1{\cal S}^{\pm}_{z_s,z_r}\Pi_0^T\Lambda^{\pm}_{z_s}P_sp^+
=-\frac{1}{2}P_r v_z^+,
\]
where $(p^+, \bv^+)$ solve the system \ref{eqn:awepm} with $f_s=0$ and
$h_s = \Lambda^{\pm}_{z_s}P_sp^+$. On the other hand, from relation
\ref{eqn:tracejump10},
\[
  P_r v_z^+ = -\frac{1}{2}\Lambda^+_{z_r}P_r p^+
\]
where
\[
  P_r p^+ = \Pi_0{\cal S}^{+}_{z_s,z_r}\Pi_0^T\Lambda^{+}_{z_s}P_s
  p^+
\]
\[
  = S^+_{z_s,z_r}\Lambda^{+}_{z_s}P_sp^+
\]
Therefore combining the last two equations with \ref{eqn:snull2},
obtain
\begin{equation}
  \label{eqn:sv}
  \frac{1}{4}\Lambda^+_{z_r}S^+_{z_s,z_r}\Lambda^{+}_{z_s} = V^+_{z_s,z_r}.
\end{equation}
This is the promised relation.


As shown in the last section, $4(V_{z_s,z_r}^+)^T$ is approximately
inverse to $S^{+}_{z_s,z_r}$. Therefore, transposing both sides of
equation \ref{eqn:sv} and using \ref{eqn:approxinv}, obtain
\begin{equation}
  \label{eqn:almostunitary}
  4(V_{z_s,z_r}^+)^TS^+_{z_s,z_r} = [ (\Lambda^+_{z_s})^T
  (S^{+}_{z_s,z_r})^T(\Lambda^+_{z_r})^T]S^{+}_{z_s,z_r} \approx I.
\end{equation}

The remarkable feature of the identity \ref{eqn:almostunitary} is that
it exhibits an approximate right inverse of $S^+$ as an adjoint with
respect to a weighted inner product - or it would, if the operators
$(\Lambda^+)$ were symmetric positive definite. As noted earlier,
these operators are only approximately symmetric, though they are
positive semi-definite. That is not a great obstacle, however:
symmetrizing them in the obvious way commits a negligible error, of
the sort that this paper already neglects wholesale. That is,
\begin{equation}
  \label{eqn:unitary}
  [ \frac{1}{2}((\Lambda^+_{z_s})^T+ \Lambda^+_{z_s})
  (S^{+}_{z_s,z_r})^T \frac{1}{2}((\Lambda^+_{z_r})^T+
  \Lambda^+_{z_r})]S^{+}_{z_s,z_r} \approx I.
\end{equation}


The symmetrized $\Lambda$ operators are at least positive
semi-definite, hence define (at least) semi-norms.
Similar relations have been derived for other scattering operators,
and have been used to accelerate iterative solutions of inverse
scatering problems: \cite{DafniSymes:SEG18b} review some of this
literature.

\section{Accelerated Iterative Inversion}

For convenience, in this section write $S$ in place of
$S^+_{z_s,z_r}$. Also abbreviate the symmetrized $\Lambda$ operators
using notation suggesting weight
operators in model and data spaces:
\begin{eqnarray}
  W_m^{-1}&=& \frac{1}{2}((\Lambda^+_{z_s})^T+
              \Lambda^+_{z_s}),\nonumber \\
  W_d &=& \frac{1}{2}((\Lambda^+_{z_r})^T+ \Lambda^+_{z_r}).
          \label{eqn:wdef}
\end{eqnarray}
The identification of the symmetrized $\Lambda^+_{z_s}$ as the inverse
of another operator $W_m$ is formal, since the former operator is
likely to have null (or nearly-null) vectors due to aperture-related
amplitude loss. Since some version of $W_m$ is essential in the
formulation for effective preconditioning, I will derive a usable
candidate to stand in for it below.

Adopting Hilbert norms defined by the operators $W_m$ and $W_d$ in its
domain and range respectively, the adjoint of $S$ is given by
\begin{equation}
\label{eqn:wadj}
S^{\dagger} = W_m^{-1}S^TW_d,
\end{equation}

In this notation, the relation \ref{eqn:unitary} takes the form
\begin{equation}
  \label{eqn:wunitary}
  S^{\dagger}S \approx I.
\end{equation}
That is to say, $S$ is approximately unitary with respect to the
weighted norms defined by $W_m$ and $W_d$. Therefore a Krylov space
method employing these norms will converge rapidly, at least for the
well-determined components of the solution.

The most convenient arrangement the Conjugate Gradient (CG) algorithm
taking advantage of the structure \ref{eqn:wadj} is the {\em
  Preconditioned CG}. Allowing that the fit error will be measured by
the data space norm, the least squares problem to be solved is not
just $Sh \approx d$, but a regularized version:
\begin{equation}
  \label{eqn:einv}
  \mbox{minimize}_h \|Sh-d\|^2_d + \alpha^2 \|Ah\|^2_m
\end{equation}

\noindent {\bf Remark:} recall that the modified data space norm $\|d\|_d^2 = \langle
d, W_d d\rangle$ has physical meaning: for acoustics, it is
proportional to the power transmitted to the fluid by the source.

The minimizer of the objective defined in equation \ref{eqn:einv}
solves the normal equation
\begin{equation}
  \label{eqn:norm0}
  (S^{\dagger}S + \alpha^2 A^{\dagger}A)h = S^{\dagger}d 
\end{equation}
where the weighted adjoint $S^{\dagger}$ has already been defined in equation \ref{eqn:wadj}, and $A^{\dagger}$ is the adjoint of $A$ in the weighted model space norm defined by $W_m$, namely
\begin{equation}
  \label{eqn:aadj}
  A^{\dagger} = W_m^{-1}A^TW_m.
\end{equation}

Note that the normal operator appearing on the left-hand side of
\ref{eqn:norm0} is not an approximate identity, due to the presence of
the regularization term: the spectrum increases in spread with
increasing $\alpha$, leading to slower convergence. Fortunately for
the present setting, the operators $W_m^{-1}$, $A$, and $W_m$
approximately commute (they are scalar {\em pseudodifferential}, once
the difficulties with the definition of $W_m$, mentioned above, are
taken care of). Scalar pseudodifferential operators approximately
commute, so $A^{\dagger} \approx A^T$. Therefore
\begin{equation}
  \label{eqn:normapprox}
  S^{\dagger}S + \alpha^2 A^{\dagger}A \approx I + \alpha^2A^TA
\end{equation}
Recall that $A$ is simply multiplication by the Euclidean distance to
the physical source point $\bx_s$: $A u (\bx) = |\bx-\bx_s|u(\bx),
A^TAu(\bx) = |\bx-\bx_s|^2u(\bx)$. So the equation $(I+\alpha^2
A^TA)u=b$ is trivial to solve, and this is a key characteristic of a
good preconditioner. However this observation must be combined with
the weighted norm structure.

Rewrite the normal equation \ref{eqn:norm0} as
\begin{equation}
  \label{eqn:norm1}
  W_m^{-1}(S^TW_dS + \alpha^2 A^TW_mA)h = W_m^{-1}S^TW_md 
\end{equation}
Since $W_m$ is self-adjoint and positive semidefinite, the common factor on both sides of \ref{eqn:norm1} can be re-written as
\begin{equation}
  \label{eqn:normpart}
  Nh = (S^*S + \alpha^2 A^*A)h = S^*d 
\end{equation}
in which $S^*, A^*$ are the adjoints with the original (Euclidean)
inner product in the domains but the weighted inner product in data
space:
\begin{eqnarray}
  \label{eqn:sadjwt}
  S^* &=& S^T W_d,\\
  A^* &=& A^T W_m.
\end{eqnarray}
Note the $S^*S$ and $A^*A$ are symmetric in the Euclidean sense, so
equation \ref{eqn:normpart} is a symmetric positive (semi-)definite
linear system, just the sort of thing for which the 
The Preconditioned Conjugate Gradient (``PCG'') algorithm was
designed. PCG for solution
of equation \ref{eqn:normpart} with preconditioner $M$ is usually
written as Algorithm 1 (see for example \cite{Golub:2012}):

\begin{algorithm}[H]
\caption{Preconditioned Conjugate Gradient Algorithm, Standard Version}
\begin{algorithmic}[1]
\State Choose $h_0=0$ 
  \State $r_0 \gets S^*d$
  \State $p_0 \gets M^{-1} r_0$
  \State $g_0 \gets p_0$
  \State $q_0 \gets Np_0$
  \State $k \gets 0$
  \Repeat
  \State $\alpha_k \gets \frac{\langle g_k,r_k \rangle}{\langle p_k,q_k\rangle}$
  \State $h_{k+1} \gets h_k + \alpha_k p_k$
  \State $r_{k+1} \gets r_k - \alpha_kq_k$
  \State $g_{k+1} \gets M^{-1} r_{k+1}$
  \State $\beta_{k+1} \gets \frac{\langle g_{k+1},r_{k+1}\rangle}{\langle g_k,r_k\rangle}$
  \State $p_{k+1}\gets g_{k+1}+\beta_{k+1}p_k$
  \State $q_{k+1} \gets Np_{k+1}$
  \State $k \gets k+1$
  \Until{Error is sufficiently small, or max iteration count exceeded} 
\end{algorithmic}
\end{algorithm}
The iteration converges rapidly if $M^{-1}N \approx I$. This is true
if and only if the symmetrized operator $M^{-1/2}NM^{-1/2} \approx I$,
which is in turn true if the eigenvalues of $M^{-1/2}NM^{-1/2}$ are
close to 1 (actually works well is most of these eigenvalues are close
to 1, and the rest are small - which is the case for the current
problem)..  Further, PCG is computationally effective is M is easy to
invert.

From \ref{eqn:normapprox} and \ref{eqn:norm1}, it follows that
\[
  W_m^{-1}(S^TW_dS + \alpha^2 A^TW_mA) \approx I + \alpha^2 A^TA.
\]
This observation suggests using $M=W_m(I+\alpha^2A^TA)$. This choice
is not symmetric, but since the operators on the right-hand side are
scalar pseudodifferential hence commute, it is equivalent to use of
\begin{eqnarray}
  M         &=&(I+\alpha^2A^TA)^{1/2}W_m(I+\alpha^2A^TA)^{1/2},\nonumber \\
  M^{-1}
            &=&(I+\alpha^2A^TA)^{-1/2}W_m^{-1}(I+\alpha^2A^TA)^{-1/2}.
                \label{eqn:defprecond}
\end{eqnarray}
With this choice, \ref{eqn:normapprox} implies that
$M^{-1}N \approx I$, also $M$ is symmetric. As already mentioned,
powers of $I + \alpha^2A^TA$ are trivial to compute, given the choice
of $A$ made here. We will examine fast algorithms for computing
$W_m^{-1}$ = the symmetrized pressure-to-source operator in the next
section. Note that only $M^{-1}$, hence only $W_m^{-1}$, appears in
Algorithm 1.


\section{Computing and Symmetrizing $\Lambda$}

Computations of $\Lambda^{\pm}_{z_s}$ and its transpose are clearly critical steps in an
implementation of the PCG algorithm outlined in the preceding section.
Direct computation of the pressure-to-source operator $\Lambda^{\pm}_{z_s}$, for instance by solving
\ref{eqn:awepm} and reading off $P_sv^{\pm}_{z}$, turns out to be
numerically ill-behaved. The relation \ref{eqn:snull} provides and
alternative approach, taking advantage of the accurate approximate
inverse to $S^+_{z_s,z_r}$ constructed above. The first row of
\ref{eqn:snull}, slightly rearranged, is
\begin{equation}
  \label{eqn:lamidea}
  \Pi_0{\cal S}^+_{z_s,z_r}\Pi_1^T f_s \approx  -\frac{1}{2} S^+_{z_s,z_r}\Lambda^+_{z_s}f_s.
\end{equation}
The approximate inverse construction for $S^++_{z_s,z_r}$ permits (approximate)
solution of this equation for $\Lambda^+_{z_s}f_s$: apply
$4(V^{+}_{z_s,z_r})^T$ to both sides of equation \ref{eqn:lamidea} and
use the first equation in the list \ref{eqn:approxinv} to get
\begin{equation}
  \label{eqn:lamident}
  \Lambda^+_{z_s} \approx -8(V^{+}_{z_s,z_r})^T\Pi_0 {\cal S}^+_{z_s,z_r}\Pi_1^T.
\end{equation}
This identity is the major result of this section: it shows how
to compute that action of $\Lambda^+_{z_s}$ by propagating the input
pressure trace, identified as a source for the velocity evolution,
forward in time from $z_s$ to $z_r$
reading off the pressure trace on $z=z_r$, identifying it once more as
a point load (source for velocity), propagating it backwards in time from $z_r$ to
$z_s$, and finally reading off the velocity trace, interpreted as a
pressure evolution source on $z_s$. 

The importance of this result lies in the failure of the obvious
method for computing the action of $\Lambda^{\pm}_{z_s}$, namely to
employ the pressure trace as a source in the velocity equation ($f_s$,
in the notation used above) at $z=z_s$, and read off the velocity
field also at $z=z_s$. This difficulty is related to the existence of
tangentially propagating waves and the lack of continuity of the trace
operator. The method implicit in equation \ref{eqn:lamident} avoids
this difficulty by propagating the fields a positive distance in $z$:
assuming as always that the causal fields are downgoing, this step
eliminates any tangentially propagating fields from consideration.


A deeper study of the pressure-to-source operator (or of the closely
related Dirichlet-to-Neumann operator for the second order wave
equation, see \cite{StefUhl:05}) shows that it is approximately
dependent only on the model coefficients near the source surface
($z=z_s$ in this case). Since the homogeneous and lens models are
identical near this surface, it is unsurprising that these figures are
very close to the previous two.  However an even more useful
observation is that the calculations in the approximation
\ref{eqn:lamident} could just as well be carried out in a much smaller
region around the source surface, and produce a result that is
functionally identical in that it will serve as a source for the same
acoustic fields globally, with small error. In effect, equation
\ref{eqn:lamident} involving propagation from source ($z=z_s$) to
receiver ($z=z_r$) surfaces is altered by replacing $z_r$ with a
receiver datum $z_s+\Delta z$ considerably closers to $z_s$:
\begin{equation}
  \label{eqn:lamnear}
  \Lambda^+_{z_s} \approx -8(V^{+}_{z_s,z_s+\Delta z})^T\Pi_0 {\cal
    S}^+_{z_s,z_s+\Delta z}\Pi_1^T.
\end{equation}
Using a receiver datum closer to the source surface has two favorable consequences:
\begin{itemize}
\item The computational domain can be smaller than is necessary to
  simulate the target data, as it need only contain the source
  surface and the receiver datum implicit in
  equation \ref{eqn:lamident}. This shrinkage of the computational
  domain can lead to substantial improvements in computational
  efficiency.
\item Since the receiver data may be chosen much closer to the
  source surface that is the case for the target data, the effective
  aperture active in the relation \ref{eqn:lamident} can be much
  larger, producing an estimated source gather much less affected by
  aperture limitation.
\end{itemize}


As mentioned in the last section, computation of the transpose of
$\Lambda^+$ (exact, not approximate in the high frequency sense) is
critical to the successful construction of the preconditioner. The
relation \ref{eqn:lamident} does not provide a computation for this
operator. However set
\begin{equation}
  \label{eqn:lamtilde}
  \tilde{\Lambda}^+_{z_s} = -8(V^{+}_{z_s,z_s+\Delta z})^T\Pi_0 {\cal
    S}^+_{z_s,z_s+\Delta z}\Pi_1^T.
\end{equation}
Then \ref{eqn:lamnear} can be rewritten
\[
  \Lambda^+_{z_s} \approx \tilde{\Lambda}^+_{z_s}.
\]
Of course, all of the examples so far show images of
$\tilde{\Lambda}^+_{z_s}$.

Since successful preconditioning requires only approximate inversion,
use of $\tilde{\Lambda}^+_{z_s} $ in place of $\Lambda^+_{z_s}$ will
still yield a working preconditioner, and the former can be transposed
to machine precision via the definition \ref{eqn:lamtilde} and the adjoint state
method (equations \ref{eqn:trtr} \ref{eqn:trtrcomp}):
\begin{equation}
  \label{eqn:lamtransp}
  (\tilde{\Lambda}^+_{z_s})^T = -8 \Pi_1 ({\cal S}^+_{z_s,z_s+\Delta
    z})^T\Pi_0^T V^{+}_{z_s,z_s+\Delta z}
\end{equation}


The model space weight operator $W_m^{-1}$ introduced in the last section is
replaced by its asymptotic approximation
\[
\frac{1}{2}(\tilde{\Lambda}^+_{z_s} +
  (\tilde{\Lambda}^+_{z_s})^T)
\]
\[
  \approx -8 \left((V^{+}_{z_s,z_s+\Delta z})^T\Pi_0 {\cal
    S}^+_{z_s,z_s+\Delta z}\Pi_1^T \right.+
\]
\[
  \left. \Pi_1 ({\cal S}^+_{z_s,z_s+\Delta
      z})^T\Pi_0^T V^{+}_{z_s,z_s+\Delta z}\right)
\]
\begin{equation}
  \label{eqn:winvcomp}
  = -4(\Pi_1 ({\cal S}^+_{z_s,z_s+\Delta z})^T(\Pi_0^T\Pi_1 +
  \Pi_1^T\Pi_0) {\cal S}^+_{z_s,z_s+\Delta z}\Pi_1^T) = \tilde{W}_m^{-1} .
\end{equation}
with a similar definition for the replacement $\tilde{W}_d$ of $W_d$.

This identity shows that only one forward and one adjoint simulation
are necessary to compute the action of $\tilde{W}_{m,d}$. The operator
in the center of the expression on the right-hand side, $\Pi_0^T\Pi_1
+ \Pi_1^T\Pi_0$, simply exchanges the components of the acoustic
fields, passing the velocity field as a pressure source and the
pressure field as a velocity source.


One more computation is required for the full implementation of the
preconditioning strategy explained in the last section: $W_m$ is
required, not just $W_m^{-1}$. Note that $W_m$ plays two roles in the
second term in equation \ref{eqn:norm1}: it is the weight matrix for
both the domain and range norms for $A$. It is perfectly OK for one of
these to be replaced by an asymptotic approximation, so long as it is
symmetric and computable (and at least semi-definite). The second row
in equation \ref{eqn:snull} appears as \ref{eqn:snull2} above:
introducing (formally) the inverse of $\Lambda^+$,
\begin{equation}
  \label{eqn:snull2mod}
-\frac{1}{2}\Pi_1{\cal S}^{+}_{z_s,z_r}\Pi_0^T  \approx
V^{+}_{z_s,z_r}(\Lambda^{+}_{z_s})^{-1}
\end{equation}
whence from the second line in display \ref{eqn:approxinv}
\begin{equation}
  \label{eqn:laminv}
-\frac{1}{8}(S^+_{z_s,z_r})^T\Pi_1{\cal S}^{+}_{z_s,z_r}\Pi_0^T  
\approx (\Lambda^{+}_{z_s})^{-1}
\end{equation} 
and
\begin{equation}
  \label{eqn:laminvtr}
-\frac{1}{8}\Pi_0({\cal S}^{+}_{z_s,z_r})^T\Pi_1^T S^+_{z_s,z_r}
\approx ((\Lambda^{+}_{z_s})^{-1})^T.
\end{equation}
Using the definition \ref{eqn:sdef} of $S^+_{z_s,z_r}$, the
symmetrized $\Lambda^{-1}$ is
\begin{equation}
  \label{wcomp}
 \tilde{W}_m = -\frac{1}{16}\left(\Pi_0 ({\cal S}^{+}_{z_s,z_r})^T
   (\Pi_1^T \Pi_0 + \Pi_0^T \Pi_1) {\cal S}^{+}_{z_s,z_r} \Pi_0^T\right)
   \approx \frac{1}{2}((\Lambda^{+}_{z_s})^{-1}
   +((\Lambda^{+}_{z_s})^{-1})^T) .
\end{equation}
Comparison with the definition \ref{eqn:winvcomp} shows that
$\tilde{W}_m$ and $\tilde{W}_m^{-1}$ differ only in the initial and
final projection factors (and overall scale), and in particular either
can be computed for the cost of a forward/adjoint operator pair. Note
that $\tilde{W}_m^{-1}$ is inverse to $\tilde{W}_m$ only in an
approximate (asymptotic, aperture-limited) sense.

\section{Numerical Examples}
This section illustrates the most important conclusions developed in
the preceding sections by finite difference wavefield simulation.

\subsection{Synthetic models and simulation}
To illustrate the structure described in the preceding section, I
introduce two 2D acoustic models, one spatially homogeneous, the other
highly refractive. The first, homogenous model has $\kappa = 4$
GPa and $\rho = 1$ g/cm$^3$ throughout a rectangular domain of size 8 km ($x$) $\times$ 4 km
($z$). The second, refractive, model is a perturbation of the first by
a low-velocity acoustic lens positioned in the center of the
rectangle (Figure \ref{fig:bml0}. To produce this structure, the density is chosen
homogeneous as in the first model, while the bulk modulus decreases to
from 4 GPa outside the lens to 1.6 GPA in its center, as shown in Figure \ref{fig:bml0}.

\plot{bml0}{width=\textwidth}{Bulk modulus, lens model. Color scale is 
in GPa. Positions of point source and receiver line indicated.}

Discretization is conventional, with a rectangular grid and staggered
finite difference scheme \cite[]{Vir:84} of order 2 in time and
2$k$ in space; for most of the experiments reported below,
$k=4$. Absorbing boudary conditions of perfectly matched layer type
are applied at all boundaries of the simulation rectangle \cite[]{Habashy:07}.
Sampling operators such as $P_r$ are implemented via linear
interpolation, and source insertion via adjoint linear interpolation
(as noted above, in the continuum limit, sources are represented via
adjoint sampling). Steps in $x$ and $z$ are the same. In the following
examples, $\Delta x = 20$ m. This choice limits the temporal frequency
of accurately computed fields to rougly 12 Hz.

\cite{GeoPros:11} gives a description of the code
implementation, out-of-date in a few respects but overall accurate.
The implementation uses the discrete adjoint state method and
auto-generated code \cite[]{TapenadeRef13}, to assure that the
computed adjoint operators are adjoint at the level of machine
precision to the computed operators. The reverse-time storage issue is
resolved through the optimal checkpointing technique
\cite[]{Griewank:book,Symes:06a-pub}, again without loss of
precision. This procedure results in computed adjoints for
$S^+_{z_s,z_r}$ and other operators that pass usual test for adjoint
accuracy, comparing inner products with pseudorandom input vectors,
with errors well under machine precision.

%Typical results with pseudorandom input traces $d_r, h_z, w_r, f_s$ are:
%\begin{itemize}
%\item computed $\langle d_r, S^+_{z_s,z_r}h_s\rangle = -2.41069174,
%  \langle (S^+_{z_s,z_r})^Td_r, h_s \rangle = -2.41069388.$
%\item  computed $\langle w_r, V^+_{z_s,z_r}f_s\rangle = -2.73362470,
%  \langle (V^+_{z_s,z_r})^Tw_r, f_s \rangle = -2.73362136.$
%\end{itemize}
%Ideally, the differences of these inner products should be at most a
%relatively small multiple of machine precision, {\em relative} to the
%products $\|d_r\|\| S^+_{z_s,z_r}\|\|h_s\|$ and
%$\|w_r\|\|V^+_{z_s,z_r}\|\|f_s\|$ (division by these quantities makes the result
%dimensionless and scale-independent). The operator norm $\| S^+_{z_s,z_r}\|$ is
%computationally inaccessible, so instead I used the smaller quantities
%$\|d_r\|\| S^+_{z_s,z_r}h_s\|$ etc. as stand-ins - thereby
%overestimating the relative error between the inner products. In all
%cases, over a very large number of random inputs, the largest observed
%relative error estimate was $O(10^{-9})$, well under the appropriate limit for
%single precision.

%We also explore the dependence of a few results on frequency, using
%$\Delta x = 10$ m and $5$ m, to accomodate 25 and 50 Hz respectively,
%maintaining 8 samples per wavelength. All computations are carried out
%in single precision.

The horizontal line of receivers sits at depth $z_r = $ 1000 m, the
(extended) sources at $z_s=3000$ m.  %as
% shown in Figure \ref{fig:bml0}.
Source and receiver $x$ ranges from $2000$ to
$6000$ m. Note that we have
reversed the order relation between $z_s$ and $z_r$ described in the
text ($z_s<z_r$). This difference is immaterial for the purpose of
illustrating the mathematical structures developed in the preceding paragraphs.
%A single point (physical) source appears in these
%experiments, located at $x_s=3500$ m, $z_s=3000$ m, as also indicated
%in the figure.

%Extended sources are confined to the horizontal
%line through the physical source position, that is $z_s = 3000$ m,
%over a 4 km interval starting at $x_r=$ 2000 m..

%The point source pressure data generated by this configuration for the
%homogeneous model is displayed in Figure \ref{fig:recphh0}, for the
%lens model in \ref{fig:recplh0}. Triplication of arrivals is evident in the latter.

%While inversion of pressure data is the main object of this exercise,
%normal velocity ($v_z$) data will play an important role, so I display
%the corresponding gathers in Figures \ref{fig:recvzhh0} and \ref{fig:recvzlh0}.

\subsection{Creating downgoing fields}
The downgoing condition constrains high-frequency energy of localized
plane wave components, hence could be enforced by dip>
filtering. However, a simpler approach is to construct fields that
must be entirely downgoing at the source and receiver surfaces by
virtue of ray geometry.

Note that a point source
on $z=z_s$ creates high frequency energy traveling on rays parallel
and nearly parallel to $z=z_s$, so that won't do. However, placing a
point source at a depth $z_d<z_s$ will work. Since the examples used here are
homogeneous in $z<z_s$, and the sampling region for extended sources
is a finite interval, all rays carrying high frequency energy cross
the source surface $z=z_s$ at a postive angle, and the field and its
traces are {\em a priori} downgoing. The same is obviously true at the
receiver surface for the homogeneous model, but is also true for the
lens model, as no rays are refracted horizontally {\em at the receiver surface}.

The choice of a point source at
$z_d=3500$ m, $x_d=3500$ m,  bandpass filter wavelet with
corner frequencies $1, 2.5, 7.5, 12.5$ Hz, gives the causal pressure and
velocity gathers at $z=z_s=3000$ m
shown in Figures \ref{fig:dsrcphh0} and \ref{fig:dsrcvzhh0}.  Since the
mechanical parameters in the homogeneous and lens models are the same
for $z<z_s$, and no rays return to this zone in either model, these
data are asymptotically the same for both models, and I show only the
homogenous medium results.
%for the lowest frequency
%source, and scaled as appropriate for examples with higher frequency
%and finer sampling.
%wavelet as used for prior examples,

\plot{dsrcphh0}{width=\textwidth}{Trace $P_sp^+$ on $z=z_s=3000$ m of
  pressure field from point source at $z_d=3500$ m, $x_d=3500$ m,
  bandpass filter source. }

\plot{dsrcvzhh0}{width=\textwidth}{Trace $P_sv_z^+$ on $z=z_s=3000$ m of
  vertical velocity field from point source at $z_d=3500$ m, $x_d=3500$ m,
  bandpass filter source.}

\subsection{Equivalence of pressure, velocity sources}
These gathers are the pressure and
velocity traces
$(P_sp^+,P_sv^+_z)$ on $z=z_s$ of a downgoing acoustic field in
$z<z_s$, hence related by the operator $\Lambda^+_{z_s}$.
Equations \ref{eqn:snull}, \ref{eqn:tracejump10} and
\ref{eqn:tracejump20} show that these differ by a factor of -2 from
source functions $f_s$ and $h_s$ in the system \ref{eqn:awepm},
with $h_s=0$ and $f_s=0$ respectively, that generate the same acoustic
field in $z<z_s$, and in particular the same receiver traces on
$z=z_r$, at least asymptotically.

%To give an example of \ref{eqn:snull} in action, it is necessary to
%create downgoing data and the action of the operator $\Lambda^{\pm}$
%on a downdoing field data $f_s$ on $z=z_s$.

%These source functions satisfy
%the relation \ref{eqn:hfcondn}, therefore source vectors $(h_s,0)$ and
%$(0, f_s)$ generate {\em the same acoustic field} $(p^+,\bv^+)$ in
%$z>z_s$.
Figures
%\ref{fig:drecphh0}, \ref{fig:dfwdphh0}, \ref{fig:daltphh0} (homogeneous model) and
\ref{fig:drecplh0}, \ref{fig:dfwdplh0}, \ref{fig:daltplh0} show the pressure
gathers extracted at $z_r=1000$ m for the point source at $z=z_d$ and
for the two choices of extended source at $z=z_s$, on the same color
scale. The obvious similarity between the fields generated by the two
extended sources,
predicted by equation \ref{eqn:snull}, is confirmed by trace
comparisons in figures %\ref{fig:drecphh0tr21}-\ref{fig:daltphh0tr81}
%and
\ref{fig:drecplh0tr81},\ref{fig:daltplh0tr81}. Other traces are
equally similar.


\plot{drecplh0}{width=\textwidth}{Pressure gather at receiver depth 
  $z_r=1000$ m from field generated by causal solution of acoustic 
  system \ref{eqn:awepm} in the lens model described in the 
  text, with point pressure source (constitutive defect) at $z_d=3500$
  m, $x_d=3500$ m.}

\plot{dfwdplh0}{width=\textwidth}{Pressure gather at receiver depth 
  $z_r=1000$ m from field generated by causal solution of acoustic 
  system \ref{eqn:awepm} in the lens model described in the 
  text, with extended pressure source (constitutive defect) on
  $z=z_s=3000$ m given bythe field depicted in Figure
  \ref{fig:dsrcvzhh0} scaled by -2 
   ($h_s=-2P_sv_z^+=\Lambda^+_{z_s}P_sp^+$) and zero velocity source (vertical load) 
  ($f_s=0$).}

\plot{daltplh0}{width=\textwidth}{Pressure gather at receiver depth 
  $z_r=1000$ m from field generated by causal solution of acoustic 
  system \ref{eqn:awepm} in the lens model described in the 
  text, with extended velocity source (vertical load) on $z=z_s=3000$
  m given by the field depicted in Figure \ref{fig:dsrcphh0} scaled by -2
  ($f_s=-2P_sp^+$) and zero pressure source (constitutive defect)
  ($h_s=0$).}

\plot{drecplh0tr81}{width=\textwidth}{Overplot of traces 81 ($x=3600$) 
  from gathers shown in \ref{fig:drecplh0} (blue), \ref{fig:dfwdplh0}
  (red).}

\plot{daltplh0tr81}{width=\textwidth}{Overplot of traces 81 ($x=3600$)
  from gathers shown in \ref{fig:drecplh0} (blue), \ref{fig:daltplh0}
  (red).}
% time rev

\subsection{Inversion by time reversal}
I have applied the approximate inversion procedure suggested in
equation \ref{eqn:approxinv} to the pressure gather shown in Figure
% \ref{fig:drecphh0} and
\ref{fig:drecplh0}, generated by a point source
at $z_d=3500$ m, $x_s=3500$ m, propagating in the %homogeneous and
lens
model (Figure \ref{fig:bml0}). I choose this example for two
reasons. First, the success of the inversion demostrates the
insensitivity of the time reversal method to ray multipathing
(triplication), evident in the data (Figure \ref{fig:drecplh0}).
%s respectively
Second, I will invert this data in the homogeneous model, that is,
construct sources that (approximately) reproduce the data using a
different material model than the one in which it was produced. This
capability is critically important in the application of the
approximate inversion in nonlinear inversion, where the early iterations
involve solution of the source estimation problem \ref{eqn:esis} at
(possibly very) wrong material models ${\bf c}$. Successful extension
methods maintain data fit throughout the course of the inversion.
%The source time dependence is again a [1, 2.5,
%10, 12.5] Hz trapezoidal bandpass filter.

As noted earlier, the acoustic field in this example is downgoing
throughout the simulation range. It can be
regarded as the result of either pressure or velocity source at $z=z_s$: the
pressure source gather $h_s$ (Figure \ref{fig:dhshh0})  is -2 times the
vertical velocity gather depicted in Figure \ref{fig:dsrcvzhh0}, the
velocity source gather $f_s$ (Figure \ref{fig:dfshh0}) is -2 times the
pressure gather depicted in Figure \ref{fig:dsrcphh0}.

\plot{dhshh0}{width=\textwidth}{Pressure source gather = -2 $\times$
  vertical velocity gather (Figure \ref{fig:dsrcvzhh0}).}

\plot{dfshh0}{width=\textwidth}{Velocity source gather = -2 $\times$
  pressure gather (Figure \ref{fig:dsrcphh0}).}

Figure \ref{fig:dinvhslh0} shows the approximate inversion (via the
first equation in display \ref{eqn:approxinv}) of the
pressure gather shown in Figure \ref{fig:drecplh0}, inverted in the homogeneous model
(rather than in the lens model used to generate the data). The result
differs greatly from the pressure source shown in Figure
\ref{fig:dhshh0}, as it must since it results from inversion in the
wrong material model.
Some dip filter effect is unavoidable and is caused by the aperture
limitation of the acquisition geometry: the steeper dips in the source
gather (Figure \ref{fig:dhshh0}) do not contribute to the data, nor to
the inversion. Also, the limited receiver aperture causes truncation
artifacts in the inversion. However, this result is an
accurate inversion: re-simulation (application of $S^{+}_{z_s,z_r}$) {\em using the same (homogeneous)
  model as used in the inversion} results in accurate recovery (Figure \ref{fig:drerecplh0}) of the
input pressure gather (Figure \ref{fig:drecplh0}). The difference is
shown on the same color scale in Figure \ref{fig:ddiffrecplh0}.

\plot{dinvhslh0}{width=\textwidth}{Approximate inversion via first 
  equation in display \ref{eqn:approxinv}. Inversion in homogenous model of 
  the pressure gather in Figure \ref{fig:drecplh0}, simulated with 
  lens model. Scaled version of output $v_z$ field obtained by 
  applying transpose of $4V^+_{z_s,z_r}$. Quite different from 
  pressure source gather (Figure \ref{fig:dhshh0}) used to generate
  data - since inversion takes place in a different material model!}

\plot{drerecplh0}{width=\textwidth}{Re-simulated pressure gather produced from
  inverted source
  shown in Figure \ref{fig:dinvhslh0}. Simulation in homogenous model
  used for inversion.}

\plot{ddiffrecplh0}{width=\textwidth}{Difference between gathers
  displayed in Figures \ref{fig:drecplh0} and \ref{fig:drerecplh0},
  plotted on same color scale.}

%To see that this is indeed an inversion, I apply $S^{+}_{z_s,z_r}$ to
%the source shown in Figure \ref{fig:dinvhshh0}, to produce the
%re-simulation shown in Figure \ref{fig:drerecphh0}. Simulation is carried
%out in the homogeneous model used in the inversion. Comparison with the
%gather in Figure \ref{fig:drecphh0} via a difference plot
%\ref{fig:ddiffrecphh0} verifies that a very close match has been
%obtained, so the source in Figure \ref{fig:dinvhshh0} is indeed a
%satisfactory inversion, even it differs considerably from the original
%point source: the difference lies in an approximate null space of the modeling
%operator, arising from the limited aperture of the acquisition
%geometry.

%The remainder of this section is devoted to illustrating two important
%features of the approximate inversion developed above. The
%first is the validity of the inversion approach given in display
%\ref{eqn:approxinv} so long as the fields generated by the
%model used in simulation and inversion are downgoing. The models
%may be quite far from homogeneous, and even generate triplications,
%whereas the inversions remain accurate modulo the dip filtering effect
%of aperture. Figure \ref{fig:dinvhsll0} shows the pressure source at $z=z_s$
%inverted from the data Figure
%\ref{fig:drecplh0}, in which a triplication is obvious. Recall that
%this data is generated by propagating a point source at $z_d=3500$ m
%in the lens model. The inversion uses the same model. Figure
%\ref{fig:dinvhsll0} is an aperture-limited version of the pressure
%source at $z=z_s=3000$ m
%(Figure \ref{fig:dhshh0}) that generates this data by a dip filter, in
%the same way that the sources in Figures \ref{fig:dhshh0} and
%\ref{fig:dinvhshh0} are related. The inverted source is an accurate
%inversion: it generates the data depicted in Figure
%\ref{fig:drerecpll0}, which is very close to the input data (Figure
%\ref{fig:drecplh0}). The difference is displayed on the same color
%scale in Figure \ref{fig:ddiffrecpll0}.

%The second feature is a decoupling of the models used (implicitly or
%otherwise) in simulation and those used in inversion: it is possible
%to {\em accurately invert pressure data for pressure source using the
%  wrong model}, in the sense that the inverted source will
%generate an accurate recovery of the input data re-simulated in the
%same (wrong) model. For example, inversion of the data of the previous
%example (Figure \ref{fig:drecplh0}) {\em in the homogeneous model}
%results in an inverted source gather at $z=z_s$ (Figure \ref{fig:dinvhslh0}
%that is very far from a dip-filtered version of the source used
%generate the data (Figure \ref{fig:dhshh0}). 

% unitarity

\subsection{Quasi-unitary property of the modeling operator}
The identities \ref{eqn:approxinv} and \ref{eqn:sv} would together
establish the approximately unitary property of $S^+_{z_s,z_r}$, if
$\Lambda$ were symmetric. Identity \ref{eqn:approxinv} was illustrated
in the last subsection. Setting the symmetric issue aside for the
moment, an illustration of the relation \ref{eqn:sv} proceeds as follows.
%A minor extension of an earlier example, based on the gathers
%displayed in Figures \ref{fig:dsrcphh0} ($P_sp^+$) and
%\ref{fig:dsrcvzhh0} ($P_sv_z^+$) illustrates the identity
%\ref{eqn:sv}.
%The pressure gathers at $z=z_r$ produced by application of
%$S^+_{z_r,z_s}$ to $h_s=-2P_s v^+_z$, and $\Pi_0{\cal S}^+_{z_s,z_r}\Pi_1^T$ to $f_s
%=-2P_sp^+$, have already been displayed (Figures \ref{fig:dfwdphh0}
%and \ref{fig:daltphh0}), and are identical to several digits. The same 
%is true of the vertical velocity gathers, shown in \ref{fig:dfwdvzhh0}
%and \ref{fig:daltvzhh0}; their difference, plotted on the same scale,
%appears in Figure \ref{fig:dsvcomphh0}. Once again, this relation
%expresses the interchangeability of pressure and velocity sources,
%related by $\Lambda$.

Relation \ref{eqn:tracejump10} characterizes
$\Lambda^+_{z_s}$ as connecting the pressure and velocity components
of downgoing fields restricted to $z=z_s$. That is, the pressure
source gather
$h_s=\Lambda^{+}_{z_s}P_s p^+ = -2
P_{z_s}v_z$, displayed in Figure \ref{fig:dhshh0}, is the image of the
pressure gather in Figure \ref{fig:dsrcphh0} under
$\Lambda^{+}_{z_s}$.
The pressure gather
\ref{fig:dfwdplh0} is the image of this pressure source gather under
$S^+_{z_s,z_r}$ (using the lens model). The corresponding vertical velocity gather
(Figure \ref{fig:dfwdvzlh0}) is $-1/2$ times $\Lambda^+_{z_r}P_r
p$. Therefore scaling the data in Figure \ref{fig:dfwdvzlh0} by
$-2$ produces
$\Lambda^+_{z_r}S^+_{z_s,z_r}\Lambda^{+}_{z_s}P_s p^+$. On the other
hand, figure \ref{fig:daltvzlh0} shows the result of applying
$V^+_{z_s,z_r}$ to $f_s=-2P_s p^+$. Therefore scaling the gather in
Figure \ref{fig:daltvzlh0} by $-\frac{1}{2}$ produces $V^+_{z_s,z_r}P_zp^+$.
Since the data in Figures \ref{fig:dfwdvzlh0} and \ref{fig:daltvzlh0}
are essentially identical, the relation \ref{eqn:sv} holds for this
example.

\plot{dfwdvzlh0}{width=\textwidth}{Vertical velocity gather, generated
  with a pressure source in the lens model, 
  corresponding to pressure gather \ref{fig:dfwdplh0}.}

\plot{daltvzlh0}{width=\textwidth}{Vertical velocity gather, generated
  with a velocity source in the lens model, corresponding to
  pressure gather \ref{fig:daltplh0}.}

\plot{dsvcomplh0}{width=\textwidth}{Difference between velocity gathers
  shown in Figures \ref{fig:dfwdvzlh0} and \ref{fig:daltvzlh0},
  plotted on the same color scale as these figures.}

%Figure \ref{fig:preddinvhshh0} shows the pressure source gather
%produced from the pressure data gather in Figure \ref{fig:dsrcphh0} by
%application of the operator on the right-hand side of formula
%\ref{eqn:lamident} to the source surface pressure gather depicted in
%\ref{fig:dsrcphh0}. The homogeneous model is used in all propagations
%implicit in the prescription \ref{eqn:lamident}. The scale is the same
%as for Figures \ref{fig:dinvhshh0} and \ref{fig:dhshh0}. In fact
%Figure \ref{fig:preddinvhshh0} appears to be nearly identical to
%Figure \ref{fig:dinvhshh0}, and both appear to be dip-filtered
%versions of Figure \ref{fig:dhshh0}. The difference plot (Figure
 % \ref{fig:ddiffinvhshh0}), using the same color scale, shows that the
 % similarity is quantitative. The recovered source gather (Figure
 % \ref{fig:preddinvhshh0}) also generates the same acoustic fields as
  %the point source at $z=z_d$: the pressure gather at $z=z_r=1000$ m
%  depth (propagated in the lens model) is shown in Figure \ref{fig:dpredhsrecplh0}, and its
%  difference with the point source gather (Figure \ref{fig:drecplh0})
%  plotted on the same color scale in Figure
%  \ref{fig:ddiffpredhsrecplh0}.

%Figures \ref{fig:preddinvhsll0}, \ref{fig:ddiffinvhsll0},
%\ref{fig:dpredhsrecpll0}, and \ref{fig:ddiffpredhsrecpll0} accomplish
%the same comparison for the $\Lambda^+_{z_s}$ approximation in
%equation \ref{eqn:lamident}, with all propagation taking place in the
%lens model.

\subsection{Economical computation of $\Lambda$ in ``thin'' subdomain}
Equation \ref{eqn:lamnear} suggests a thin-slab computation of the
$\Lambda$ action, which is both accurate and economical. 
This calculations
place a receiver array at $z_s+\Delta z=2900$ m depth, just 100 m above the
source surface at $z_s=3000$ m. For the discretization used to create
the examples shown so far, that is just a 5 gridpoint difference in
depth, as opposed to 100 gridpoints between the source and receiver
depths for examples such as shown in Figure \ref{fig:drecplh0}.

Asymptotically, $\Lambda^{\pm}_{z_s}$ depends only on the medium
coefficients ${\bf c}$ in an arbirarily small region containing the
source surface $z=z_s$. In this example, the homogeneous and lens models are identical in the depth
range $2900 < z < 3000$ m, so the computed $\Lambda^+_{z_s}$ operators
will be precisely the same for both models. Hence I show only results for the
homogenous model.

The approximation to $\Lambda_{z_s}^+$ via equation \ref{eqn:lamnear}
for this configuration is evaluated in
Figures \ref{fig:preddnshshh0}, \ref{fig:ddiffnshshh0},
\ref{fig:dprednshsrecpll0}, and \ref{fig:ddiffprednshsrecpll0}. The effect of aperture
limitation is clearly diminished: the second figure in this series
compares the full-aperture pressure source gather (Figure
\ref{fig:dhshh0}) with the 
image of the corresponding pressure gather (Figure \ref{fig:dsrcphh0})
under the approximation to $\Lambda_{z_s}^+$, and the last figures
show that the approximated source gathers accurately predict the
point-source pressure gather at the receiver datum $z_r=1000$ m.

\plot{preddnshshh0}{width=\textwidth}{Pressure source gather = image
  under pressure-to-source operator $\Lambda^+_{z_s}$ of pressure gather
  shown in Figure \ref{fig:dsrcphh0}, homogeneous model, using
  ``near'' receiver traces at $z=2900$ m. Compare Figure
  \ref{fig:dhshh0}: because the sources and
  receivers are close, little aperture is lost in this case.}

\plot{ddiffnshshh0}{width=\textwidth}{Difference between (a) image
  (Figure \ref{fig:preddnshshh0}) of $\Lambda^+_{z_s}$ applied to
  pressure gather (Figure \ref{fig:dsrcphh0}) using a near receiver
  array to implement formula \ref{eqn:lamident}, and (b) source gather
  (Figure \ref{fig:dhshh0}) inferred from vertical velocity.
  Homogeneous model used in all
  propagations. Same color scale as in Figure
  \ref{fig:preddnshshh0}. }

%\plot{dprednshsrecplh0}{width=\textwidth}{Pressure gather at receiver
%  datum $z=z_r=1000$ m simulated in lens model from source gather shown in Figure
%  \ref{fig:preddnshshh0}. Compare with point source pressure gather
%  (Figure \ref{fig:drecplh0}).}

%\plot{ddiffprednshsrecplh0}{width=\textwidth}{Plot of difference 
%  between data shown in Figures \ref{fig:drecplh0} and 
%  \ref{fig:dprednshsrecplh0}, plotted on same color scale as the latter 
% two figures.}

\plot{dprednshsrecpll0}{width=\textwidth}{Pressure gather at receiver
  datum $z=z_r=1000$ m simulated in lens model from source gather shown in Figure
  \ref{fig:preddnshshh0}. Compare with point source pressure gather
  (Figure \ref{fig:drecplh0}).}

\plot{ddiffprednshsrecpll0}{width=\textwidth}{Plot of difference 
  between data shown in Figures \ref{fig:drecplh0} and 
  \ref{fig:dprednshsrecpll0}, plotted on same color scale as the latter 
  two figures.}
\subsection{Symmetrizing $\Lambda$}

Figure \ref{fig:preddnshstrhh0} shows the image of the pressure gather
in Figure \ref{fig:dsrcphh0} under $(\tilde{\Lambda}^+_{z_s})^T$,
using the ``near'' traces at $z=2900$, that is, $\Delta z = 100$ m in
expression \ref{eqn:lamtransp}, and propagation in the
homogeneous model. Note
the close resemblance to the image of the same pressure gather under
$\tilde{\Lambda}^+_{z_s}$ displayed in Figure
\ref{fig:preddnshshh0}. The difference of these two images is
displayed in \ref{fig:ddiffnslamtrhh0}, on the same color scale as the
images themselves. Since the propagation takes place entirely in a
region where all of the mechanical parameters are homogenous, I do not
offer a similar comparison for the lens model.

\plot{preddnshstrhh0}{width=\textwidth}{Pressure source gather = image
  under {\em transpose} of pressure-to-source operator
  $\Lambda^+_{z_s}$ of pressure gather shown in Figure
  \ref{fig:dsrcphh0}, homogeneous model, using ``near'' receiver
  traces at $z=2900$ m. Compare Figure \ref{fig:dhshh0} and
  \ref{fig:preddnshshh0}: as noted in the text, $\Lambda^+_{z_s}$ is
  asymptotically symmetric, so the resemblance is not a surprise.}

\plot{ddiffnslamtrhh0}{width=\textwidth}{Difference between data in
  Figures \ref{fig:preddnshshh0} and \ref{fig:preddnshstrhh0}, plotted
  on the same scale as these figures, showing that the asymptotic
  symmetry of $\Lambda^+_{z_s}$ is actually quantitative for the
  length, time and frequency scales of these examples.}

\subsection{Asymptotic symmetry of $\Lambda$}

Figure \ref{fig:symmdnshshh0} shows the output of the symmetrized
approximate source-to-pressure operator per equation \ref{eqn:winvcomp},
applied once again to the pressure data in Figure
\ref{fig:dsrcphh0}. Note the resemblance to Figures
\ref{fig:preddnshshh0} and \ref{fig:preddnshstrhh0}. These are all
asymptotic approximations of each other. Figure
\ref{fig:ddiffsymmdnshsll0} shows the
 the difference between the pressure gather at $z=z_r$ produced from
 the pressure source output by the symmetrized $\Lambda$, and the point source
simulation (Figure \ref{fig:dfwdplh0}), plotted on the same scale as
the latter, in both cases with all propagations in the lens model.

\plot{symmdnshshh0}{width=\textwidth}{Pressure source gather = image
    under {\em symmetrized} pressure-to-source operator
    $\frac{1}{2}\left(\Lambda^+_{z_s}+(\Lambda^+_{z_s})^T\right)$ of
    pressure gather shown in Figure \ref{fig:dsrcphh0}, homogeneous
    model, using ``near'' receiver traces at $z=2900$ m. Compare
    Figure \ref{fig:preddnshshh0}.}

\plot{ddiffsymmdnshsll0}{width=\textwidth}{Difference between point
  source simulation (Figure \ref{fig:dfwdplh0}) and pressure gather at
  $z=z_r=1000$ m produced by simulation with the source shown in
  Figure \ref{fig:symmdnshshh0},
  propagation in the lens model.}

\subsection{Unitary property of modeling operator}
To illustrate this unitary property of $S^+_{z_s,z_r}$, I apply the
operator
\[
  \frac{1}{2}((\Lambda^+_{z_s})^T+ \Lambda^+_{z_s})
  (S^{+}_{z_s,z_r})^T \frac{1}{2}((\Lambda^+_{z_r})^T+
  \Lambda^+_{z_r})
\]
to the data $S^+_{z_s,z_r}h_s$ (Figure \ref{fig:drerecplh0}), in which
$h_s$ is the %version (Figure \ref{fig:dinvhslh0}) of the
downgoing source created earlier (Figure \ref{fig:dhshh0})%, produced
%by inversion of the data in Figure \ref{fig:drecplh0} via the first
%equation in display \ref{eqn:approxinv}, propagation in the homogenous
%model.
The operator above is computed via the technique explained in
the preceding subsection, below, using
auxiliary receiver arrays 100 m above the data source and receiver arrays.

The output is shown in Figure
\ref{fig:lamsstlamrdrecplh0}. The difference with the actual source is
shown in Figure \ref{fig:difflamsstlamrdrecplh0}.
%A similar exercise
%using the lens data \ref{fig:drecplh0} but inversion in the homogenous
%model produces the extended source depicted in
%\ref{fig:lamsstlamrdrecplh0}, which differs from the inversion result
%via equation \ref{eqn:approxinv} by the gather displayed in Figure
%\ref{fig:difflamsstlamrdrecplh0}.

\plot{lamsstlamrdrecplh0}{width=\textwidth}{Inversion of data shown in
  Figure \ref{fig:drecplh0}, simulated in lens model, using the
  approximate unitarity relation \ref{eqn:unitary} and propagation in
  homogenous model.}

\plot{difflamsstlamrdrecplh0}{width=\textwidth}{Difference between
  data displayed in Figures \ref{fig:dinvhslh0} and
  \ref{fig:lamsstlamrdrecplh0}, plotted on the same color scale.}

\subsection{Preconditioned CG iteration}
This final subsection shows that result of Conjugate Gradient
iteration, with and without preconditioning, applied to the source
estimation problem \ref{eqn:esis}, with zero and non-zero penalty
weight $\alpha$. The data $d$ is the gather shown in
\ref{fig:drecplh0}, simulated using the lens model with source shown
in Figure \ref{fig:dhshh0}, or, alternatively, a point source with
bandpass filter wavelet located at $x_d=3500$ m, $z_d=3500$ m. In the inversion,
the material model is taken to be homogeneous, as has been the case in
all of the previous examples. 

Figure \ref{fig:compnres0lh0} shows the progress of the normal residual
(Euclidean norm of the difference of the two sides of equation \ref{eqn:norm1}),
for Conjugate Gradient and Preconditioned Conjugate Gradient
(Algorithm 1) iterations, applied to solution of the optimization
problem \ref{eqn:einv} with $\alpha=0$. For CG, the norms are both the ordinary
Euclidean norm, $W_m=W_d=I$. For PCG, $W_m$ and $W_d$ are given in
display \ref{eqn:wdef}, with the symmetrized $\Lambda$s computed as
indicated in the preceding subsections. Convergence for the
preconditioned algorithm is roughly 4 times as fast.

\plot{compnres0lh0}{width=\textwidth}{Comparison of normal residual 
  (gradient) Euclidean norms: CG (blue), PCG (red), plotted 
  vs. iteration. Data = lens model, point source (Figure 
  \ref{fig:drecplh0}), inversion in homogenous model. Penalty weight 
  $\alpha=0$.}

Figure \ref{fig:compnres1lh0} shows the same comparison with non-zero
penalty weight, $\alpha=10^{-3}$. The PCG normal residual curve is
almost identical with that in the $\alpha=0$ case, wheras the CG
convergence has slowed down noticeably, being about five times as slow
as the preconditioned algorithm.

\plot{compnres1lh0}{width=\textwidth}{Comparison of normal residual 
  (gradient) Euclidean norms: CG (blue), PCG (red), plotted 
  vs. iteration. Data = lens model, point source (Figure 
  \ref{fig:drecplh0}), inversion in homogenous model. Penalty weight 
  $\alpha=10^{-3}$.}

\section{Conclusion}

The linear modeling operator of surface source extended acoustic
waveform inversion is approximately invertible, and this paper has
shown how to approximately invert it. The construction is based on
reverse time propagation of data, as inspired by the literature on
photoacoustic tomography. However, since the input energy comes from a
surface source, rather than a pressure boundary value, the
pressure-to-source operator intervenes. It provides not just an
approximate inverse, but a definition of weighted norms in domain and
range spaces of the modeling operator, in terms of which that operator
is approximately unitary. Accordingly, Krylov space iteration defined
in terms of these weighted norms, or equivalently preconditioned
Conjugate Gradient iteration, gives a rapidly convergent solution
method for the linear subproblem.

The existence of an approximate unitary representation of the modeling
operator is not merely a computational convenience, however. It
reveals fundamental aspects of the operator's structure that enable
an explanation for the mitigation of cycle-skipping, a feature of the
{\em nonlinear} extended inverse problem. This fact echoes earlier observations
concerned a reflected wave inverse problem, involving a modeling
operator with a similar approximate inverse
\cite[]{tenKroode:IPTA14,Symes:IPTA14}. Also, the approximate inverse
leads to a stable computation of the gradient of the nonlinear
objective function \ref{eqn:esi}, resolving a difficulty first noted
also for reflected wave inversion \cite[]{KerSy:94}.

The transmission inverse problem figuring most prominently in
contemporary applied seismology is surely the Full Waveform Inversion
(FWI) of diving wave data. This is essentially the same problem as the
one discussed in this paper, and can be formulated and treated the
same way, at least for acoustic wave physics. All of the topics
treated here are open for elastic wave physics - the analogue of the
pressure-to-source map would is the map from surface velocity field to
corresponding constitutive defect, analogous to the elastic
Dirichlet-to-Neumann map investigated by \cite{Rachele:00}.

The underlying tool in the ideas developed here is geometric optics
(or ray theory), without which the very concept of downgoing waves
would be meaningless. The physics of actual earth materials includes
material heterogeneity on all scales, which appears to leave little
room for the assumption of scale separation underlying geometric
optics. Moreover, earth materials are anelastic, with elastic wave
energy being converted to and from thermal excitation, pore fluid
motion, and so on. A truly satisfactory understanding of inverse wave
problems will eventually need to accommodate heterogenity and
anelasticity beyond the current capabilities of the ray-based theory.

\append{Adjoint Computation}

The adjoint of ${\cal S}^+_{z_s,z_r}$ can be computed by a variant of
the adjoint state method, in this case a by-product of the
conservation of energy. This calculation leads to equation
\ref{eqn:sadj1}, from which the other statements about adjoints made in the second
section of the paper follow.

Suppose that $p^-,\bv^-$ solve \ref{eqn:awepm}
with $(h_s,f_s\bf{e}_z)\delta(z-z_s)$ replaced by
$ (h_r,f_r\bf{e}_z) \delta(z-z_r)$. Then
\[
0 = 
\left(\int\, dx\,dy\,dz\, \frac{p^+ p^-}{\kappa} +  
\rho \bv^+ \cdot \bv^- \right)|_{t \rightarrow \infty}
-
\left(\int\, dx\,dy\,dz\, \frac{p^+ p^-}{\kappa} +  \rho \bv^+ \cdot \bv^- \right)|_{t \rightarrow -\infty}
\]
\[
= 
\int_{-\infty}^{\infty} \,dt\, \frac{d}{dt}\left(\int\, dx\,dy\,dz\, \frac{p^+ p^-}{\kappa} +  \rho \bv^+ \cdot \bv^- \right)
\]
\[
= 
\int_{-\infty}^{\infty} \,dt\, \left(\int\, dx\,dy\,dz\, \frac{1}{\kappa} \frac{\partial p^+}{\partial t} p^- +  p^+ \frac{1}{\kappa}\frac{\partial p^-}{\partial t} \right.
\]
\[
+
\left. \rho \frac{\partial \bv^+}{\partial t} \cdot \bv^- + \rho \bv^+ \cdot \frac{\partial \bv^-}{\partial t} \right)
\]
\[
= 
\int_{-\infty}^{\infty} \,dt\, \left(\int\, dx\,dy\,dz\, \left(- \nabla \cdot \bv^+ + 
 h_s \delta(z-z_s)\right) p^- + p^+ \left(- \nabla \cdot \bv^- + 
 h_r \delta(z-z_r)\right) \right.
\]
\[
+
\left.  (- \nabla p^++f_s\bf{e}_z) \cdot \bv^- + \bv^+ \cdot (-\nabla
  p^- + f_r \bf{e_z}) \right)
\]
\[
= 
\int_{-\infty}^{\infty}\,dt\, \left(\int\, dx\,dy\,dz\, \left(- \nabla \cdot \bv^+ + 
 h_s \delta(z-z_s)\right) p^- + p^+ \left(- \nabla \cdot \bv^- + 
 h_r \delta(z-z_r)\right) \right.
\]
\[
+
\left.  p^+ (\nabla \cdot \bv^-) + (\nabla \cdot \bv^+) p^- 
  +f_s \delta(z-z_s) v_z^- + v_z^+f_r \delta(z-z_r) \right)
\]
after integration by parts in the last two terms. Most of what is left cancels, leaving 
\[
0 = \int_{-\infty}^{\infty}\,dt\,dx\,dy\, (h_sP_sp^-+f_zP_sv_z^-) +
( h_rP_rp^++f_rP_rv_z^+)
\]
\[
  = \langle (h_s,f_s), {\cal S}^-(h_r,f_r) \rangle+ \langle (h_r,f_r), {\cal S}^+_{z_s,z_r}(h_s,f_s) \rangle
\]
whence \ref{eqn:sadj1} follows immediately.


\section{Declarations}
\subsection{Funding}
The author did not receive support from any organization for the
submitted work.
\subsection{Competing Interests}
The author certifies that he has no affiliations with or involvement
in any organization or entity with any financial interest or
non-financial interest in the subject matter or materials discussed in
this manuscript.
\subsection{Data, Material, and Code Availability}
The computational examples reported in this work were written in the
Madacascar reproducible research framework ({\tt
  http:www.reproducibility.org}). Code and data source is available
from the author on request.


\bibliographystyle{seg}
%\bibliography{../../bib/masterref}

\title{Solution of an Inverse Problem for the Wave Equation by means of the Source-Receiver Extension}
%\author{Guanghui Huang and William. W. Symes}

%\address{The Rice Inversion Project,
%Rice University,
%Houston TX 77251-1892 USA, email {\tt symes@caam.rice.edu}.}

\author{
Guanghui Huang\thanks{Department of Computational and Applied Mathematics, Rice University,
Houston, TX, 77005, USA,
{\tt ghhuang@rice.edu}} \ and \ William Symes\thanks{Department of Computational and Applied Mathematics, Rice University,
Houston, TX, 77005, USA,
{\tt symes@rice.edu}}
}

\lefthead{Huang and Symes}

\righthead{Source-Receiver Extension}

\maketitle
\parskip 12pt

\begin{abstract}
  Iterative gradient-based minimization of mean-square misfit often
  results in suboptimal estimation of sound speed from remote
  recordings of waves. Model extension, that
  is, enlargement of the model space (domain of the simulation
  operator), coupled with a penalty for deviation from the original
  domain, permits descent methods to converge to a global minimizer of
  mean-square misfit. This paper applies the model extension concept
  to a simple transmission inverse problem for the wave equation modeled on crosswell
  tomography. The model extension studied here permits the energy
  source to depend source and receiver coordinates, whereas the
  modeled physics dictates that all energy sources should be the same
  (and given a priori). Minimization of a suitable penalty for
  deviation from the prescribed source model drives the sound velocity
  towards a global solution of the least squares data fit problem. An
  analysis based on geometric optics explains why the straightforward
  least-squares formulation fails to yield a useful algorithm, and why the
  extended formulation succeeds. The analysis requires the absence of
  caustics; we show that a substantial modification of the algorithm
  is required if these are present, and suggest such a modification
  for which numerical experiments provide some evidence of
  effectiveness.
\end{abstract}

\section{Introduction}
Estimation of sound speed in the wave equation from by least-squares
fitting of time-like seismic trace data (``full waveform inversion''
or FWI, in the seismic literature) has demonstrated potential to
produce highly detailed maps of the earth's sound speed
distribution. As an optimization problem, however, FWI is notoriously
difficult. Many realistic examples, using both synthetic and field
data, appear to show that iterative descent methods tend to stagnate
quite far from any global minimizer, at points in parameter space
having little to do with any accurate rendition of earth
structure. This commonly-observed stagnation has led many authors to
infer the existence of local minimizers of the least-squares
objective, caused by serendipitous partial fits of oscillatory data,
better than those produced by any nearby model, and preventing
improvement of fit by local optimization. A combination of data accuracy
at sufficiently low frequency and sufficiently accurate initial
velocity estimates will overcome this obstacle, but such combinations
are not always available, and in any case difficult to verify.
% This ``cycle
%skipping'' phenomenon has been conclusively demonstrated in only
%trivial cases, but is widely believed to be generic.  In any case,
%premature stagnation of iterative optimization methods is the biggest
%impediment to routine use of least-squares data fitting. It may be
%overcome by use of either sufficiently low-frequency data that the
%correct cycles in the simulated data overlap those in the measured
%data for reasonable initial estimate of sound velocity, or
%sufficiently accurate initial estimate that the correct cycles overlap
%for the measured frequency range. However frequency content of seismic
%data is limited by experimental design, and a priori knowledge of
%earth structure on the required scales is limited. It is not
%necessarily possible to achieve the right combination of frequency
%content and initial sound velocity accuracy, or even to know whether
%it has been achieved. 
See \cite{VirieuxOperto:09} for a good overview
%of FWI as currently construed by industrial and academic
seismologists.

This paper explores another solution of the FWI stagnation problem,
via extending the modeling operator to a larger  {\em infeasible} domain, so that the
data is easy to fit (in principle perfectly) by local optimization. We discuss the structure of FWI, and of its
{\em source-receiver} extension, for a simple acoustic transmission
problem, idealizing crosswell waveform tomography. Extended
(infeasible) solutions must be driven towards feasibility somehow.
We introduce measures of infeasibility, in the form of {\em
  annihilation} operators, the kernels of which are the original (feasible, physical)
domain of the modeling operator. We use these operators to define
auxiliary constrained minimization problems, and show that these
problems are convex over much larger domains than is the feasible
least squares problem (FWI).

The data of the idealized crosswell tomography problem
are functions of time, representing the acoustic
fields recorded at receivers position on a line in the plane. Source
processes generating these fields lie on another line in the plane,
and the sound velocity of the region between the lines is to  be
recovered from the data. It is assumed a priori that source processes
differ only by position on the line, and that this unique source
process is known. This assumption is an idealization: the effective
source process in actual crosswell surveys may well change from one
source position to another, even if the energy source apparatus
remains the same, due to coupling with the rock formation and other
near-field effects. Also, this source is seldom measured directly in
any usable way. For the purpose of this paper, however, we set aside
these issues.

The {\em source-receiver} extension discussed here models each
function of time, recording the acoustic field for each source and
receiver posiiton, as resulting from an independent source
process. That is, the source processes are permitted to depend on both
source and receiver coordinates, whereas physically they should be
independent of these coordinates. The larger model space consists of
the sound velocity field together with a set of source/receiver
dependent filters, one for each time function (``trace'') in the
data), which modify the output of the simulation operator at each
source/receiver combination. Unsurprisingly, the addition of so many
parameters to the domain of the modeling operator makes data fitting
considerably easier - in fact, for essentially any distribution of
sound velocity, it is possible to adjust the filters provided by the
extension to fit the data. Of the myriad possible measures of
non-physicality, we single out two for careful study - the mean square
of the image under multiplication by $t$, and the mean square of the derivative with respect to source and
receiver coordinates. Forcing either
quantity toward zero, while maintaining data fit, implicitly recovers a minimizer of the original
mean-square misfit function, for model-consistent data, provided that
none of the velocity fields encountered along the way produce
conjugate points, or caustics. Our main result in this case is that
the objective functions described above closely approximate objectives
for the traveltime inverse problem (``tomography''), hence inherit
their good properties, including in some cases global convergence.  

In the presence of caustics, the close relation between tomography and
the source-receiver extended source inverse problems breaks down. We describe the
mechanism of this failure: it results from implicit attempts to match
different branches of traveltime. We show how to overcome this
problem, at least to some extent, via exponential weighting of data, a
technique introduced by Pratt (REF) to force emphasize fitting of
first-arriving waveforms in FWI.  

\section{A model problem based on crosswell tomography'}

Crosswell seismic tomography  aims to  estimate acoustic structure between
two wells from seismic signals (``traces'') recorded in one of the wells due to seismic
energy sources in the other. The simplest useful realization of this
inverse problem uses linear acoustics to describe the generation and
propagation of seismic waves in the earth between the source and
receiver wells. In this paper we adopt the further simplifying
assumption that material density is constant and can be absorbed in
other quantities. The remaining parameter of linear acoustics is the
sound velocity, which varies with position, to reflect the
spatial heterogeniety of sedimentary rocks. With these assumptions,
crosswell tomography seeks to recover the sound velocity, or an
equivalent quantity, in the region
between the wells from measurements of acoustic fields at the wells.

Linear acoustics is a gross approximation to actual seismic wave
physics. We will make quite a few other approximations and
idealizations. The discussion section will offer some thoughts about
how important or limiting these assumptions are, and what prospects
exist for transcending them. 

To describe the problem setting more precisely, we introduce
notation that will be used throughout the discussion. Coordinates in
$\bR^3$ are $\bx=(x,y,z)$, $z$ representing depth. We assume that the
source and receiver wells are 
perfectly straight and vertical line segments in $\bR^3$(certainly an
idealization!), lying in the plane $y=0$: the
source ``well'' $W_s$and receiver ``well'' $W_r$ are defined by the
limit depths $z_s^{\rm min}$,$z_s^{\rm max}$,$z_r^{\rm min}$,$z_r^{\rm
  max}$: 
\begin{eqnarray}
\label{eqn:wells} 
W_s & = & \{\bx_s = (x_s,0,z_s): z_s^{\rm min} \le z_s \le z_s^{\rm 
          max}\} \nonumber \\
W_r & = & \{\bx_r = (x_r,0,z_r): z_r^{\rm min} \le z_r \le z_r^{\rm 
          max}\} 
\end{eqnarray}
As we will frequently need to refer to the depth ranges in the wells,
introduce
\begin{eqnarray}
\label{eqn:depths} 
Z_s & = & [z_s^{\rm min}, z_s^{\rm max}] \nonumber \\
Z_r & = & [z_r^{\rm min},z_r^{\rm max}] 
\end{eqnarray}
We idealize energy sources as isotropic point radiators. A radiator at
$\bx=\bx_s$ appears as the right-hand side in the acoustic wave
equation, of the form $\delta(\bx-\bx_s)f(t)$. 
The function $f(t)$ is causal (vanishes on a negative half-axis, that
is,  for $t<<0$), and has units of
mass/time/time. The idealized model posits that every
source is has the same mechanism, modeled by the
isotropic point radiator with the same time dependence $f$, differing only in
depth $z=z_s$ in the source well.

Slowness (reciprocal velocity) turns out to be a more convenient
choice than velocity to represent the spatially varying acoustic
properties of the interwell material. We denote the slowness by $m$.
For physical reasons, $m$ should be positive and bounded uniformly above and 
below, say by $0 < m_{\rm min} < m_{\rm max}$.

We assume that the
material is {\em acoustically transparent}, that is, generates no
reflected waves of significant energy in the seismic band. The geometric
optics approximation, to be summarized below, shows that a convenient
mathematical proxy for transparency is the assumption that $m$ is
smooth, with derivatives up to
order 2 bounded by another constant $\mu > 0$. These conditions are
summarized in the definition of the feasible set of slowness models $M$:
\begin{equation}
\label{eqn:vfeas}
M=\{m \in C^{\infty}(\bR^3): m_{\rm min} < m(\bx) < m_{\rm max},
|D^{\alpha}m(\bx)| \le \mu,
\bx \in \bR^3, |\alpha| \le 2\}
\end{equation}

The pressure field 
$p(\bx,t;\bx_s)$ due to the isotropic point radiator source at 
$\bx_s=(x_s,0,z_s)$, with time dependence $f(t)$, is the solution of the initial value problem 
\begin{eqnarray}
\left(m^2 \frac{\pa^2 p}{\pa t^2} - \nabla^2 p\right)(\bx,t) &=&
     \delta(\bx-\bx_s) f(t), \nonumber \\
 p&=&0, \, t<<0 \label{eqn:pressure}
\end{eqnarray}
We will establish the existance, uniqueness, and essential properties
of solutions $p$ in the next section. Proceeding formally for the moment,
define the modeling operator $S[m]$ to relate the slowness $m$ and
source function $f$ to the pressure field at the receiver locations
$\bx_r$ and source locations $\bx_s$, for a finite time interval $0
\le t \le t^{\rm max}$. Since the horizontal coordinates of sources and
receivers are fixed as part of the idealized cross-well acquisition
geometry, we regard the output of $S[m]$ as a function of $z_s,z_r,$
and $t$, with domain
\begin{equation}
\label{eqn:omdef}
\Omega = Z_r\times Z_s \times \bR
\end{equation}
 A smooth cutoff factor provides
convenient way to incorporate the limited time duration of measured
signals, and to smoothly taper the measured data to zero near the
boundary of the measurement domain, with favorable signal-processing
consequences. We denote this factor by
\begin{equation}
\label{eqn:wdef}
w \in C_0^{\infty}( \Omega): \,{\rm supp}\,w \subset 
Z_r\times Z_s \times (0,t^{\rm max}).
\end{equation}
Proceeding formally, define
\begin{equation}
\label{eqn:sdef}
S[m]f(z_r,t;z_s) = w(z_r,t;z_s) p(\bx_r,t;\bx_s), 
\end{equation}
The crosswell waveform inversion problem, in the highly idealized form 
studied here, is: given data $d$, a function of on $\Omega$,
find slowness $m \in M$ and source function $f$ such that 
$S[m]f \approx wd$. 


\section{Properties of the Modeling Operator}

The problem formulation just stated begs several questions. For
example, what sort of functions $d$, $f$, and $m$ should be admitted?
Under what circumstances is the restriction of the pressure field $p$
to the submanifold $W_s \times W_r \times \bR$ well-defined?

For numerical convenience, and on physical grounds
\cite[]{SantosaSymes:00}, $f$ and $d$ should be square-integrable and
the natural (and conventional) measure for the data misfit $S[m]f-d$
is $L^2$. Indeed, iff $m \in M$ is constant, $f \in C_0^{\infty}(\bR)$,
$\bx \ne \bx_s$, then
\be\label{eqn:cv} 
p(\bx,t;\bx_s) = 
\frac{f(t-mr)}{4\pi r}, \,\,r=|\bx-\bx_s|.  
\ee 
The distribution on the right-hand side of equation \ref{eqn:cv} has a
well-defined restriction to $\bx=\bx_r$, for reasons to be reviewed
shortly. so that $S[m]$ given by
equation \ref{eqn:sdef} is well-defined, and extends continuously to a
map $f \in L^2(\bR) \mapsto S[m]f \in L^2(\bar{\Omega})$. 

More generally, 

\begin{theorem}
\label{thm:trace}
For any $m \in M$, $f \in {\cal E}'(\bR)$, there exists a unique solution $p \in {\cal D}'(\bR^4)$ of problem
\ref{eqn:pressure}. Define $\chi_{\bx_r}: \bR \rightarrow \bR^4$ by
$\chi_{\bx_r}(t) = (\bx_r,t)$. For any $\bx_r \in \bR^3\setminus
\{\bx_s\}$ , the map $f\mapsto \chi_{\bx_r}^*p$ is continuous: ${\cal E}'(\bR) \rightarrow
{\cal D}'(\bR)$. Denote by $G(\cdot,\cdot;\bx_s) \in {\cal D}'(\bR^4)$
the solution of the initial value problem
\ref{eqn:pressure} with $f = \delta$. Then $\chi_{\bx_r}^*p =
(\chi_{\bx_r}^*G(\cdot,\cdot;\bx_s))*f =
\chi_{\bx_r}(G(\cdot,\cdot;\bx_s)*(\delta_{\bR^3}\otimes f))$. 
\end{theorem}
\begin{proof}
  A proof of the existence and uniqueness of causal distribution
  solutions to \ref{eqn:pressure} may be found in \cite{Lax:PDENotes},
  Chapter 6. Alternatively, since $\delta(\cdot-\bx_s)\otimes f \in
  H^s(\bR^4)$ for some $s < -2$, \cite{Tay:81} 2.1-2.2 shows
  how to construct local solutions, which can be patched together
  taking advantage of the uniformly bounded speed of propagation
  implicit in $m \in M$. From the definition (\cite{Tay:81}, Chapter VI), the wave front set
  $WF(p)$ is contained in the set
\[
C \cup \Pi^{-1}({\{\bx_s\} \times \bR})
\]
in which $C$ is the characteristic variety of the wave operator,
\begin{equation}
\label{eqn:charvar}
C[m] = \{(\bx,t,\xi,\omega) \in T^*(\bR^4): m^2(\bx)\omega^2 =
|\xi|^2\}
\end{equation}
and  $\Pi: T^*(\bR^4) \rightarrow \bR^4$ is the fiber projection.

The normal bundle of $\{\bx_r\} \times \bR$ consists of covectors with
coordinates $(\bx_r,t,\xi,0)$, hence has trivial intersection with
$C[m]$. Therefore the restriction of $p$ to $\{\bx_r\} \times \bR$ is
well-defined (\cite{Dui:95}, Proposition 1.3.3), as is the pull-back
by $\chi_{\bx_r}$. Convolution in time
with $f$ is by definition convolution in space-time with
$\delta_{\bR^3} \otimes f \in {\cal E}'(\bR^4)$, a continuous map on ${\cal
  D'}(\bR^4)$,  and this latter convolution commutes with
the wave operator in \ref{eqn:pressure}. This convolution also maps
the right-hand side for $G$ to the right-hand-side for $p$.
\end{proof}

\begin{rem} $G$ is of course the causal Green's function, or distribution
  kernel of the retarded fundamental solution \cite[]{CourHil:62}.
\end{rem}

An immediate consequence of this result is a proper definition of the
map $S[m]$.

\begin{theorem}
\label{thm:gather}
Denote by $\Delta = \{(\bx,\bx): \bx \in \bR^3\}$ the diagonal. With
notation as in Theorem \ref{thm:trace}, the map $(\bR^3 \times \bR^3)
\setminus \Delta \times {\cal E}'(\bR) \rightarrow {\cal D}'(\bR)$ defined by $(\bx_r,\bx_s,f)
\mapsto \chi_{\bx_r}^*p(\cdot,\cdot;\bx_s)$ is continuous. 
\end{theorem}
\begin{proof}
The right-hand side of \ref{eqn:pressure} is a continuous
${\cal E}'(\bR^4)$- valued function of $\bx_s$. Denote by $N$ the
normal bundle of the fibration $\{\{\bx\} \times \bR: \bx \in \bR^3\}$.
Denote by $\Gamma$ a closed conic submanifold of
$T^*(\bR^4)$ containing $C$ in its interior. chosen so that $\Gamma
\cap N=\{0\}$. The solution of the wave equation is a continuous
${\cal D}'(\bR^4)$-valued function of the ${\cal E}'(\bR^4)$- valued
right-hand side. Therefore solution of \ref{eqn:pressure} is a continuous
${\cal D}'(\bR^4)$-valued function of $\bx_s\in \bR^3$ and $f\in {\cal
  E}'(\bR)$, and also continous in the sense of 
${\cal D}_{\Gamma}'(\bR^4\setminus \{\bx_s\})$ as a consequence of the
propagation of singularities theorem. Since $N \cap \Pi^{-1}(\bx_r) =\{\bx_r\} \times \bR$,
$\chi_{\bx_r}^*$ is continuous: $\bR^3 \times {\cal D_{\Gamma}}'(\bR^4) \rightarrow
{\cal D}'(\bR)$. Putting all of this together, obtain the 
statement of the theorem. 
\end{proof}

Theorem \ref{thm:gather} permits a precise rewrite of the
formal definition \ref{eqn:sdef}: define $S[m]: {\cal E}'(\bR)
\rightarrow C(Z_r \times [z_x^{\rm min},z_s^{\rm max}], {\cal E}'(\bR))$ by 
\[
S[m]f(z_r,\cdot;z_s) = w(z_r,\cdot;z_s) \chi_{\bx_r}^*p(\cdot,\cdot;\bx_s)
\]

To formulate the inverse problem as envisioned in the last section, an
analogous result is needed with $L^2$ replacing ${\cal E}'$. This step
requires the introduction of geometric optics.

Recall that the rays of geometric optics are trajectories
$(X(t),P(t))$ in $\bR^3 \times (\bR^3 \setminus \{0\})$ satisfy 
\begin{eqnarray}
\label{eqn:ham}
\frac{dX}{dt} &=&m(X)^{-2}P,\nonumber \\
\frac{dP}{dt} &=&\nabla \log m(X), \nonumber\\
|P| &=& m(X).
\end{eqnarray}
The condition $m \in M$ implies that trajectories exist that solve the
first two conditions in \ref{eqn:ham} for all $t>0$, for any initial
values of $X$ and $P$. The third condition is compatible with the
other two: if the initial $X$, $P$ satisfy it, then it is satisfied
along the entire ray.

Define the {\em geodesic exponential map}
\[
\exp(t,P_s; \bx_s) = X(t)
\]
where $(X(t),P(t))$ solves \ref{eqn:ham} with $X(0)=\bx_s$,
$P(0)=P_s, |P_s|=m(\bx_s)$: $(t,P_s)$ are {\em
  geodesic normal coordinates} centered at $\bx_s$. Denote by
$S_m(\bR^3)$ the sphere bundle defined by the Riemannian metric $m (dx
\otimes dx + dy \otimes dy + dz \otimes dz)$ defined by the slowness
$m \in M$. It is a standard
result that $t^*:S_m(\bR^3) \rightarrow \bR_+$ exists for which
$\exp$ is a diffeomorphism from 
\begin{equation}
\label{eqn:prenorm}
\tilde{{\cal N}} = \{(\bx_s,t,P_s) \in
\bR^3 \times \bR_+ \times m(\bx_s)S^2: 0 < t < t^*(\bx,P_s)\}
\end{equation}
 onto a punctured
neighborhood ${\cal N}(\bx_s) \setminus \{\bx_s\}$ of $\bx_s$. A {\em normal neighborhood}
${\cal N}$ is an open subset of $\bR^3$ for which ${\cal N} \subset
{\cal N}(\bx)$ for every $\bx \in {\cal N}$. It is also a standard
result that every point $\bx \in \bR^3$ has a normal neighborhood (\cite{Friedlander:75}, Chapter 2). 

In a normal neighborhood of $\bx_s$, the inverse of the exponential
map expresses $t$ as a function of $\bx$: this function is the
{\em travel time} $\tau(\bx,\bx_s)$.
The travel time is differentiable and satisifies the eikonal equation in the punctured normal
neighborhood (and in a suitable generalized sense in the entire normal
neighborhood):
\begin{eqnarray}
\label{eqn:eik}
|\nabla_{\bx} \tau(\bx,\bx_s)|&=&m(\bx), \bx \in {\cal N}, \nonumber \\
\tau(\bx,\bx_s) & \sim & m(\bx_s)|\bx-\bx_s| \mbox{ as } \bx \rightarrow 
\bx_s. 
\end{eqnarray}

A slowness field $m \in M$ satisfies the {\em simple ray geometry}
hypothesis iff for every $\bx_s \in W_s$, the
the receiver well $W_r$ is contained in a
normal neighborhood ${\cal N}(\bx_s)$ centered at $\bx_s$. In this case, there is a common
neighborhood ${\cal N}$, normal for all points $\bx_s$ in the source
well and containing the closure of the receiver well.


A slowness fails to satisfy simple ray geometry hypothesis if the
exponential map at some source point $\bx_s$ fails to be a
diffeomorphism in a neighborhood of the receiver well, by failing to
be injective, that is, more than one ray (``path'') connects $\bx_s$
with another point,

In the simple ray geometry case, the Green's function has a useful
decomposition. 

\begin{theorem}
\label{thm:simplegreen}
Suppose that $m$ satisfies the simple ray geometry assumption. Then there exists $t^{\rm max} >
\max\{\tau(\bx_r,\bx_s): \bx_r \in W_r, \bx_s \in W_s\}$, and 
functions $a \in C^{\infty}(W_r\times W_s), b\in
C^{\infty}(W_r,W_s,(0,t^{\rm max})$ so that for
$\bx_r \in W_r, \bx_s \in W_s, t <t^{\rm max}$,
\begin{equation}
\label{eqn:simplegreen}
G(\bx_r,t;\bx_s) = a(\bx_r,\bx_s)\delta(t-\tau(\bx_r,\bx_s)) + b(\bx_r,t;\bx_s)H(t-\tau(\bx_r,\bx_s)).
\end{equation}
\end{theorem}
\begin{proof}
The starting point is the parametrix contruction due to Hadamard, as described
by \cite{Friedlander:75}, Theorem 4.3.2. In the
notation used here, for $\bx,\bx_s \in {\cal N}$ (a normal
neighborhood), there exist $U, V \in C^{\infty}({\cal N}\times{\cal N}
\times \bR)$ so that
\begin{equation} 
\label{eqn:parametrix}
\tilde{G}(\bx,t;\bx_s) = U(\bx,t;\bx_s)\delta^+(t^2-\tau^2(\bx,\bx_s))
+ V(\bx,t;\bx_s)H^+(t^2-\tau(\bx,\bx_s))
\end{equation}
Here $\delta^+(t^2-\tau^2)$ is defined as the limit
\[
\delta^+(t^2-\tau^2) = \lim_{\epsilon \rightarrow
  0^+}H(t-\tau)\delta(t^2-\tau^2-\epsilon)
\]
The product on the right is well-defined, as the singular supports of
the factors are disjoint, and picks out the positive-time sheet of the
distorted hyperboloid $t^2 = \tau^2$. The limit exists in the sense of
${\cal D}'$.

$\tilde{G}$ is a $C^{\infty}$ parametrix, in the sense that 
\[
\left(m\frac{\partial^2}{\partial
    t^2}-\nabla^2\right)\tilde{G}(\bx,t;\bx_s) =
\delta(\bx-\bx_s) + R(\bx,t;\bx_s), \,\,R \in C^{\infty}({\cal
  N}\times{\cal N}\times \bR),\,R=0 \mbox{ for } t<\tau.
\]
Define $\tilde{R}$ to be the solution of the initial value problem
\[
\left(m\frac{\partial^2}{\partial
    t^2}-\nabla^2\right)\tilde{R} = -R; \,\,\tilde{R}=0, t<0
\]
This problem has a unique $C^{\infty}$ solution. The
domain-of-dependence property implies that $\tilde{R}=0$ for $t <
\tau$ also. 

Any singularity of $G$ in ${\cal N}\times{\cal N}\times \bR$ lies
either on the light cone $t=\tau$ or must occur in $t > t^{\rm
  max}$. So on ${\cal N}\times{\cal N}\times (0,t^{\rm max})$, 
\[
G = \tilde{G} + \tilde{R}.
\]

Finally, since 
\[
\delta^+(t^2-\tau^2) = \frac{\delta(t-\tau)}{2\tau}
\]
for $t>0$, or equivalently, $\tau>0$, set $a = U/2\tau, b = V +
\tilde{R}$ to obtain \ref{eqn:simplegreen}.
\end{proof}

\begin{rem}
In the notation used here, 
\[
U(\bx,\bx_s) = \frac{m(\bx_s)}{2\pi} + O(|\bx-\bx_s|)
\]
so for constant $m$,
\[
a(\bx_r,\bx_s) = \frac{m}{4\pi\tau(\bx,\bx_s)} +
O(1)=\frac{1}{4\pi|\bx-\bx_s|} + O(1)
\]
which recovers \ref{eqn:cv} approximately for $\bx_r$ close to $\bx_s$
- of course, the error term actually vanishes in this case.
\end{rem}

\begin{cor}
\label{thm:gosimple}
Under the assumptions of Theorem \ref{thm:simplegreen}, 
for $\bx_r \in W_r, \bx_s \in W_s$, and $t < t^{\rm max}$, set $A(z_r,z_s) = a(\bx_r,\bx_s),
T(z_r,z_s)=\tau(\bx_r,\bx_s), B(z_r,t;z_s) = b(\bx_r,t;\bx_s)$. Then
$A,T \in C^{\infty}(Z_r \times 
Z_s)$,
$B \in \in C^{\infty}(Z_r \times Z_s\times (-\infty,t^{\rm max}))$, $T>0$. For $f \in
L^2(\bR)$,
\begin{equation}
\label{eqn:gosimple} 
S[m]f (z_r,t;z_s) = w(z_r,t;z_s)\left(A(z_r,z_s)f(t-T(z_r,z_s)) + \int_{T(z_r,z_s)}^{\infty}
\,ds\,B(z_r,s;z_s)f(t-s)\right). 
\end{equation}
\end{cor}

For highly oscillatory $f$, the first term in (\ref{eqn:gosimple}) is much
larger than the second, and can be taken as a high frequency
asymptotic approximation of $S[m]f$. 

Analysis of the inverse problem requires some information on 
the smooth dependence of $S$ on $m$. 

\begin{theorem}
\label{thm:smdep}
Denote by $M^k$ be the interior of the closure of $M$ in $C^k({\cal
  N})$; $M^k$ is defined by the same conditions as $M$ but with
$C^{\infty}$ replaces by $C^k$.  Then
\begin{itemize}

\item[1. ] If $N$ is a normal neighborhood for $m_0 \in M^k$,
  $k \ge 2$, and $\tilde{\cal N}$ is its pre-image under the
  exponential map for $m_0$ as defined in \ref{eqn:prenorm}, then there is an open neighborhood ${\cal U}$ of $m_0$ in
  $M^k$ so that ${\cal N}$ is a normal neighborhood for every
  $m \in {\cal U}$.
\item[2. ] For ${\cal N}, \tilde{\cal N}$, and ${\cal U}$ as defined
  in item 1, 
\begin{equation}
\label{eqn:expreg}
\exp \in C^0({\cal U}, C^k(\tilde{\cal N},{\cal N})) \cap C^1({\cal
  U}, C^{k-1}(\tilde{\cal N},{\cal N}))
\end{equation}
\item[3. ] Suppose that the simple ray geometry assumption holds for
an open set ${\cal U} \subset M^k$ (that is, for every $m \in {\cal
  U}$ and that ${\cal N}$ is a normal neighborhood for every $m \in
{\cal U}$ containing $W_r$ and $W_s$. Then $T, A,$ and
$B$ extend to functions on ${\cal U}$ for $k \ge 2$ so that
\begin{itemize}
\item[3.1 ]$T \in C^1({\cal U},C^{k-2}({\cal N}\times{\cal N}))$ 
\item[3.2 ]$A \in C^1({\cal U},C^{k-2}({\cal N}\times{\cal N}))$ 
\item[3.3 ]$B \in C^1({\cal U},,C^{k-2}({\cal N}\times{\cal N}\times(-\infty,t^{\rm
    max})))$
\item[3.4 ]$S \in C^0({\cal U},{\cal L}_s(L^2(\bR),
C^0(Z_r\times Z_s))),
  L^2(\bR))))$
\item[3.5 ]$S \in G^1({\cal U},{\cal L}_s(H^1_0(\bR),
C^0(Z_r\times Z_s))),
  L^2(\bR))))$
\end{itemize} 
\end{itemize}
\end{theorem}
\begin{proof}
The items 1., 2., 3.1, 3.2, and 3.3 follow from the standard theorem
  on continuous dependence of solutions of ordinary differential
  equations on parameters applied to the Hamiltonian system
  \ref{eqn:ham}, and to the transport equations defining the functions
  $U$ and $V$ appearing in the Hadamard construction, and the inverse
  function theorem applied to the exponential map. Note that $B$ is
  implicitly extended smoothly to $t < \tau(\bx,\bx_s)$, possible
  because the intersection of the light cone $t=\tau(\bx,\bx_s)$ with
  $W_r$ is a smooth curve. In items 3.4 and
  3.5, ${\cal L}_s({\cal B}_1,{\cal B}_2)$ denotes the topological
  vector space of continuous linear maps from Banach space
  ${\cal B}_1$ to Banach space $ {\cal B}_2$, endowed with the strong
  operator topology, that is, the topology of pointwise
  convergence. That is, if $m_j \rightarrow m$ in
  ${\cal U} \subset M^k$, then for each $f \in L^2(\bR)$,
  $S[m_j]f \rightarrow S[m]f$ in the sense indicated in item
  3.4. In item 3.5, $G^1$ indicates the class of
  G\^{a}teaux-differentiable functions: that is, for each $m \in {\cal
    U}$, there exists a linear map 
\[
DS[m]: M^k \rightarrow {\cal L}_s(H^1_0(\bR),
C^0(Z_r\times Z_s,L^2(\bR))),
\]
so that for each $\delta m \in M^k$, $f \in L^2(\bR)$,
\[
\|(S[m + h\delta m]-S[m]-hDS[m]\delta m) f \| = o(h) \mbox{ as }
h\rightarrow 0.
\]
Items 3.4 and 3.5 follow directly from the the other
  conclusions and Corollary \ref{thm:gosimple}.
\end{proof}

\begin{rem}
\label{rem:nounif}
  It is not possible to replace ${\cal L}_s$ with ${\cal L}$, the same
  vector space of continuous linear maps with the (Banach) topology of
  uniform convergence, in the statements of items 3.4 and 3.5. To see
  this, suppose that $m_j \rightarrow m$ in $M^k$. Assume that $m$
  satisfies the simple ray geometry condition with normal neighborhood
  ${\cal N}$ containing $W_r$ and $W_s$, and that the sequence
  $\{m_j\}$ lies inside an open ball ${\cal U}$ sufficiently small
  that ${\cal N}$ is a normal neighborhood for every member of this
  sequence. Finally suppose that
\[
\inf\{\tau[m_j](z_s,z_r) -\tau[m](z_s,z_r)\}=\epsilon_j > 0.
\]
This condition can be arranged, for example, by taking
$m_j=(1+c2^{-j})m$ for sufficiently small $c>0$. For each $j$, choose
$f_j \in L^2(\bR)$ so that 
\[
\|f_j\|=1,\,\,{\rm supp}\,f_j \subset [0,\epsilon_j/2].
\]
Then ${\rm supp}\,f_j(\cdot-\tau[m_j])$ and ${\rm supp}\,f_j(\cdot-\tau[m])$ are
disjoint. It is easy to see that the integrated term in the expansion
\ref{eqn:gosimple} tends to zero,  and item 2 above implies that
$A[m_j] \rightarrow A[m]$, so 
\[
\|S[m_j]f_j-S[m]f_j\|^2\rightarrow \|A[m_j]f_j(\cdot-\tau[m_j])\|^2 + \|A[m]f_j(\cdot-\tau[m])\|^2 
\]
\[
\rightarrow 2 \int_{z_r^{\rm min}}^{z_r^{\rm
    max}}\,dz_r\,\int_{z_s^{\rm min}}^{z_s^{\rm
    max}}\,dz_s\,|A[m](z_r,z_s)|^2 > 0
\]
Thus $S[m_j]$ does not tend to $S[m]$ in operator norm. 

This observation captures a very important characteristic of all
inverse problems in wave propagation in which wave velocity is an
unknown: the data prediction map is not uniformly continuous in
material parameters and source parameters jointly. This fact explains
the character of the full waveform inversion problem, to be analyzed
in the next section.
\end{rem}

\begin{rem}
\label{rem:derivloss}
Note that for $m, \delta m \in M^k$, $DS[m]\delta m$ is continuous
on a smaller domain than is the case for $S[m]$. In fact, it follows
from Corollary \ref{thm:gosimple} that for $f \in H^1_0(\bR)$,
\[
D(S[m]f)\delta m  = w\left(-A[m](DT[m]\delta m)\frac{df}{dt}(t-T[m]) +
((DA[m]\delta m)\right.
\]
\begin{equation}
\label{eqn:preq}
\left. -B[m](DT[m]\delta m))f(t-T[m]) + \int_{T(z_r,z_s)}^{\infty} \,ds\,
(DB[m]\delta m)(s)f(t-s)\right)
\end{equation}
This observation shows that the statement of item 5 is sharp.
\end{rem}

\begin{rem}
\label{rem:multipath}
In general, geometric optics predicts more than one traveltime between
each source and receiver (the multipath case), hence a principal term
involving multiple shifted copies f. It is possible to show that in a
subset of $W_r \times W_s$ of second Baire category, smooth travel times
$T_i$, amplitudes $a_i$, and remainders $B_i$ exist so that
\begin{equation}
\label{eqn:gomultipath} 
S[m]f (z_r,t;z_s) = w(z_r,t;z_s)\left(\sum_{i=0}^NA_i(z_r,z_s)f(t-T_i(z_r,z_s)) + \int_{T_i(z_r,z_s)}^{\infty} 
\,ds\,B_i(z_r,s;z_s)f(t-s)\right). 
\end{equation}
We will not give a detailed justification of this expansion, but will
note one of its consequences later.
\end{rem}

\section{Full Waveform Inversion (FWI)}

In view of the mapping property described in the previous section, a
natural explicit formulation of the inverse problem is via least
squares: given $d\in L^2(\Omega)$ and $f\in L^2(\bR)$, find $m \in M$ to minimize 
\be\label{eqn:ls}
 J_{\rm LS}[m;f,d] =
\frac{1}{2}\|(S[m]f-d)\|^2.  
\ee 

The first consequence to draw from the expansion \ref{eqn:gosimple}
is a very strong limitation on the possible convexity of $J_{\rm
  LS}$. To state this fact conveniently, introduce a family of source
functions based on a ``mother'' function $f_1 \in
C^{\infty}_0(\bR)$, for which
\begin{equation}
\label{eqn:mutha}
\|f_1\|_{L^2(\bR)} = 1.
\end{equation}
For $\epsilon>0$, define
\be\label{eqn:srcfam}
f_{\epsilon} =
\frac{1}{\sqrt{\epsilon}}f_1\left(\frac{t}{\epsilon}\right).
\ee

\begin{theorem}\label{thm:FWIconv}
Suppose that $k \ge 2$, ${\cal U} \subset M^k$ is open and convex and that the
simple ray geometry hypothesis (Definition \ref{def:srg}) is satisfied
at every $m \in {\cal U}$.  Suppose that $m_*, m_0 \in {\cal U}$, and
for some $\rho < 1$,
\be\label{eqn:ttgap}
T[m_*](z_s,z_r) \le \rho T[m_0](z_s,z_r).
\ee
 for every $\bx_s \in W_s, \bx_r \in W_r$,  Define $d_{\epsilon} = S[m_*]f_{\epsilon}$. Then there exists
$\epsilon_*>0$ so that for $0 <\epsilon \le \epsilon_*$, $J_{\rm LS}(\cdot,d_{\epsilon},f_{\epsilon})$
is {\em not} convex on ${\cal U}$.
\end{theorem}
\begin{rem}
\label{rem:scale}
The relation \ref{eqn:ttgap} is easy to achieve. If $m_0 \in {\cal U}$ and
$1-\rho$ is sufficiently small that $m_* = \rho m_0 \in {\cal U}$, then
\ref{eqn:ttgap} holds.
\end{rem} 
\begin{proof}
It sufficies to show that the function $j:[0,1] \rightarrow \bR$ defined by 
\[
j(\sigma) = J_{\rm LS}[\sigma m_* + (1-\sigma)m_0;d_{\epsilon},f_{\epsilon}]
\]
is not convex.

The crux of the matter is the representation \ref{eqn:gosimple}. To signify the dependence of the quantities $A, T,$ and $B$ from
equation \ref{eqn:gosimple} on the slowness field, throughout this
argument and subsequent discussion we write $A=A[m]$ and so on.

Theorem \ref{thm:smdep} that implies that given $\alpha>0$, there
exists $\sigma_0 \in [0,1)$ for which 
\[
(1+\alpha) \int_{z_r^{\rm min}}^{z_r^{\rm max}}\,dz_r\, \int_{z_s^{\rm
    min}}^{z_s^{\rm max}}\,dz_s\,|A[m_*](z_r,z_s)|^2.
\]
\[
\ge \int_{z_r^{\rm min}}^{z_r^{\rm max}}\,dz_r\, \int_{z_s^{\rm
    min}}^{z_s^{\rm max}}\,dz_s\,|A[\sigma m_* +
(1-\sigma)m_0](z_r,z_s)|^2 
\]
\[
\ge (1-\alpha) \int_{z_r^{\rm min}}^{z_r^{\rm max}}\,dz_r\, \int_{z_s^{\rm
    min}}^{z_s^{\rm max}}\,dz_s\,|A[m_*](z_r,z_s)|^2.
\]
for $\sigma \in [\sigma_0,1]$.

Define $S_0$ to be the first term in the expansion \ref{eqn:gosimple},
\[
S_0[m]f(z_r,t;z_s) = A[m](z_r,z_s)f(t-T(z_r,z_s)).
\]
Condition \ref{eqn:ttgap} implies that there exists $\epsilon_*>0$,
depending on $\alpha$, so that if $0 <\epsilon\le \epsilon_*$, then the supports of $S_0[\sigma m_* + (1-\sigma)m_0]f_{\epsilon}$ and
$S_0[m_*]f_{\epsilon}$ are disjoint if $\sigma_0 \le \sigma \le \frac{1}{2}(1+\sigma_0)$, so
\[
\|S_0[\sigma m_* + (1-\sigma)m_0]f_{\epsilon}-S_0[m_*]f_{\epsilon}\|^2 =
\|S_0[\sigma m_* + (1-\sigma)m_0]f_{\epsilon}\|^2 + \|S_0[m_*]f_{\epsilon}\|^2 
\]
\[
= \int_{z_r^{\rm min}}^{z_r^{\rm max}}\,dz_r\, \int_{z_s^{\rm
    min}}^{z_s^{\rm max}}\,dz_s\,|A[\sigma m_* + (1-\sigma)m_0](z_r,z_s)|^2 +
|A[m_*](z_r,z_s)|^2 
\]
\begin{equation}
\label{eqn:orthominus}
\ge (2-\alpha)\left(\int_{z_r^{\rm min}}^{z_r^{\rm max}}\,dz_r\, \int_{z_s^{\rm
    min}}^{z_s^{\rm max}}\,dz_s\,|A[m_*](z_r,z_s)|^2\right) 
\end{equation}
Similarly, for $\sigma \in [\sigma_0, \frac{1}{2}(1+\sigma_0)]$,
\begin{equation}
\label{eqn:orthoplus}
\|S_0[\sigma m_* + (1-\sigma)m_0]f_{\epsilon}-S_0[m_*]f_{\epsilon}\|^2 
\le (2+\alpha)\left(\int_{z_r^{\rm min}}^{z_r^{\rm max}}\,dz_r\, \int_{z_s^{\rm
    min}}^{z_s^{\rm max}}\,dz_s\,|A[m_*](z_r,z_s)|^2\right) 
\end{equation}

Since $ B[\sigma m_* + (1-\sigma)m_0]$ is uniformly bounded in
$C^{k-2}$,
\[
\sup_{t \in (0,t^{\rm max})} \left|\int_{T(z_r,z_s)}^{\infty}
  \,ds\,B[\sigma m_* + (1-\sigma)m_0](z_r,s;z_s)f_{\epsilon}(t-s)\right| = O(\sqrt{\epsilon})
\]
If $\epsilon_*>0$ is chosen possibly smaller yet, again depending on
$\alpha$, then for $0 <\epsilon\le\epsilon_*$,
\[
\|S[\sigma m_* + (1-\sigma)m_0]f_{\rm epsilon}-S_0[\sigma m_* +
(1-\sigma)m_0]f_{\rm epsilon}\|^2 
\]
\begin{equation}
\label{eqn:lot}
\le \alpha \int_{z_r^{\rm min}}^{z_r^{\rm max}}\,dz_r\,
  \int_{z_s^{\rm min}}^{z_s^{\rm max}}\,dz_s\,|A[m_*](z_r,z_s)|^2
\end{equation}
Combining \ref{eqn:orthoplus} and \ref{eqn:lot} for $\sigma=\sigma_0$, obtain
\begin{equation}
\label{eqn:ub}
\|S[\sigma_0 m_* + (1-\sigma_0)m_0]f_{\epsilon}-S[m_*]f_{\epsilon}\|^2 
\le (\sqrt{2+\alpha}+2\sqrt{\alpha})^2 \int_{z_r^{\rm min}}^{z_r^{\rm max}}\,dz_r\,
  \int_{z_s^{\rm min}}^{z_s^{\rm max}}\,dz_s\,|A[m_*](z_r,z_s)|^2 
\end{equation}
Similarly, combining \ref{eqn:orthominus} and \ref{eqn:lot} for
$\sigma=\frac{1}{2}(1+\sigma_0)$, 
\[
\|S[\frac{1}{2}(1+\sigma_0) m_* + (1-\frac{1}{2}(1+\sigma_0))m_0]f_{\epsilon}-S[m_*]f_{\epsilon}\|^2 
\]
\begin{equation}
\label{eqn:lb}
\ge (\sqrt{2-\alpha}-2\sqrt{\alpha})^2 \int_{z_r^{\rm min}}^{z_r^{\rm max}}\,dz_r\,
  \int_{z_s^{\rm min}}^{z_s^{\rm max}}\,dz_s\,|A[m_*](z_r,z_s)|^2 
\end{equation}
Since 
\[
\frac{\sqrt{2-\alpha}-2\sqrt{\alpha}}{\sqrt{2+\alpha}+2\sqrt{\alpha}}
\rightarrow 1 \mbox{ as } \alpha \rightarrow 0,
\]
inequalities \ref{eqn:ub} and \ref{eqn:lb} imply that for $\alpha$
small enough (and consequenty $\epsilon_*$ small enough,
\[
\frac{1}{2}j(\sigma_0) = \frac{1}{2}(j(\sigma_0) + j(1)) <
j\left(\frac{1}{2}(1+\sigma_0)\right),
\]
\end{proof}

\begin{rem} Since 
\[
\frac{\int d\omega \omega^2 |\hat{f}_{\epsilon}(\omega)|^2}{\int
  d\omega |\hat{f}_{\epsilon}(\omega)|^2}
= \frac{\int dt \left(\frac{df_{\epsilon}}{dt}(t)\right)^2}{\int dt
  \left(f_{\epsilon}(t)\right)^2}
\]
\[
=\epsilon^{-2}\frac{\int dt \left(\frac{df_{1}}{dt}(t)\right)^2}{\int dt
  \left(f_{1}(t)\right)^2}
\]
and the left-hand side can be interpreted as mean square frequency in
$f_{\epsilon}$, $\epsilon$ can be interpreted as relative wavelength
(relative to the root mean square wavelength in $f_1$). Thus the maxiumum
diameter of an inscribed sphere in the domain of convexity of $J_{\rm
  LS}$ is proportional to a wavelength. This fact is often expressed
in the literature by stating that to ensure success of FWI, the
initial estimate of slowness must be sufficient to predict traveltimes
within a half-wavelength. Of course, Theorem \ref{thm:FWIconv} does not show that
local optimization would not succeed, that is, that stationary points
other than the global minimizer exist, merely that the domain of
convexity has diameter on the order of a mean wavelength. 

Note that the conclusion of Theorem \ref{thm:FWIconv} holds regardless of whether
$f_1$ is oscillatory or mean-zero, so the common description of this
phenomenon as ``cycle-skipping'' is somewhat misleading.
\end{rem}

\section{Source-Receiver Extension}

As illustrated in the previous section, mean-square data misfit tends
to be large, and instensitive to small
velocity changes, except in the immediate vicinity of its global
minimizer. This misfit saturation is the root cause of the poor
performance of local optimization methods for $J_{\rm LS}$..

Several cures for this pathology have been explored. The state of the
art relies on small enough error in the initial velocity estimate,
relative to the longest wavelengths reliably present in the data. The non-overlapping
signal supports of the example given in the last section are
impossible if enough low-frequency signal exists in the
data. Unfortunately frequency content is limited by acquisition
technology. Sometimes it is possible to supply a sufficiently
accurate initial model in view of spectral limitations, and sometimes
it is not, or even to know
whether one has been successfully supplied

Another class of alternatives extends the simulation map
$(m,f) \mapsto S[m]f$ to a larger domain, so chosen that the ``relaxed''
inverse problem set in this larger domain always has a small-residual
solution. Some alternative objective must then be supplied, for which
a minimizer (at least for model-consistent data) is a model of the
original type, which by construction also fits the data.
This section introduces and analyzes such a data-fitting {\em
  extension} for the idealized crosswell tomography problem.

Before we introduce the extended simulation map, we redefine $S$ so
that $S[m], m\in M$ is injective, and the linear least squares problem
for $f$ coercive.

Suppose that ${\cal U} \subset M$ consists of slowness fields
satisfying the simple ray geometry hypothesis, having a common normal
neighborhood ${\cal N}$ containing $W_s$ and $W_r$. Suppose also that
${\cal U}$ contains at least one constant.
For typical $\bx_s \in W_s, \bx_r \in W_r$, denote by $r$ and $r_0$
the ray generated by the exponential map connecting $\bx_s,\bx_r$, respectively
the straight line segment between $\bx_s$ and $\bx_r$. Note that there
may be other connecting rays, but any such rays must exit ${\cal N}$ and be associated to larger
travel times than $\tau[m](\bx_s,\bx_r)$, which is the travel time
along $r$. Also, Fermat's principle holds: the travel time along $r$
is minimum amongst travel time along all paths from $\bx_s$ to
$\bx_r$, lying entirely inside of ${\cal N}$. Write
\[
\tau[m](\bx_s,\bx_r) = \int_r \,dl\, m,
\]
with $dl$ denoting the Euclidean arc length element. it follows that
\[
\tau[m](\bx_s,\bx_r) \le \int_{r_0} \,dl, m \le m_{\rm max}|\bx_r -\bx_s|
\bx_s|.
\]
Also, 
\[
\tau[m](\bx_s,\bx_r) \ge \int_r \,dl m_{\rm min} \ge m_{\rm min}
|\bx_r-\bx_s|
\]
since $r_0$ is geodesic for the Euclidean metric. Since we have
presumed that ${\cal U}$ contains at least one constant slowness
field, both bounds can be approximated arbitrarily well.
Set
\begin{eqnarray}
\label{eqn:rdef}
r_{\rm max} &=& \sup \{|\bx_r-\bx_s|: \bx_s \in W_s, \bx_r \in
                W_r\},\nonumber\\
r_{\rm min} &=& \inf \{|\bx_r-\bx_s|: \bx_s \in W_s, \bx_r \in W_r\}.
\end{eqnarray}
Putting these simple results together, it follows that
\begin{eqnarray}
\label{eqn:tauglob}
m_{\rm max}r_{\rm max} & = & \sup \{\tau[m](\bx_s,\bx_r): \bx_s \in W_s, \bx_r \in 
  W_r, m \in {\cal U}\} \nonumber\\
m_{\rm min}r_{\rm min} |& = & \inf \{\tau[m](\bx_s,\bx_r): \bx_s \in W_s, \bx_r \in 
  W_r, m \in {\cal U}\}
\end{eqnarray}
As these quantities will recur throughout the arguments to come,
define
\begin{eqnarray}
\label{eqn:tauextr}
\tau_{\rm max}& = & m_{\rm max}r_{\rm max} \nonumber\\
\tau_{\rm min}& = & m_{\rm min}r_{\rm min}
\end{eqnarray}

We now add the only hypothesis on ${\cal U}$ pertaining to the
behaviour of slowness fields outside of the common normal
neighborhood:

The maximum measurement time $t^{\rm max}$ satisfies the {\em
  no-return hypothesis}:
\begin{itemize}
\item
\begin{equation}
\label{eqn:tmaxbd}
t^{\rm max} >  \tau_{\rm max}-\tau_{\rm min}
\end{equation}
\item for any $m \in {\cal U}$ no ray with initial $X=\bx_s \in W_s$ passes over $\bx_r \in W_r$
at any time in the interval $(\tau[m](\bx_s,\bx_r), t^{\rm max}+\tau_{\rm max})$.
\end{itemize}
Note that the second condition means that $t^{\rm max}$ as specified 
here will may be used in the statement of Theorems \ref{thm:gosimple}
and \ref{thm:smdep}. In fact, the larger limit $t^{\rm max} +\tau_{\rm max})$ may be so used. The extra margin will be needed to
make some later estimates uniform over large sets of
slowness fields.

Since rays passing over $W_r$ a second time necessarily leave ${\cal
  N}$, this hypothesis actually concerns the global behaviour of the
slowness field. It is satisfied for any constant slowness in ${\cal
  U}$, and is clearly open. Therefore beginning with an open set of
slownesses in $M^k$, $k \ge 2$, satisfying the simple ray geometry
hypothesis and containing a constant slowness, there must be a subset
satisfying also the no-return hypothesis.

One might naturally ask how a slowness field may be
extended outside the image domain ${\cal N}$ of the exponential map to
satisfy the no-return hypothesis with the largest possible $t^{\rm
  max}$, however we shall not address this question here. 

Choose $\Delta > 0$ and define
\begin{equation}
\label{eqn:fwddomdef}
D_{\Delta} = \{f \in L^2(\bR): {\rm supp}\,f \subset (-\tau_{\rm min} +
\Delta, t^{\rm max} -\Delta - \tau_{\rm max})\}
\end{equation}

\begin{theorem}
\label{thm:fwdinj}
Suppose that ${\cal U} \subset M$, $t^{\rm max}$ satisfy the simple 
ray geometry and no-return hypotheses, $\Delta >0$, and $w=1$ on 
$[\Delta, t^{\rm max}-\Delta]$. Then for $m \in {\cal U}$,
$S[m]:D_{\Delta} \rightarrow L^2(\Omega)$ is coercive, and 
\[
f \in D_{\Delta} \,\Rightarrow \, S[m]f (z_r,t;z_s) =
A(z_r,z_s)f(t-T(z_r,z_s)) 
\]
\begin{equation}
\label{eqn:gosimplemod} 
+ w(z_r,t;z_s)\int_{T(z_r,z_s)}^{\infty}
\,ds\,B(z_r,s;z_s)f(t-s). 
\end{equation}
\end{theorem}
\begin{proof}
Inequalities \ref{eqn:tauglob} and \ref{eqn:tmaxbd} imply that if
$t \in (-\tau_{\rm min} +
\Delta, t^{\rm max} -\Delta - \tau_{\rm max})$,  then $t+\tau[m](\bx_s,\bx_r) \in (\Delta ,t^{\rm
  max}-\Delta)$. Thus if $f \in D_{\Delta}$, then $w=1$ on ${\rm
  supp}\,f(\cdot-\tau[m](\bx_s,\bx_r))$, establishing
\ref{eqn:gosimplemod}. The injectivity of $S[m]$ follows from Picard
iteration, applied to \ref{eqn:gosimplemod}.
\end{proof}

\begin{definition} (Source-receiver extension)
Suppose that ${\cal U} \subset M$, $t^{\rm max}$ satisfy the simple
ray geometry and no-return hypotheses. Define $\bar{w}_0$ to be the
characteristic function of $\{t<t^{\rm max}\}$ in $\Omega$.  

Define the
{\em source-receiver} extended source domain
\[
\oD_0 = \{\of \in L^2(\Omega): {\rm supp}\,f \subset 
Z_r\times Z_s 
\]
\begin{equation}
\label{eqn:extdomdef}
\times (-\tau_{\rm max}, t^{\rm max} - \tau_{\rm  min})\}
\end{equation}
and extended modeling operator 
\be\label{eqn:extdom}
\oS[m]: \oD_{0} \rightarrow L^2(\Omega) 
\ee
by
\be\label{eqn:extmap}
\oS[m]\of (z_r,t;z_s) = \bar{w}_0(z_r,t;z_s)\bar{p}(\bx_r,t;\bx_s).
\ee
\end{definition}
in which $\bar{p}$ is the solution of the wave equation
\ref{eqn:pressure} with $f(t) = \of(z_r,t;z_s)$.  That is, $\oS$ acts
by computing each data trace $\bar{p}(\bx_r,t;\bx_s)$ as if it were
generated by an indepedent source function $\of(z_r,\cdot;z_s)$ for
each source and receiver location.

Note that the time interval in the definition of $\oD_0$ strictly
contains that in the definition of $D_{\Delta}$.
Define the extension map $E: D_{\Delta} \rightarrow \oD_0$ by 
$(Ef)(z_t,t;z_s) = f(t)$, with $f$ on the RHS being regarded as
a constant function of $z_sz_r$. Then $\oS$ is an
extension of $S$, in the sense that
\be\label{eqn:extprop}
\oS[m](Ef) = S[m]f,\,f \in D_{\Delta}.
\ee

At first blush, it would appear that computing $\oS$ would be a very
expensive proposition, even after discretization in $z_s$ and $z_r$,
as each trace would have to be simulated independently. However this
is not the case, as \be\label{eqn:extgreen} \oS[m]\of(z_r,t;z_s) =
\bar{w}_0(z_r,t;z_s(G[m](\bx_r,\cdot;\bx_s) *_t \of(,z_r,\cdot;z_s))(t), \ee
with $G[m]$ being the Green's function of the problem
(\ref{eqn:pressure}), defined above. $G[m]$ can be approximated
numerically, one numerical simulation for each value of $z_s$. Once
this is done, the evaluation of $\oS[m]\of$ is a relatively
inexpensive convolution for each trace.

The main advantage of the extended model is that any data can be fit -
for any slowness model (within the class specified in the last
section)! More precisely, define 
\begin{equation}
\label{eqn:extrange}
\oF_0 = \{d \in L^2(\Omega): d=0 \mbox{ if } t<0 \mbox{ or } t>t^{\rm 
  max} \}. 
\end{equation}
\begin{theorem}
\label{thm:extsur}
Suppose that ${\cal U} \subset M$, $t^{\rm max}$ satisfy the simple 
ray geometry and no-return hypotheses
Then for $m \in {\cal U}$,
\begin{equation}
\label{eqn:extsur}
\oF_0 \subset \oS[m]\oD_0.
\end{equation}
Define $\oS_0[m]: L^2(\Omega) \rightarrow L^2(\Omega)$ and its inverse by
\begin{eqnarray}
\label{eqn:approxrinv}
\oS_0[m]\of(z_r,t;z_s) & = &
A[m](z_r,z_s)\of(z_r,t-T[m](z_r,z_s);z_s),\nonumber\\
\oS_0^{-1}[m]d(z_r,t;z_s) &=&
A[m]^{-1}(z_r,z_s)d(z_r,t+T[m](z_r,z_s);z_s),
\end{eqnarray}
and $\bar{R}_0[m]: L^2_{\rm comp}(\Omega) \rightarrow L^2_{\rm comp}$
by
\[
\bar{R}_0[m]d(z_r,t;z_s) 
\]
\begin{equation}
\label{eqn:approxres}
= \frac{\bar{w}_0(z_r,t+T[m](z_r,z_s);z_s)}{A(z_r,z_s)}
\int_0^t\,ds\,B[m](z_r,t+T[m](z_r,z_s)-s;z_s)d(z_r,s;z_s)
\end{equation}
Then 
\begin{equation}
\label{eqn:rinvdef}
\oS^R[m] = \oS_0^{-1}
\sum_{k=0}^{\infty}(-\bar{R}_0[m])^k \bar{w}_0
\end{equation}
converges in norm to a bounded operator $\oS^R[m]: \oF_0
\rightarrow \oD_0$, and $\oS^R[m]$ is a right inverse
for $\oS[m]$ on $\oF_0$, that is, for $d \in \oF_0$,
\begin{equation}
\label{eqn:invreln}
\oS[m]\oS[m]^Rd = d.
\end{equation}
\end{theorem}

\begin{proof} 
Observe that  the definition \ref{eqn:extmap} can be re-written using
the Green's function components $A, T,$ and $B$ and the cutoff function
$\bar{w}_0$ in the form
\[
\oS[m]\of(z_r,t;z_s) = \bar{w}_0(t) \left(
A[m](z_r,z_s)\of(z_r.t-T[m](z_r,z_s);z_s) \right.
\]
\begin{equation}
\label{eqn:extmapexp0}
\left.+ \int_{T[m](z_r,z_s)}^{\infty}\,ds\,B[m](z_r,s;z_s)\of(z_r,t-s;z_s)\right)
\end{equation}
According to the no-return hypothesis, $B[m]$ is well-defined and
smooth for $t \le t^{\rm max} + \tau_{\rm max}$. For $\of \in
\oD_0$, the upper limit of integration may be replaced by $t + \tau_{\rm max}$. Since only
$t\le t^{\rm max}$ is significant on the right-hand side of
\ref{eqn:extmapexp0}, the integrand in the last summand is
well-defined, and the integral defines an
operator $\bar{B}[m]: \oD_0  \rightarrow L^2(\Omega)$,
\begin{equation}
\label{eqn:barb}
\bar{B}[m]\of(z_r,t;z_s) = 
\int_{T[m](z_r,z_s)}^{t^{\rm max}+\tau_{\rm max}}\,ds\,B[m](z_r,s;z_s)\of(z_r,t-s;z_s)
\end{equation}
Then equation \ref{eqn:extmapexp0} can be abbreviated as
\begin{equation}
\label{eqn:extmapexpabbrev}
\oS[m] = \bar{w}_0 (\oS_0[m] + \bar{B}[m]).
\end{equation}
From the definitions of $\oS_0,\bar{R}_0$,
\[
\bar{w}_0\bar{B}[m] = \bar{R}_0[m]\oS_0[m]
\]
whence
\begin{equation}
\label{eqn:extmapexp}
\oS[m]\of = \bar{w}_0(I + \bar{R}_0[m]) \oS_0[m]\of.
\end{equation}


Note that $\bar{R}_0$ preserves supports of the form $t_0 \le t \le
t^{\rm max}$, in particular maps $\oF_0$ into itself.
Straightforward estimates using a weighted $L^2$ norm with decreasing
exponential weight in $t$ shows that as an operator on $\oF_0$, $I + \bar{R}_0[m]$ is invertible,
and its inverse is given by the Neumann series appearing in the
definition \ref{eqn:rinvdef} of $\oS^R[m]$.

For $d \in \oF_0$, then, $\bar{w}_0 d = d $, and
$\sum_{k=0}^{\infty}(-\bar{R}_0[m])^kd  \in \oF_0$. The definition of
$\oD_0$ is concocted to guarantee that if $d \in \oF_0$, then
$\oS[m]^{-1}d \in \oD_0$ for any $m \in {\cal U}$. Thus equation
\ref{eqn:rinvdef} defines a map from $\oF_0$ to $\oD_0$. While in
general $\oS_0[m]\oD_0$ is not a subset of $\oF_0$,
$\oS_0[m]\oS_0[m]^{-1} = I $ so membership in $\oF_0$ is preserved:
\[
\oS_0[m]\oS_0[m]^{-1}\sum_{k=0}^{\infty}(-\bar{R}_0[m])^kd  \in \oF_0.
\]
Tacking on the other factors in the expansion \ref{eqn:extmapexp} of
$\oS[m]$,  the conclusion \ref{eqn:invreln} follows, which in turn
implies the inclusion \ref{eqn:extsur}.
\end{proof}

\section{Extended Full Waveform Inversion}
Theorem \ref{thm:extsur} shows that data fitting with $\oS$, rather
than $S$, cannot suffer from the difficulty identified in Theorem
\ref{thm:FWIconv}, that is, failure to fit the data until the slowness
field predicts the right kinematics within a wavelength or so: in
fact, data can be fit via $\oS$ with essentiallly any slowness, however
kinematically inaccurate.. However
this pleasant property does not in itself yield a solution of the
inverse problem, because it is achieved at the price of adding a large
number of parameters to the models space. This enlargement of the
model space has two negative consequecnes:
\begin{itemize}
\item[1. ] The extended modeling map $\oS$ is not
injective - it is not possible to define the domain and range so that
this map is bijective for an open set ${\cal U}$ of slowness
models. 
\item[2. ] More fundamentally, since data
can be fit with any slowness model at all, data fitting no longer
constrains the slowness model.
\end{itemize}
Item 2 is a
consequence of the unphysical nature of the extended source, with each
data trace modeled by an independent source trace. The large number of
parameters added in extending the model must somehow be controlled, and 
forced to assume trivial values in order to recover  a slowness $m$
and physical source $f$ that together explain the data.

Compensation for the null space of $\oS[m]$ is straightforward via
Tihonov regularization. That is, choose a regularization parameter
$\alpha >0$ and define the inverted extended source $\ofa[m,d], m \in
M, d \in L^2(\Omega)$, by
\begin{equation}
\label{eqn:regextinv}
\ofa[m,d] = \mbox{ argmin}_{\of}\,\frac{1}{2}(\|\oS[m]\of - d\|^2 + \alpha^2\|\of\|^2).
\end{equation}
If ${\cal U}$ and $t^{\rm max}$ are chosen as in the preceding
discussion, that is, to satisfy the simple ray geometry and no-return
hypotheses, and $m \in {\cal U}, d \in \oF_0$, then
as $\alpha \rightarrow 0$, $\ofa[m,d] \rightarrow \oS[m]^Rd$.

A hint about how the second issue might be addressed comes from
further examination of the synthetic example defined earlier. Recall
that $m_*$ is the target slowness 
(corresponding to the velocity depicted in Figure \ref{fig:velbig})
used to generate the data $d=S[m_*]f$ (Figure \ref{fig:shot6kmbig}).
Figures \ref{fig:shot6kmbig_bigdecon} and \ref{fig:shot6kmbig_decon} display
$\ofa[m,d]$ for $d=S[m_*]f$ and $m=m_*$ and $m=m_0 = 0.5 s/km$
respectively. In both cases, the regularization parameter $\alpha$ is
set = $10^{-3}$. The non-extended or
physical source $f$ is localized near $t=0$. The inverted extended
source for correct slowness (Figure \ref{fig:shot6kmbig_bigdecon})
yields a result very close to the extension $Ef$, and has the same
quality (energy focused at $t=0$, independent of $z_r$). Inversion with the incorrect slowness
shifts the extended source into $t \ne 0$. Also, the level surfaces of
$\ofa[m_0,d]$ (Figure \ref{fig:shot6kmbig_decon}) are not
constant in $z_r$, but vary significantly. If this were not true, then
any $z_r=const.$ trace could be taken as the source function $f$, and
the inverse problem would be solved. 

These observations suggest two alternative methods for measuring the
correctness of $m$ from the properties of $\ofa[m,d]$. First,
concentration near $t=0$ may be measurred by multplying $\ofa[m,d]$ by $t$ and taking the $L^2$ norm of the result. If the velocity
is correct, {\em and} the target source function actually is supported
near $t=0$, this norm will be small, as the support of $\ofa[m,d]$
will be close to $t=0$ also (see Figure \ref{fig:shot6kmbig_bigdecon}). Second, the variance with respect to $z_r$ may
be measured by taking the norm of the $z_r$ derivative, which will be
zero (not just small) for an extended model in the range of the
extension operator. That is, the range of the extension operator (the
``physical'' extended models is the null space of the $z_r$ derivative.
Either case may be captured by small values of the objective function
\be\label{eqn:dsdef}
J_{\rm DS}[m,d] = \frac{1}{2}\|A\ofa[m,d]\|_{L^2(\Omega)}^2,
\ee
in which
\begin{itemize}
\item $A\of(z_s,z_r,t) = t\of(z_s,z_r,t)$, or
\item $A\of(z_s,z_r,t) = \nabla_{z_s,z_r}\of(z_s,z_r,t)$.
\end{itemize}
Operators $A$ which indirectly detect the kinematic correctness of a
wave propagation model via qualities of auxiliary data generated with
that model (in this case, the extended source estimate $\ofa[m,d]$)
have come to be known as {\em annihilators}, since in many cases the
auxiliary data corresponding to correct models is in (or near) the
null space. Model estimation via minimizations of functions such as
$J_{\rm DS}$ is sometimes called {\em differential semblance}
optimization, after a name introduced in some of the early papers on
this concept \cite[]{SymesCar:91,KerSy:94} to reflect the use of
differential operators as annihilators. A differential annihilator
(the second option above) was used in the first papers known to us on the
source-receiver extension \cite[]{SongSymes:94b,Symes:94c}. A brief
survey of the literature on this topic appeared above, in the
Introduction.

Both choices have drawbacks. The first choice (multiplication by $t$)
defines a bounded operator on $D_0$, however without a null space;
without some further compactness-inducing constraint on $\of$, a
minimizer of $J_{\rm DS}$ may not exist. For computational purposes,
this defect appears to be minor, as discretization effectively forces
the search for $\ofa$ into a compact subspace of
$L^2(\Omega)$. However this effect has not been analyzed; the applied
literature uses it without mathematical justification (but with
considerable success - see for example \cite{Warner:16}). The second
choice has a complementary problem: the derivative with respect to
$z_r,z_s$ is of course unbounded, rendering $J_{\rm DS} $ only densely
defined. The obvious remedy is to use a subspace of $H^1(\Omega)$ as
the domain of $\oS$ - see \cite{KerSy:94} for more discussion of this
option, and implementation details. 

We will accept the drawbacks of the first option above (multiplication
by $t$) in the remainder of our discussion, simply because the
implementation is simpler and is the approach taken in the
computational examples accompanying this paper.

\section{Gradient and Hessian}

Throughout this section we will assume without further comment that
${\cal U}$ and $t^{\rm max}$ are chosen to satisfy the simple ray
geometry and no-return hypotheses, as explained earlier.

Recall from Theorem \ref{thm:smdep}, and surrounding discussion,
especially Remark \ref{rem:derivloss}, that $(m,f) \mapsto S[m]f$ is
only differentiable in $m$ if $f$ has one square integrable derivative
or better. The same is true fore $\oS$. The local
behavioiur of $J_{\rm DS}$ follows from unpacking the directional derivative of
$\oS$ at $m \in {\cal U}$ in a direction $\delta m \in
C^{\infty}(\bR^3)$ into a leading term with explicit dependence on $\partial
\of/ \partial t$, plus an $L^2$-continuous remainder.
The leading term factors as $\oS[m]Q[m]\delta m$, in which $Q$ is a
first order differential operator:
\begin{equation}
\label{eqn:qdef}
Q[m]\delta m =\frac{1}{A[m](z_r,z_s)}\left(-(DT[m]\delta
m) (z_r,z_s)\frac{\partial}{\partial t} + (D A[m]\delta m)(z_r,z_s)\right)
\end{equation}
A suitable domain for $Q[m]\delta m$ is
\begin{equation}
\label{eqn:d1def}
\oD_1 = H^1_0(Z_r \times Z_s \times [-\tau_{\rm max},t^{\rm
  max}-\tau_{\rm min}])
\end{equation}
Note that $Q[m]\delta m$ is {\em essentially skew-symmetric}, that is
\begin{equation}
\label{eqn:qess}
(Q[m]\delta m) + (Q[m]\delta m)^T =2 DA[m]\delta m.
\end{equation}
The right-hand side of equation \ref{eqn:qess} is a differential
operator of order zero, in particular bounded on $L^2(\Omega)$,
whereas $Q$ is first order.
\begin{theorem}
\label{thm:q}
There exists continuous map $P: {\cal U} \times C^{\infty}(\bR^3)
\rightarrow {\cal L}(\oD_0,\oD_0)$, linear in its second argument, 
so that for $\of \in \oD_1$,
\begin{equation}
\label{eqn:paramderiv}
D(\oS[m]\of)\delta m = \oS[m]((Q[m]\delta m) + (P[m]\delta
m))\of
\end{equation}
$P$ is a Volterra convolution operator of the first kind. In
particular, the operators $P$ and $Q$ commute.
\end{theorem}
\begin{proof}
Suppressing $z_r,z_s$ for the moment, recall the definition of $\oS_0[m]$
\[
(\oS_0[m]\of)(t) = A[m]\of(t-T[m]).
\]
Also, from the defintion \ref{eqn:barb}, 
\[
(\bar{B}[m]\of)(t) = 
\int_{T[m]}^{\infty}\,ds\,B[m](s)\of(t-s) 
\]
\[
=\int_{-\infty}^{t-T[m]}\,ds\,B[m](s)(t-s)\of(s)
\]
\begin{equation}
\label{eqn:barbalt}
= (\oS_0[m]\tilde{B}[m])\of(t),
\end{equation}
where
\begin{equation}
\label{eqn:tildeb}
(\tilde{B}[m])\of(t) = \int_{-\infty}^{t}\,ds\,B[m](t+T[m]-s)\of(s) 
\end{equation}
In order to view $\tilde{B}[m]$ as a Volterra integral operator on $D_0$ with
smooth kernel, it is necessary to extend $B[m]$ smoothly to $t >
t^{\rm max} + \tau_{\rm max}$. The extended values do not play
any role in the result of the computation \ref{eqn:barbalt} for $t >
t^{\rm max}$. 

These manipulations combine to yield a re-write of the expression
\ref{eqn:extmapexpabbrev} for $\oS[m]$:
\begin{equation}
\label{eqn:extmapexpabbrevalt}
\oS[m] = \bar{w}_0 \oS_0[m](I + \tilde{B}[m]).
\end{equation}
Its directional derivative at $m \in {\cal U}$ in the direction
$\delta m \in C^{\infty}(\bR^3)$ is
\[
D(\oS[m]\of)\delta m = \bar{w}_0( (D(\oS_0[m])\delta m) (I +
\tilde{B}[m])\of+\oS_0[m]D(\tilde{B}[m]\of)\delta m)
\]
Note that 
\[
(D(\oS_0[m]\of)\delta m)(t) = (DA[m]\delta m)\of(t-T[m]) -A[m](DT[m]\delta
m)\frac{\partial \of}{\partial t}(t-T[m])
\]
\[
=\oS_0[m]\left(-\frac{(DT[m]\delta m)}{A[m]}
  \frac{\partial \of}{\partial t} + (D\log A[m]\delta m)\of\right)(t)
\]
so 
\[
\bar{w}_0(D(\oS_0[m]\of)\delta m) =  \bar{w}_0\oS_0[m]Q[m]\delta m
\]
whence
\[
D(\oS[m]\of)\delta m = \bar{w}_0\oS_0[m] (Q[m]\delta m)(I+\tilde{B}[m])\of 
\]
\[
+ \bar{w}_0\oS_0[m]D(\tilde{B}[m]\of)\delta m 
\]
Since $\tilde{B}$ is a convolution operator in $t$, it commutes with
multiplication by functions of $z_s,z_r$, and with $\partial/\partial
t$, so this is actually equation \ref{eqn:paramderiv}, provided that
$P$ is defined as 
\begin{equation}
\label{eqn:pdef}
P[m]\delta m = (I + \tilde{B}[m])^{-1}D\tilde{B}[m] \delta m
\end{equation}
Since both factors on the right-hand side of equation \ref{eqn:pdef}
are Volterra convolution operators (of second and first types
respectively), so their product is a Volterra convolution operator of
the first kind, and $P$ commutes with $Q$ as claimed.
\end{proof}

In the remainder of this section we will use the equivalence between $D(\oS[m]\of)\delta
m$, the directional derivative of the vector-valued function $m
\mapsto \oS[m]\of$,  and $(D\oS[m]\delta m)\of$, the directional
derivative of the operator-valued function $m \mapsto \oS[m]$,
evaluated at the vector $\of$, as is valid if $\of \in \oD_1$.

Just as existence of a derivative with respect to $m$ of $\oS[m]\of$
requires that $\of$ have a square-integrable time derivative, so
existence of a derivative with respecto $m$ of $\oS[m]^Td$ requires
that $d$ have a time derivative. Accordingly, define
\[
\oF_1 = \oF_0 \cap H^1(\Omega)
\]
and metrize $\oF_1$ with the Sobolev 1-norm.

\begin{theorem}
\label{thm:qcomm}
For $d \in \oF_1$, $m \in {\cal U}$,
\[
DJ_{\rm DS}[m,d]\delta m =\alpha^2 \langle ((P[m]\delta m)^T +(P[m]\delta
m))\ofa[m,d],\oga[m,d]\rangle 
\]
\begin{equation}
\label{eqn:qderiv}
-\langle (P[m]\delta
m)\ofa[m,d],A^TA\ofa[m,d]\rangle + \langle [Q[m]\delta m,A^TA]\ofa[m,d],\ofa[m,d]\rangle.
\end{equation}
For $m \in {\cal U}$, $\delta m \in C^{\infty}(\bR^3)$, the linear
functional on $\oF_1$ defined by $d \mapsto DJ_{\rm DS}[m,d]\delta m$
extends by continuity to $\oF_0$.
\end{theorem}
\begin{proof}
The normal equation equivalent to \ref{eqn:regextinv} are:
\begin{eqnarray}
\label{eqn:normal}
N_{\alpha}[m]\of &=& \oS[m]^Td\nonumber\\
N_{\alpha}[m] & = & \oS[m]^T\oS[m] + \alpha^2 I
\end{eqnarray}
So
\begin{equation}
\label{eqn:ofanorm}
\ofa[m,d] = N_{\alpha}[m]^{-1}\oS[m]^Td
\end{equation}
and 
\[
J_{\rm DS}[m,d] = \frac{1}{2}\|A N_{\alpha}[m]^{-1}\oS[m]^Td\|^2
\]
For $d \in \oF_1$,
\[
DJ_{\rm DS}[m,d] \delta m = \langle A (DN_{\alpha}[m]^{-1}\delta m)
\oS[m]^Td + AN_{\alpha}[m]^{-1}D(\oS[m]^Td)\delta m,A\ofa[m,d]\rangle
\]
\[
=\langle - N_{\alpha}[m]^{-1}(DN_{\alpha}[m]\delta m)\ofa[m,d]
  +N_{\alpha}[m]^{-1}D(\oS[m]^Td)\delta m,A^TA\ofa[m,d]\rangle
\]
Introduce $\oga[m,d]$, defined by
\begin{equation}
\label{eqn:gdef}
\oga[m,d] = N_{\alpha}[m]^{-1}A^TA\ofa[m,d].
\end{equation}
Then
\[
DJ_{\rm DS}[m,d] \delta m = \langle - (DN_{\alpha}[m]\delta m)\ofa[m,d] +
  D(S[m]^Td)\delta m, \oga[m,d]\rangle
\]
\begin{equation}
\label{eqn:pregrad1}
= \langle (DS[m]^T\delta m)(d-S[m]\ofa[m,d]) - S[m]^T (DS[m]\delta m)\ofa[m,d],\oga[m,d]\rangle
\end{equation}
From Theorem \ref{thm:q}, this is
\[
=\langle (Q[m]\delta m + P[m]\delta m)^TS[m]^T(d-S[m]
\ofa[m,d]) - S[m]^TS[m]
(Q[m]\delta m + P[m]\delta m)\ofa[m,d],\oga[m,d]\rangle
\]
which because of the normal equation \ref{eqn:normal} is
\[
= \langle (Q[m]\delta m + P[m]\delta m)^T\alpha^2\ofa[m,d],\oga[m,d]\rangle 
-\langle (Q[m]\delta m + P[m]\delta m)\ofa[m,d], (N_{\alpha} -
\alpha^2I)\oga[m,d]\rangle
\]
\[
= \alpha^2 \langle (Q[m]\delta m + P[m]\delta
m)^T\ofa[m,d],\oga[m,d]\rangle 
+ \alpha^2 \langle (Q[m]\delta m + P[m]\delta
m)\ofa[m,d],\oga[m,d]\rangle
\]
\[
-\langle (Q[m]\delta m + P[m]\delta
m)\ofa[m,d],A^TA\ofa[m,d]\rangle
\]
\[
=\alpha^2 \langle ((P[m]\delta m)^T +(P[m]\delta
m))\ofa[m,d],\oga[m,d]\rangle 
\]
\[
-\langle (Q[m]\delta m + P[m]\delta
m)\ofa[m,d],A^TA\ofa[m,d]\rangle
\]
by virtue of the defintion \ref{eqn:gdef}. Finally, the essential skew-symmetry of
$Q[m]\delta m$ (equation \ref{eqn:qess}) implies that
\[
-\langle (Q[m]\delta m)\ofa[m,d],A^TA\ofa[m,d]\rangle
\]
\[
=-\langle [A^TA,Q[m]\delta m]\ofa[m,d],\ofa[m,d]\rangle- \langle A^TA
\ofa[m,d],(Q[m]\delta m)^T\ofa[m,d]\rangle
\]
\[
=-\langle [A^TA,Q[m]\delta m]\ofa[m,d],\ofa[m,d]\rangle +\langle
(Q[m]\delta m - 2DA[m]\delta m)\ofa[m,d],A^TA\ofa[m,d]\rangle
\]
whence
\begin{equation}
\label{eqn:qcomm}
-\langle (Q[m]\delta m)\ofa[m,d],A^TA\ofa[m,d]\rangle =
\langle\left(\frac{1}{2} [Q[m]\delta m,A^TA] + A^TA DA[m]\delta m\right)\ofa[m,d],\ofa[m,d]\rangle,
\end{equation}
and equation \ref{eqn:qderiv} is established.

Recall that $Q[m]\delta m$ is a $t$-independent multiple of
$\partial/\partial t$, and $A = $ multiplication by $t$. Therefore
\begin{equation}
\label{eqn:qcommdt}
[Q[m]\delta m,A^TA] = 2t\frac{DT[m]\delta m}{A[m]}
\end{equation}
defines a bounded multiplier on $\oD_0$. Since all of the other operators
appearing in equation \ref{eqn:qderiv} are bounded on $\oD_0$, and
$\ofa[m,d]$ and $\oga[m,d]$ are continuous in $d \in \oF_0$, 
continuous extension of $DJ_{\rm DS}[m,d]\delta m$ to $d \in \oF_0$ follows.
\end{proof}

To understand how the term involving the commutator $[Q,A^TA]$ may
dominate the derivative of $J_{\rm DS}$, we introduce a measure of
oscillation in time:

\begin{definition} For $u \in L^2(\Omega)$, $a < b$,
\begin{equation}
\label{eqn:lamdef}
\lambda_{[a,b]}(u) =
\left(\frac{\int_{Z_r}\,dz_r\,\int_{Z_s}\,dz_s\,\int_a^b\,dt,\,\left(\int_a^t\,ds\,u(z_r,s;z_s)\right)^2}{\int_{Z_r}\,dz_r\,\int_{Z_s}\,dz_s\,\int_a^b\,dt\,u^2}\right)^{\frac{1}{2}}.
\end{equation}
\end{definition}
Essentially, $\lambda$ is the ratio of the partial Sobolev $W^{-1,2}$ norm to
the $L^2$ norm, and measures oscillation in $t$. If $u(t) = \sin \pi \omega
t$, then $\lambda_{[a,b]}(u) = \frac{C}{ \omega} + O(\omega^{-2})$, so
$\lambda$ is more or less wavelength, hence the notation. Its
importance for the analysis of the gradient lies in a simple lemma,
the proof of which we leave to the reader:

\begin{lem}
\label{lem:lamlem}
Suppose that $k \in C^1(\Omega)$, $a < b$, and define for $u \in L^2(\Omega)$
\[
Ku(z_r,z_s,t) = \int_a^t\,ds\,k(z_r,t-s;z_s) u(z_r,s;z_s).
\]
Then 
\[
\|Ku\|_{L^2(Z_r\times Z_s \times [a,b])} \le
(2+(b-a)^2)^{\frac{1}{2}}\|k\|_{C^1(\Omega)}\lambda_{[a,b]}(u)\|u\|.
\]
Also, if ${\rm supp} \,u \subset [a,b]$ and $\int_a^bu = 0$, then $a' \le a < b \le b'$
implies that
\[
\lambda_{[a',b']}(u) = \lambda_{[a,b]}(u)
\]
and
\[
\|K^Tu\|_{L^2(Z_r\times Z_s \times [a,b])} \le
(2+(b-a)^2)^{\frac{1}{2}}\|k\|_{C^1(\Omega)}\lambda_{[a,b]}(u)\|u\|.
\]
\end{lem}

\begin{theorem}
\label{thm:qerr}
Under the hypotheses of Theorem \ref{thm:qcomm}, for $d \in \oF_0$ for
which
\[
\int_0^{t^{\rm max}}\,d(z_r,t;z_s) = 0, z_r \in Z_r, z_s \in Z_s,
\]
there is $C >0$
depending on ${\cal U}$ and $t^{\rm max}$ so that
\[
|DJ_{\rm DS}[m,d]\delta m - \langle [Q[m]\delta
m,A^TA]\ofa[m,d],\ofa[m,d]\rangle| 
\]
\[
\le C \alpha^{-2}\lambda_{[0,t^{\rm max}]}(d)(1+\alpha^{-2}\lambda_{[0,t^{\rm max}]}(d))\|d\|^2
\]
\end{theorem}
\begin{proof}
Every operator appearing in the following expressions depends on $m
\in {\cal U}$, and $\ofa$ and $\oga$ depend on $d$ as well. Since these
dependencies are universal, we suppress them from the notation.

Using the expansion \ref{eqn:extmapexpabbrevalt}, rewrite the normal
equation \ref{eqn:normal} as
\[
(\alpha^2 I + (I+\tilde{B})^TS_0^t \bar{w}_0^2 S_0 (I+\tilde{B})) \ofa
= (I+\tilde{B})^TS_0^T \bar{w}_0d
\]
\[
=S_0^T d +\tilde{B}^TS_0^T d
\] 
from the definition of $\oF_0$. The left-hand side is
\[
= (\alpha^2 + S_0^T\bar{w}_0^2S_0) \ofa  + K \ofa,
\]
where
\[
K = \tilde{B}^T S_0^T\bar{w}_0^2S_0 + S_0^T\bar{w}_0^2S_0\tilde{B}
+ \tilde{B}^T S_0^T\bar{w}_0^2S_0\tilde{B}.
\]
Decompose $\ofa = \ofa^0 + \ofa^1$, where
\[
\ofa^0 = (\alpha^2 + A^2)^{-1}S_0^T d.
\]
Tracking supports, you see that
\[
S_0^T\bar{w}_0^2S_0 \ofa^0 = \S_0^TS_0\ofa^0 = A^2 \ofa^0,
\]
so 
\[
(\alpha^2 + S_0^T\bar{w}_0^2S_0) \ofa^0 =S_0^T d.
\]
Therefore, 
\[
N_{\alpha}\ofa^1 = \tilde{B}^TS_0^T \bar{w}_0d - K\ofa^0
\]
Since $\ofa^0$ is a multiple of a translate of $d$, from Lemma \ref{lem:lamlem} 
\[
\lambda_{[-\tau_{\rm max},t^{\rm max}-\tau_{\rm min}]}(\ofa^0) =
  \lambda_{[0,t^{\rm max}]}(d).
\]
Every summand in $K$ has a factor of $\tilde{B}$ or $\tilde{B}^T$ (or
both). Lemma \ref{lem:lamlem} implies that
\[
\|N_{\alpha}\ofa^1\| \le C \lambda_{[0,t^{\rm max}]}\|d\|
\] 
for suitable $C$ depending on $A, B$ hence on ${\cal U}$, $t^{\rm
  max}$. Since $N_{\alpha}-\alpha^2I \ge 0$ and $\|\ofa^0\|\le C\|d\|$,
obtain
\begin{eqnarray}
\label{eqn:f1barbd}
\|\ofa^1\| & \le & C \alpha^{-2}\lambda_{[0,t^{\rm max}]}(d))\|d\|\\
\label{eqn:fbarbd}
\|\ofa\| & \le & C(1 + \alpha^{-2}\lambda_{[0,t^{\rm max}]}(d))\|d\|
\end{eqnarray}
hence
\[
\alpha^2\|\oga\| \le C(1 + \alpha^{-2}\lambda_{[0,t^{\rm max}]})\|d\|.
\]
Finally, estimate $P \ofa = P \ofa^0  + P \ofa^1$ by applying Lemma
\ref{lem:lamlem} again to the first summand on the right, and using
the already-established bound for $\|\ofa^1\|$ to bound the second
term. Similar reasoning applied to $P^T \ofa$ and appeal to Theorem
\ref{thm:qcomm} establishes the result.
\end{proof}



\section{Transmission Caustics}

\section{Discussion}

\section{Conclusion}

\newpage

\inputdir{project}
%\plot{csqlens10}{width=\textwidth}{Lens: diameter = 0.5 km, thickness = 1.0 km, background velocity = 2 km/s, velocity at center = 1.4142  km/s.}
%\plot{shot6kmlens10conv}{width=\textwidth}{Shot gather - source at x=5100 m, z=10 m. Receivers between 4 km and 6km at depth = 3.0 km.}
%\plot{awicurve}{width=\textwidth}{RMS amplitudes of inversions for
%trace-dependent source (matched source), scaled by t (second moment
%objective). Convex combinations of homogeneous (2 km/s) and lens models.}
%\plot{./velbig_dec}{width=\textwidth}{Example: $v$ = 2 km/s for $z \le
%  0.5 km$, $=2.0 + (z-0.5)*0.5$ km/s for $z>0.5$ km. Source location
%  $x=5100$, $z=10$ m and receiver locations $x=3000 - 6000$ m,
%$z=3000$ m indicated.}
%\plot{shot6kmhom}{width=\textwidth}{Shot gather for geometry indicated
%  in Figure \ref{fig:velbig_dec_exp}, generated with homogeneous velocity:
%  $v_0=2$ km/s}
%\plot{shot6kmbig}{width=\textwidth}{Shot gather for geometry and
%  $v$ as in Figure \ref{fig:velbig_dec_exp}.}
%\plot{shot6kmbig_decon}{width=\textwidth}{Inverse image of data in 
%  Figure (\ref{shot6kmbig}) under extended modeling operator at $v_0$
%  = 2 km/s}
%\plot{shot6kmbig_bigdecon}{width=\textwidth}{Inverse image of data in 
%  Figure \ref{shot6kmbig} under extended modeling operator at target
%  $v$ (Figure \ref{fig:velbig_dec_exp}).}
%\plot{fwigrad6kmbighom}{width=\textwidth}{FWI gradient for shot gather
%defined in Figure \ref{fig:velbig_dec_exp}, at $v=v_0$ = 2 km/s. Scaled by
%10$^6$ for plotting. Note very
%small values, and sign is positive, indicating that FWI objective {\em
%increases} in the direction of the target velocity.}
%\plot{fwiscan}{width=\textwidth}{Plot of values of the FWI objective
%  along the line segment between $v_0$ and $v$, for the shot gather
%  described in Figure \ref{fig:velbig_dec_exp}. For abscissa $h$, value
%  plotted is FWI objective at $h v_0 + (1-h) v$. Note that objective
%  does in fact initially increase from its value at $v_0$, consistent
%  with the positive average value of the gradient as shown in Figure \ref{fig:fwigrad6km%bighom}. }
%\plot{ivagrad6kmbighom}{width=\textwidth}{Source-receiver Extended FWI
%  gradient, same geometry, data, and velocity model as in Figure
%  \ref{fig:fwigrad6kmbighom}. Note that overall sign is negative, indicating
% correct conclusion that $v_0$ is slow.}
%\plot{fwistack}{width=\textwidth}{FWI gradient at $v=v_0$ for line of shots 
%  extending from 2100 m to 8000 m. Scaled by
%10$^6$ for plotting. Receiver locations are windowed so 
%  that the range of offsets is the same for all shots. In effect, the 
%  gradient is a stack of translates of the gradient in Figure
%  \ref{fig:fwigrad6kmbighom}.}
%\plot{ivastack}{width=\textwidth}{Source-receiver Extended FWI gradient at $v=v_0$ for line of shots 
%  extending from 2100 m to 8000 m. Receiver locations are windowed so 
%  that the range of offsets is the same for all shots. In effect, the 
%  gradient is a stack of translates of the gradient in Figure
%  \ref{fig:ivagrad6kmbighom}.}

\bibliographystyle{seg}
\bibliography{../../bib/masterref}


\end{document}
