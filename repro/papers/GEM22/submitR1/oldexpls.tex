\section{Numerical Examples}
This section illustrates the most important conclusions developed in
the preceding sections by finite difference wavefield simulation.

\subsection{Synthetic models and simulation}
To illustrate the structure described in the preceding section, I
introduce two 2D acoustic models, one spatially homogeneous, the other
highly refractive. The first, homogenous model has $\kappa = 4$
GPa and $\rho = 1$ g/cm$^3$ throughout a rectangular domain of size 8 km ($x$) $\times$ 4 km
($z$). The second, refractive, model is a perturbation of the first by
a low-velocity acoustic lens positioned in the center of the
rectangle (Figure \ref{fig:bml0}. To produce this structure, the density is chosen
homogeneous as in the first model, while the bulk modulus decreases to
from 4 GPa outside the lens to 1.6 GPA in its center, as shown in Figure \ref{fig:bml0}.

\plot{bml0}{width=\textwidth}{Bulk modulus, lens model. Color scale is 
in GPa. Positions of point source and receiver line indicated.}

Discretization is conventional, with a rectangular grid and staggered
finite difference scheme \cite[]{Vir:84} of order 2 in time and
2$k$ in space; for most of the experiments reported below,
$k=4$. Absorbing boudary conditions of perfectly matched layer type
are applied at all boundaries of the simulation rectangle \cite[]{Habashy:07}.
Sampling operators such as $P_r$ are implemented via linear
interpolation, and source insertion via adjoint linear interpolation
(as noted above, in the continuum limit, sources are represented via
adjoint sampling). Steps in $x$ and $z$ are the same. In the following
examples, $\Delta x = 20$ m. This choice limits the temporal frequency
of accurately computed fields to rougly 12 Hz.

\cite{GeoPros:11} gives a description of the code
implementation, out-of-date in a few respects but overall accurate.
The implementation uses the discrete adjoint state method and
auto-generated code \cite[]{TapenadeRef13}, to assure that the
computed adjoint operators are adjoint at the level of machine
precision to the computed operators. The reverse-time storage issue is
resolved through the optimal checkpointing technique
\cite[]{Griewank:book,Symes:06a-pub}, again without loss of
precision. This procedure results in computed adjoints for
$S^+_{z_s,z_r}$ and other operators that pass usual test for adjoint
accuracy, comparing inner products with pseudorandom input vectors,
with errors well under machine precision.

%Typical results with pseudorandom input traces $d_r, h_z, w_r, f_s$ are:
%\begin{itemize}
%\item computed $\langle d_r, S^+_{z_s,z_r}h_s\rangle = -2.41069174,
%  \langle (S^+_{z_s,z_r})^Td_r, h_s \rangle = -2.41069388.$
%\item  computed $\langle w_r, V^+_{z_s,z_r}f_s\rangle = -2.73362470,
%  \langle (V^+_{z_s,z_r})^Tw_r, f_s \rangle = -2.73362136.$
%\end{itemize}
%Ideally, the differences of these inner products should be at most a
%relatively small multiple of machine precision, {\em relative} to the
%products $\|d_r\|\| S^+_{z_s,z_r}\|\|h_s\|$ and
%$\|w_r\|\|V^+_{z_s,z_r}\|\|f_s\|$ (division by these quantities makes the result
%dimensionless and scale-independent). The operator norm $\| S^+_{z_s,z_r}\|$ is
%computationally inaccessible, so instead I used the smaller quantities
%$\|d_r\|\| S^+_{z_s,z_r}h_s\|$ etc. as stand-ins - thereby
%overestimating the relative error between the inner products. In all
%cases, over a very large number of random inputs, the largest observed
%relative error estimate was $O(10^{-9})$, well under the appropriate limit for
%single precision.

%We also explore the dependence of a few results on frequency, using
%$\Delta x = 10$ m and $5$ m, to accomodate 25 and 50 Hz respectively,
%maintaining 8 samples per wavelength. All computations are carried out
%in single precision.

The horizontal line of receivers sits at depth $z_r = $ 1000 m, the
(extended) sources at $z_s=3000$ m.  %as
% shown in Figure \ref{fig:bml0}.
Source and receiver $x$ ranges from $2000$ to
$6000$ m. Note that we have
reversed the order relation between $z_s$ and $z_r$ described in the
text ($z_s<z_r$). This difference is immaterial for the purpose of
illustrating the mathematical structures developed in the preceding paragraphs.
%A single point (physical) source appears in these
%experiments, located at $x_s=3500$ m, $z_s=3000$ m, as also indicated
%in the figure.

%Extended sources are confined to the horizontal
%line through the physical source position, that is $z_s = 3000$ m,
%over a 4 km interval starting at $x_r=$ 2000 m..

%The point source pressure data generated by this configuration for the
%homogeneous model is displayed in Figure \ref{fig:recphh0}, for the
%lens model in \ref{fig:recplh0}. Triplication of arrivals is evident in the latter.

%While inversion of pressure data is the main object of this exercise,
%normal velocity ($v_z$) data will play an important role, so I display
%the corresponding gathers in Figures \ref{fig:recvzhh0} and \ref{fig:recvzlh0}.

\subsection{Creating downgoing fields}
The downgoing condition constrains high-frequency energy of localized
plane wave components, hence could be enforced by dip>
filtering. However, a simpler approach is to construct fields that
must be entirely downgoing at the source and receiver surfaces by
virtue of ray geometry.

Note that a point source
on $z=z_s$ creates high frequency energy traveling on rays parallel
and nearly parallel to $z=z_s$, so that won't do. However, placing a
point source at a depth $z_d<z_s$ will work. Since the examples used here are
homogeneous in $z<z_s$, and the sampling region for extended sources
is a finite interval, all rays carrying high frequency energy cross
the source surface $z=z_s$ at a postive angle, and the field and its
traces are {\em a priori} downgoing. The same is obviously true at the
receiver surface for the homogeneous model, but is also true for the
lens model, as no rays are refracted horizontally {\em at the receiver surface}.

The choice of a point source at
$z_d=3500$ m, $x_d=3500$ m,  bandpass filter wavelet with
corner frequencies $1, 2.5, 7.5, 12.5$ Hz, gives the causal pressure and
velocity gathers at $z=z_s=3000$ m
shown in Figures \ref{fig:dsrcphh0} and \ref{fig:dsrcvzhh0}.  Since the
mechanical parameters in the homogeneous and lens models are the same
for $z<z_s$, and no rays return to this zone in either model, these
data are asymptotically the same for both models, and I show only the
homogenous medium results.
%for the lowest frequency
%source, and scaled as appropriate for examples with higher frequency
%and finer sampling.
%wavelet as used for prior examples,

\plot{dsrcphh0}{width=\textwidth}{Trace $P_sp^+$ on $z=z_s=3000$ m of
  pressure field from point source at $z_d=3500$ m, $x_d=3500$ m,
  bandpass filter source. }

\plot{dsrcvzhh0}{width=\textwidth}{Trace $P_sv_z^+$ on $z=z_s=3000$ m of
  vertical velocity field from point source at $z_d=3500$ m, $x_d=3500$ m,
  bandpass filter source.}

\subsection{Equivalence of pressure, velocity sources}
These gathers are the pressure and
velocity traces
$(P_sp^+,P_sv^+_z)$ on $z=z_s$ of a downgoing acoustic field in
$z<z_s$, hence related by the operator $\Lambda^+_{z_s}$.
Equations \ref{eqn:snull}, \ref{eqn:tracejump10} and
\ref{eqn:tracejump20} show that these differ by a factor of -2 from
source functions $f_s$ and $h_s$ in the system \ref{eqn:awepm},
with $h_s=0$ and $f_s=0$ respectively, that generate the same acoustic
field in $z<z_s$, and in particular the same receiver traces on
$z=z_r$, at least asymptotically.

%To give an example of \ref{eqn:snull} in action, it is necessary to
%create downgoing data and the action of the operator $\Lambda^{\pm}$
%on a downdoing field data $f_s$ on $z=z_s$.

%These source functions satisfy
%the relation \ref{eqn:hfcondn}, therefore source vectors $(h_s,0)$ and
%$(0, f_s)$ generate {\em the same acoustic field} $(p^+,\bv^+)$ in
%$z>z_s$.
Figures
%\ref{fig:drecphh0}, \ref{fig:dfwdphh0}, \ref{fig:daltphh0} (homogeneous model) and
\ref{fig:drecplh0}, \ref{fig:dfwdplh0}, \ref{fig:daltplh0} show the pressure
gathers extracted at $z_r=1000$ m for the point source at $z=z_d$ and
for the two choices of extended source at $z=z_s$, on the same color
scale. The obvious similarity between the fields generated by the two
extended sources,
predicted by equation \ref{eqn:snull}, is confirmed by trace
comparisons in figures %\ref{fig:drecphh0tr21}-\ref{fig:daltphh0tr81}
%and
\ref{fig:drecplh0tr81},\ref{fig:daltplh0tr81}. Other traces are
equally similar.


\plot{drecplh0}{width=\textwidth}{Pressure gather at receiver depth 
  $z_r=1000$ m from field generated by causal solution of acoustic 
  system \ref{eqn:awepm} in the lens model described in the 
  text, with point pressure source (constitutive defect) at $z_d=3500$
  m, $x_d=3500$ m.}

\plot{dfwdplh0}{width=\textwidth}{Pressure gather at receiver depth 
  $z_r=1000$ m from field generated by causal solution of acoustic 
  system \ref{eqn:awepm} in the lens model described in the 
  text, with extended pressure source (constitutive defect) on
  $z=z_s=3000$ m given bythe field depicted in Figure
  \ref{fig:dsrcvzhh0} scaled by -2 
   ($h_s=-2P_sv_z^+=\Lambda^+_{z_s}P_sp^+$) and zero velocity source (vertical load) 
  ($f_s=0$).}

\plot{daltplh0}{width=\textwidth}{Pressure gather at receiver depth 
  $z_r=1000$ m from field generated by causal solution of acoustic 
  system \ref{eqn:awepm} in the lens model described in the 
  text, with extended velocity source (vertical load) on $z=z_s=3000$
  m given by the field depicted in Figure \ref{fig:dsrcphh0} scaled by -2
  ($f_s=-2P_sp^+$) and zero pressure source (constitutive defect)
  ($h_s=0$).}

\plot{drecplh0tr81}{width=\textwidth}{Overplot of traces 81 ($x=3600$) 
  from gathers shown in \ref{fig:drecplh0} (blue), \ref{fig:dfwdplh0}
  (red).}

\plot{daltplh0tr81}{width=\textwidth}{Overplot of traces 81 ($x=3600$)
  from gathers shown in \ref{fig:drecplh0} (blue), \ref{fig:daltplh0}
  (red).}
% time rev

\subsection{Inversion by time reversal}
I have applied the approximate inversion procedure suggested in
equation \ref{eqn:approxinv} to the pressure gather shown in Figure
% \ref{fig:drecphh0} and
\ref{fig:drecplh0}, generated by a point source
at $z_d=3500$ m, $x_s=3500$ m, propagating in the %homogeneous and
lens
model (Figure \ref{fig:bml0}). I choose this example for two
reasons. First, the success of the inversion demostrates the
insensitivity of the time reversal method to ray multipathing
(triplication), evident in the data (Figure \ref{fig:drecplh0}).
%s respectively
Second, I will invert this data in the homogeneous model, that is,
construct sources that (approximately) reproduce the data using a
different material model than the one in which it was produced. This
capability is critically important in the application of the
approximate inversion in nonlinear inversion, where the early iterations
involve solution of the source estimation problem \ref{eqn:esis} at
(possibly very) wrong material models ${\bf c}$. Successful extension
methods maintain data fit throughout the course of the inversion.
%The source time dependence is again a [1, 2.5,
%10, 12.5] Hz trapezoidal bandpass filter.

As noted earlier, the acoustic field in this example is downgoing
throughout the simulation range. It can be
regarded as the result of either pressure or velocity source at $z=z_s$: the
pressure source gather $h_s$ (Figure \ref{fig:dhshh0})  is -2 times the
vertical velocity gather depicted in Figure \ref{fig:dsrcvzhh0}, the
velocity source gather $f_s$ (Figure \ref{fig:dfshh0}) is -2 times the
pressure gather depicted in Figure \ref{fig:dsrcphh0}.

\plot{dhshh0}{width=\textwidth}{Pressure source gather = -2 $\times$
  vertical velocity gather (Figure \ref{fig:dsrcvzhh0}).}

\plot{dfshh0}{width=\textwidth}{Velocity source gather = -2 $\times$
  pressure gather (Figure \ref{fig:dsrcphh0}).}

Figure \ref{fig:dinvhslh0} shows the approximate inversion (via the
first equation in display \ref{eqn:approxinv}) of the
pressure gather shown in Figure \ref{fig:drecplh0}, inverted in the homogeneous model
(rather than in the lens model used to generate the data). The result
differs greatly from the pressure source shown in Figure
\ref{fig:dhshh0}, as it must since it results from inversion in the
wrong material model.
Some dip filter effect is unavoidable and is caused by the aperture
limitation of the acquisition geometry: the steeper dips in the source
gather (Figure \ref{fig:dhshh0}) do not contribute to the data, nor to
the inversion. Also, the limited receiver aperture causes truncation
artifacts in the inversion. However, this result is an
accurate inversion: re-simulation (application of $S^{+}_{z_s,z_r}$) {\em using the same (homogeneous)
  model as used in the inversion} results in accurate recovery (Figure \ref{fig:drerecplh0}) of the
input pressure gather (Figure \ref{fig:drecplh0}). The difference is
shown on the same color scale in Figure \ref{fig:ddiffrecplh0}.

\plot{dinvhslh0}{width=\textwidth}{Approximate inversion via first 
  equation in display \ref{eqn:approxinv}. Inversion in homogenous model of 
  the pressure gather in Figure \ref{fig:drecplh0}, simulated with 
  lens model. Scaled version of output $v_z$ field obtained by 
  applying transpose of $4V^+_{z_s,z_r}$. Quite different from 
  pressure source gather (Figure \ref{fig:dhshh0}) used to generate
  data - since inversion takes place in a different material model!}

\plot{drerecplh0}{width=\textwidth}{Re-simulated pressure gather produced from
  inverted source
  shown in Figure \ref{fig:dinvhslh0}. Simulation in homogenous model
  used for inversion.}

\plot{ddiffrecplh0}{width=\textwidth}{Difference between gathers
  displayed in Figures \ref{fig:drecplh0} and \ref{fig:drerecplh0},
  plotted on same color scale.}

%To see that this is indeed an inversion, I apply $S^{+}_{z_s,z_r}$ to
%the source shown in Figure \ref{fig:dinvhshh0}, to produce the
%re-simulation shown in Figure \ref{fig:drerecphh0}. Simulation is carried
%out in the homogeneous model used in the inversion. Comparison with the
%gather in Figure \ref{fig:drecphh0} via a difference plot
%\ref{fig:ddiffrecphh0} verifies that a very close match has been
%obtained, so the source in Figure \ref{fig:dinvhshh0} is indeed a
%satisfactory inversion, even it differs considerably from the original
%point source: the difference lies in an approximate null space of the modeling
%operator, arising from the limited aperture of the acquisition
%geometry.

%The remainder of this section is devoted to illustrating two important
%features of the approximate inversion developed above. The
%first is the validity of the inversion approach given in display
%\ref{eqn:approxinv} so long as the fields generated by the
%model used in simulation and inversion are downgoing. The models
%may be quite far from homogeneous, and even generate triplications,
%whereas the inversions remain accurate modulo the dip filtering effect
%of aperture. Figure \ref{fig:dinvhsll0} shows the pressure source at $z=z_s$
%inverted from the data Figure
%\ref{fig:drecplh0}, in which a triplication is obvious. Recall that
%this data is generated by propagating a point source at $z_d=3500$ m
%in the lens model. The inversion uses the same model. Figure
%\ref{fig:dinvhsll0} is an aperture-limited version of the pressure
%source at $z=z_s=3000$ m
%(Figure \ref{fig:dhshh0}) that generates this data by a dip filter, in
%the same way that the sources in Figures \ref{fig:dhshh0} and
%\ref{fig:dinvhshh0} are related. The inverted source is an accurate
%inversion: it generates the data depicted in Figure
%\ref{fig:drerecpll0}, which is very close to the input data (Figure
%\ref{fig:drecplh0}). The difference is displayed on the same color
%scale in Figure \ref{fig:ddiffrecpll0}.

%The second feature is a decoupling of the models used (implicitly or
%otherwise) in simulation and those used in inversion: it is possible
%to {\em accurately invert pressure data for pressure source using the
%  wrong model}, in the sense that the inverted source will
%generate an accurate recovery of the input data re-simulated in the
%same (wrong) model. For example, inversion of the data of the previous
%example (Figure \ref{fig:drecplh0}) {\em in the homogeneous model}
%results in an inverted source gather at $z=z_s$ (Figure \ref{fig:dinvhslh0}
%that is very far from a dip-filtered version of the source used
%generate the data (Figure \ref{fig:dhshh0}). 

% unitarity

\subsection{Quasi-unitary property of the modeling operator}
The identities \ref{eqn:approxinv} and \ref{eqn:sv} would together
establish the approximately unitary property of $S^+_{z_s,z_r}$, if
$\Lambda$ were symmetric. Identity \ref{eqn:approxinv} was illustrated
in the last subsection. Setting the symmetric issue aside for the
moment, an illustration of the relation \ref{eqn:sv} proceeds as follows.
%A minor extension of an earlier example, based on the gathers
%displayed in Figures \ref{fig:dsrcphh0} ($P_sp^+$) and
%\ref{fig:dsrcvzhh0} ($P_sv_z^+$) illustrates the identity
%\ref{eqn:sv}.
%The pressure gathers at $z=z_r$ produced by application of
%$S^+_{z_r,z_s}$ to $h_s=-2P_s v^+_z$, and $\Pi_0{\cal S}^+_{z_s,z_r}\Pi_1^T$ to $f_s
%=-2P_sp^+$, have already been displayed (Figures \ref{fig:dfwdphh0}
%and \ref{fig:daltphh0}), and are identical to several digits. The same 
%is true of the vertical velocity gathers, shown in \ref{fig:dfwdvzhh0}
%and \ref{fig:daltvzhh0}; their difference, plotted on the same scale,
%appears in Figure \ref{fig:dsvcomphh0}. Once again, this relation
%expresses the interchangeability of pressure and velocity sources,
%related by $\Lambda$.

Relation \ref{eqn:tracejump10} characterizes
$\Lambda^+_{z_s}$ as connecting the pressure and velocity components
of downgoing fields restricted to $z=z_s$. That is, the pressure
source gather
$h_s=\Lambda^{+}_{z_s}P_s p^+ = -2
P_{z_s}v_z$, displayed in Figure \ref{fig:dhshh0}, is the image of the
pressure gather in Figure \ref{fig:dsrcphh0} under
$\Lambda^{+}_{z_s}$.
The pressure gather
\ref{fig:dfwdplh0} is the image of this pressure source gather under
$S^+_{z_s,z_r}$ (using the lens model). The corresponding vertical velocity gather
(Figure \ref{fig:dfwdvzlh0}) is $-1/2$ times $\Lambda^+_{z_r}P_r
p$. Therefore scaling the data in Figure \ref{fig:dfwdvzlh0} by
$-2$ produces
$\Lambda^+_{z_r}S^+_{z_s,z_r}\Lambda^{+}_{z_s}P_s p^+$. On the other
hand, figure \ref{fig:daltvzlh0} shows the result of applying
$V^+_{z_s,z_r}$ to $f_s=-2P_s p^+$. Therefore scaling the gather in
Figure \ref{fig:daltvzlh0} by $-\frac{1}{2}$ produces $V^+_{z_s,z_r}P_zp^+$.
Since the data in Figures \ref{fig:dfwdvzlh0} and \ref{fig:daltvzlh0}
are essentially identical, the relation \ref{eqn:sv} holds for this
example.

\plot{dfwdvzlh0}{width=\textwidth}{Vertical velocity gather, generated
  with a pressure source in the lens model, 
  corresponding to pressure gather \ref{fig:dfwdplh0}.}

\plot{daltvzlh0}{width=\textwidth}{Vertical velocity gather, generated
  with a velocity source in the lens model, corresponding to
  pressure gather \ref{fig:daltplh0}.}

\plot{dsvcomplh0}{width=\textwidth}{Difference between velocity gathers
  shown in Figures \ref{fig:dfwdvzlh0} and \ref{fig:daltvzlh0},
  plotted on the same color scale as these figures.}

%Figure \ref{fig:preddinvhshh0} shows the pressure source gather
%produced from the pressure data gather in Figure \ref{fig:dsrcphh0} by
%application of the operator on the right-hand side of formula
%\ref{eqn:lamident} to the source surface pressure gather depicted in
%\ref{fig:dsrcphh0}. The homogeneous model is used in all propagations
%implicit in the prescription \ref{eqn:lamident}. The scale is the same
%as for Figures \ref{fig:dinvhshh0} and \ref{fig:dhshh0}. In fact
%Figure \ref{fig:preddinvhshh0} appears to be nearly identical to
%Figure \ref{fig:dinvhshh0}, and both appear to be dip-filtered
%versions of Figure \ref{fig:dhshh0}. The difference plot (Figure
 % \ref{fig:ddiffinvhshh0}), using the same color scale, shows that the
 % similarity is quantitative. The recovered source gather (Figure
 % \ref{fig:preddinvhshh0}) also generates the same acoustic fields as
  %the point source at $z=z_d$: the pressure gather at $z=z_r=1000$ m
%  depth (propagated in the lens model) is shown in Figure \ref{fig:dpredhsrecplh0}, and its
%  difference with the point source gather (Figure \ref{fig:drecplh0})
%  plotted on the same color scale in Figure
%  \ref{fig:ddiffpredhsrecplh0}.

%Figures \ref{fig:preddinvhsll0}, \ref{fig:ddiffinvhsll0},
%\ref{fig:dpredhsrecpll0}, and \ref{fig:ddiffpredhsrecpll0} accomplish
%the same comparison for the $\Lambda^+_{z_s}$ approximation in
%equation \ref{eqn:lamident}, with all propagation taking place in the
%lens model.

\subsection{Economical computation of $\Lambda$ in ``thin'' subdomain}
Equation \ref{eqn:lamnear} suggests a thin-slab computation of the
$\Lambda$ action, which is both accurate and economical. 
This calculations
place a receiver array at $z_s+\Delta z=2900$ m depth, just 100 m above the
source surface at $z_s=3000$ m. For the discretization used to create
the examples shown so far, that is just a 5 gridpoint difference in
depth, as opposed to 100 gridpoints between the source and receiver
depths for examples such as shown in Figure \ref{fig:drecplh0}.

Asymptotically, $\Lambda^{\pm}_{z_s}$ depends only on the medium
coefficients ${\bf c}$ in an arbirarily small region containing the
source surface $z=z_s$. In this example, the homogeneous and lens models are identical in the depth
range $2900 < z < 3000$ m, so the computed $\Lambda^+_{z_s}$ operators
will be precisely the same for both models. Hence I show only results for the
homogenous model.

The approximation to $\Lambda_{z_s}^+$ via equation \ref{eqn:lamnear}
for this configuration is evaluated in
Figures \ref{fig:preddnshshh0}, \ref{fig:ddiffnshshh0},
\ref{fig:dprednshsrecpll0}, and \ref{fig:ddiffprednshsrecpll0}. The effect of aperture
limitation is clearly diminished: the second figure in this series
compares the full-aperture pressure source gather (Figure
\ref{fig:dhshh0}) with the 
image of the corresponding pressure gather (Figure \ref{fig:dsrcphh0})
under the approximation to $\Lambda_{z_s}^+$, and the last figures
show that the approximated source gathers accurately predict the
point-source pressure gather at the receiver datum $z_r=1000$ m.

\plot{preddnshshh0}{width=\textwidth}{Pressure source gather = image
  under pressure-to-source operator $\Lambda^+_{z_s}$ of pressure gather
  shown in Figure \ref{fig:dsrcphh0}, homogeneous model, using
  ``near'' receiver traces at $z=2900$ m. Compare Figure
  \ref{fig:dhshh0}: because the sources and
  receivers are close, little aperture is lost in this case.}

\plot{ddiffnshshh0}{width=\textwidth}{Difference between (a) image
  (Figure \ref{fig:preddnshshh0}) of $\Lambda^+_{z_s}$ applied to
  pressure gather (Figure \ref{fig:dsrcphh0}) using a near receiver
  array to implement formula \ref{eqn:lamident}, and (b) source gather
  (Figure \ref{fig:dhshh0}) inferred from vertical velocity.
  Homogeneous model used in all
  propagations. Same color scale as in Figure
  \ref{fig:preddnshshh0}. }

%\plot{dprednshsrecplh0}{width=\textwidth}{Pressure gather at receiver
%  datum $z=z_r=1000$ m simulated in lens model from source gather shown in Figure
%  \ref{fig:preddnshshh0}. Compare with point source pressure gather
%  (Figure \ref{fig:drecplh0}).}

%\plot{ddiffprednshsrecplh0}{width=\textwidth}{Plot of difference 
%  between data shown in Figures \ref{fig:drecplh0} and 
%  \ref{fig:dprednshsrecplh0}, plotted on same color scale as the latter 
% two figures.}

\plot{dprednshsrecpll0}{width=\textwidth}{Pressure gather at receiver
  datum $z=z_r=1000$ m simulated in lens model from source gather shown in Figure
  \ref{fig:preddnshshh0}. Compare with point source pressure gather
  (Figure \ref{fig:drecplh0}).}

\plot{ddiffprednshsrecpll0}{width=\textwidth}{Plot of difference 
  between data shown in Figures \ref{fig:drecplh0} and 
  \ref{fig:dprednshsrecpll0}, plotted on same color scale as the latter 
  two figures.}
\subsection{Symmetrizing $\Lambda$}

Figure \ref{fig:preddnshstrhh0} shows the image of the pressure gather
in Figure \ref{fig:dsrcphh0} under $(\tilde{\Lambda}^+_{z_s})^T$,
using the ``near'' traces at $z=2900$, that is, $\Delta z = 100$ m in
expression \ref{eqn:lamtransp}, and propagation in the
homogeneous model. Note
the close resemblance to the image of the same pressure gather under
$\tilde{\Lambda}^+_{z_s}$ displayed in Figure
\ref{fig:preddnshshh0}. The difference of these two images is
displayed in \ref{fig:ddiffnslamtrhh0}, on the same color scale as the
images themselves. Since the propagation takes place entirely in a
region where all of the mechanical parameters are homogenous, I do not
offer a similar comparison for the lens model.

\plot{preddnshstrhh0}{width=\textwidth}{Pressure source gather = image
  under {\em transpose} of pressure-to-source operator
  $\Lambda^+_{z_s}$ of pressure gather shown in Figure
  \ref{fig:dsrcphh0}, homogeneous model, using ``near'' receiver
  traces at $z=2900$ m. Compare Figure \ref{fig:dhshh0} and
  \ref{fig:preddnshshh0}: as noted in the text, $\Lambda^+_{z_s}$ is
  asymptotically symmetric, so the resemblance is not a surprise.}

\plot{ddiffnslamtrhh0}{width=\textwidth}{Difference between data in
  Figures \ref{fig:preddnshshh0} and \ref{fig:preddnshstrhh0}, plotted
  on the same scale as these figures, showing that the asymptotic
  symmetry of $\Lambda^+_{z_s}$ is actually quantitative for the
  length, time and frequency scales of these examples.}

\subsection{Asymptotic symmetry of $\Lambda$}

Figure \ref{fig:symmdnshshh0} shows the output of the symmetrized
approximate source-to-pressure operator per equation \ref{eqn:winvcomp},
applied once again to the pressure data in Figure
\ref{fig:dsrcphh0}. Note the resemblance to Figures
\ref{fig:preddnshshh0} and \ref{fig:preddnshstrhh0}. These are all
asymptotic approximations of each other. Figure
\ref{fig:ddiffsymmdnshsll0} shows the
 the difference between the pressure gather at $z=z_r$ produced from
 the pressure source output by the symmetrized $\Lambda$, and the point source
simulation (Figure \ref{fig:dfwdplh0}), plotted on the same scale as
the latter, in both cases with all propagations in the lens model.

\plot{symmdnshshh0}{width=\textwidth}{Pressure source gather = image
    under {\em symmetrized} pressure-to-source operator
    $\frac{1}{2}\left(\Lambda^+_{z_s}+(\Lambda^+_{z_s})^T\right)$ of
    pressure gather shown in Figure \ref{fig:dsrcphh0}, homogeneous
    model, using ``near'' receiver traces at $z=2900$ m. Compare
    Figure \ref{fig:preddnshshh0}.}

\plot{ddiffsymmdnshsll0}{width=\textwidth}{Difference between point
  source simulation (Figure \ref{fig:dfwdplh0}) and pressure gather at
  $z=z_r=1000$ m produced by simulation with the source shown in
  Figure \ref{fig:symmdnshshh0},
  propagation in the lens model.}

\subsection{Unitary property of modeling operator}
To illustrate this unitary property of $S^+_{z_s,z_r}$, I apply the
operator
\[
  \frac{1}{2}((\Lambda^+_{z_s})^T+ \Lambda^+_{z_s})
  (S^{+}_{z_s,z_r})^T \frac{1}{2}((\Lambda^+_{z_r})^T+
  \Lambda^+_{z_r})
\]
to the data $S^+_{z_s,z_r}h_s$ (Figure \ref{fig:drerecplh0}), in which
$h_s$ is the %version (Figure \ref{fig:dinvhslh0}) of the
downgoing source created earlier (Figure \ref{fig:dhshh0})%, produced
%by inversion of the data in Figure \ref{fig:drecplh0} via the first
%equation in display \ref{eqn:approxinv}, propagation in the homogenous
%model.
The operator above is computed via the technique explained in
the preceding subsection, below, using
auxiliary receiver arrays 100 m above the data source and receiver arrays.

The output is shown in Figure
\ref{fig:lamsstlamrdrecplh0}. The difference with the actual source is
shown in Figure \ref{fig:difflamsstlamrdrecplh0}.
%A similar exercise
%using the lens data \ref{fig:drecplh0} but inversion in the homogenous
%model produces the extended source depicted in
%\ref{fig:lamsstlamrdrecplh0}, which differs from the inversion result
%via equation \ref{eqn:approxinv} by the gather displayed in Figure
%\ref{fig:difflamsstlamrdrecplh0}.

\plot{lamsstlamrdrecplh0}{width=\textwidth}{Inversion of data shown in
  Figure \ref{fig:drecplh0}, simulated in lens model, using the
  approximate unitarity relation \ref{eqn:unitary} and propagation in
  homogenous model.}

\plot{difflamsstlamrdrecplh0}{width=\textwidth}{Difference between
  data displayed in Figures \ref{fig:dinvhslh0} and
  \ref{fig:lamsstlamrdrecplh0}, plotted on the same color scale.}

\subsection{Preconditioned CG iteration}
This final subsection shows that result of Conjugate Gradient
iteration, with and without preconditioning, applied to the source
estimation problem \ref{eqn:esis}, with zero and non-zero penalty
weight $\alpha$. The data $d$ is the gather shown in
\ref{fig:drecplh0}, simulated using the lens model with source shown
in Figure \ref{fig:dhshh0}, or, alternatively, a point source with
bandpass filter wavelet located at $x_d=3500$ m, $z_d=3500$ m. In the inversion,
the material model is taken to be homogeneous, as has been the case in
all of the previous examples. 

Figure \ref{fig:compnres0lh0} shows the progress of the normal residual
(Euclidean norm of the difference of the two sides of equation \ref{eqn:norm1}),
for Conjugate Gradient and Preconditioned Conjugate Gradient
(Algorithm 1) iterations, applied to solution of the optimization
problem \ref{eqn:einv} with $\alpha=0$. For CG, the norms are both the ordinary
Euclidean norm, $W_m=W_d=I$. For PCG, $W_m$ and $W_d$ are given in
display \ref{eqn:wdef}, with the symmetrized $\Lambda$s computed as
indicated in the preceding subsections. Convergence for the
preconditioned algorithm is roughly 4 times as fast.

\plot{compnres0lh0}{width=\textwidth}{Comparison of normal residual 
  (gradient) Euclidean norms: CG (blue), PCG (red), plotted 
  vs. iteration. Data = lens model, point source (Figure 
  \ref{fig:drecplh0}), inversion in homogenous model. Penalty weight 
  $\alpha=0$.}

Figure \ref{fig:compnres1lh0} shows the same comparison with non-zero
penalty weight, $\alpha=10^{-3}$. The PCG normal residual curve is
almost identical with that in the $\alpha=0$ case, wheras the CG
convergence has slowed down noticeably, being about five times as slow
as the preconditioned algorithm.

\plot{compnres1lh0}{width=\textwidth}{Comparison of normal residual 
  (gradient) Euclidean norms: CG (blue), PCG (red), plotted 
  vs. iteration. Data = lens model, point source (Figure 
  \ref{fig:drecplh0}), inversion in homogenous model. Penalty weight 
  $\alpha=10^{-3}$.}
