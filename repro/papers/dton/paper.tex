\title{Waveform Inversion via Blended Source Extension}
\author{William. W. Symes \thanks{The Rice Inversion Project,
Department of Computational and Applied Mathematics, Rice University,
Houston TX 77251-1892 USA, email {\tt symes@caam.rice.edu}.}}

\lefthead{Symes}

\righthead{Blended Source Approximate Inverse}

\maketitle
\begin{abstract}
Define the blended source extension, introduce annihilator, waveform inversion based on it
and variable projection.
Explain how approximate inverse via energy decay  provides a
substitute for the $\Psi$DO property of the normal operator for smooth
medium transmission problems.
\end{abstract}
\section{Introduction}
\section{Acoustics - sketch}
Acoustic system with pressure source:
\begin{eqnarray}
\label{eqn:awe}
\gamma \frac{\partial p}{\partial t} & = & - \nabla \cdot \bv +
F, \nonumber \\
\rho \frac{\partial \bv}{\partial t} & = & - \nabla p + {\bf G},\nonumber \\
p & =& p_0 \mbox{ for } t= t_0\nonumber\\ 
\bv & = & \bv_0 \mbox{ for } t= t_0 
\end{eqnarray}

$\Omega$  is a smoothly bounded open set in $\bR^d$. Denote by
$T^{\rm int}_{\Omega}$ and $T^{\rm ext}_{\Omega}$ the traces on $\partial
\Omega$ from interior and exterior of $\Omega$, and use the same notation
for functions of time. These maps are a priori defined with domains
$C^{\infty}_0(\Omega \times \bR)$ and $C^{\infty}_0((\bR^d\setminus
\Omega) \times \bR)$ and ranges $C^{\infty}_0(\partial \Omega \times \bR)$.

For the moment, assume that $p_0=0, \bv_0=0$. This restriction will be
removed later. The initial system can then be re-stated as
\begin{eqnarray}
\label{eqn:ichom}
\gamma \frac{\partial p}{\partial t} & = & - \nabla \cdot \bv +
F, \nonumber \\
\rho \frac{\partial \bv}{\partial t} & = & - \nabla p + {\bf G},\nonumber \\
p & = & 0, t < 0\\
\bv & = & 0, t < 0 
\end{eqnarray}

The system \ref{eqn:ichom} takes the form described in
\cite{BlazekStolkSymes:13}, with $P(\nabla) (p,\bv) = (-\nabla p,
-\nabla \cdot \bv)$ (use $P(\nabla)$ here rather than $p(\nabla)$ as
in \cite{BlazekStolkSymes:13} to avoid conflict with the notation for
the pressure field. As shown in Appendix A of
\cite{BlazekStolkSymes:13}, $P(\nabla)$ is densely defined and skew adjoint  on $L^2(\Omega)^{d+1}$ with domain
$H^1_0(\Omega) \times H^1_{\rm div}(\Omega)$. Therefore
Theorem 1 of  \cite{BlazekStolkSymes:13} applies to the system
\ref{eqn:ichom}: if $(F, {\bf G}) \in H^1_{\rm loc}(\bR,(L^2(\Omega))^{d+1})$
and $= 0$ for $t<0$, then there exists a unique strong solution
$(p,\bv) \in C^1(\bR,(L^2(\Omega))^{d+1}) \cap C^0(\bR,  H^1_0(\Omega) 
\times H^1_{\rm div}(\Omega))$.

The same reasoning applies to the exterior domain $\bR \times (\bR^d
\setminus \Omega)$, to yield a unique strong solution with similar conditions.

The trace operators $\Ti$ and $\Te$ extend to domains $C^k(\bR,
H^1(\Omega))$ and $C^k(\bR,H^1_{\rm loc}(\bR,\bR^d \setminus
\Omega))$
and range $C^k(\bR,L^2(\partial \Omega))$, $k \ge 0$, per the standard trace
theorem for Sobolev spaces. Define the interior (exterior) Dirichlet
problem for $d \in C^0(\bR,L^2(\partial \Omega))$ by
 
\begin{thm}
  Suppose that
  $d \in H^1([0,T], H^1(\partial \Omega)) \cap H^2([0,T],L^2(\partial
  \Omega))$. Then there exist unique 
$(\ip,\ibv) \in C^1(\bR,(L^2(\Omega))^{d+1}) \cap C^0(\bR,  H^1(\Omega) 
\times H^1_{\rm div}(\Omega))$.
for which
\begin{eqnarray}
\label{eqn:iawebv}
\gamma \frac{\partial \ip}{\partial t} & = & - \nabla \cdot \ibv, \nonumber \\
\rho \frac{\partial \ibv}{\partial t} & = & - \nabla \ip,\nonumber \\
\Ti \ip & = & d \\
\ip & = & 0, t < 0\\
\ibv & = & 0, t < 0.
\end{eqnarray}
Also, there extst unique 
$(\ep,\ebv) \in C^1(\bR,(L^2(\bR^d \setminus \Omega))^{d+1}) \cap C^0(\bR,  H^1(\bR^d \setminus \Omega) 
\times H^1_{\rm div}(\bR^d \setminus \Omega))$.
for which
\begin{eqnarray}
\label{eqn:eawebv}
\gamma \frac{\partial \ep}{\partial t} & = & - \nabla \cdot \ebv, \nonumber \\
\rho \frac{\partial \ebv}{\partial t} & = & - \nabla \ep,\nonumber \\
\Te \ep & = & d \\
\ep & = & 0, t < 0\\
\ebv & = & 0, t < 0.
\end{eqnarray}
\end{thm}
\begin{proof}
Since $\Omega$ is precompact, it is possible to construct a collar
neigborhood of $\partial \Omega$ in which a finite number of local
coordinate systems may be defined, covering $\partial \Omega$. Using
these coordinate systems, it is possible to construct an extension of 
$d \in H^2_{\rm loc}(\bR,L^2_0(\bR^d)) \cap
H^1_{\rm loc}(\bR,H^1_0(\bR^d))$. For this extension,
\[
\gamma\frac{\partial d}{\partial t} \in H^1_{\rm
  loc}(\bR,L^2_0(\bR^d)), \,
\nabla d \in H^1_{\rm
  loc}(\bR,L^2_0(\bR^d))^d
\]
Define $\ip_1, \ibv_1$ to be the strong solutions of the system
\ref{eqn:ichom} in $\bR \times \Omega$, respectively $\bR \times \bR^d
\setminus \Omega$, with 
\[
F = -\gamma\frac{\partial d}{\partial t}, {\bf G} = -\nabla d.
\]
Then $\ip = \ip_1 + d, \ibv = \ibv_1$ is the unique strong solution of
system \ref{eqn:iawebv}, $\ep = \ep_1 + d, \ebv = \ebv_1$ the unique
strong solution of system \ref{eqn:eawebv}.
\end{proof}

Note that
\begin{equation}
\label{eqn:trace}
T^{\rm  int}\ip = T^{\rm ext}\ep = d.
\end{equation}
 
Denote by $\bn$ the outward unit normal on $\partial \Omega$. note
that $\bn \cdot \Ti$ extends continuously to  $C^0(\bR,H_{\rm
  div}(\Omega))$. Similarly,
$\bn \cdot \Te$ extends continuously to $C^0(\bR, H_{\rm div}(\bR^d \setminus \Omega))$.
Therefore Theorem 2 implies that if $d \in H^1([0,T], H^1(\partial \Omega)) \cap H^2([0,T],L^2(\partial
  \Omega))$
\begin{equation}
\label{eqn:dton}
D[\kappa,\rho]d =[v_n] = \Te \ebv - \Ti \ibv)
\end{equation}
is well defined.

\begin{thm}
Suppose that $d \in H^1([0,T], H^1(\partial \Omega)) \cap H^2([0,T],L^2(\partial
  \Omega))$, and denote by $(\ip,\ibv)$ and $(\ep,\ebv)$ solutions of
  \ref{eqn:iawebv} and \ref{eqn:eawebv} respectively. Define 
\begin{equation}
\label{eqn:both}
(p,\bv)= \left\{
\begin{array}{c}
(\ip,\ibv) \mbox{ in } \Omega \times [0,T]\\
(\ep,\ebv) \mbox{ in } (\bR^d \setminus \Omega) \times [0,T] 
\end{array}
\right.
\end{equation}
Then $(p,\bv)$ is a weak solution of 
\begin{eqnarray}
\label{eqn:awe1}
\gamma \frac{\partial p}{\partial t} & = & - \nabla \cdot \bv +
D[\kappa,\rho]d \delta_{\partial \Omega}, \nonumber \\
\rho \frac{\partial \bv}{\partial t} & = & - \nabla p,\nonumber \\
p & =& 0 \mbox{ for } t= 0\nonumber\\ 
\bv & = & 0 \mbox{ for } t= 0
\end{eqnarray}
\end{thm}

\begin{proof}: (sketch) choose smooth test functions $\phi, {\bf \psi}$ vanishing for $t=T$. Then
\[
\int_{\Omega \times [0,T]}\left (\gamma \frac{\partial \phi}{\partial t} +
\nabla \cdot {\bf \psi}\right)p + \left(\rho\frac{\partial {\bf
    \psi}}{\partial t}
  + \nabla \phi\right) \cdot \bv
\]
\[
= - \int_{\Omega \times [0,T]}\left (\gamma \frac{\partial p}{\partial t} +
\nabla \cdot \bv\right) \phi + \left(\rho\frac{\partial \bv}{\partial t}
  + \nabla p\right) \cdot {\bf \psi}
\]
\begin{equation}
\label{eqn:rhs1}
+ \int_{\partial \Omega \times [0,T]} \bn \cdot {\bf \psi} \ip + \bn
  \cdot \ibv \phi
\end{equation}
Similarly,
\[
\int_{(\bR^d\setminus \Omega) \times [0,T]}\left (\gamma \frac{\partial \phi}{\partial t} +
\nabla \cdot {\bf \psi}\right)p + \left(\rho\frac{\partial {\bf
    \psi}}{\partial t}
  + \nabla \phi\right) \cdot \bv
\]
\[
= - \int_{(\bR^d \setminus\Omega) \times [0,T]}\left (\gamma \frac{\partial p}{\partial t} +
\nabla \cdot \bv\right) \phi + \left(\rho\frac{\partial \bv}{\partial t}
  + \nabla p\right) \cdot {\bf \psi}
\]
\begin{equation}
\label{eqn:rhs2}
- \int_{\partial \Omega \times [0,T]} \bn \cdot {\bf \psi} \ep + \bn
  \cdot \ebv \phi
\end{equation}
Adding equations \ref{eqn:rhs1} and \ref{eqn:rhs2} and taking into
account that $\ip$ and $\ep$ coincide (by construction) on $\partial
\Omega \times [0,T]$, obtain
\[
\int_{(\bR^d \times [0,T]}\left (\gamma \frac{\partial \phi}{\partial t} +
\nabla \cdot {\bf \psi}\right)p + \left(\rho\frac{\partial {\bf
    \psi}}{\partial t}
  + \nabla \phi\right) \cdot \bv
\]
\[
= \int_{\partial \Omega \times [0,T]} (\bn \cdot \ibv -  \bn
  \cdot \ebv) \phi
\]
$\phi$ and ${\bf \psi}$ being arbitrary, other than vanishing at $t=T$,
\end{proof}

Conversely,
\begin{thm}: Suppose $f \in  H^2([0,T],L^2(\partial \Omega)) \cap
  H^1([0,T],H^1(\partial \Omega))$, and $(p,\bv)$ is a weak solution of
\begin{eqnarray}
\label{eqn:awe2}
\gamma \frac{\partial p}{\partial t} & = & - \nabla \cdot \bv +
f \delta_{\partial \Omega}, \nonumber \\
\rho \frac{\partial \bv}{\partial t} & = & - \nabla p,\nonumber \\
p & =& 0 \mbox{ for } t= 0\nonumber\\ 
\bv & = & 0 \mbox{ for } t= 0
\end{eqnarray}
Then $T^{\rm ext}_{\Omega} p = T^{\rm int}_{\Omega}p$.
% and $f =
%D[\kappa,\rho]T^{\rm int}_{\Omega}p$.
\end{thm}

\begin{proof}: (sketch) Choosing test functions $\phi,\psi$ as before, the
system \ref{eqn:awe2} means that
\[
\int_{(\bR^d \times [0,T]}\left (\gamma \frac{\partial \phi}{\partial t} +
\nabla \cdot {\bf \psi}\right)p + \left(\rho\frac{\partial {\bf
    \psi}}{\partial t}
  + \nabla \phi\right) \cdot \bv
\]
\[
= \int_{\partial \Omega \times [0,T]} \phi f
\]
Extend $\bn$ to a collar neighborhood of $\partial \Omega$ and choose
smooth $\chi$ supported in that neighborhood and $=1$ on $\partial
\Omega$. Also extend $f$ so that it remains in the same class. Set
\[
\bw = \chi f \bn \left({\bf 1}_{\Omega} - \frac{1}{2}\right)
\]
\[
\int (\nabla \cdot \bw)  \phi = \int_{\Omega \times [0,T]}(...) + \int_{(\bR^d
  \setminus \Omega) \times [0,T]}(...) 
\]
\[
= \int_{\partial \Omega \times [0,T]} \bn \cdot (\chi f \bn) \phi
\]
\[
= \int_{\partial \Omega \times [0,T]} f  \phi
\]
That is,
\[
 \nabla \cdot \bw = f \delta_{\partial \Omega \times [0,T]}
\]
Then $(p,\bv-\bw))$ is a weak solution of 
\begin{eqnarray}
\label{eqn:awe12}
\gamma \frac{\partial p}{\partial t} & = & - \nabla \cdot \bv \nonumber \\
\rho \frac{\partial (\bv-\bw)}{\partial t} & = & -\rho \frac{\partial\bw}{\partial t} - \nabla p,\nonumber \\
p & =& 0 \mbox{ for } t= 0\nonumber\\ 
\bv & = & 0 \mbox{ for } t= 0
\end{eqnarray}
The RHS of the second equation in the system \ref{eqn:awe12} is of class $H^1([0,T],L^2(\bR^d))$ so Theorem 1 in
\cite{BlazekStolkSymes:13} implies that $(p, \bv-\bw)$ is a strong
solution of system \ref{eqn:awe12}, in particular that $p \in C^0([0,T],H^1(\bR^d))$,
whence the assertion about continuity at the boundary follows from the
trace theorem.

The remaining statement now follows from the previous theorem.
\end{proof}

\section{Ins and Outs}
Assume that $\kappa, \rho$ are of class $C^{\infty}$ in a neighborhood
of $\partial \Omega$. 

Denote by $P(\bx,\omega,\bk)$ symbol of the acoustic system,
\[
P(\bx,\Omega,\bk) = 
i\left(
\begin{array}{cc}
\gamma(\bx)i\omega & \bk^T\\
\bk &  \rho(\bx)\omega I_{3\times3}
\end{array}
\right)
\]
Determinant is zero iff $\omega=0$ or $\omega^2 = c^2|\bk|^2$.
At a point $\bx$ on $\partial \Omega$, write $\bk = \bk^{\perp} +
k_n\bn$. The characteristic eqn has two solutions for $k_n$, namely
\[
k_n = \pm \sqrt{\frac{\omega^2}{c^2}-|\bk^{\perp}|^2}
\]
and correspondingly, the normal component of $\bv$ satisfies
\[
\rho v_n = \bn \cdot \bv = \int \bn \cdot \nabla p
\]
or in the Fourier domain 
\[
c \rho v_n = \frac{k_n}{\omega} p = \pm
\sqrt{1-\frac{c^2|\bk^{\perp}|^2}{\omega^2}}p
\]
Define $V$ to be the $\Psi$DO on $\partial \Omega \times \bR$ with symbol
\[
V(\bx,\bk^{\perp})=\frac{1}{c\rho}\sqrt{1-\frac{c^2|\bk^{\perp}|^2}{\omega^2}}
\]
Then if $(p,\bv)$ is incoming/outgoing, $T^{\rm int}v_n = \pm VT^{\rm int}p$ 
The incoming/outgoing subspace corresponds to $\pm$.
Note that if $(p, \bv^{\perp} + v_n\bn)$ is incoming, then $(p,
\bv^{\perp} - v_n\bn)$ is outgoing, and vis-versa. 

Projections onto Incoming/outgoing data: $P_{\pm}$. Data is outgoing,
source incoming. So to extract data from overlapping signal with 
source, apply $P_-$. Then solution backwards in time has source field 
as its incoming component. 

Decomposition of $(p,v_n)$: 
\[
(p,v_n) = (p_+, Vp_+) + (p_-,-Vp_-) = (p_++p_-, V(p_+-p_-)) \Rightarrow
\]
\[
(p,V^{-1}v_n) = (p_++p_-,p_+-p_-)
\]
\[
p_+ = \frac{1}{2}(p + V^{-1}v_n), p_-=\frac{1}{2}(p-V^{-1}v_n)
\]
\[
P_{\pm}(p,v_n) = \left( \frac{1}{2}(p \pm V^{-1}v_n),  \frac{1}{2}(Vp \pm
  v_n)\right)
\]

Ray theory assumption about exterior: all rays leaving $\partial
\Omega$ keep going, never to return. Then the outer solution
$(\ep,\ebv)$ defined in Theorem 1 is purely outgoing, i.e. 
\[
P_-T^{\rm ext}(\ep,\ebv) = T^{\rm ext}(\ep,\ebv), \,P_+T^{\rm ext}(\ep,\ebv) = 0
\]
hence
\[
v_n^{\rm ext} = -Vp 
\]
and
\begin{equation}
\label{eqn:did}
D[\kappa,\rho]p = v_n^{\rm int}-v_n^{\rm ext} = v_n^{\rm int}+Vp =
2 P_+(p,v^{\rm int}_n)_2
\end{equation}
that is, $D[\kappa,\rho]p$ is the normal v-component of the incoming projection of the
solution of system \ref{eqn:awe2}.
[Since $\ip=\ep$ on boundary, and initial conditions are same, follows
that the tangential components of $\ibv, \ebv$ are the same.]

Equation \ref{eqn:did} relates $D[\kappa,\rho]$ to the $\Psi$DO
$V$. Note however that in general it is not a $\Psi$DO itself, unless
$v_n=Vp$, that is, unless the entire field $(p,\bv)$ is incoming at
the boundary. This occurs under certain circumstances, but in general
both components must be computed, and the relation between $p$ and
$D[\kappa,\rho]p$ is global.

1D example. 
 
\section{Linear Inverse Problem}
%Problem: Given uniformly positive, bounded, and measureable $\kappa,\rho$ and
%$d \in H^2([0,T],L^2(\partial \Omega)) \cap  H^1([0,T],H^1(\partial
%\Omega))$, find $f \in H^2([0,T],L^2(\partial \Omega)) \cap
% H^1([0,T],H^1(\partial \Omega))$ so that the solution $(p,\bv)$ of 
%the system \ref{eqn:awe2} satisfies $d = T^{\rm int} p$.

Suppose that $(p,\bv)$ is the solution of \ref{eqn:awe2} as described
in Theorem 2, and define 
\[
S[\kappa,\rho] f = T^{\rm int}p^{\rm int}.
\]
By standard uniqueness thm, if you record final data at
$t=T$, say $p_T(\bx)=p(\bx,T), \bv_T(\bx)=\bv(\bx,T)$ for $\bx \in
\Omega$, and $d = T^{\rm int}p$, then $(p,\bv)=(p_r,\bv_r)$, the
solution of the time-reversed problem
\begin{eqnarray}
\label{eqn:awer}
\gamma \frac{\partial p_r}{\partial t} & = & - \nabla \cdot \bv_r \nonumber \\
\rho \frac{\partial \bv_r}{\partial t} & = & - \nabla p_r \nonumber \\
p_r & =& p_T \mbox{ for } t= T\nonumber\\ 
\bv_r & = & \bv_T \mbox{ for } t= T \\
p_r & = & d \mbox{ on } \partial \Omega \times [0,T]
\end{eqnarray}
In general, suppose $d$ is given on $\partial \Omega \times [0,T]$,
and define
\[
S[\kappa,\rho]^{\dagger}d = v_{r,n} + Vp_r.
\]
If specifically $d = S[\kappa,\rho]f$, then the last section together
with Theorem 2 implies that
\begin{equation}
\label{eqn:idealinv}
S[\kappa,\rho]^{\dagger}S[\kappa,\rho] = I
\end{equation}

Implicitly, $S[\kappa,\rho]^{\dagger}$ depends on the final data
$(p_T,\bv_T)$. Suppose that local energy decays: 

\noindent {\bf Local Decay Assumption:} There exist $C, \alpha > 0$ so
that the solution of system \ref{eqn:awe} with $F = {\bf G} = 0$
satisfies
\[
E(T) \le Ce^{-\alpha T} E(0),
\]
where $E$ is the acoustic strain energy:in $\Omega$:
\[
E(t) = \frac{1}{2}\int_{\Omega} dx \left(\rho |\bv(\bx,t)|^2 +
  \frac{1}{\kappa} p(\bx,t)^2\right).
\]
In that case, for any $\epsilon > 0$ there is $T>0$ so that $E(T) <
\epsilon \sup_{0 \le t \le T}E(t)$. Define $(p_a,\bv_a)$ to be the
solution of
\begin{eqnarray}
\label{eqn:awer}
\gamma \frac{\partial p_a}{\partial t} & = & - \nabla \cdot \bv_a \nonumber \\
\rho \frac{\partial \bv_a}{\partial t} & = & - \nabla p_a \nonumber \\
p_a & =& 0 \mbox{ for } t= T\nonumber\\ 
\bv_a & = & 0 \mbox{ for } t= T \\
p_a & = & d \mbox{ on } \partial \Omega \times [0,T]
\end{eqnarray}
$(p_a-p_r,\bv_a-\bv_r)$ satisfies a similar system with zero Dirichlet
data. Since the energy in the final data is small, it follows that the
predicted Neumann data is small. NOTE this requires a collar
neighborhood with smooth coefficients and microlocal restrictions on
the sources, so that restriction to the boundary acts like order zero op.

(then relation with adjoint: $S^{\dagger} = S^TW$)


Slightly less trivially, suppose that the boundary $\partial \Omega
\times [0,T]$ is divided into two subdomains, $\Gamma$ and its
complement $\Gamma'$, and $d$ is given only on $\Gamma$.

\section{Gradient}
\[
J[\kappa] = \frac{1}{2}\|AS[\kappa]^{\dagger}d\|^2=\frac{1}{2}\|AS[\kappa]^TWd\|^2
\]
Note that if $\kappa$ is not varied near wells then W is independent
of constant.
\[
DJ[\kappa]\delta \kappa= = \langle(DS[\kappa]\delta \kappa)^T Wd, A^TA 
S[\kappa]^TWd\rangle
\]
\begin{equation}
\label{eqn:deriv}
=\langle( Wd, DS[\kappa]\delta \kappa) A^TA S[\kappa]^TWd\rangle
\end{equation}
\begin{equation}
\label{eqn:grad}
=\langle DS[\kappa]^*(Wd,A^TAS[\kappa]^TWd\rangle_{\rm model}
\end{equation}
Here $DS[\kappa]^*(Wd,A^TAS[\kappa]^TWd)$ is the reverse time
migration of $Wd$ with source field $A^TA S[\kappa]^TWd$. The model
inner produce may incorporate a weight (usually should, eg. power of
the Laplacian, in which case its inverse postmultiplies the Gradient.

\section{Hessian}
Differentiating equation \ref{eqn:deriv} again,
\[
D^2J[\kappa](\delta \kappa_1,\delta \kappa_2)
= \langle( Wd, D^2S[\kappa](\delta \kappa_1,\delta \kappa_2) A^TA
S[\kappa]^TWd\rangle
\]
\[
+ \langle( Wd, DS[\kappa](\delta \kappa_1) A^TA DS[\kappa](\delta 
\kappa_2)^TWd\rangle. 
\]
Now suppose that $d = S[\kappa]f$ with physical $f$, that is
$Af=0$. Thus $AS[\kappa]^TWd \approx 0$ and the Hessian is
\begin{equation}
\label{eqn:hess1}
+ \langle A(DS[\kappa](\delta \kappa_1))^T Wd, A (DS[\kappa](\delta 
\kappa_2))^TWd\rangle. 
\end{equation}

 Also suppose that the factorization lemma holds:
\begin{equation}
\label{eqn:fact}
DS[\kappa]\delta \kappa \approx S[\kappa]Q[\kappa,\delta \kappa]
\end{equation}
with $Q$ an essentiall skew-adjoint $\Psi$DO of order 1.

Remark. It may not be necessary to assume this much: setting 
$Q = S^{\dagger}DS$ may do. That even works when geometric optics is
not valid. Need to better understand structure.

Then as noted elsewhere
\begin{equation}
\label{eqn:hess2}
D^2J[\kappa](\delta \kappa_1,\delta \kappa_2) 
\approx \langle [A,Q[\kappa,\delta \kappa_1]]S[\kappa]^TWd,
[A,Q[\kappa,\delta \kappa_2]]S[\kappa]^TWd\rangle
\end{equation}

\section{Asymptotics}
Assume 3D, single arrival. Then with source surface = $\{z=z_s\}$ and
similarly for receivers,
\[
S[\kappa]f(\bx_r,t) = \int dx_sdy_s a(\bx_r,\bx_s)
f(t-T(\bx_s,\bx_r),\bx_s)
\]
\[
(DS[\kappa]\delta \kappa)f(\bx_r,t) = -\int dx_sdy_s f'(t-T(\bx_s,\bx_r),\bx_s)
(DT[\kappa]\delta \kappa)(\bx_s,\bx_r)
\]
\[
S[\kappa]^T(DS[\kappa]\delta \kappa)f(\bx_s,t) = -\int dx_r dy_r \int
dx'_s dy'_s a(\bx_r,\bx_s) a(\bx_r,\bx'_s) f'(t+T(\bx_s,\bx_r) -
T(\bx_r,\bx'_s)) (DT[\kappa]\delta \kappa)(\bx_s',\bx_r)
\]
\[
=\int dx_r dy_r dx'_s dy'_s d\omega  a(\bx_r,\bx_s) a(\bx_r,\bx'_s) \hat{f}(\omega)e^{i\omega(t +T(\bx_s,\bx_r) -T(\bx_r,\bx'_s))}
\]
Inner ($\bx_r$) integral, stationary phase (for 3D, so integral is
over $x_r,y_r$:
\[
\approx
\sum
\frac{2\pi}{\omega}|\det\left(\nabla_{\bx_r}\nabla_{\bx_r}(T(\bx_s,\bx_r)-T(\bx_r,\bx_s'))\right)|^{-\frac{1}{2}}
  e^{i\omega(t +T(\bx_s,\bx_r) -T(\bx_r,\bx'_s))} e^{i\pi\sigma/4}(DT[\kappa]\delta \kappa)(\bx_s',\bx_r)
\]
$\sigma$ is the signature of the Hessian - take it to be zero. The sum is over $\bx_r$ satisfying the stationary phase condition
\[
\nabla_{\bx_r} T(\bx_s,\bx_r) = \nabla_{\bx_r}T(\bx'_s,\bx_r)
\]
which may be interpreted as stating that the ray from $\bx_s$ to
$\bx_r$ arrives at $\bx_r$ with the same tangential slowness vector as the ray
from $\bx_s'$ to $\bx_r$. Assuming that all such rays cross the
surface transversally, and using the eikonal equation to determine the
$z$ component, conclude that the slowness vectors are the same. Thus
these must be the same rays, hence $\bx_s$ and $\bx'_s$ must lie on
the same ray. Assuming that rays cross the source surface only once,
this implies that $\bx_s=\bx_s'$. 
\section{General Case}
Begin with an abstract hyperbolic system of the sort decribed by
\cite{BlazekStolkSymes:13}, equation (1):
\begin{equation}
\label{symmhyp}
a \frac{\partial u}{\partial t} + p(\nabla) u  = f\mbox{ in } \bR^d \times \bR; \, u = u_0 \mbox{ for } t = 0,
\end{equation}
without the lower-order terms considered in that paper. Here $a$ is a
bounded measurable positive definite symmetric matrix-valued time-independent function, and $p(\nabla)$ is
a skew-symmetric first order PDO. 

Assume that 
$f$ is supported in an open domain $\Omega \in \bR^d$, and that $a$ is 
smooth in a neighborhood of  $\bR^d \setminus \Omega$. 

The key assumption on $a$ is that it has the {\em local energy decay
  property}: there exist $C, \alpha >0$ so that for any $f \in
L^2_{\rm comp}(\Omega \times (-\infty,0)), u_0 \in L^2_{\rm comp}(\Omega)$,
\begin{equation}
\label{eqn:decay}
\|u(t)\|_{L^2(\Omega)} \le Ce^{-\alpha t} (\|u_0\|_{L^2(\Omega)} +
\|f\|_{L^2(\Omega \times \bR)}).
\end{equation}

D-to-N map: suppose $\Gamma \subset \bR^d$ is a smooth submanifold of
dimension $d-1$, and define the trace map $S[m]$ as in the discussion
around Theorem 2 in \cite{BlazekStolkSymes:13}. Let $V$ be the
subspace as defined there, and assume that $S[m]$ extends to $V$.




\section{Adjoint}
\begin{equation}
\label{eqn:dfwd1d}
\frac{p^{n+1}-p^n}{\kappa\Delta t} = D^T \bv^{n+1/2} + \frac{f^n}{\kappa} \delta_{z_s},\,\frac{\bv^{n+1/2}-\bv^{n-1/2}}{\beta \Delta t} = - Dp^n,\, p^0=0, \bv^{1/2}=\bf{0}
\end{equation}
\begin{equation}
\label{eqn:dadj1d}
\frac{q^{n+1}-q^n}{\kappa \Delta t} = D^T \bw^{n+1/2} + \frac{g^{n+1}}{\kappa} \delta_{z_r},\,\frac{\bw^{n+1/2}-\bw^{n-1/2}}{\beta \Delta t} = - Dq^n, q^N=0, \bw^{N-1/2}=\bf{0}
\end{equation}
\[
\sum^N_{n=1} \frac{\Delta t}{\kappa} p^n g^n|_{z=z_r} = \sum_{\bx,n} dV \frac{\Delta t}{\kappa} p^n g^n \delta_{z_r}
\]
\[
= \sum_{\bx,n} dV \Delta t p^n\left(\frac{q^{n}-q^{n-1}}{\kappa\Delta t} -D^T \bw^{n-1/2}\right)
\]
\[
= \sum_{\bx,n} dV \Delta t \left(\frac{p^{n}-p^{n+1}}{\kappa\Delta t}\right)q^n - D p^n \cdot \bw^{n-1/2}
\]
\[
= \sum_{\bx,n} dV \Delta t \left(-D^T\bv^{n+1/2} - \frac{f^n}{\kappa}\delta_{z_s}\right)q^n + \frac{\bv^{n+1/2}-\bv^{n-1/2}}{\beta \Delta t}\bw^{n-1/2}
\]
\[
=-\sum_{\bx,n} dV \frac{\Delta t}{\kappa} f^nq^n \delta_{z_s} 
-\bv^{n+1/2} \cdot \left(D q^n - \frac{\bw^{n-1/2}-\bw^{n+1/2}}{\beta \Delta t} \right) 
\]
\[
=-\sum_{\bx,n} dV \frac{\Delta t}{\kappa} f^nq^n \delta_{z_s} 
-\bv^{n+1/2}\cdot  \left(D q^n + \frac{\bw^{n+1/2}-\bw^{n-1/2}}{\beta \Delta t} \right) 
\]
\begin{equation}
\label{eqn:adjreln}
=-\sum^{N-1}_{n=0} \frac{\Delta t}{\kappa} f^nq^n|_{z=z_s}
\end{equation}
\begin{equation}
\label{eqn:opdef}
S[\kappa,\beta]f = p|_{z=z_r}, \,S[\kappa,\beta]^*g = -q|_{z=z_s}
\end{equation}
\begin{equation}
\label{eqn:ipdef}
\langle u, v \rangle_{\kappa,z_0} = \sum_{\bx',n} \frac{dV' \Delta t}{\kappa(\bx',z_0)} u(\bx',z_0,n\Delta t) v(\bx',z_0,n\Delta t)
\end{equation}
\begin{equation}
\langle S[\kappa,\beta]f, g \rangle_{\kappa,z_s} = \langle f, S[\kappa,\beta]^*g \rangle_{\kappa,z_r}
\end{equation}

\bibliographystyle{seg}
\bibliography{../../bib/masterref}
