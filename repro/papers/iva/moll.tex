
The asymptotic property of $\de$ and dependent quantities will be
expressed in terms of a {\em mollifier family} $\Be$, whose
properties abstract those of Friedrichs mollifiers (\cite{Tay:81},
II.7):
\begin{enumerate}
\item $\{\Be:\epsilon>0\} \subset Op^{-\infty}(\oM,\oM)$ 
\item $\{\Be:\epsilon>0\} \subset Op^0(\oM,\oM)$ is bounded 
\item $\Be \rightarrow I$ strongly
\end{enumerate}

The essential properties of mollifier families are captured in the
following lemma, which abstracts \cite{Tay:81},
Chapter II section 7 and 
problem II.6.7.

\begin{lem}\label{lem:moll} Suppose that $\{\Be\}$ is a mollifier family,
and $P \in Op^{-1}(\oM,\oM)$. Then
\begin{enumerate}
\item $P\Be, \Be P \in Op^{-\infty}(\oM,\oM)$ for all
  $\epsilon > 0$
\item $\|P\Be - P\|_{{\cal L}(\oM^0,\oM^0)}, \|\Be P-P\|_{{\cal L}(\oM^0,\oM^0)} \rightarrow 0$ as
$\epsilon \rightarrow 0$  
\item $\|[P,\Be]\|_{{\cal L}(\oM^0,\oM^0)} \rightarrow 0$ as
  $\epsilon \rightarrow 0$.
\end{enumerate}
\end{lem}
\begin{proof}
Item 1. is clear. To prove item 2, note that for a sequence $\{\epsilon_j >0\}$ with
$\epsilon_j \rightarrow 0$, $T_j \equiv PB_{\epsilon_j}-P \rightarrow
0$ strongly in ${\cal L}(\oM^0,\oM^0)$. Since $\{\Be:\epsilon > 0\}$
is bounded in ${\cal L}(\oM^{-1},\oM^{-1})$ and $P \in {\cal
  L}(\oM^{-1},\oM^{0})$, $\{T_j\}$ is bounded in ${\cal
  L}(\oM^{-1},\oM^{0})$, whence $\{T_j\Lambda\}$ is bounded in ${\cal
L}(\oM^0,\oM^0)$. $\oM^1 \subset \oM^0$ is dense: for any
$x \in \oM^0$ and $\delta > 0$, choose $x_{\delta} \in \oM^1$ so that
$\|x-x_{\delta}\|_0 < \delta$. Then $\|(T_j\Lambda)x\|_0 \le \|T_j
(\Lambda x_{\delta})\| + C\delta$ with $C \ge 0$ independent of
$j$. Since $T_j \rightarrow 0$ strongly, for $j$ sufficiently large, 
$\|T_j(\Lambda x_{\delta})\| < \delta$. Since $\delta > 0$ is
arbitrary, conclude that $T_j \Lambda \rightarrow 0$ strongly. Since
$\Lambda^{-1}$ is compact on $\oM^0$ and composition of a strongly
convergent sequence of operators with a compact map produces a uniformly convergent
sequence, conclude that $\|P\Be - P\|_{{\cal L}(\oM^0,\oM^0)}
\rightarrow 0$. A similar argument shows that  $\|\Be P - P\|_{{\cal L}(\oM^0,\oM^0)}
\rightarrow 0$. Item 3 follows immediately from item 2.
\end{proof}

\begin{thm} \label{thm:innerbd} Suppose that $\brea,\bqea$ are families of maps $U \times
  D \times (0,e_{\rm max}] \mapsto \oM^0$ satisfying
\begin{eqnarray}
\label{eqn:priorbd1}
\|e_{m,\lambda}\|_m & \le & e\\
\label{eqn:priorbd2}
\|e_{q,\lambda}\|_m &\le& e 
\end{eqnarray}
where
\begin{eqnarray}
\label{eqn:mres}
e_{m,\lambda} & = & \Ne[m]\brea[m,d,e]-\odF[m]\de,\\
\label{eqn:qres}
e_{q,\lambda} & = & \Ne[m]\bqea[m,d,e]-A^TA\brea[m,d,e] 
\end{eqnarray}
Then there exist $K_1,K_2 > 0$ and an increasing function $\epsilon: \bR^+
\rightarrow (0,e_{\rm max}]$ for which $\lim_{\lambda \rightarrow 0}
\epsilon(\lambda)=0$ and
\begin{eqnarray}
\label{eqn:postbd1}
\|\brea[m,d,e] -\bre[m,d]\|_m & \le & K_1e + \frac{K_2}{\lambda^2}
                                      \|B_{\epsilon(\lambda)}e_{m,\lambda}\|_m\\
\label{eqn:postbd2}
\|\bqea[m,d,e] -\bqe[m,d]\|_m & \le & K_1e + \frac{K_2}{\lambda^2}
\|B_{\epsilon(\lambda)}e_{m,\lambda}\|_m + \frac{K_2}{\lambda^2} \|B_{\epsilon(\lambda)}e_{q,\lambda}\|_m.
\end{eqnarray}
\end{thm}

\begin{proof} For convenience, set
\[
\dbr = \brea[m,d,e] -\bre[m,d].
\]
Then
\[
\Ne[m] \dbr = e_{m,\lambda}
\]
from which follows the basic estimate ($m$ and $0$ norms equivalent):
\[
\|\dbr\|_0 \le \frac{C}{\lambda^2}\| e_{m,\lambda}\|_0.
\]
(Here and in the following arguments, $C$ stands for a scalar
independent of $\lambda$, which may change from expression to
expression.)

First estimate $\Be \dbr$:
\[
\Ne[m] \Be \dbr = \Be e_{m,\lambda} + [S[m],\Be]\dbr
\]
From Lemma \ref{lem:moll}, $\|[S[m],\Be]\|_{{\cal L}(\oM^0,\oM^0)} \rightarrow
  0$. Choose $\epsilon(\lambda)$ so that $\|[S[m],\Be]\|_{{\cal
      L}(\oM^0,\oM^0)} \le \lambda^2/2$. Then above implies that
\[
\|B_{\epsilon(\lambda)} \dbr\| \le \frac{C}{\lambda^2}
\|B_{\epsilon(\lambda)} e_{m,\lambda}\|
\]
Also
\[
(I-\Be)\dbr + (I-\Be)S[m](I-\Be)\dbr + (I-\Be)S[m]\Be\dbr =
(I-\Be)e_{m,\lambda}
\]
Also from Lemma \ref{lem:moll}, $\|(I-\Be)S\|_{{\cal L}(\oM^0,\oM^0)} \rightarrow
  0$. Choose $\epsilon(\lambda)$ if necessary smaller so that $\|(I-\Be)S\|_{{\cal
      L}(\oM^0,\oM^0)} \le \min(\lambda^2,1/2)$. Then
\[
\|(I-B_{\epsilon(\lambda)})\dbr\|_0 \le
C\|B_{\epsilon(\lambda)} \dbr\|_0 + 2e
\]
from which inequality \ref{eqn:postbd1} follows directly.

\end{proof}
