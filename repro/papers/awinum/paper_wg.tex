\title{AWI-WG Examples}
\author{Huiyi Chen, Susan E. Minkoff, and William W. Symes}

\lefthead{Symes}

\righthead{AWI-WG Examples}

\maketitle
\parskip 12pt

\begin{abstract}
abstract goes here
\end{abstract}
\setlength{\parindent}{0cm}

\section{Summary}
Scans of the WG objective function over the line segment between a homogeneous slowness model $m_0$ and the target model $m$  (the model used to generate the data) show smooth decrease towards a minimum. However the minimizer is often not located at the point on the line segment corresponding to the target model. The line segment is
$$ t \mapsto (1-t)m_0 + t m $$
so $t=1$ is the target.  Depending on exactly which choices of regularization parameter $\sigma$ and normal residual tolerance $\rho$ are used in the computation of the objective function, the minimizer can be close to $t=1$ or in error by more than 10\%, that is, $t > 1.1$.

This error represents a limitation of the resolution obtainable from the AWI-WG objective as a means of estimating slowness. For examples in which the starting model is farther from the target model, the error appears to be proportionally smaller. Since the argument connecting AWI-WG to travel time tomography depends on geometric asymptotic theory, which is approximate, this result is not surprising. It also suggests that the penalty version of AWI, which incorporates a controlled component of the FWI objective, may offer better resolution than the AWI-WG.

\section{Results}

\inputdir{project}

The first example uses for $m$ the ``lens'' model that we have used in many other tests, generated by
\begin{verbatim}
data.model(bulkfile='m0l.rsf', bulk=4.0, nx=401, nz=201,
   dx=20, dz=20, lensfac=0.7)
\end{verbatim}
and plotted in Figure \ref{fig:m0l}:
\plot{m0l}{width=\textwidth}{Lens model, lensfac = 0.7, high aspect ratio'}

The data for these two models  ($m$ and $m_0$) differs mostly by a half-wavelength phase shift: the waveform apparent in both gathers is similar to the first derivative of a gaussian, whereas the waveform of the difference is closer to a second derivative.

\multiplot{2}{d0l,dlhom}{width=0.45\textwidth}{Left: data for target model $m$; Right: data for homogeneous model $m_0$.}
\plot{d0ldiff}{width=0.45\textwidth}{DIfference between gathers plotted in Figures \ref{fig:d0l} and \ref{fig:dlhom}}

The following two tables show the values of the AWI-WG objective function sampled at $\Delta t = 0.1$, for $\sigma=0.001$ and three different (decreasing) choices of $\rho$, resulting in increasingly accurate solution of the normal equations and decreasing error in the data fit by the adaptive filter.

\begin{table}
  \begin{center}
  \begin{tabular}{c|c|c}
    \hline
    t & v & e \\
0.0 &  1.1593e+06 & 2.9150e-02 \\ 
0.1 & 1.0909e+06 & 2.9959e-02 \\
0.2 & 1.0149e+06 & 3.1390e-02 \\
0.3 & 9.7392e+05 & 2.9837e-02 \\
0.4 & 9.2755e+05 & 2.9033e-02 \\
0.5 & 8.5273e+05 & 3.1508e-02 \\
0.6 & 8.2288e+05 & 2.8947e-02 \\
0.7 & 7.8012e+05 & 2.9248e-02 \\
0.8 & 7.5644e+05 & 2.7974e-02 \\
0.9 & 7.2089e+05 & 2.9301e-02 \\
1.0 & 7.0420e+05 & 2.9180e-02 \\
1.1 & 6.8927e+05 & 2.9798e-02 \\
    1.2 & 7.0310e+05 & 2.9222e-02 \\
    \hline 
  \end{tabular}

  \caption{Values (v) of AWI-WG objective and adaptive filter error (e)  based on data shown in Figure \ref{fig:d0l} at $m_{\rm test} = (1-t)m_0 + t m$, computed with $\sigma=0.001, \rho = 0.01$.}
    \label{table:rho01}
  \end{center}
  \end{table}


  \begin{table}
  \begin{center}
  \begin{tabular}{c|c|c}
    \hline
    t & v & e \\
0.0 & 1.2419e+06 & 6.7093e-03 \\
0.1 & 1.1755e+06 & 6.4195e-03 \\
0.2 & 1.1334e+06 & 6.7615e-03 \\
0.3 & 1.1132e+06 & 7.3999e-03 \\
0.4 & 1.0664e+06 & 7.4529e-03 \\
0.5 & 1.0245e+06 & 7.3856e-03 \\
0.6 & 9.9281e+05 & 7.4972e-03 \\
0.7 & 9.4065e+05 & 7.5612e-03 \\
0.8 & 7.6815e+05 & 5.5490e-03 \\
0.9 & 7.5830e+05 & 5.5697e-03 \\
1.0 & 7.5452e+05 & 6.0200e-03 \\
1.1 & 7.6785e+05 & 6.4036e-03 \\
1.2 & 8.1194e+05 & 6.8795e-03 \\
    \hline 
  \end{tabular}
   \caption{Values (v) of AWI-WG objective and adaptive filter error (e) based on data shown in Figure \ref{fig:d0l} at $m_{\rm test} = (1-t)m_0 + t m$, computed with $\sigma=0.001, \rho = 0.001$.}
  \label{table:rho001}    
  \end{center}
\end{table}

\begin{table}
  \begin{center}
  \begin{tabular}{c|c|c}
    \hline
    t & v & e \\
0.0 & 1.4052e+06 & 1.4151e-03 \\
0.1 & 1.3288e+06 & 1.3979e-03 \\
0.2 & 1.2528e+06 & 1.4231e-03 \\
0.3 & 1.1850e+06 & 1.4625e-03 \\
0.4 & 1.1325e+06 & 1.4531e-03 \\
0.5 & 1.0860e+06 & 1.4492e-03 \\
0.6 & 1.0351e+06 & 1.4641e-03 \\
0.7 & 1.0033e+06 & 1.3976e-03 \\
0.8 & 9.9407e+05 & 1.3310e-03 \\
0.9 & 9.8442e+05 & 1.3020e-03 \\
1.0 & 9.6528e+05 & 1.3018e-03 \\
1.1 & 9.4995e+05 & 1.3066e-03 \\
    1.2 & 9.3124e+05 & 1.3467e-03 \\
        \hline 
  \end{tabular}
   \caption{Values (v) of AWI-WG objective and adaptive filter error (e) based on data shown in Figure \ref{fig:d0l} at $m_{\rm test} = (1-t)m_0 + t m$, computed with $\sigma=0.001, \rho = 0.0001$.}
    \label{table:rho0001}
  \end{center}
\end{table}
    
As $\rho$ decreases by two orders of magnitude, the fit error decreases from about 3 \% ($\rho=0.01$), to 0.7\% ($\rho=0.001$), to 0.15\% $\rho=0.0001$. The last result is probably near-optimal: it requires several hundred CG iterations values of $t$ near 1.0. the positive regularization weight bounds the lowest error achievable away from zero.

Note that the first scan (Table \ref{table:rho01} with $\rho=0.01$) has its minimum sample at $t=1.1$. Assuming that the objective is monotone increasing (decreasing) for $t > 1.2$ (respectively $t < 1.0$), the most one can conclude is that the minimizer on this line segment lies in the open interval $(1.0, 1.2)$. Quadratic fit suggests that it may be close to $t=1.1$. However the precise location of the line minimizer is not the point: that is, rather, that the minimum in any set containing this line segment is not at $t=1.0$, that is not at $m_{\rm test} = m$, the target model.

One might think that the normal equation has not been solved precisely enough to approximate the actual value of the AWI-WG objective. The scan with $\rho=0.001$ (Table \ref{table:rho001}) supports this supposition: the data is fit more accurately by the constructed adaptive filters, and the minimum sample occurs at $t=1.0$ (which of course means only that the line minimum is likely between $0.9$ and $1.1$). However this conclusion would be incorrect: Table \ref{table:rho0001} with an even more accurate solution of the normal equation ($\rho=0.0001$) now shows a minimum at the maximum $t = 1.2$, from which one can only conclude that the line minimizer is $> 1.1$.

These examples lead to the conclusion that the resolution of the AWI-WG slowness estimate is strictly limited: in this example, the error relative to the overall slowness error at the initial estimate ($m_0$) is at least 10\%.

Looked at another way, the difference between $m$ and $m_0$ (with the other acquisition choices and modeling choices made in creating the data and computing the objective) are to small for AWI-WG to detect accurately. To test this hypothesis, I introduce another lens model $m_1$, differing from the previous model in having a fatter and deeper lens:

\plot{m1l}{width=\textwidth}{Lens model $m_1$, lensfac = 0.5, low aspect ratio'}

The command to generate this model is
\begin{verbatim}
data.model(bulkfile='m1.rsf', bulk=4.0, nx=401, nz=201,
   dx=20, dz=20, lensfac=0.5, lensradd=0.4)
\end{verbatim}

(Note the new argument {\tt lensradd} - using this command requires the latest version of the {\tt data} module.)

The time shift between data built with the homogeneous model $m_0$ and the corresponding data built with $m_1$ is considerably larger:

\multiplot{2}{d1l,d1ldiff}{width=0.45\textwidth}{Left: data for target model $m_1$ (Figure \ref{fig:m1l}); Right: difference with data for homogeneous model $m_0$.}

The next three tables show the same experiments as the first three, but the samples are taken from the line segment $t \mapsto (1-t)m_0 + t m_1, t \in [0.0,1.2]$. The choices of $\sigma$ and $\rho$ are the same. 

\begin{table}
  \begin{center}
  \begin{tabular}{c|c|c}
    \hline
    t & v & e \\
0.0 & 3.3591e+06 & 2.8394e-02 \\
0.1 & 2.9599e+06 & 2.8493e-02 \\
0.2 & 2.5653e+06 & 2.9214e-02 \\
0.3 & 2.1924e+06 & 3.0009e-02 \\
0.4 & 1.8761e+06 & 2.7319e-02 \\
0.5 & 1.5692e+06 & 2.8048e-02 \\
0.6 & 1.2820e+06 & 2.9795e-02 \\
0.7 & 1.0505e+06 & 2.9463e-02 \\
0.8 & 8.5406e+05 & 2.9257e-02 \\
0.9 & 7.1680e+05 & 3.0119e-02 \\
1.0 & 6.6482e+05 & 2.7938e-02 \\
1.1 & 6.8210e+05 & 2.8238e-02 \\
1.2 & 7.8490e+05 & 2.8498e-02 \\
        \hline 
  \end{tabular}
   \caption{Values (v) of AWI-WG objective and adaptive filter error (e) based on data shown in Figure \ref{fig:d0l} at $m_{\rm test} = (1-t)m_0 + t m_1$, computed with $\sigma=0.001, \rho = 0.01$.}
    \label{table:m1rho01}
  \end{center}
\end{table}
\begin{table}
  \begin{center}
  \begin{tabular}{c|c|c}
    \hline
    t & v & e \\
    0.0 & 3.5153e+06 & 5.9678e-03 \\
0.1 & 3.1136e+06 & 6.1703e-03 \\
0.2 & 2.7146e+06 & 5.8284e-03 \\
0.3 & 2.3423e+06 & 5.9218e-03 \\
0.4 & 2.0156e+06 & 6.2422e-03 \\
0.5 & 1.7010e+06 & 6.1459e-03 \\
0.6 & 1.4255e+06 & 6.3031e-03 \\
0.7 & 1.2071e+06 & 6.4979e-03 \\
0.8 & 1.0679e+06 & 6.8024e-03 \\
0.9 & 9.3836e+05 & 7.2389e-03 \\
1.0 & 7.8422e+05 & 6.3987e-03 \\
1.1 & 8.8863e+05 & 7.0049e-03 \\
1.2 & 1.0187e+06 & 6.3996e-03 \\
        \hline 
  \end{tabular}
   \caption{Values (v) of AWI-WG objective and adaptive filter error (e) based on data shown in Figure \ref{fig:d0l} at $m_{\rm test} = (1-t)m_0 + t m_1$, computed with $\sigma=0.001, \rho = 0.001$.}
    \label{table:m1rho001}
  \end{center}
\end{table}
                   
\begin{table}
  \begin{center}
  \begin{tabular}{c|c|c}
    \hline
    t & v & e \\
    0.0 & 3.5287e+06 & 1.6961e-03 \\
0.1 & 3.0967e+06 & 1.3654e-03 \\
0.2 & 2.7877e+06 & 1.6737e-03 \\
0.3 & 2.3827e+06 & 1.4092e-03 \\
0.4 & 2.0719e+06 & 1.5689e-03 \\
0.5 & 1.7617e+06 & 1.3423e-03 \\
0.6 & 1.4830e+06 & 1.3776e-03 \\
0.7 & 1.2629e+06 & 1.3850e-03 \\
0.8 & 1.0936e+06 & 1.3775e-03 \\
0.9 & 9.4892e+05 & 1.4197e-03 \\
1.0 & 9.0077e+05 & 1.4474e-03 \\
1.1 & 8.9294e+05 & 1.3807e-03 \\
1.2 & 1.0196e+06 & 1.3239e-03 \\
        \hline 
  \end{tabular}
   \caption{Values (v) of AWI-WG objective and adaptive filter error (e) based on data shown in Figure \ref{fig:d0l} at $m_{\rm test} = (1-t)m_0 + t m_1$, computed with $\sigma=0.001, \rho = 0.0001$.}
    \label{table:m1rho0001}
  \end{center}
\end{table}

Qualitatively, these results are similar to those for the closer-to-homogeneous target model $m$. However, this time the scan for gradient tolerance $\rho=0.001$ still has the minimum sample at $t=1.0$, and the scan for $\rho=0.0001$ has its minimum sample at $t=1.1$ rather than $t=1.2$. That is, the error in the location of the minimum sample - or, the ``solution'' of the inverse problem via AWI-WG, if the search is limited to this line segment - is closer, as a proportion of the difference with the starting model, than was the case in the previous example.

These examples and other similar examples suggest that the observed error is a resolution error stemming from the difference between AWI-WG and travel time tomography: the relation is based on geometric asymptotics, which provides approximate solutions of the wave equation.

