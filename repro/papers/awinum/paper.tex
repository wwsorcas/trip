\title{Matched Source Waveform Inversion for Transmitted Waves}
\author{Huiyi Chen, Susan E. Minkoff, and William W. Symes}

\lefthead{Symes}

\righthead{MSWI}

\maketitle
\parskip 12pt

\begin{abstract}
Matched Source Waveform Inversion applied to transmission data
produces an estimate of refractive index similar to the result of
travel time inversion, but without explicit identification of travel
times. This paper reviews the theoretical justification of this result
and its limitations, and exhibits a numerical illustration. 
\end{abstract}
\setlength{\parindent}{0cm}

\section{Introduction}
Matched Source Waveform Inversion (MSWI) is a variant of Full Waveform
Inversion (FWI) \cite[]{VirieuxOperto:09}, that sometimes overcomes
one of FWI's impediments, namely its tendency to stagnate at
suboptimal model estimates.

(...)
,
demonstrably far from global optima in controlled settings and often
uninformative of material structure in field use. It is possible that
such local descent algorithms are trapped in regions around local,
\section{Summary}
The version of MSWI discussed here uses acoustic wave propagation with
isotropic point sources and receivers.
The pressure and velocity fields $p({\bf x},t;{\bf x}_s)$, ${\bf v}({\bf x},t;{\bf x}_s)$ for the source location ${\bf x}_s$ depend on the bulk modulus $\kappa({\bf x})$, buoyancy $\beta({\bf x})$ (reciprocal of the density $\rho({\bf x})$), and wavelet $w(t;{\bf x}_s)$ through the acoustic system
\begin{eqnarray}
  \label{eqn:awe}
 \frac{\partial p}{\partial t} & = &- \kappa \nabla \cdot {\bf v} +
w(t;{\bf x}_s) \delta({\bf x}-{\bf x}_s); \nonumber \\
\frac{\partial {\bf v}}{\partial t} & = & - \beta \nabla p; \\ 
p, {\bf v} & = & 0 \mbox{ for }  t \ll 0.
\end{eqnarray}
The model vectors $m=(\kappa,\rho)$ make up the domain of the forward
map or {\em modeling operator} is $F[m]w = \{p({\bf x}_r,t;{\bf
  x}_s)\}$, for specified source and receiver positions ${\bf x}_s, {\bf x}_r$ and
recording interval $[0,t_d]$. 

In this context, FWI means: given source wavelet
$w(t;\bx_s)$ and data traces $d(\bx_r,\cdot;\bx_s)$, find a model $m$
so that $F[m]w \approx d$. The
simplest version of FWI concretizes this task by asking for a model
$m$ minimizing the mean square error
\begin{equation}
  \label{eqn:fwi}
  J_{\rm FWI}[m;d]= \frac{1}{2}\|F[m]w-d\|^2.
\end{equation}
Note that $w$ is assumed known and treated as a parameter in this
statement of the FWI task. In practice, it is not known (nor are the
sources and receviers necessarily isotropic), and should be estimated
along with the model $m$.

The approach to local optimization taken here (and in most work on FWI
and related topics) is based on the gradient of the objective defined
in equiation \ref{eqn:fwi}:
\begin{equation}
  \label{eqn:fwigrad}
  g = \nabla  J_{\rm FWI}[m;d] = DF[m]^T(F[m]w-d).
\end{equation}
In this formula, $DF[m]$ is the derivative of $F$ at $m$. 
This is the Euclidean (or $L^2$) gradient, that is, the vector $g$ for
which the Euclidean inner product
\begin{equation}
  \label{eqn:eucip}
  \langle g, \delta m\rangle = g^T\delta m
\end{equation}
with any other vector $\delta m$ gives the
rate of change in the direction of that vector of $J_{\rm FWI}$ at m.

For optimization within a set of slowly varying models (on the
wavelength scale),  it is appropriate to penalize oscillation of the
search vector. A convenient way to accompish this goal is the use of a
{\em weighted inner product} to define the gradient, rather than the
Euclidean inner product. A weight operator $W$ should be symmetric and
positive definite: then
\begin{equation}
  \label{eqn:wip}
  \langle g, \delta m\rangle_W = g^TW\delta m
\end{equation}
defines an alternative inner product. Comparing the definitions
\ref{eqn:fwigrad}, \ref{eqn:eucip}, and \ref{eqn:wip}, clearly the
vector $g_W$ for which $\langle g_W, \delta m \rangle_W$ gives the
rate of change of $J_{\rm FWI}$ at $m$ in the direction $\delta m$ is
\begin{equation}
  \label{eqn:fwiwgrad}
  g_W = W^{-1}\nabla  J_{\rm FWI}[m;d] =W^{-1} DF[m]^T(F[m]w-d).
\end{equation}

If $W$ is chosen to greatly
amplify oscillatory components of the vector to which it is applied,
then those components of $g_W$ must be suppressed relative to the
corresponding components of $g$, hence $g_W$ represents a
non-oscillatory search direction. Note that only the inverse operator
$W^{-1}$ appears in the formula \ref{eqn:fwiwgrad}.

As mentioned earlier, application of local optimization methods
directly to $J_{\rm FWI}$ tends to produce unsatisfactory model
estimates. MSWI modifies the measure of distance between predicted and
observed data by inserting an adaptive filter field $u$, consisting of
one filter per trace. Since only finite time intervals of $u$ and $d$ are available in
practice, introduce the {\em truncated filter operator} $K[u]$. This
operator acts by extending its filter $u$ (given on a symmetric
interval $[-t_u,t_u]$ for each source-receiver pair) and the function to which
it is applied (given on the data interval $[0,t_d]$ for each
source-receiver pair) to be zero outside their domains of definition, convolving the
resulting functions on $\bR$, and finally restricting or truncating
the result to the time interval of the input function, that is,
$[0,t_d]$. This operator is applied to the predicted data $F[m]w$ to
produce the filtered predicted data $K[u]F[m]w$.

It is possible to make the error between filtered predicted data and
observed data as small as one likes by choosing an appropriate filter
field $u$, so this error by itself is useless for estimating the
model. If $u(\bx_s,\bx_r,t)=\delta(t)$, on the other hand the filtered
predicted data is identical to the predicted data.
Therefore, some penalty for divergence of the filter $u$ from
$\delta(t)$ needs to supplement the filtered prediction error.. The works referenced in the Introduction mostly use the
mean-square of the filter scaled by $t$, and add it to the mean-square
of the filtered prediction error. This sum is the MSWI objective function:
\begin{equation}
  \label{eqn:filtpen}
  J_{\alpha,\sigma}[m,u;d]=\frac{1}{2}(\|K[u]F[m]w-d\|^2
  +\alpha^2\|tu\|^2 + \sigma^2\|u\|^2).
\end{equation}


\section{A numerical illustration}

\inputdir{project}

This section presents the inversion of acoustic data with dimensions
typical of crustal seismic exploration. The target bulk modulus field ($\kappa$)
is depicted in Figure \ref{fig:m}, and contains an acoustic ``lens''
positioned in the center between 1000 m and 3000 m depth
(``$z$''). The background level outside the lens is 4 GPa; at the
center it is 2 GPa. The
buoyancy ($\beta$)is spatially homogeneous at 1 cm$^3/g$, so the background
wave speed is 2000 m/s.

\plot{m}{width=\textwidth}{Lens model.}

This field is sampled on a 20 $\times$ 20 m grid. Acoustic wave
propagation is simulated via a staggered grid finite difference method
\cite[]{vir86,lev88,Cohen:01}. The time
step is chosen to be safely stable, given the parameter fields,
difference formulae, and
spatial sampling. 

Receivers are spaced on the line at depth 1000 m, from 2000 m to 6000
m horizontal coordinate (``$x$''), spaced 20 m apart. Eleven source
positions lie on the line $z=3000$ m, spaced 200 m apart, from
$x=3000$ m to $x = 5000$ m. The source wavelet $w$ (the same for every
source ) is a $[1.0, 2.5, 7.5, 12.5]$ Hz trapezoidal bandpass filter,
with a median frequency of 5.875 Hz corresponding to a median wavelength of
$\approx$ 340 m, and shortest wavelength of 160 m. It is centered at
$t=1$ s.

The discrete pressure field is
sampled at the externally specified time grid and spatial locations
via piecewise linear interpolation in space and cubic spline
interpolation in time. The isotropic point source is added
into the acoustic fields at each time step via the adjoints of these
interpolation operations.

Simulated data from this configuration appears as Figure \ref{fig:d11}.

The initial model $m_0$ for inversion is homogeneous, with $\kappa = $ 4 GPa,
$\beta = $ 1 cm$^3$/g. The corresponding data is depicted in Figure \ref{fig:sim0}.
These plots show the time shift between the homogeneous model data and
the target data, for the central source positions.

\multiplot{2}{d11,sim0}{width=0.45\textwidth}{a: Data for target model $m$
  (Figure \ref{fig:m}). b: Data for homogeneous model $m_0$.}

We apply a version of weighted steepest descent optimization to the
minimization of $J_{\rm FWI}$. The inverse weight operator ($W^{-1}$
in formula \ref{eqn:fwiwgrad}) is a
10-point moving average in both spatial directions, repeated once. As
noted above, the weight operator itself is not required. The weighted
gradient is computed via formula \ref{eqn:fwigrad}. The adjoint
derivative $DF[m]^T$ is computed via the {\em adjoint state method}
\cite[]{Chavent:74,GauTarVir:86}, with time reversal of the acoustic fields
implemented via optimal checkpointing
\cite[]{Griewank:92,Griewank:book,Symes:06a-pub}. The negative of the weighted
gradient ($g_W$, formula \ref{eqn:fwiwgrad}) is the search direction,
and the optimal step in this direction is approximated by a simple
backtracking line search algorithm \cite[]{NocedalWright}.

The value of $J_{\rm FWI}[m_0,d] \approx 3.04$, and the rate of
increase in the weighted gradient direction at $m_0$ is 1.74 $\times
10^{-6}$. Note that the rate of increase is the same as the weighted
gradient norm. After 10 steepest descent steps, the objective value has
decreased by about 30\%, to 1.9, and the weighted gradient norm by
more than an order of magnitude, to $\approx 9.6 \times 10^{-8}$. The
final model appears in Figure \ref{fig:mestfwi0}, and the
corresponding data in Figure \ref{fig:resimfwi0}.

%\multiplot{2}{mestfwi0,resimfwi0}{width=0.45\textwidth}{a: Bulk
%  modulus produced by 10 steepest descent steps to minimize the FWI
%  objective $J_{\rm FWI}[\cdot;d]$ with data $d$ depicted in Figure
%  \ref{fig:sim11}, starting with homogeneous model. b: Data simulated
%  from FWI inversion result.}

\plot{mestfwi0}{width=\textwidth}{a: Bulk modulus produced by 10
  steepest descent steps to minimize the FWI objective \ref{eqn:fwi},
  using the data shown in \ref{fig:d11}.}

\multiplot{2}{resimfwi0,residfwi0}{width=0.45\textwidth}{a: Data
  corresponding to FWI inversion result shown in Figure
  \ref{fig:mestfwi0}. b: Residual or data error - difference between
  data shown in Figure \ref{fig:resimfwi0} and target data
  \ref{fig:d11}. Note the failure to match the later signal in the
  central part of the display.}

While the reduction in the (weighted) gradient norm indicates progress
towards a stationary point, it is not possible to claim that the final
estimate is in the vicinity of a local minimizer. The fit to data
obtained is not at all satisfactory. More discussion of this result
will be found later in the paper.

\bibliographystyle{seg}
\bibliography{../../bib/masterref}
