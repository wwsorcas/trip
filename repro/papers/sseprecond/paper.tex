\title{Efficient Computation of Extended Sources}
\author{William. W. Symes \thanks{The Rice Inversion Project,
Department of Computational and Applied Mathematics, Rice University,
Houston TX 77251-1892 USA, email {\tt symes@caam.rice.edu}.}}

\lefthead{Symes}

\righthead{Approximate Source Inversion}

\maketitle
\begin{abstract}
Source extension is a reformulation of inverse problems in wave propagation, that at least in some cases leads to computationally tractable iterative solution methods. The core subproblem in all source extension methods is the solution of a linear inverse problem for a source (right hand side in a system of wave equations) through minimization of data error in the least squares sense with soft imposition of physical constraints on the source via an additive quadratic penalty. A variant of the time reversal method from photoacoustic tomography provides an approximate solution that can be used to precondition Krylov space iteration for rapid convergence to the solution of this subproblem. An acoustic 2D example for sources supported on a surface, with a soft contraint enforcing point support, illustrates the effectiveness of this preconditioner.
\end{abstract}

\section{Introduction}
Full Waveform Inversion (FWI) can be described in terms of 
\begin{enumerate}
\item a linear wave operator $L[{\bf c}]$, depending on a vector of
  space-dependent coefficients ${\bf c}$ and acting on causal vector wavefields $\bu$ vanishing in negative time:
\begin{equation}
\label{eqn:init}
\bu \equiv 0, t \ll 0; 
\end{equation}
\item a trace sampling operator $P$ acting on wavefields and producing data traces;
\item and a (vector) source function (of space and time) $\bff$ representing energy input to the system. 
\end{enumerate}
The basic FWI problem is: given data $d$, find ${\bf c}$ so that 
\begin{equation}
\label{eqn:fwi}
P\bu \approx d \mbox{ and } L[\bf{c}]\bu = \bff.
\end{equation}
In this formulation, the source function $\bff$ may be given, or
to be determined subject to some constraints.

%The energy source $\bff$ may also be largely undetermined, apart from some known characteristics such as localization in space and/or time. In fact, additional source degrees of freedom, beyond those needed to describe physically realized sources, may be useful in rendering the FWI problem \ref{eqn:fwi} more amenable to numerical solution, via so-called extended modeling (see \cite{geoprosp:2008}, \cite{LeeuwenHerrmannWRI:13}, \cite{HuangNammourSymesDollizal:SEG19}, and many references cited there). Therefore it is natural to view $\bff$ as also an unknown in formulating the problem \ref{eqn:fwi} via nonlinear least squares:
A simple nonlinear least squares formulation is:
\begin{equation}
\label{eqn:ols}
\mbox{choose } {\bf c} \mbox{ to minimize } \|PL[{\bf c}]^{-1}\bff -d \|^2.
\end{equation}
Practical optimization formulations typically augment the objective in
\ref{eqn:ols} by additive penalties or other constraints.

As is well-known, local optimization methods are the only feasible
approach given the dimensions of a typical instance of \ref{eqn:fwi},
and those have a tendency to stall due to ``cycle-skipping''. See for
exampe \cite{VirieuxOperto:09} and many references cited there. Source
extension is one approach to avoiding this problem. It consists in
imposing the wave equation as a soft as opposed to hard constraint, by
allowing the source field $\bff$ to have more degrees of freedom than
is permitted by a faithful model model of the seismic experiment, and
constraining these additional degrees of freedom by means of an
additive penalty modifying the probem \ref{eqn:ols}:
\begin{equation}
\label{eqn:esi}
\mbox{choose } {\bf c}, \bff \mbox{ to minimize } \|PL[{\bf c}]^{-1}\bff -d \|^2 + \alpha^2 \|A\bff\|^2 
\end{equation}
The operator $A$ penalizes deviation from known (or assumed)
characteristics of the source function - its null space consists of
feasible (or ``physical'') source models.

\cite{HuangNammourSymesDollizal:SEG19} present an overview of the
literature on source extension methods, describing a variety of
methods to add degrees of freedom to physical source model. The present paper
concerns {\em surface source extension}: physical sources are
presumed to be concentrated at points $\bx_s$ in space, whereas their extended
counterparts are permitted to spread energy over surfaces containing
the physical source locations. A simple choice for the penalty
operator $A$ is then multiplication by the distance $|\bx-\bx_s|$ to the physical
source location:
\begin{equation}
  \label{eqn:penop}
  (A\bff)(\bx,t) = |\bx-\bx_s|\bff(\bx,t)
\end{equation}
I shall use this choice of penalty operator whenever a specific choice
is necessary in the development of the theory below.

This paper presents a numerically efficient approach to solving the
{\em source subproblem} of problem \ref{eqn:esi}:
\begin{equation}
\label{eqn:esis}
\mbox{given } {\bf c}, \mbox{ choose } \bff \mbox{ to minimize }
\|PL[{\bf c}]^{-1}\bff -d \|^2 + \alpha \|A\bff\|^2 
\end{equation}
Solution of this subproblem is an essential component of {\em variable
  projection} algorithms for solution of the nonlinear inverse problem
\ref{eqn:esi}. Variable projection is not merely a convenient choice
of algorithm for this purpose: it is in some sense essential, see for
example \cite{Symes:SEG20}. It replaces the nonlinear
least squares problem \ref{eqn:esi} with a {\em reduced} problem, to
be solved iteratively. Each iteration involves solution of the
subproblem \ref{eqn:esis}. Therefore efficient solution of the
subproblem is essential to efficient solution of the nonlinear problem
via variable projection.

The penalty operator $A$ defined in \ref{eqn:penop} is linear, so the source
subproblem is a linear least squares problem. Under some additional
assumptions to be described below, I shall show how to construct an
accurate approximate solution operator for problem
\ref{eqn:esis}. This approximate solution operator may be used to
accelerate Krylov space methods for the solution of the source
subproblem \ref{eqn:esis}. Numerical examples suggest the
effectiveness of this acceleration.

I will fully describe a preconditioner for a special
case of the source subproblem \ref{eqn:esis}, in which ${\bf u}$ is an
acoustic field, $L[{\bf c}]$ is the wave operator of linear
acoustodynamics, the spatial positions of traces extracted by $P$ lie
on a plane, and the positions at which the extended
source $\bff$ is nonzero lie on another, parallel, plane. The data and
sources are further limited to pressure traces and constitutive law
defects (``pressure sources''). While this
``crosswell'' configuration simplifies the analysis underlying the
construction of approximate solutions for the source subproblem
\ref{eqn:esis}, it is only one of many transmission configurations for
which similar developments are possible. Perhaps the most important
alternative example is the diving wave configuration, which plays a
central role in contemporary FWI. I will discuss the generalization of
surface source extention to diving wave inversion at the end of the
paper.

The preconditioner construction is very similar to the time-reversal
method in photoacoustic tomography
\cite[]{StefanovUhlmannIP:09}. Preconditioning amounts to a change of
norm in the domain and range spaces of the modeling operator. In this
case, the modfied norms are weighted $L^2$, and the weight operators
map pressure to corresponding surface source on the source and
receiver planes. These are essentially the same as the ``hyperbolic
Dirichlet-to-Neumann'' map that plays a prominent role in
thermoacoustic tomography and other wave inverse problems
\cite[]{Rachele:00,StefUhl:05}. \cite{HouSymes:EAGE16} demonstrated a
very similar preconditioner for Least Squares Migration, also for its
subsurface offset extension \cite[]{HouSymes:16}, motivated by
\cite{tenKroode:12}. These constructions also involve the
Dirichlet-to-Neumann operator. This concept also turns up in hidden
form in the work of Yu Zhang and collaborators on true amplitude
migration
\cite[]{YuZhang:14,TangXuZhang:13,XuWang:2012,XuZhangTang:11,Zhang:SEG09}.

The obvious computation of the pressure-to-source map
- prescribe the pressure, solve the wave equation with this boundary
condition, read off the equivalent source - suffers from intrinsic
numerical inaccuracy. I suggest an alternative computationally
feasible approach.

\section{Operators}

For acoustic wave physics, the coefficient vector is
$\bf{c}=(\kappa,\rho)^T$, with components bulk modulus $\kappa$ and
density $\rho$, and the state vector $\bu=(p,\bv)^T$ consists of
pressure $p$ (a scalar space-time field) and particle velocity $\bv$
(a vector space-time field). The wave operator $L[\bf{c}]$ is:
\begin{equation}
\label{eqn:aweop}
L[\bf{c}]\bf{u} = 
\left(
\begin{array}{c}
\frac{1}{\kappa}\frac{\partial p}{\partial t}  + \nabla \cdot \bv, \\
\rho\frac{\partial \bv}{\partial t} + \nabla p.
\end{array}
\right) 
\end{equation}
That is,
\begin{equation}
  \label{eqn:awemat}
  L[{\bf c}] = \left(
    \begin{array}{cc}
      \frac{1}{\kappa}\frac{\partial}{\partial t} & \nabla \cdot \\
      \nabla & \rho \frac{\partial}{\partial t}
    \end{array}
  \right)
\end{equation}
$L[{\bf c}]$ has a well-defined inverse if it is restricted to either
causal or anti-causal vector wavefields.

Most of what follows is valid for any space dimension $n >0$. The
coefficient vector $\bf{c}=(\kappa,\rho)$ is defined throughout space
$\bR^n$, the state vector $\bu$ throughout space-time
$\bR^{n+1}$. Whenever convenient for mathematical manipulations,
$n=3$: for instance, I will write $\bx=(x,y,z)^T$ for the spatial
coordinate vector, and refer to the third (vertical) coordinate of
particle velocity as $v_z$. Examples
later in this paper will use $n=2$ for computational convenience.

Since all of the operators in the discussion that follows depend on
the coefficient vector 
$\bf{c}$, I will suppress it from the notation, for example, $L=L[\bf{c}]$.

The surface source extension replaces point sources on or near a
surface in $\bR^3$ with source functions confined to the same
surface. The simplest example of this extended geometry specifies a
plane $\{(x,y,z,t): z=z_s\}$ at source depth $z_s$ as the surface. For
acoustic modeling, surface sources are combinations of constitutive law
defects and loads normal to the surface, localized on $z=z_s$. That
is, right-hand sides in the system $L\bu=\bff$ take the form
$\bff(\bx,t) = (h_s(x,y,t)\delta(z-z_s),
f_s(x,y,t)\bf{e}_z\delta(z-z_s))^T$ for scalar defect $h_s$ and normal
force $f_s$ ($\bf{e}_z=(0,0,1)$). With the choice $L$ given in
\ref{eqn:awemat}, the causal/anti-causal wave system $L\bu^{\pm}=\bff$
takes the form
\begin{eqnarray}
\label{eqn:awepm}
\frac{1}{\kappa}\frac{\partial p^{\pm}}{\partial t} & = & - \nabla \cdot \bv^{\pm} +
h_s \delta(z-z_s), \nonumber \\
\rho\frac{\partial \bv^{\pm}}{\partial t} & = & - \nabla p^{\pm} +
                                                f_s{\bf e} \delta(z-z_s),\nonumber \\
p^{\pm} & =& 0 \mbox{ for } \pm t \ll 0,\nonumber\\ 
\bv^{\pm} & = & 0 \mbox{ for } \pm t \ll 0.
\end{eqnarray}

\noindent {\bf Remark:} In system \ref{eqn:awepm} and many similar
systems to follow, I will use the shorthand
\[
  p^+ = 0 \mbox{ for } t \ll 0 
\]
to mean that $p^+$ is {\em causal}, that is,
\[
  \mbox{For some } T \in \bR, p^+(\cdot,t) = 0 \mbox{ for all } t <
  T.
\]
Similarly,
\[
  p^- = 0 \mbox{ for } t \gg 0 
\]
signifies that $p^-$ is anti-causal.

Extended forward modeling consists in solving \ref{eqn:awepm} and
sampling the solution components at receiver locations. For
simplicity, throughout this paper I will assume that the receivers are
located on another spatial hyperplane $\{(x,y,z,t): z=z_r\}$ at
receiver depth $z_r>z_s$. The constructions to follow involve interchange
of the roles of $z_s$ and $z_r$ (that is, locating sources on $z=z_r$
and receivers at $z=z_s$), so rather than the sampling operator $P$ of
the introduction, I will denote by $P_s,P_r$ the sampling 
operators on $z=z_s$, $z=z_r$ respectively. In practice, sampling
occurs at a discrete array of points (trace locations) on these
surfaces, and over a zone of finite extent. In this theoretical
discussion, I will neglect both finite sample rate and extent, and
regard the data, for example $P_rp^+$, as continuously sampled and
extending over the entire plane $z=z_r$.

A technical difficulty with spatial sampling must be
addressed. Acoustic field energy is defined in terms of the mean
squares of the pressure and particle velocity fields. However fields
with finite mean square do not in general have well-defined
restrictions to lower-dimensional sets: for example, the pressure
field of a finite-energy acoustic vector field may not have a
well-defined restriction to the space-time plane $z=z_r$. This
phenomenon is related to the ill-posedness of wave equations as
evolution equations in spatial variables, an observation attributed to
Hadamard (see \cite{CourHil:62}, Chapter 6, section 17, also
\cite{Payn:75,Symes:83}). Some constraint on the acoustic field,
beyond finite energy, is mandatory in formulating the inverse problems
\ref{eqn:esi} and \ref{eqn:esis}. In fact, the natural constraint in
the ``crosswell'' geometry of this paper is that high-frequency energy
{\em not} travel along rays parallel or nearly parallel to the
surfaces $z=z_s, z=z_r$. I will call fields with this property {\em
  downgoing} (even though the concept also encompasses {\em upcoming}
fields). \cite{BaoSy:91b} give a mathematically complete discussion of
downgoing wave field properties.

For downgoing solutions of system \ref{eqn:awepm}, the key
components ($p^{\pm}$ and $v^{\pm}_z$) are continuous functions of $z$
in the open slab $z_s<z<z_r$ with well-defined limits at the boundary
planes, but may be discontinuous at the source plane
$z=z_s$. Similarly, the roles of $z_s$ and $z_r$ will be interchanged
in some of the constructions to come, and the corresponding solutions
may be discontinuous at $z=z_r$. Accordingly, interpret $P_s$, $P_r$
as the limit from right and left respectively: for $u=p^{\pm}$ or
$v^{\pm}_z$,
\begin{eqnarray}
  \label{eqn:defsamp}
  P_su(x,y,t) &=& \lim_{z \rightarrow z_s^+} u(x,y,z,t),\nonumber \\
  P_ru(x,y,t) &=& \lim_{z \rightarrow z_r^-} u(x,y,z,t).                  
\end{eqnarray}


The causal/anti-causal vector
modeling operators ${\cal S}^{\pm}_{z_s,z_r}$ are defined in terms of
the solutions $(p^{\pm},\bv^{\pm})$ of the systems \ref{eqn:awepm} by
\begin{equation}
  {\cal S}^{\pm}_{z_s,z_r}(h_s,f_s)^T  = (P_rp^{\pm},P_r v_z^{\pm})^T,
  \label{eqn:fwd}
\end{equation}
The subscript signifies that sources are located on $z=z_s$, the
receivers on $z=z_r$. It is necessary to include this information in
the notation, as versions of ${\cal S}^{\pm}$ with sources and receivers in
several locations will be needed in the discussion below.

\noindent {\bf Remark:} To connect with the formulation presented in
the introduction, note that for continuous $u$,
$P_su(x,y,t)=u(x,y,z_s,t)$, and therefore the adjoint of $P_s$ (in the
sense of distributions) is $P_s^Th(x,y,z,t) =
h(x,y,t)\delta(z-z_s)$. Write ${\cal P}_s = \mbox{diag }(P_s,P_s)$ and
similarly for ${\cal P}_r$. Then
\[
  {\cal S}^{+}_{z_s,z_r} = {\cal P}_r L^{-1}({\cal P}_s)^T,
\]
in which $L^{-1}$ is interpreted in the causal sense, and similarly
for ${\cal S}^{-}$. Sources confined to $z=z_s$ are precisely those
functions (distributions, really) output by ${\cal P}_s^T$, so the
problem statements \ref{eqn:esi} and \ref{eqn:esis} can be rewritten
in terms of ${\cal S}^+_{z_s,z_r}$, with $P$ identified with ${\cal P}_r$.

${\cal S}^{\pm}$ is not stably invertible: its columns are
approximately linearly dependent, as will be verified below. The
diagonal components of ${\cal S}^{\pm}$ thus carry essentially all of
its information, and it is in terms of these that a sensible inverse problem
is defined.

Denote by $\Pi_i, i=0,1$ the projection on the first,
respectively second, component of a vector in $\bR^2$. The 
forward modeling operator from pressure source to pressure trace is
\begin{equation}
  \label{eqn:sdef}
  S^{\pm}_{z_s,z_r} = \Pi_0 {\cal S}^{\pm}_{z_s,z_r} \Pi_0^T 
\end{equation}
and the forward modeling operator from velocity source (normal force)
to velocity trace is
\begin{equation}
  \label{eqn:vdef}
  V^{\pm}_{z_s,z_r} = \Pi_1 {\cal S}^{\pm}_{z_s,z_r} \Pi_1^T 
\end{equation}

With these conventions, we can write the version of the source
subproblem \ref{eqn:esis} studied in this paper as
\begin{equation}
  \label{eqn:esisp}
  \mbox{find }h_s\mbox{ to minimize }\|S^{+}_{z_s,z_r}h_s- d\|^2 +
  \alpha^2\|Ah_s\|^2.
\end{equation}

\section{2D Examples}

\inputdir{project}

To illustrate the structure described in the preceding section, I
introduce two 2D acoustic models, one spatially homogeneous, the other
with highly refractive. The first, homogenous model has $\kappa = 4$
GPa and $\rho = 1$ g/cm$^3$ throughout a rectangular domain of size 8 km ($x$) $\times$ 4 km
($z$). The second, refractive, model is a perturbation of the first by
a low-velocity acoustic lens positioned in the center of the
rectangle. To produce this structure, the density is chosen
homogeneous as in the first model, while the bulk modulus decreases to
from 4 GPa outside the lens to 1.6 GPA in its center, as shown in Figure \ref{fig:bml0}.

Discretization is conventional, with a rectangular grid and staggered
finite difference scheme \cite[]{Vir:84} of order 2 in time and
2$k$ in space; for most of the experiments reported below, $k=4$.
Sampling operators such as $P_r$ are implemented via linear
interpolation, and source insertion via adjoint linear interpolation
(as noted above, in the continuum limit, sources are represented via
adjoint sampling). Steps in $x$ and $z$ are the same. Most results use
$\Delta x = 20$ m, and limit the temporal frequency of the computed
traces to 12.5 Hz. We also explore the dependence of a few results on frequency, using
$\Delta x = 10$ m and $5$ m, to accomodate 25 and 50 Hz respectively,
maintaining 8 samples per wavelength. All computations are carried out
in single precision.

\cite{GeoPros:11} gives a description of the code
implementation, out-of-date in a few respects but overall accurate.

The horizontal line of receivers sits at depth $z_r = $ 1000 m, as
shown in Figure \ref{fig:bml0}. Receiver $x$ ranges from $2000$ to
$6000$ m. A single point (physical) source appears in these
experiments, located at $x_s=3500$ m, $z_s=3000$ m, as also indicated
in the figure. The source time function is a bandpass filter with
corner frequencies $1, 2.5, 7.5, 12.5$ Hz for the lowest frequency
source, and scaled as appropriate for examples with higher frequency
and finer sampling.  Extended sources are confined to the horizontal
line through the physical source position, that is $z_s = 3000$ m,
over a 4 km interval starting at $x_r=$ 2000 m. Note that we have
reversed the order relation between $z_s$ and $z_r$ described in the
text ($z_s<z_r$). This difference is immaterial.

The point source pressure data generated by this configuration for the
homogeneous model is displayed in Figure \ref{fig:recphh0}, for the
lens model in \ref{fig:recplh0}. Triplication of arrivals is evident in the latter.



While inversion of pressure data is the main object of this exercise,
normal velocity ($v_z$) data will play an important role, so I display
the corresponding gathers in Figures \ref{fig:recvzhh0} and \ref{fig:recvzlh0}.


\section{Adjoints}

It follows from the adjoint state method (see Appendix A for details) that
\begin{equation}
  \label{eqn:sadj1}
  ({\cal S}^{\pm}_{z_s, z_r})^T = -{\cal S}^{\mp}_{z_r,z_s}
\end{equation}

Define $R$ to be the {\em time-reversal operator} on functions of
space-time, $Rf(\bx,t) = f(\bx,-t)$, and ${\cal R}$ to be the {\em
  acoustic field time-reversal operator}
\begin{equation}
  \label{eqn:trdef}
  {\cal R} \left(
    \begin{array}{c}
      p\\
      \bv
    \end{array}
  \right) =
  \left(
    \begin{array}{c}
      Rp\\
      -R\bv
    \end{array}
  \right)
\end{equation}
Then 
\begin{equation}
  \label{eqn:trsadj}
  {\cal R}{\cal S}^{\mp} = -{\cal S}^{\pm}_{z_r,z_s}{\cal R}
\end{equation}
Since $R^2 = I$ and ${\cal R}^2 = I$, the identities \ref{eqn:sadj} and \ref{eqn:trsadj} imply that

\begin{equation} 
  \label{eqn:trtr}
 ({\cal S}^{\pm}_{z_s,z_r})^T = {\cal R}{\cal S}_{z_r,z_s}^{\pm}{\cal R}=
 -{\cal S}^{\mp}_{z_r,z_s}.
\end{equation}

The relation \ref{eqn:trtr} implies that
\begin{eqnarray}
  (S^{\pm}_{z_s,z_r})^T &=& -S^{\mp}_{z_r,z_s} \nonumber\\
                        &=& R S^{\pm}_{z_r,z_s}R, \nonumber\\
    (V^{\pm}_{z_s,z_r})^T &=& -V^{\mp}_{z_r,z_s} \nonumber\\
                        &=& R V^{\pm}_{z_r,z_s}R.
                            \label{eqn:trtrcomp}
\end{eqnarray}

The implemented code uses the discrete adjoint state method and auto-generated code \cite[]{TapenadeRef13}, to
assure that the computed adjoint operators are adjoint at the level of
machine precision to the computed operators. The reverse-time storage
issue is resolved through the optimal checkpointing technique
\cite[]{Griewank:book,Symes:06a-pub}, again without loss of precision.

Typical results with pseudorandom input traces $d_r, h_z, w_r, f_s$ are:
\begin{itemize}
\item computed $\langle d_r, S^+_{z_s,z_r}h_s\rangle = -2.41069174,
  \langle (S^+_{z_s,z_r})^Td_r, h_s \rangle = -2.41069388.$
\item  computed $\langle w_r, V^+_{z_s,z_r}f_s\rangle = -2.73362470,
  \langle (V^+_{z_s,z_r})^Tw_r, f_s \rangle = -2.73362136.$
\end{itemize}
Ideally, the differences of these inner products should be at most a
relatively small multiple of machine precision, {\em relative} to the
products $\|d_r\|\| S^+_{z_s,z_r}\|\|h_s\|$ and
$\|w_r\|\|V^+_{z_s,z_r}\|\|f_s\|$ (division by these quantities makes the result
dimensionless and scale-independent). The operator norm $\| S^+_{z_s,z_r}\|$ is
computationally inaccessible, so instead I used the smaller quantities
$\|d_r\|\| S^+_{z_s,z_r}h_s\|$ etc. as stand-ins - thereby
overestimating the relative error between the inner products. In all
cases, over a very large number of random inputs, the largest observed
relative error estimate was $O(10^{-9})$, well under the appropriate limit for
single precision.

\section{Pressure-to-Source}

Since the system \ref{eqn:awepm} has a unique solution by standard
theory \cite[]{Lax:PDENotes}, the source vector field $(h_s,f_s)$
determines the acoustic field $(p^{\pm},\bv^{\pm})$ in space time, and
in particular the limits from the right at $z=z_s$, $P_sp^{\pm}$ and
$P_sv_z^{\pm}$. This relation is not invertible: it is not possible to
prescribe both pressure and normal velocity on a surface such as
$z=z_s$. So the columns of the matrix operator
${\cal S}^{\pm}_{z_s,z_r}$ must satisfy a linear relation. In this
section I will explain this relation; it involves the {\em
  pressure-to-source} map. This operator also turns out to be the
principle component of a preconditioning strategy for iterative
solution of the optimization problem \ref{eqn:esis}, so I will devote
some effort to its proper definition. It is closely related to the
Dirichlet-to-Neumann operator mentioned in the introduction.

While it is not possible to prescribe both pressure and velocity on
$z=z_s$ in solutions of \ref{eqn:awepm}, it is possible to
prescribe pressure only, for instance: if the function $\phi$ on
the surface $z=z_s$ satisfies suitable conditions, for
example the downgoing constraint mentioned earlier, a unique solution
exists for the acoustic system in both half-spaces $\pm z > z_s$:
\begin{eqnarray}
\label{eqn:awe0}
  \frac{1}{\kappa}\frac{\partial p_{\pm}}{\partial t} & = & - \nabla \cdot \bv_{\pm}, \nonumber \\
  \rho\frac{\partial \bv_{\pm}}{\partial t} & = & - \nabla
                                                    p_{\pm}, \nonumber \\
  p_{\pm} & =& 0,  \mbox{ for } t \ll 0, \nonumber\\ 
  \bv_{\pm} & = & 0 \mbox{ for } t \ll 0, \nonumber\\
  \lim_{z \rightarrow z_s^{\pm}}p_{\pm}(x,y,t,z)& =& \phi(x,y,t).
\end{eqnarray}
Note that the subscript $\pm$ here refers to the sign of $z-z_s$, as opposed
to the superscript ${\pm}$, which refers to the sign of $t$ throughout
this paper.

From the boundary condition (last equation in \ref{eqn:awe0}), one
sees that the pressures $p_{\pm}$ in the two half-spaces have the same
limit at the boundary $z=z_s$. Stick the two half-space
solutions together to form an acoustic field $(p^+,\bv^+)$ in all of
space-time, that is,
\begin{equation}
  \label{eqn:awealt}
  p^+(x,y,z,t) =
  \left\{
    \begin{array}{c}
      p_+(x,y,z,t) \mbox{ if } z>0,\\
      p_-(x,y,z,t) \mbox{ if } z<0,
    \end{array}
  \right.
\end{equation}
and a similar definition for $\bv^+$. Then $p^+$ is continuous across
$z=z_s$, and the boundary condition in system \ref{eqn:awe0} may be
written as $P_sp^+=\phi$.

The same construction can be carried out in the anti-causal sense,
with anti-causal half-space solutions glued together to form a
full-space distribution solution $(p^-,\bv^-)$, with the property that
$p^-$ is continuous across $z=z_s$ and $P_sp^-=\phi$.

The reader may object that the notation $(p^\pm,\bv^\pm)$ is already in
use, for the solution of \ref{eqn:awepm}. This objection is
valid. However, {\em in the sense
  of distributions}, $(p^{\pm},\bv^{\pm})$ as defined in display
\ref{eqn:awealt}, is {\em exactly} the causal solution of \ref{eqn:awepm}
for the choice $h_s = -[v^{\pm}_{z}]|_{z=z_s}, f_s=0$, as follows from a
simple integration-by-parts calculation. So the notation is consistent!

The negative jump $-[v^{\pm}_{z}]|_{z=z_s}$ is thus a function of $\phi$. Define
the {\em pressure-to-source} operator $\Lambda^{\pm}_{z_s}$ by
\begin{equation}
  \label{eqn:deflam}
  \Lambda^{\pm}_{z_s}\phi = -[v^{\pm}_{z}]|_{z=z_s}
\end{equation}
The conclusion: if $h_s = \Lambda^{\pm}_{z_s}\phi$ and $f_s=0$ in the
system \ref{eqn:awepm}, then $\phi=P_sp^{\pm}$.

Otherwise put, $S^{\pm}_{z_s,z_s}\Lambda^{\pm}_{z_s} \phi = \phi$, so
$\Lambda^{\pm}_{z_s}$ is inverse to $S^{\pm}_{z_s,z_s}$. The relation
\ref{eqn:trtrcomp} implies in turn that
\begin{equation}
  \label{eqn:lamadj}
  (\Lambda^{\pm}_{z_s})^T = - \Lambda^{\mp}_{z_s}
\end{equation}

There is also a {\em velocity-to-source} operator. For the solution
$(p^{\pm},\bv^{\pm})$ of system \ref{eqn:awepm} with $h_s=0$, the
normal component of velocity, $v^{\pm}_z$, is continuous across
$z=z_s$, and the velocity source (vertical load)
$f_s=-[p^{\pm}]_{z=z_s}$. I will not name the velocity-to-source
operator, as it does not appear explicitly in the developments to
follow. As will be seen, it is essentially the inverse of the
pressure-to-source operator.

The quadratic form defined by $\Lambda^{\pm}_{z_s}$ has fundamental
physical significance. Define the total acoustic energy $E^{\pm}(t)$ of the
field $(p^{\pm},\bv^{\pm})$, at time $t$ by
\begin{equation}
  \label{eqn:defae0}
  E^{\pm}(t) = \frac{1}{2} \int \,d\bx \, \left(\frac{(p^{\pm})^2}{\kappa} + \rho |\bv^{\pm}|^2\right)
\end{equation}
Then
\begin{equation}
  \label{eqn:elim}
  \pm \lim_{\pm t \rightarrow \infty} E^{\pm}(t) =  \langle P_sp^{\pm},
  (\Lambda^{\pm}_{z_s} P_sp^{\pm}) \rangle_{L^2(z=z_s)}.
\end{equation}
That is, the value of the quadratic form defined by
$\Lambda^{\pm}_{z_s}$, evaluated at the pressure trace on $z=z_s$,
gives the total energy transferred from the source to the
acoustic field over time. Since $E$ is itself a positive definite
quadratic form in the acoustic field, it follows that $\pm
\Lambda^{\pm}_{z_s}$ is positive semi-definite. 

While $\Lambda^{\pm}_{z_s}$ is positive semi-definite, it is not
symmetric. However, it is {\em approximately symmetric} in the
high-frequency sense. This fact follows from a
geometric optics analysis of the half-space solution,
assuming (as always) downgoing data:
\begin{equation}
  \label{eqn:lamappsim}
  (\Lambda^{\pm}_{z_s})^T \approx \Lambda^{\pm}_{z_s}.
\end{equation}

The analysis also reveals that the solution components not continuous
at $z=z_s$ are odd there:
\begin{equation}
  \label{eqn:odd1}
  \lim_{z\rightarrow z_s^+} v^{\pm}_{z} \approx - \lim_{z\rightarrow z_s^-}
  v^{\pm}_{z}
\end{equation}
for the solution of \ref{eqn:awepm} with $f_s=0$.
Similarly, 
\begin{equation}
  \label{eqn:odd2}
  \lim_{z\rightarrow z_s^+} p^{\pm}\approx - \lim_{z\rightarrow z_s^-}
  p^{\pm}
\end{equation}
for the solution of \ref{eqn:awepm} with $h_s=0$. Here ``$\approx$''
means in the sense of high frequency asymptotics, that is, that the
difference between the two sides is relatively smooth, hence small if
the data is highly oscillatory. Therefore if $f_s=0$ in system \ref{eqn:awepm},
\begin{equation}
  h_s = \Lambda^{\pm}_{z_s}P_sp^{\pm} = -[v^{\pm}_{z}]|_{z=z_s} \approx -2
  P_sv^{\pm}_{z}
  \label{eqn:tracejump10}
\end{equation}
Similarly, if $h_s=0$ in system \ref{eqn:awepm}, then
\begin{equation}
  \label{eqn:tracejump20}
  f_s = -[p^{\pm}]|_{z=z_s} \approx -2 P_s p^{\pm}.
\end{equation}
Thus $f_s$ determines approximately the boundary value of $p^{\pm}$,
as a solution of the acoustic wave system in the half-space
$z>z_s$. However, as repeated in equation \ref{eqn:tracejump10}, a
solution with this boundary value is also the restriction to $z>z_s$
of a solution to \ref{eqn:awepm} with $f_s=0$ and $h_s=
\Lambda^{\pm}_{z_s}P_sp^{\pm}$. Therefore if
\begin{equation}
  \label{eqn:hfcondn}
  h_s =-\frac{1}{2}\Lambda^{\pm}_{z_s}f_s,
\end{equation}
then the pressure boundary value $P_sp^{\pm}$ is the
same for the solutions of \ref{eqn:awepm} for source vectors $(h_s,0)$
and $(0,f_s)$. Since the pressure boundary values are the same, the solutions
in $z>z_s$ are the same. In particular, since $z_r>z_s$ and ${\cal
  S}^{\pm}_{z_s,z_r}(h_s,f_s)^T = (P_rp^{\pm},P_rv^{\pm}_z)^T$, it follows
that
\begin{equation}
  \label{eqn:snull}
  {\cal S}^{\pm}_{z_s,z_r}\left(\frac{1}{2}\Lambda^{\pm}_{z_s}f_s,f_s\right)^T \approx 0.
\end{equation}

Equation \ref{eqn:snull} states the relation between the columns of $
{\cal S}^{\pm}_{z_s,z_r}$ mentioned in the introduction to this
section.

To give an example of \ref{eqn:snull} in action, it is necessary to
create downgoing data and the action of the operator $\Lambda^{\pm}$
on a downdoing field data $f_s$ on $z=z_s$. Note that a point source
on $z=z_s$ creates high frequency energy traveling on rays parallel
and nearly parallel to $z=z_s$, so that won't do. One simple approach
is to create pressure and normal velocity data on $z=z_s$ by placing a
point source at a depth $z_d<z_s$. Since the examples used here are
homogeneous in $z<z_s$, the traces extracted from the resulting
causal pressure and velocity data on $z=z_s$ are {\em a priori}
downgoing, hence related by the operator $\Lambda^+_{z_s}$.

Since the
mechanical parameters in the homogeneous and lens models are the same
for $z<z_s$, and no rays return to this zone in either model, these
data are asymptotically the same for both models, and I show only the
homogenous medium results. Regard these gathers as the trace
$(P_sp^+,P_sv^+_z)$ on $z=z_s$ of a downgoing acoustic field in $z>z_s$.
Equations \ref{eqn:tracejump10} and
\ref{eqn:tracejump20} show that these differ by a factor of -2 from
source functions $f_s$ and $h_s$ as in the system \ref{eqn:awepm},
with $h_s=0$ and $f_s=0$ respectively. For a point source at $z_d=3500$ m, $x_d=3500$ m, same bandpass filter
wavelet as used for prior examples, causal data on $z=z_s=3000$ m are
shown in Figures \ref{fig:dsrcphh0} and \ref{fig:dsrcvzhh0}. 

These source functions satisfy
the relation \ref{eqn:hfcondn}, therefore source vectors $(h_s,0)$ and
$(0, f_s)$ generate {\em the same acoustic field} $(p^+,\bv^+)$ in $z>z_s$. Figures
\ref{fig:drecphh0}, \ref{fig:dfwdphh0}, \ref{fig:daltphh0} (homogeneous model) and
\ref{fig:drecplh0}, \ref{fig:dfwdplh0}, \ref{fig:daltplh0} (lens model) show the pressure
gathers extracted at $z_r=1000$ m for the point source at $z=z_d$ and
for the two choices of extended source at $z=z_s$, on the same color
scale. The obvious similarity between the fields generated by the two
extended sources,
predicted by equation \ref{eqn:snull}, is confirmed by trace
comparisons in figures \ref{fig:drecphh0tr21}-\ref{fig:daltphh0tr81}
and
\ref{fig:drecplh0tr21}-\ref{fig:daltplh0tr81}.


\section{Time Reversal}

Recall that the source vector $(h_s,f_s)$ is assumed to produce a
downgoing field $(p^+,\bv^+)$, that is, emanates high-frequency energy only along
rays that make an angle with the vertical bounded below by a common
minimum angle. Such rays leave $\Omega$ within a common maximum
time. Consequently (Appendix B), in the
slab $z_s<z<z_r$, the field $(p^+,\bv^+)$ approximates the solution of an
anti-causal evolution equation. Choose $\chi(t)$ to be a smooth function
that is $= 0$ for $t \gg 0$ and $=1$ at times when near rays carrying
high-frequency energy in $(p^+,\bv^+)$ cross $z=z_r$. Define 
$(\tilde{p}^-,\tilde{\bv}^-)$ to be the solution in the half-space
$\Omega \times \bR$ of
\begin{eqnarray}
\label{eqn:revawe}
  \frac{1}{\kappa}\frac{\partial \tilde{p}^-}{\partial t} & = & - \nabla \cdot \tilde{\bv}^-, \nonumber \\
  \rho\frac{\partial \tilde{\bv}^-}{\partial t} & = & - \nabla \tilde{p}^-,\nonumber \\
  \tilde{p}^- & =& 0,  \mbox{ for } t \gg 0\\ 
  \tilde{\bv}^- & = & 0 \mbox{ for } t \gg 0\\
  P_r\tilde{p}^- &=& \chi P_rp^+ . 
\end{eqnarray}
That is, $\tilde{p}^-$ has the same boundary value on $z=z_r$ as
$p^+$, except for low-frequency residue that is muted by
$\chi$. Therefore
$p^+ \approx \tilde{p}^-, \bv^+ \approx \tilde{\bv}^-$ near
$z=z_r$. Since the right-hand sides in the system \ref{eqn:awepm} are
singular only on $z=z_s$, and the high-frequency components of
$(p^+,\bv^+)$ are carried by downgoing rays, these differ negligibly
from the the high-frequency components of
$(\tilde{p}^-,\tilde{\bv}^-)$ in the space-time slab $z_s<z<z_r$, and
the approximation holds throughout this region. In particular
$P_sv^+_z \approx P_s \tilde{v}^-_z$. In view of the relation
\ref{eqn:tracejump10},
\begin{equation}
  \label{eqn:tildevtohsubs}
  -2P_s\tilde{v}^-_z \approx h_s,
\end{equation}
so solution
of the system \ref{eqn:revawe} followed by restriction to $z=z_s$ and
multiplication by $-2$ 
approximately inverts the map $S^+_{z_s,z_r}: h_s \mapsto P_rp^+$.

Next observe that in view of the relation \ref{eqn:tracejump20}, and
the downgoing nature of the ray system carrying the high frequency
energy in $(p^+,\bv^+)$, the field $(\tilde{p}^-,\tilde{\bv}^-)$ is
actually the restriction to $z<z_r$ of the anti-causal solution of \ref{eqn:awepm}
with $z_s$ replaced by $z_r$, zero constitutive defect, and vertical
load given by the jump in pressure at $z=z_r$ - for this field, use
the same notation. Continuity of vertical
velocity $\tilde{v}^-_z$ at $z=z_r$ implies that the vertical load is
\[
  f_r = -[\tilde{p}^-]|_{z=z_r} =-(\lim_{z\rightarrow
    z_r^+}\tilde{p}^- - \lim_{z\rightarrow
    z_r^-}\tilde{p}^-)
\]
\[
  \approx 2 P_r \tilde{p}^- = 2 P_r p^+
\]
(from the definition \ref{eqn:defsamp}, $P_r$ is the limit from the
left). Thus
\[
  P_s \tilde{v^-_z} \approx V^-_{z_r,z_s}(2 P_rp^+) \approx
  2V^-_{z_r,z_s}S^+_{z_r,z_s}h_s.
\]
so
\[
  h_s \approx -2 P_s v^+_z \approx -2 P_s \tilde{v}^-_z \approx
  -4V^-_{z_r,z_s}S^+_{z_r,z_s}h_s
\]
Combine this observation with \ref{eqn:tildevtohsubs} to obtain
\[
 -4  V^-_{z_r,z_s} S^+_{z_s,z_r}  \approx  I,
\]
This relation combines with the identity \ref{eqn:trtrcomp} to
yield the first main result of this section:
\begin{eqnarray}
  \label{eqn:approxinv}
  (V^+_{z_s,z_r})^T S^+_{z_s,z_r} & \approx & \frac{1}{4}I, \nonumber\\
  (S^+_{z_s,z_r})^T V^+_{z_s,z_r} & \approx & \frac{1}{4}I, \nonumber\\
  V^+_{z_s,z_r} (S^+_{z_s,z_r})^T & \approx & \frac{1}{4}I, \nonumber\\
  S^+_{z_s,z_r} (V^+_{z_s,z_r})^T & \approx & \frac{1}{4}I.
\end{eqnarray}.
The second equation is simply the transpose of the first, and the
last two follow by by an exactly analogous argument using time
reversal and interchange of the roles of $z_s$ and$z_r$.

The conclusion is significant enough to merit restating in English:
provided that high-frequency energy in the various fields is carried
along downgoing ray fields, the transpose of $V^+$ is an approximate
inverse to $S^+$, modulo a factor of 4. To recover the pressure source
$h_s$ generating a pressure gather $P_rp$ at $z=z_r$, multiply the
latter by -2, then apply the transpose of $V^+_{z_s,z_r}$ to this
gather, reading out a vertical velocity field at $z=z_s$. Multiply
again by -2 and you have a high-frequency approximation to $h_s$.

I have applied this procedure to the pressure gathers shown in Figures
\ref{fig:drecphh0} and \ref{fig:drecplh0}, generated by a point source
at $z_d=3500$ m, $x_s=3500$ m, propagating in the homogeneous and lens
models respectively. The source time dependence is again a [1, 2.5,
10, 12.5] Hz trapezoidal bandpass filter. As noted earlier, in the
region $z < z_s=3000 m$, these fields are downgoing and can be
regarded as the result of either pressure or velocity source at $z=z_s$: the
pressure source gather $h_s$ (Figure \ref{fig:dhshh0})  is -2 times the
vertical velocity gather depicted in Figure \ref{fig:dsrcvzhh0}, the
velocity source gather $f_s$ (Figure \ref{fig:dfshh0}) is -2 times the
pressure gather depicted in Figure \ref{fig:dsrcphh0}.

Figure \ref{fig:dinvhshh0} shows the approximate inversion (via the
first equation in display \ref{eqn:approxinv}) of the
pressure gather shown in Figure \ref{fig:drecphh0}, inverted in the homogeneous model
used to generate the data. The result is a scaled, dip-filtered
version of the pressure source shown in Figure \ref{fig:dhshh0}.
The dip filter effect is unavoidable and is caused by the aperture
limitation of the acquisition geometry: the steeper dips in the source
gather (Figure \ref{fig:dhshh0}) do not contribute to the data, and are not
recovered in the inversion.

To see that this is indeed an inversion, I apply $S^{+}_{z_s,z_r}$ to
the source shown in Figure \ref{fig:dinvhshh0}, to produce the
re-simulation shown in Figure \ref{fig:drerecphh0}. Simulation is carried
out in the homogeneous model used in the inversion. Comparison with the
gather in Figure \ref{fig:drecphh0} via a difference plot
\ref{fig:ddiffrecphh0} verifies that a very close match has been
obtained, so the source in Figure \ref{fig:dinvhshh0} is indeed a
satisfactory inversion, even it differs considerably from the original
point source: the difference lies in an approximate null space of the modeling
operator, arising from the limited aperture of the acquisition
geometry.

The remainder of this section is devoted to illustrating two important
features of the approximate inversion developed above. The
first is the validity of the inversion approach given in display
\ref{eqn:approxinv} so long as the fields generated by the
model used in simulation and inversion are downgoing. The models
may be quite far from homogeneous, and even generate triplications,
whereas the inversions remain accurate modulo the dip filtering effect
of aperture. Figure \ref{fig:dinvhsll0} shows the pressure source at $z=z_s$
inverted from the data Figure
\ref{fig:drecplh0}, in which a triplication is obvious. Recall that
this data is generated by propagating a point source at $z_d=3500$ m
in the lens model. The inversion uses the same model. Figure
\ref{fig:dinvhsll0} is an aperture-limited version of the pressure
source at $z=z_s=3000$ m
(Figure \ref{fig:dhshh0}) that generates this data by a dip filter, in
the same way that the sources in Figures \ref{fig:dhshh0} and
\ref{fig:dinvhshh0} are related. The inverted source is an accurate
inversion: it generates the data depicted in Figure
\ref{fig:drerecpll0}, which is very close to the input data (Figure
\ref{fig:drecplh0}). The difference is displayed on the same color
scale in Figure \ref{fig:ddiffrecpll0}.

The second feature is a decoupling of the models used (implicitly or
otherwise) in simulation and those used in inversion: it is possible
to {\em accurately invert pressure data for pressure source using the
  wrong model}, in the sense that the inverted source will
generate an accurate recovery of the input data re-simulated in the
same (wrong) model. For example, inversion of the data of the previous
example (Figure \ref{fig:drecplh0}) {\em in the homogeneous model}
results in an inverted source gather at $z=z_s$ (Figure \ref{fig:dinvhslh0}
that is very far from a dip-filtered version of the source used
generate the data (Figure \ref{fig:dhshh0}). However, this result is an
accurate inversion: re-simulation {\em using the same (homogeneous)
  model as used in the inversion} results in accurate recovery of the
input pressure gather (Figure \ref{fig:drecplh0}). The difference is
shown on the same color scale in Figure \ref{fig:ddiffrecplh0}.

\section{Unitarity}

The next chapter in this story recognizes the relations in display
\ref{eqn:approxinv} as asserting the approximate unitarity of
$S^+_{z_s,z_r}$.

The matrix identity \ref{eqn:snull} implies a relation between $S, V,$
and $\Lambda$ of some interest in itself. After minor re-arrangement, the second row of reads
\begin{equation}
  \label{eqn:snull2}
-\frac{1}{2}\Pi_1{\cal S}^{\pm}_{z_s,z_r}\Pi_0^T\Lambda^{\pm}_{z_s}  \approx
V^{\pm}_{z_s,z_r}.
\end{equation}
In these relations, the projection on the left picks out the vertical velocity component
of a downgoing wavefield at $z=z_r$: that is,
\[
-\frac{1}{2}\Pi_1{\cal S}^{\pm}_{z_s,z_r}\Pi_0^T\Lambda^{\pm}_{z_s}P_sp^+
=-\frac{1}{2}P_r v_z^+,
\]
where $(p^+, \bv^+)$ solve the system \ref{eqn:awepm} with $f_s=0$ and
$h_s = \Lambda^{\pm}_{z_s}P_sp^+$. On the other hand, from relation
\ref{eqn:tracejump10},
\[
  P_r v_z^+ = -\frac{1}{2}\Lambda^+_{z_r}P_r p^+
\]
where
\[
  P_r p^+ = \Pi_0{\cal S}^{+}_{z_s,z_r}\Pi_0^T\Lambda^{+}_{z_s}P_s
  p^+
\]
\[
  = S^+_{z_s,z_r}\Lambda^{+}_{z_s}P_sp^+
\]
Therefore combining the last two equations with \ref{eqn:snull2},
obtain
\begin{equation}
  \label{eqn:sv}
  \frac{1}{4}\Lambda^+_{z_r}S^+_{z_s,z_r}\Lambda^{+}_{z_s} = V^+_{z_s,z_r}.
\end{equation}
This is the promised relation.

A minor extension of an earlier example, based on the gathers
displayed in Figures \ref{fig:dsrcphh0} ($P_sp^+$) and
\ref{fig:dsrcvzhh0} ($P_sv_z^+$) illustrates the identity
\ref{eqn:sv}.
The pressure gathers at $z=z_r$ produced by application of
$S^+_{z_r,z_s}$ to $h_s=-2P_s v^+_z$, and $\Pi_0{\cal S}^+_{z_s,z_r}\Pi_1^T$ to $f_s
=-2P_sp^+$, have already been displayed (Figures \ref{fig:dfwdphh0}
and \ref{fig:daltphh0}), and are identical to several digits. The same 
is true of the vertical velocity gathers, shown in \ref{fig:dfwdvzhh0}
and \ref{fig:daltvzhh0}; their difference, plotted on the same scale,
appears in Figure \ref{fig:dsvcomphh0}. Once again, this relation
expresses the interchangeability of pressure and velocity sources,
related by $\Lambda$.

Relation \ref{eqn:tracejump10} characterizes
$\Lambda^+_{z_s}$ as connecting the pressure and velocity components
of downgoing fields restricted to $z=z_s$. That is, the pressure
source gather
$h_s=\Lambda^{+}_{z_s}P_s p^+ = -2
P_{z_s}v_z$, displayed in Figure \ref{fig:dhshh0}, is the image of the
pressure gather in Figure \ref{fig:dsrcphh0} under
$\Lambda^{+}_{z_s}$.
The pressure gather
\ref{fig:dfwdphh0} is the image of this pressure source gather under
$S^+_{z_s,z_r}$. The corresponding vertical velocity gather
(Figure \ref{fig:dfwdvzhh0}) is $-1/2$ times $\Lambda^+_{z_r}P_r
p$. Therefore scaling the data in Figure \ref{fig:dfwdvzhh0} by
$-2$ produces
$\Lambda^+_{z_r}S^+_{z_s,z_r}\Lambda^{+}_{z_s}P_s p^+$. On the other
hand, figure \ref{fig:daltvzhh0} shows the result of applying
$V^+_{z_s,z_r}$ to $f_s=-2P_s p^+$. Therefore scaling the gather in
Figure \ref{fig:daltvzhh0} by $-\frac{1}{2}$ produces $V^+_{z_s,z_r}P_zp^+$.
Since the data in Figures \ref{fig:dfwdvzhh0} and \ref{fig:daltvzhh0}
are essentially identical, the relation \ref{eqn:sv} holds for this
example.

As shown in the last section, $4(V_{z_s,z_r}^+)^T$ is approximately
inverse to $S^{+}_{z_s,z_r}$. Therefore, transposing both sides of
equation \ref{eqn:sv} and using \ref{eqn:approxinv}, obtain
\begin{equation}
  \label{eqn:almostunitary}
  4(V_{z_s,z_r}^+)^TS^+_{z_s,z_r} = [ (\Lambda^+_{z_s})^T
  (S^{+}_{z_s,z_r})^T(\Lambda^+_{z_r})^T]S^{+}_{z_s,z_r} \approx I.
\end{equation}

The remarkable feature of the identity \ref{eqn:almostunitary} is that
it exhibits an approximate right inverse of $S^+$ as an adjoint with
respect to a weighted inner product - or it would, if the operators
$(\Lambda^+)$ were symmetric positive definite. As noted earlier,
these operators are only approximately symmetric, though they are
positive semi-definite. That is not a great obstacle, however:
symmetrizing them in the obvious way commits a negligible error, of
the sort that this paper already neglects wholesale. That is,
\begin{equation}
  \label{eqn:unitary}
  [ \frac{1}{2}((\Lambda^+_{z_s})^T+ \Lambda^+_{z_s})
  (S^{+}_{z_s,z_r})^T \frac{1}{2}((\Lambda^+_{z_r})^T+
  \Lambda^+_{z_r})]S^{+}_{z_s,z_r} \approx I.
\end{equation}

To illustrate this unitary property of $S^+_{z_s,z_r}$, I apply the
operator
\[
  \frac{1}{2}((\Lambda^+_{z_s})^T+ \Lambda^+_{z_s})
  (S^{+}_{z_s,z_r})^T \frac{1}{2}((\Lambda^+_{z_r})^T+
  \Lambda^+_{z_r})
\]
to the data $S^+_{z_s,z_r}h_s$ (Figure \ref{fig:drerecphh0}), in which
$h_s$ is the dip-filtered version (Figure \ref{fig:dinvhshh0}) of the
downgoing source created earlier (Figure \ref{fig:dhshh0}), produced
by inversion of the data in Figure \ref{fig:drecphh0} via the first
equation in display \ref{eqn:approxinv}, propagation in the homogenous
model.  The operator above is computed via the technique explained in
the section ``Computing and Symmetrizing $\Lambda$'', below, using
auxiliary receiver arrays 100 m above the data source and receiver arrays.

The output is shown in Figure
\ref{fig:lamsstlamrdrecphh0}. The difference with the actual source is
shown in Figure \ref{fig:difflamsstlamrdrecphh0}. A similar exercise
using the lens data \ref{fig:drecplh0} but inversion in the homogenous
model produces the extended source depicted in
\ref{fig:lamsstlamrdrecplh0}, which differs from the inversion result
via equation \ref{eqn:approxinv} by the gather displayed in Figure
\ref{fig:difflamsstlamrdrecplh0}.

The symmetrized $\Lambda$ operators are at least positive
semi-definite, hence define (at least) semi-norms.
Similar relations have been derived for other scattering operators,
and have been used to accelerate iterative solutions of inverse
scatering problems: \cite{DafniSymes:SEG18b} review some of this
literature.

\section{Accelerated Iterative Inversion}

For convenience, in this section write $S$ in place of
$S^+_{z_s,z_r}$. Also abbreviate the symmetrized $\Lambda$ operators
using notation suggesting weight
operators in model and data spaces:
\begin{eqnarray}
  W_m^{-1}&=& \frac{1}{2}((\Lambda^+_{z_s})^T+
              \Lambda^+_{z_s}),\nonumber \\
  W_d &=& \frac{1}{2}((\Lambda^+_{z_r})^T+ \Lambda^+_{z_r}).
          \label{eqn:wdef}
\end{eqnarray}
The identification of the symmetrized $\Lambda^+_{z_s}$ as the inverse
of another operator $W_m$ is formal, since the former operator is
likely to have null (or nearly-null) vectors due to aperture-related
amplitude loss. Since some version of $W_m$ is essential in the
formulation for effective preconditioning, I will derive a usable
candidate to stand in for it below.

Adopting Hilbert norms defined by the operators $W_m$ and $W_d$ in its
domain and range respectively, the adjoint of $S$ is given by
\begin{equation}
\label{eqn:wadj}
S^{\dagger} = W_m^{-1}S^TW_d,
\end{equation}

In this notation, the relation \ref{eqn:unitary} takes the form
\begin{equation}
  \label{eqn:wunitary}
  S^{\dagger}S \approx I.
\end{equation}
That is to say, $S$ is approximately unitary with respect to the
weighted norms defined by $W_m$ and $W_d$. Therefore a Krylov space
method employing these norms will converge rapidly, at least for the
well-determined components of the solution.

Similar relations have been derived for other scattering operators,  
and have been used to accelerate iterative solutions of inverse  
scattering problems: \cite{DafniSymes:SEG18b} review some of this  
literature. 
 
The most convenient arrangement the Conjugate Gradient (CG) algorithm
taking advantage of the structure \ref{eqn:wadj} is the {\em
  Preconditioned CG}. 

Allowing that the fit error will be measured by the data space norm,
the least squares problem to be solved is not just $Sh \approx d$, but
a regularized version:
\begin{equation}
  \label{eqn:einv}
  \mbox{minimize}_h \|Sh-d\|^2_d + \alpha^2 \|Ah\|^2_m
\end{equation}
Remark: recall that the modified data space norm $\|d\|_d^2 = \langle
d, W_d d\rangle$ has physical meaning: for acoustics, it is
proportional to the power transmitted to the fluid by the source.

The minimizer of the objective defined in equation \ref{eqn:einv}
solves the normal equation
\begin{equation}
  \label{eqn:norm0}
  (S^{\dagger}S + \alpha^2 A^{\dagger}A)h = S^{\dagger}d 
\end{equation}
where the weighted adjoint $S^{\dagger}$ has already been defined in equation \ref{eqn:wadj}, and $A^{\dagger}$ is the adjoint of $A$ in the weighted model space norm defined by $W_m$, namely
\begin{equation}
  \label{eqn:aadj}
  A^{\dagger} = W_m^{-1}A^TW_m.
\end{equation}

Note that the normal operator appearing on the left-hand side of
\ref{eqn:norm0} is not an approximate identity, due to the presence of
the regularization term: the spectrum increases in spread with
increasing $\alpha$, leading to slower convergence. Fortunately for
the present setting, the operators $W_m^{-1}$, $A$, and $W_m$
approximately commute (they are scalar {\em pseudodifferential}, once
the difficulties with the definition of $W_m$, mentioned above, are
taken care of), so that $A^{\dagger} \approx A^T$. Therefore
\begin{equation}
  \label{eqn:normapprox}
  S^{\dagger}S + \alpha^2 A^{\dagger}A \approx I + \alpha^2A^TA
\end{equation}
Recall that $A$ is simply multiplication by the Euclidean distance to
the physical source point $\bx_s$: $A u (\bx) = |\bx-\bx_s|u(\bx),
A^TAu(\bx) = |\bx-\bx_s|^2u(\bx)$. So the equation $(I+\alpha^2
A^TA)u=b$ is trivial to solve, and this is a key characteristic of a
good preconditioner. However this observation must be combined with
the weighted norm structure.

Rewrite the normal equation \ref{eqn:norm0} as
\begin{equation}
  \label{eqn:norm1}
  W_m^{-1}(S^TW_dS + \alpha^2 A^TW_mA)h = W_m^{-1}S^TW_md 
\end{equation}
Since $W_m$ is self-adjoint and positive definite, the common factor on both sides of \ref{eqn:norm1} can be re-written as
\begin{equation}
  \label{eqn:normpart}
  Nh = (S^*S + \alpha^2 A^*A)h = S^*d 
\end{equation}
in which $S^*, A^*$ are the adjoints with the original (Euclidean)
inner product in the domains but the weighted inner product in data
space:
\begin{eqnarray}
  \label{eqn:sadj}
  S^* &=& S^T W_d,\\
  A^* &=& A^T W_m.
\end{eqnarray}
Note the $S^*S$ and $A^*A$ are symmetric in the Euclidean sense, so
equation \ref{eqn:normpart} is a symmetric positive (semi-)definite
linear system, just the sort of thing for which the 
The Preconditioned Conjugate Gradient (``PCG'') algorithm was
designed. PCG for solution
of equation \ref{eqn:normpart} with preconditioner $M$is usually
written as Algorithm 1 (see for example \cite{Golub:2012}):

\begin{algorithm}[H]
\caption{Preconditioned Conjugate Gradient Algorithm, Standard Version}
\begin{algorithmic}[1]
\State Choose $h_0=0$ 
  \State $r_0 \gets S^*d$
  \State $p_0 \gets M^{-1} r_0$
  \State $g_0 \gets p_0$
  \State $q_0 \gets Np_0$
  \State $k \gets 0$
  \Repeat
  \State $\alpha_k \gets \frac{\langle g_k,r_k \rangle}{\langle p_k,q_k\rangle}$
  \State $h_{k+1} \gets h_k + \alpha_k p_k$
  \State $r_{k+1} \gets r_k - \alpha_kq_k$
  \State $g_{k+1} \gets M^{-1} r_{k+1}$
  \State $\beta_{k+1} \gets \frac{\langle g_{k+1},r_{k+1}\rangle}{\langle g_k,r_k\rangle}$
  \State $p_{k+1}\gets g_{k+1}+\beta_{k+1}p_k$
  \State $q_{k+1} \gets Np_{k+1}$
  \State $k \gets k+1$
  \Until{Error is sufficiently small, or max iteration count exceeded} 
\end{algorithmic}
\end{algorithm}
The iteration converges rapidly if $M^{-1}N \approx I$. This is true
if and only if the symmetrized operator $M^{-1/2}NM^{-1/2} \approx I$,
which is in turn true if the eigenvalues of $M^{-1/2}NM^{-1/2}$ are
close to 1 (actually works well is most of these eigenvalues are close
to 1, and the rest are small - which is the case for the current
problem)..  Further, PCG is computationally effective is M is easy to
invert.

From \ref{eqn:normapprox} and \ref{eqn:norm1}, it follows that
\[
  W_m^{-1}(S^TW_dS + \alpha^2 A^TW_mA) \approx I + \alpha^2 A^TA.
\]
This observation suggests using $M=W_m(I+\alpha^2A^TA)$. This choice
is not symmetric, but since the operators on the right-hand side are
scalar pseudodifferential hence commute, it is equivalent to use of
\begin{eqnarray}
  M         &=&(I+\alpha^2A^TA)^{1/2}W_m(I+\alpha^2A^TA)^{1/2},\nonumber \\
  M^{-1}
            &=&(I+\alpha^2A^TA)^{-1/2}W_m^{-1}(I+\alpha^2A^TA)^{-1/2}.
                \label{eqn:defprecond}
\end{eqnarray}
With this choice, \ref{eqn:normapprox} implies that
$M^{-1}N \approx I$, also $M$ is symmetric. As already mentioned,
powers of $I + \alpha^2A^TA$ are trivial to compute, given the choice
of $A$ made here. We will examine fast algorithms for computing
$W_m^{-1}$ = the symmetrized pressure-to-source operator in the next
section. Note that only $M^{-1}$, hence only $W_m^{-1}$, appears in
Algorithm 1.

[Example]

\section{Computing and Symmetrizing $\Lambda$}

Computations of $\Lambda^{\pm}_{z_s}$ and its transpose are clearly critical steps in an
implementation of the PCG algorithm outlined in the preceding section.
Direct computation of the pressure-to-source operator $\Lambda^{\pm}_{z_s}$, for instance by solving
\ref{eqn:awepm} and reading off $P_sv^{\pm}_{z}$, turns out to be
numerically ill-behaved. The relation \ref{eqn:snull} provides and
alternative approach, taking advantage of the accurate approximate
inverse to $S^+_{z_s,z_r}$ constructed above. The first row of
\ref{eqn:snull}, slightly rearranged, is
\begin{equation}
  \label{eqn:lamidea}
  \Pi_0{\cal S}^+_{z_s,z_r}\Pi_1^T f_s \approx  -\frac{1}{2} S^+_{z_s,z_r}\Lambda^+_{z_s}f_s.
\end{equation}
The approximate inverse construction for $S^++_{z_s,z_r}$ permits (approximate)
solution of this equation for $\Lambda^+_{z_s}f_s$: apply
$4(V^{+}_{z_s,z_r})^T$ to both sides of equation \ref{eqn:lamidea} and
use the first equation in the list \ref{eqn:approxinv} to get
\begin{equation}
  \label{eqn:lamident}
  \Lambda^+_{z_s} \approx -8(V^{+}_{z_s,z_r})^T\Pi_0 {\cal S}^+_{z_s,z_r}\Pi_1^T.
\end{equation}
This identity is the major result of this section: it shows how
to compute that action of $\Lambda^+_{z_s}$ by propagating the input
pressure trace, identified as a source for the velocity evolution,
forward in time from $z_s$ to $z_r$
reading off the pressure trace on $z=z_r$, identifying it once more as
a point load (source for velocity), propagating it backwards in time from $z_r$ to
$z_s$, and finally reading off the velocity trace, interpreted as a
pressure evolution source on $z_s$. 

The importance of this result lies in the failure of the obvious
method for computing the action of $\Lambda^{\pm}_{z_s}$, namely to
employ the pressure trace as a source in the velocity equation ($f_s$,
in the notation used above) at $z=z_s$, and read off the velocity
field also at $z=z_s$. This difficulty is related to the existence of
tangentially propagating waves and the lack of continuity of the trace
operator. The method implicit in equation \ref{eqn:lamident} avoids
this difficulty by propagating the fields a positive distance in $z$:
assuming as always that the causal fields are downgoing, this step
eliminates any tangentially propagating fields from consideration.

Figure \ref{fig:preddinvhshh0} shows the pressure source gather
produced from the pressure data gather in Figure \ref{fig:dsrcphh0} by
application of the operator on the right-hand side of formula
\ref{eqn:lamident} to the source surface pressure gather depicted in
\ref{fig:dsrcphh0}. The homogeneous model is used in all propagations
implicit in the prescription \ref{eqn:lamident}. The scale is the same
as for Figures \ref{fig:dinvhshh0} and \ref{fig:dhshh0}. In fact
Figure \ref{fig:preddinvhshh0} appears to be nearly identical to
Figure \ref{fig:dinvhshh0}, and both appear to be dip-filtered
versions of Figure \ref{fig:dhshh0}. The difference plot (Figure
  \ref{fig:ddiffinvhshh0}), using the same color scale, shows that the
  similarity is quantitative. The recovered source gather (Figure
  \ref{fig:preddinvhshh0}) also generates the same acoustic fields as
  the point source at $z=z_d$: the pressure gather at $z=z_r=1000$ m
  depth is shown in Figure \ref{fig:dpredhsrecphh0}, and its
  difference with the point source gather (Figure \ref{fig:drecphh0})
  plotted on the same color scale in Figure
  \ref{fig:ddiffpredhsrecphh0}.

Figures \ref{fig:preddinvhsll0}, \ref{fig:ddiffinvhsll0},
\ref{fig:dpredhsrecpll0}, and \ref{fig:ddiffpredhsrecpll0} accomplish
the same comparison for the $\Lambda^+_{z_s}$ approximation in
equation \ref{eqn:lamident}, with all propagation taking place in the
lens model.

A deeper study of the pressure-to-source
operator (Appendix B) shows that it is approximately dependent only on the model
coefficients near the source surface ($z=z_s$ in this case). Since the
homogeneous and lens models are identical near this surface, it is
unsurprising that these figures are very close to the previous two.
However an even more useful observation is that the calculations in
the approximation \ref{eqn:lamident} could just as well be carried out
in a much smaller region around the source surface, and produce a
result that is functionally identical in that it will serve as a
source for the same acoustic fields globally, with small error. In
effect, equation \ref{eqn:lamident} involving propagation from source
($z=z_s$) to receiver ($z=z_r$) surfaces is altered by replacing $z_r$
with a receiver datum $z_s+\Delta z$ considerably closers to $z_s$:
\begin{equation}
  \label{eqn:lamnear}
  \Lambda^+_{z_s} \approx -8(V^{+}_{z_s,z_s+\Delta z})^T\Pi_0 {\cal
    S}^+_{z_s,z_s+\Delta z}\Pi_1^T.
\end{equation}
Using a receiver datum closer to the source surface has two favorable consequences:
\begin{itemize}
\item The computational domain can be smaller than is necessary to
  simulate the target data, as it need only contain the source
  surface and the receiver datum implicit in
  equation \ref{eqn:lamident}. This shrinkage of the computational
  domain can lead to substantial improvements in computational
  efficiency.
\item Since the receiver data may be chosen much closer to the
  source surface that is the case for the target data, the effective
  aperture active in the relation \ref{eqn:lamident} can be much
  larger, producing an estimated source gather much less affected by
  aperture limitation.
\end{itemize}

To illustrate these points, I have re-computed the source gather
estimates shown in Figure \ref{fig:preddinvhshh0}. In place of the receiver geometry in the
target data (Figure \ref{fig:drecphh0}), I
place a receiver array at $z_s+\Delta z=2900$ m depth, just 100 m above the
source surface at $z_s=3000$ m. For the discretization used to create
the examples shown so far, that is just a 5 gridpoint difference in
depth, as opposed to 100 gridpoints between the source and receiver
depths for examples such as shown in Figure \ref{fig:recphh0} or
\ref{fig:drecphh0}.

Note that the homogeneous and lens models are identical in the depth
range $2900 < z < 3000$ m, so the computed $\Lambda^+_{z_s}$ operators
will be the same for both models. Hence I show only results for the
homogenous model.

The approximation to $\Lambda_{z_s}^+$ via equation \ref{eqn:lamident}
for this configuration is evaluated in the same way as before in
Figures \ref{fig:preddnshshh0}, \ref{fig:ddiffnshshh0},
\ref{fig:dprednshsrecphh0}, and \ref{fig:ddiffprednshsrecphh0}. The effect of aperture
limitation is clearly diminished: the second figure in this series
compares the full-aperture pressure source gather (Figure
\ref{fig:dhshh0}) with the 
image of the corresponding pressure gather (Figure \ref{fig:dsrcphh0})
under the approximation to $\Lambda_{z_s}^+$, and the last figures
show that the approximated source gathers accurately predict the
point-source pressure gather at the receiver datum $z_r=1000$ m.

As mentioned in the last section, computation of the transpose of
$\Lambda^+$ (exact, not approximate in the high frequency sense) is
critical to the successful construction of the preconditioner. The
relation \ref{eqn:lamident} does not provide a computation for this
operator. However set
\begin{equation}
  \label{eqn:lamtilde}
  \tilde{\Lambda}^+_{z_s} = -8(V^{+}_{z_s,z_s+\Delta z})^T\Pi_0 {\cal
    S}^+_{z_s,z_s+\Delta z}\Pi_1^T.
\end{equation}
Then \ref{eqn:lamnear} can be rewritten
\[
  \Lambda^+_{z_s} \approx \tilde{\Lambda}^+_{z_s}.
\]
Of course, all of the examples so far show images of
$\tilde{\Lambda}^+_{z_s}$.

Since successful preconditioning requires only approximate inversion,
use of $\tilde{\Lambda}^+_{z_s} $ in place of $\Lambda^+_{z_s}$ will
still yield a working preconditioner, and the former can be transposed
to machine precision via the definition \ref{eqn:lamtilde} and the adjoint state
method (equations \ref{eqn:trtr} \ref{eqn:trtrcomp}):
\begin{equation}
  \label{eqn:lamtransp}
  (\tilde{\Lambda}^+_{z_s})^T = -8 \Pi_1 ({\cal S}^+_{z_s,z_s+\Delta
    z})^T\Pi_0^T V^{+}_{z_s,z_s+\Delta z}
\end{equation}

Figure \ref{fig:preddnshstrhh0} shows the image of the pressure gather
in Figure \ref{fig:dsrcphh0} under $(\tilde{\Lambda}^+_{z_s})^T$,
using the ``near'' traces at $z=2900$, that is, $\Delta z = 100$ m in
expression \ref{eqn:lamtransp}, and propagation in the
homogeneous model. Note
the close resemblance to the image of the same pressure gather under
$\tilde{\Lambda}^+_{z_s}$ displayed in Figure
\ref{fig:preddnshshh0}. The difference of these two images is
displayed in \ref{fig:ddiffnslamtrhh0}, on the same color scale as the
images themselves. Since the propagation takes place entirely in a
region where all of the mechanical parameters are homogenous, I do not
offer a similar comparison for the lens model.

The model space weight operator $W_m$ introduced in the last section is
replaced by its asymptotic approximation
\[
  \tilde{W}_m = \frac{1}{2}(\tilde{\Lambda}^+_{z_s} +
  (\tilde{\Lambda}^+_{z_s})^T)
\]
\[
  = -8 \left((V^{+}_{z_s,z_s+\Delta z})^T\Pi_0 {\cal
    S}^+_{z_s,z_s+\Delta z}\Pi_1^T \right.+
\]
\[
  \left. \Pi_1 ({\cal S}^+_{z_s,z_s+\Delta
      z})^T\Pi_0^T V^{+}_{z_s,z_s+\Delta z}\right)
\]
\begin{equation}
  \label{eqn:wcomp}
  = -4(\Pi_1 ({\cal S}^+_{z_s,z_s+\Delta z})^T(\Pi_0^T\Pi_1 + \Pi_1^T\Pi_0) {\cal S}^+_{z_s,z_s+\Delta z}\Pi_1^T)
\end{equation}
with a similar definition for the replacement $\tilde{W}_d$ of $W_d$.

This identity shows that only one forward and one adjoint simulation
are necessary to compute the action of $\tilde{W}_{m,d}$. The operator
in the center of the expression on the right-hand side, $\Pi_0^T\Pi_1
+ \Pi_1^T\Pi_0$, simply exchanges the components of the acoustic
fields, passing the velocity field as a pressure source and the
pressure field as a velocity source.

Figure \ref{fig:symmdnshshh0} shows the output of the symmetrized
approximate source-to-pressure operator per equation \ref{eqn:wcomp},
applied once again to the pressure data in Figure
\ref{fig:dsrcphh0}. Note the resemblance to Figures
\ref{fig:preddnshshh0} and \ref{fig:preddnshstrhh0}. These are all
asymptotic approximations of each other.

 \newpage
 
\section{Figures}


\plot{bml0}{width=\textwidth}{Bulk modulus, lens model. Color scale is 
in GPa. Positions of point source and receiver line indicated.}

\plot{recphh0}{width=\textwidth}{Output pressure gather for point source in 
  homogeneous medium, configuration as described in Figure \ref{fig:bml0} and in
  the text.}

\plot{recplh0}{width=\textwidth}{Output pressure gather for point source in 
  refractive (lens) medium, configuration as described in Figure \ref{fig:bml0} and in
  the text.}

\plot{recvzhh0}{width=\textwidth}{Output vertical velocity gather for point source in 
  homogeneous medium, configuration as described in Figure \ref{fig:bml0} and in
  the text.}

\plot{recvzlh0}{width=\textwidth}{Output vertical velocity gather for point source in 
  refractive (lens) medium, configuration as described in Figure \ref{fig:bml0} and in
  the text.}

\plot{dsrcphh0}{width=\textwidth}{Trace $P_sp^+$ on $z=z_s=3000$ m of
  pressure field from point source at $z_d=3500$ m, $x_d=3500$ m,
  bandpass filter source. }

\plot{dsrcvzhh0}{width=\textwidth}{Trace $P_sv_z^+$ on $z=z_s=3000$ m of
  vertical velocity field from point source at $z_d=3500$ m, $x_d=3500$ m,
  bandpass filter source.}

\plot{drecphh0}{width=\textwidth}{Pressure gather at receiver depth 
  $z_r=1000$ m from field generated by causal solution of acoustic 
  system \ref{eqn:awepm} in the homogeneous model described in the 
  text, with point pressure source (constitutive defect) at $z_d=3500$
  m, $x_d=3500$ m.}
  
\plot{dfwdphh0}{width=\textwidth}{Pressure gather at receiver depth
  $z_r=1000$ m from field generated by causal solution of acoustic
  system \ref{eqn:awepm} in the homogeneous model described in the
  text, with extended pressure source (constitutive defect)
  on $z=z_s=3000$ m given by $h_s=-2P_sv_z^+=\Lambda^+_{z_s}P_sp^+$,
  where $P_sp^+$ is shown in Figure \ref{fig:dsrcphh0} and $P_sv^+_z$
  in Figure \ref{fig:dsrcvzhh0}. In this simulation, the velocity
  velocity source (vertical load) vanishes: $f_s=0$.}

\plot{daltphh0}{width=\textwidth}{Pressure gather at receiver depth
  $z_r=1000$ m from field generated by causal solution of acoustic
  system \ref{eqn:awepm} in the homogeneous model described in the
  text, with extended velocity source (vertical load) $f_s=-2P_sp^+$
  on $z=z_s=3000$ m, where $P_sp^+$ is shown in Figure
  \ref{fig:dsrcphh0}.  In this simulation, the zero pressure source
  (constitutive defect vanishes: $h_s=0$.}

\plot{drecplh0}{width=\textwidth}{Pressure gather at receiver depth 
  $z_r=1000$ m from field generated by causal solution of acoustic 
  system \ref{eqn:awepm} in the lens model described in the 
  text, with point pressure source (constitutive defect) at $z_d=3500$
  m, $x_d=3500$ m.}

\plot{drecphh0tr21}{width=\textwidth}{Overplot of traces 21 ($x=2400$) 
  from gathers shown in \ref{fig:drecphh0} (blue), \ref{fig:dfwdphh0}
  (red).}

\plot{daltphh0tr21}{width=\textwidth}{Overplot of traces 21 ($x=2400$) 
  from gathers shown in \ref{fig:drecphh0} (blue), \ref{fig:daltphh0}
  (red).}

\plot{drecphh0tr51}{width=\textwidth}{Overplot of traces 51 ($x=3000$) 
  from gathers shown in \ref{fig:drecphh0} (blue), \ref{fig:dfwdphh0}
  (red).}

\plot{daltphh0tr51}{width=\textwidth}{Overplot of traces 51 ($x=3000$) 
  from gathers shown in \ref{fig:drecphh0} (blue), \ref{fig:daltphh0}
  (red).}

\plot{drecphh0tr81}{width=\textwidth}{Overplot of traces 81 ($x=3600$) 
  from gathers shown in \ref{fig:drecphh0} (blue), \ref{fig:dfwdphh0}
  (red).}

\plot{daltphh0tr81}{width=\textwidth}{Overplot of traces 81 ($x=3600$) 
  from gathers shown in \ref{fig:drecphh0} (blue), \ref{fig:daltphh0}
  (red).}

\plot{dfwdplh0}{width=\textwidth}{Pressure gather at receiver depth 
  $z_r=1000$ m from field generated by causal solution of acoustic 
  system \ref{eqn:awepm} in the lens model described in the 
  text, with extended pressure source (constitutive defect) on
  $z=z_s=3000$ m given bythe field depicted in Figure
  \ref{fig:dsrcvzhh0} scaled by -2 
   ($h_s=-2P_sv_z^+=\Lambda^+_{z_s}P_sp^+$) and zero velocity source (vertical load) 
  ($f_s=0$).}

\plot{daltplh0}{width=\textwidth}{Pressure gather at receiver depth 
  $z_r=1000$ m from field generated by causal solution of acoustic 
  system \ref{eqn:awepm} in the lens model described in the 
  text, with extended velocity source (vertical load) on $z=z_s=3000$
  m given by the field depicted in Figure \ref{fig:dsrcphh0} scaled by -2
  ($f_s=-2P_sp^+$) and zero pressure source (constitutive defect)
  ($h_s=0$).}

\plot{drecplh0tr21}{width=\textwidth}{Overplot of traces 21 ($x=2400$) 
  from gathers shown in \ref{fig:drecplh0} (blue), \ref{fig:dfwdplh0}
  (red).}

\plot{daltplh0tr21}{width=\textwidth}{Overplot of traces 21 ($x=2400$) 
  from gathers shown in \ref{fig:drecplh0} (blue), \ref{fig:daltplh0}
  (red).}

\plot{drecplh0tr51}{width=\textwidth}{Overplot of traces 51 ($x=3000$) 
  from gathers shown in \ref{fig:drecplh0} (blue), \ref{fig:dfwdplh0}
  (red).}

\plot{daltplh0tr51}{width=\textwidth}{Overplot of traces 51 ($x=3000$) 
  from gathers shown in \ref{fig:drecplh0} (blue), \ref{fig:daltplh0}
  (red).}

\plot{drecplh0tr81}{width=\textwidth}{Overplot of traces 81 ($x=3600$) 
  from gathers shown in \ref{fig:drecplh0} (blue), \ref{fig:dfwdplh0}
  (red).}

\plot{daltplh0tr81}{width=\textwidth}{Overplot of traces 81 ($x=3600$)
  from gathers shown in \ref{fig:drecplh0} (blue), \ref{fig:daltplh0}
  (red).}

\plot{dhshh0}{width=\textwidth}{Pressure source gather = -2 $\times$
  vertical velocity gather (Figure \ref{fig:dsrcvzhh0}).}

\plot{dfshh0}{width=\textwidth}{Velocity source gather = -2 $\times$
  pressure gather (Figure \ref{fig:dsrcphh0}).}

\plot{dinvhshh0}{width=\textwidth}{Approximate inversion via first 
  equation in display \ref{eqn:approxinv}. Inversion in homogenous model of 
  the pressure gather in Figure \ref{fig:drecphh0}, simulated with 
  homogeneous model. Scaled version of output $v_z$ field obtained by 
  applying transpose of $4V^+_{z_s,z_r}$. Approximates dip-filtered
  version of pressure source gather (Figure \ref{fig:dhshh0}).}

\plot{drerecphh0}{width=\textwidth}{Re-simulated pressure gather
  produced from inverted source
  shown in Figure \ref{fig:dinvhshh0}. Simulation in homogeneous model.}

\plot{ddiffrecphh0}{width=\textwidth}{Difference between gathers
  displayed in Figures \ref{fig:drecphh0} and \ref{fig:drerecphh0},
  plotted on same color scale.}

\plot{dinvhsll0}{width=\textwidth}{Approximate inversion via first 
  equation in display \ref{eqn:approxinv}. Inversion in lens model of 
  the pressure gather in Figure \ref{fig:drecplh0}, simulated with 
  lens model. Scaled version of output $v_z$ field obtained by 
  applying transpose of $4V^+_{z_s,z_r}$. Approximates dip-filtered
  version of pressure source gather (Figure \ref{fig:dhshh0}).}

\plot{drerecpll0}{width=\textwidth}{Re-simulated pressure gather
  produced from inverted source
  shown in Figure \ref{fig:dinvhsll0}. Simulation in lens model.}

\plot{ddiffrecpll0}{width=\textwidth}{Difference between gathers
  displayed in Figures \ref{fig:drecpll0} and \ref{fig:drerecpll0},
  plotted on same color scale.}

\plot{dinvhslh0}{width=\textwidth}{Approximate inversion via first 
  equation in display \ref{eqn:approxinv}. Inversion in homogenous model of 
  the pressure gather in Figure \ref{fig:drecplh0}, simulated with 
  lens model. Scaled version of output $v_z$ field obtained by 
  applying transpose of $4V^+_{z_s,z_r}$. Approximates dip-filtered
  version of pressure source gather (Figure \ref{fig:dhshh0})}

\plot{drerecplh0}{width=\textwidth}{Re-simulated pressure gather produced from
  inverted source
  shown in Figure \ref{fig:dinvhslh0}. Simulation in lens model.}

\plot{ddiffrecplh0}{width=\textwidth}{Difference between gathers
  displayed in Figures \ref{fig:drecplh0} and \ref{fig:drerecplh0},
  plotted on same color scale.}

\plot{dfwdvzhh0}{width=\textwidth}{Vertical velocity gather, generated
  with a pressure source in the homogeneous model, 
  corresponding to pressure gather \ref{fig:dfwdphh0}.}

\plot{daltvzhh0}{width=\textwidth}{Vertical velocity gather, generated
  with a velocity source in the homogenous model, corresponding to
  pressure gather \ref{fig:daltphh0}.}

\plot{dsvcomphh0}{width=\textwidth}{Difference between velocity gathers
  shown in Figures \ref{fig:dfwdvzhh0} and \ref{fig:daltvzhh0},
  plotted on the same color scale as these figures.}

\plot{lamsstlamrdrecphh0}{width=\textwidth}{Inversion of data shown in
  Figure \ref{fig:drecphh0}, simulated in homogeneous model, using the
  approximate unitarity relation \ref{eqn:unitary} and propagation in
  homogenous model.}

\plot{difflamsstlamrdrecphh0}{width=\textwidth}{Difference between
  data displayed in Figures \ref{fig:dinvhshh0} and
  \ref{fig:lamsstlamrdrecphh0}, plotted on the same color scale.}
  
\plot{lamsstlamrdrecplh0}{width=\textwidth}{Inversion of data shown in
  Figure \ref{fig:drecplh0}, simulated in lens model, using the
  approximate unitarity relation \ref{eqn:unitary} and propagation in
  homogenous model.}

\plot{difflamsstlamrdrecplh0}{width=\textwidth}{Difference between
  data displayed in Figures \ref{fig:dinvhslh0} and
  \ref{fig:lamsstlamrdrecplh0}, plotted on the same color scale.}

\plot{preddinvhshh0}{width=\textwidth}{Pressure source gather = image
  of pressure-to-source operator $\Lambda^+_{z_s}$ applied to pressure gather
  shown in Figure \ref{fig:dsrcphh0}, homogeneous model. Compare Figure
  \ref{fig:dinvhshh0}.}

\plot{ddiffinvhshh0}{width=\textwidth}{Difference between (a) image
  (Figure \ref{fig:preddinvhshh0}) of $\Lambda^+_{z_s}$ applied to
  pressure gather (Figure \ref{fig:dsrcphh0}), and (b) source gather
  (Figure \ref{fig:dinvhshh0}) inverted from receiver pressure gather
  (Figure \ref{fig:drecphh0}). Homogeneous model used in all
  propagations. Same color scale as in Figure
  \ref{fig:preddinvhshh0}. }

\plot{dpredhsrecphh0}{width=\textwidth}{Pressure gather at receiver 
  datum $z=z_r=1000$ m simulated in homogeneous model from source 
  gather shown in Figure \ref{fig:preddinvhshh0}, also based on the 
  homogeneous model. Compare with point source pressure gather (Figure 
  \ref{fig:drecphh0}).}

\clearpage

\plot{ddiffpredhsrecphh0}{width=\textwidth}{Plot of difference 
  between data shown in Figures \ref{fig:drecphh0} and 
  \ref{fig:dpredhsrecphh0}, plotted on same color scale as the latter 
  two figures.}

\plot{preddinvhsll0}{width=\textwidth}{Pressure source gather = image
  of pressure-to-source operator $\Lambda^+_{z_s}$ applied to pressure gather
  shown in Figure \ref{fig:dsrcphh0}, lens model. Compare Figure
  \ref{fig:dinvhsll0}.}

\plot{ddiffinvhsll0}{width=\textwidth}{Difference between (a) image
  (Figure \ref{fig:preddinvhsll0}) of $\Lambda^+_{z_s}$ applied to
  pressure gather (Figure \ref{fig:dsrcphh0}), and (b) source gather
  (Figure \ref{fig:dinvhsll0}) inverted from receiver pressure gather
  (Figure \ref{fig:drecplh0}). Lens model used in all propagations.
  Same color scale as in Figure \ref{fig:preddinvhsll0}.}

\plot{dpredhsrecpll0}{width=\textwidth}{Pressure gather at receiver 
  datum $z=z_r=1000$ m simulated in lens model from source 
  gather shown in Figure \ref{fig:preddinvhsll0}, also based on the 
  homogeneous model. Compare with point source pressure gather (Figure 
  \ref{fig:drecplh0}).}

\plot{ddiffpredhsrecpll0}{width=\textwidth}{Plot of difference 
  between data shown in Figures \ref{fig:drecplh0} and 
  \ref{fig:dpredhsrecpll0}, plotted on same color scale as the latter 
  two figures.}

\clearpage

\plot{preddnshshh0}{width=\textwidth}{Pressure source gather = image
  under pressure-to-source operator $\Lambda^+_{z_s}$ of pressure gather
  shown in Figure \ref{fig:dsrcphh0}, homogeneous model, using
  ``near'' receiver traces at $z=2900$ m. Compare Figure
  \ref{fig:dhshh0} and \ref{fig:dinvhshh0}: because the sources and
  receivers are close, little aperture is lost in this case.}

\plot{ddiffnshshh0}{width=\textwidth}{Difference between (a) image
  (Figure \ref{fig:preddnshshh0}) of $\Lambda^+_{z_s}$ applied to
  pressure gather (Figure \ref{fig:dsrcphh0}) using a near receiver
  array to implement formula \ref{eqn:lamident}, and (b) source gather
1650  (Figure \ref{fig:dhshh0}) inferred from vertical velocity.
  Homogeneous model used in all
  propagations. Same color scale as in Figure
  \ref{fig:preddnshshh0}. }

\plot{dprednshsrecphh0}{width=\textwidth}{Pressure gather at receiver
  datum $z=z_r=1000$ m simulated from source gather shown in Figure
  \ref{fig:preddnshshh0}. Compare with point source pressure gather
  (Figure \ref{fig:drecplh0}).}

\clearpage

\plot{ddiffprednshsrecphh0}{width=\textwidth}{Plot of difference 
  between data shown in Figures \ref{fig:drecphh0} and 
  \ref{fig:dpredhsrecphh0}, plotted on same color scale as the latter 
  two figures.}

\plot{preddnshstrhh0}{width=\textwidth}{Pressure source gather = image
  under {\em transpose} of pressure-to-source operator
  $\Lambda^+_{z_s}$ of pressure gather shown in Figure
  \ref{fig:dsrcphh0}, homogeneous model, using ``near'' receiver
  traces at $z=2900$ m. Compare Figure \ref{fig:dhshh0} and
  \ref{fig:preddnshshh0}: as noted in the text, $\Lambda^+_{z_s}$ is
  asymptotically symmetric, so the resemblance is not a surprise.}

\plot{ddiffnslamtrhh0}{width=\textwidth}{Difference between data in
  Figures \ref{fig:preddnshshh0} and \ref{fig:preddnshstrhh0}, plotted
  on the same scale as these figures, showing that the asymptotic
  symmetry of $\Lambda^+_{z_s}$ is actually quantitative for the
  length, time and frequency scales of these examples.}

\plot{symmdnshshh0}{width=\textwidth}{Pressure source gather = image
    under {\em symmetrized} pressure-to-source operator
    $\frac{1}{2}\left(\Lambda^+_{z_s}+(\Lambda^+_{z_s})^T\right)$ of
    pressure gather shown in Figure \ref{fig:dsrcphh0}, homogeneous
    model, using ``near'' receiver traces at $z=2900$ m. Compare
    Figure \ref{fig:preddnshshh0}.}

\plot{compreshh0}{width=\textwidth}{Comparison of residual Euclidean 
  norms: CG (blue), PCG (red), plotted vs. iteration. Data =
  homogeneous model, point source (Figure \ref{fig:recphh0}), inversion 
  in homogeneous model.}

\plot{compreslh0}{width=\textwidth}{Comparison of residual Euclidean 
  norms: CG (blue), PCG (red), plotted vs. iteration. Data =
  lens model, point source (Figure \ref{fig:recplh0}), inversion 
  in homogenous model.}

\plot{compresll0}{width=\textwidth}{Comparison of residual Euclidean 
  norms: CG (blue), PCG (red), plotted vs. iteration. Data =
  lens model, point source (Figure \ref{fig:recplh0}), inversion 
  in lens model.}

\plot{compnreshh0}{width=\textwidth}{Comparison of normal residual
  (gradient) Euclidean norms: CG (blue), PCG (red), plotted
  vs. iteration. Data = homogeneous model, point source (Figure
  \ref{fig:recphh0}), inversion in homogenous model.}

\plot{compnreslh0}{width=\textwidth}{Comparison of normal residual 
  (gradient) Euclidean norms: CG (blue), PCG (red), plotted 
  vs. iteration. Data = lens model, point source (Figure 
  \ref{fig:recplh0}), inversion in homogenous model.}

\plot{compnresll0}{width=\textwidth}{Comparison of normal residual 
  (gradient) Euclidean norms: CG (blue), PCG (red), plotted 
  vs. iteration. Data = lens model, point source (Figure 
  \ref{fig:recplh0}), inversion in lens model.}

\newpage

\append{Adjoint Computation}

The adjoint of ${\cal S}^+_{z_s,z_r}$ can be computed by a variant of
the adjoint state method, in this case a by-product of the
conservation of energy. Suppose that $p^-,\bv^-$ solve \ref{eqn:awepm}
with $(h_s,f_s\bf{e}_z)\delta(z-z_s)$ replaced by
$ (h_r,f_r\bf{e}_z) \delta(z-z_r)$. Then
\[
0 = 
\left(\int\, dx\,dy\,dz\, \frac{p^+ p^-}{\kappa} +  
\rho \bv^+ \cdot \bv^- \right)|_{t \rightarrow \infty}
-
\left(\int\, dx\,dy\,dz\, \frac{p^+ p^-}{\kappa} +  \rho \bv^+ \cdot \bv^- \right)|_{t \rightarrow -\infty}
\]
\[
= 
\int_{-\infty}^{\infty} \,dt\, \frac{d}{dt}\left(\int\, dx\,dy\,dz\, \frac{p^+ p^-}{\kappa} +  \rho \bv^+ \cdot \bv^- \right)
\]
\[
= 
\int_{-\infty}^{\infty} \,dt\, \left(\int\, dx\,dy\,dz\, \frac{1}{\kappa} \frac{\partial p^+}{\partial t} p^- +  p^+ \frac{1}{\kappa}\frac{\partial p^-}{\partial t} \right.
\]
\[
+
\left. \rho \frac{\partial \bv^+}{\partial t} \cdot \bv^- + \rho \bv^+ \cdot \frac{\partial \bv^-}{\partial t} \right)
\]
\[
= 
\int_{-\infty}^{\infty} \,dt\, \left(\int\, dx\,dy\,dz\, \left(- \nabla \cdot \bv^+ + 
 h_s \delta(z-z_s)\right) p^- + p^+ \left(- \nabla \cdot \bv^- + 
 h_r \delta(z-z_r)\right) \right.
\]
\[
+
\left.  (- \nabla p^++f_s\bf{e}_z) \cdot \bv^- + \bv^+ \cdot (-\nabla
  p^- + f_r \bf{e_z}) \right)
\]
\[
= 
\int_{-\infty}^{\infty}\,dt\, \left(\int\, dx\,dy\,dz\, \left(- \nabla \cdot \bv^+ + 
 h_s \delta(z-z_s)\right) p^- + p^+ \left(- \nabla \cdot \bv^- + 
 h_r \delta(z-z_r)\right) \right.
\]
\[
+
\left.  p^+ (\nabla \cdot \bv^-) + (\nabla \cdot \bv^+) p^- 
  +f_s \delta(z-z_s) v_z^- + v_z^+f_r \delta(z-z_r) \right)
\]
after integration by parts in the last two terms. Most of what is left cancels, leaving 
\[
0 = \int_{-\infty}^{\infty}\,dt\,dx\,dy\, (h_sP_sp^-+f_zP_sv_z^-) +
( h_rP_rp^++f_rP_rv_z^+) = \langle (h_s,f_s), {\cal S}^-(h_r,f_r) \rangle+ \langle (h_r,f_r), {\cal S}^+_{z_s,z_r}(h_s,f_s) \rangle
\]
whence
\begin{equation}
\label{eqn:sadj}
 ({\cal S}^+_{z_s,z_r})^T = -{\cal S}^{-}_{z_r,z_s}.
\end{equation}

The systems \ref{eqn:awepm} for $(p^{\pm},\bv^{\pm})$ with the two
signs of superscript differ only in the direction of time evolution (causal vs. anti-causal). Define $R$ to be the {\em time-reversal operator} on functions of space-time,
$Rf(\bx,t) = f(\bx,-t)$, and ${\cal R}$ to be the {\em acoustic field time-reversal operator} 
\begin{equation}
  \label{eqn:trdef}
  {\cal R} \left(
    \begin{array}{c}
      p\\
      \bv
    \end{array}
  \right) =
  \left(
    \begin{array}{c}
      Rp\\
      -R\bv
    \end{array}
  \right)
\end{equation}
Then ${\cal R}(p^-,\bv^-)$ solves \ref{eqn:awepm} with
$(h_s\delta(z-z_s),f_s\bf{e_z}\delta(z-z_s))$ replaced by
$(-Rh_r\delta(z-z_r), Rf_r\bf{e_z}\delta(z-z_r))$. That is,
\begin{equation}
  \label{eqn:trsadj}
  {\cal R}{\cal S}^- = -{\cal S}^+_{z_r,z_s}{\cal R}
\end{equation}
Since $R^2 = I$ and ${\cal R}^2 = I$, the identities \ref{eqn:sadj}
and \ref{eqn:trsadj} imply equation \ref{eqn:trtr}.


\bibliographystyle{seg}
\bibliography{../../bib/masterref}


