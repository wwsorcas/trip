\section{Discussion}

The source subproblem is a necessary step in solving the overall FWI problem \ref{eqn:fwi}. In particular, the nonlinear least squares realization \ref{eqn:esi} of source-extended FWI is naturally approached via the {\em variable projection method} \cite[]{GolubPereyra:73,GolubPereyra:03,vanLeeuwenMulder:09,Rickett:SEG12}, in which the minimum over $\bff$ in \ref{eqn:esis} is treated as a function of $\bf{c}$, which is in turn minimized. Variable projection thus treats the source subproblem as an inner problem, which must be solved for an estimate of $\bff$ at every step of an iterative method for the outer problem of minimization over $\bf{c}$. Efficiency in solving the source subproblem \ref{eqn:esis} is critical to efficient implementation of variable projection for solving the regularized FWI problem \ref{eqn:esi}.

This paper restricts the choice of wave physics embodied in
$L[\bf{c}]$ to linear acoustics.  Thus $\bf{c}=(\kappa,\rho)$ is the
pair (bulk modulus, density) of positive functions of spatial
position. The principal characteristics assumed of sources $\bff=$ are localization and isotropy. That is, physical sources will be assumed to be isotropic point radiators, described by the functional form
\begin{equation}
\label{eqn:ptsrc}
f_{\rm pt}(\bx,t;\bx_s) = w(t)\delta(\bx-\bx_s). 
\end{equation}
Source locations $\bx_s$ are assumed known (as is natural for active source methods), and the source {\em wavelet} $w(t)$ is to be found as part of the solution of \ref{eqn:esi}. A convenient choice of penalty operator $A$ is then multiplication by the {\em offset} $|\bx-\bx_s|$: its null space consists precisely of distributions of the form \ref{eqn:ptsrc}. 

More complex radiation characteristics may be accommodated with similar but more complex choice for the operator $A$. Since the core analysis is similar, this paper deals only with the simplest case \ref{eqn:ptsrc}.

The wavefield $u$ in this work is presumed to exist in all of Euclicean space-time: in other words, boundary effects have been removed from $u$, by surface-related multiple elimination or other techniques. Predicted data traces are simply the samples of one or more components of $u$ at receiver points $\bx_r$.

%The domain of wave propagation is denoted $\Omega$, and two subsets of its boundary are identified: $\Gamma_s$ contains the support (non-zeroes) of the (extended) sources and in particular the locations $\bx_s$ of physical sources. Data trace locations $\bx_r$ lie in another subset $\Gamma_r$. The data projection operator $P$ samples the acoustic field $u$ on $\Gamma_r$, and includes a finite aperture mute.  

Achieving a small residual in the source subproblem \ref{eqn:esis} for arbitrary $c$ is a critical part of the convergence mechanism for the extended source approach to FWI \cite[]{HuangNammourSymesDollizal:SEG19}. However, for arbitrary data $d$, coefficient array $c$, and source location $\bx_s$, there does not exist an isotropic point source $f_{\rm pt}$ of the form \ref{eqn:ptsrc} for which $PL[c]^{-1}f_{\rm pt} \approx d$. This is so even if there does exist a coefficient array $c^*$ and point source $f_{\rm pt}^*$ located at $\bx_s$ with wavelet $w^*(t)$ for which $d = PL[c^*]^{-1}f_{\rm pt}^*$, if $c$ differs substantially from $c^*$. It is this fact that underlies the effectiveness of the extended source approach to FWI:  driving the residual in the problem \ref{eqn:esi} towards zero necessarily requires modifying $c$ to resemble $c^*$, assuming the latter exists, as well as pushing the optimal $f$ towards the null space of $A$. Conversely, the source subproblem \ref{eqn:esis} does not have a small residual solution of point source form \ref{eqn:ptsrc}, in general, for a given choice of $c$, even if that were the case for a different choice ($c^*$). Therefore a more general class of source models than that specified by the condition \ref{eqn:ptsrc} must be admitted to the feasible set for the source subproblem \ref{eqn:esis} if it is to serve as the inner problem in a variable projection formulation of extended source FWI.

The class of non-point sources investigated here are those confined a hypersurface assumed to contain the locations of physical (point) sources, but not {\em a priori} required to have point support. These {\em surface extended sources} are able to yield a relatively small residual in the source subproblem \ref{eqn:esis} for more or less arbitrary coefficient array $c$. This observation is a by-product of the approximate solution for \ref{eqn:esis} constructed below. The construction of the approximate solution is closely related to the {\em time reversal} method of photoacoustic tomography (see \cite{StefanovUhlmannIP:09} and references cited there).

The next section gives a precise description of an acoustic version of problem \ref{eqn:esis} for an idealized {\em crosswell} configuration, in which source located on a surface $z=z_s$ generate waves that propagate to to a receiver surface $z=z_r > z_s$. In principle, {\em diving waves} can be described with similar mathematics, but with some additional complications, so I do not treat the diving wave case explicitly in this paper. The assumptions detailed here imply that the data $d$ of the inverse problems \ref{eqn:esi}, \ref{eqn:esis} have the physical character of {\em transmitted waves}. There are no mathematical results at present concerning reflected wave inverse problems treated by source extension methods, though there are tantalizing numerical clues \cite[]{LeeuwenHerrmannWRI:13,Warner:14,Warner:16,LeeuwenHerrmann:16,HuangSymes:Geo17,HuangSymes:Geo18a,HuangSymes:Geo18b}.

The following section describes the construction of the approximate
inverse and explains the conditions under which accuracy should be expected. I observe that the approximate inverse is approximately the adjoint of the modeling operator $PL[c]^{-1}$ with respect to weighted norms in its domain and range (see \cite{HouSymes:15} for a similar observation in a different context). This fact immediately suggests use of the approximate inverse as a preconditioner to accelerate convergence of Krylov space methods such as conjugate gradient iteration applied to the least squares problem \ref{eqn:esis}. However, its straightforward application in the preconditioned conjugate gradient (PCG) method \cite[]{Golub:2012} involves explicitly a version the so called Dirichlet-to-Neumann (D2N) operator, which is computationally awkward. I show how to reorganize the PCG iteration so that only solutions of $L[c]u=f$ are required. The penultimate section presents some 2D examples in the crosswell configuration, illustrating the accuracy of the approximate inversion and the accelerated convergence of PCG. The examples are chosen to emphasize that surface extended sources can be constructed to fit more or less arbitrary data, unlike point sources - as mentioned before, this approximate invertibility property of the modeling operator is of critical importance in the application to extended FWI. The paper ends with a discussion of several important matters not addressed here, and a restatement of the conclusions.

The intent of the main body of this paper is to present the formal structure of the extended source subproblem and the way in which this structure leads to accelerated numerical solution. This formal structure is however supported by a rigorous foundation. In order to avoid disrupting the formal account, this foundation material is relegated to an appendix.


\section{Prototype Numerical Examples}
I present a collection of simple examples that illustrate the features of the surface source extension claimed in preceding sections. 

I used the IWAVE acoustic staggered grid package to carry out these calculations. This package implements (2,2k) schemes for k=1,2,..., and outputs traces (of either velocity or pressure at any point in space via multilinear interpolation. The discretized modeling operator is thus of second order accuracy, though as usual I have used higher order in space to reduce grid dispersion.

Source injection is implemented as the adjoint of trace sampling, resulting in another second-order error [REFERENCES]. 

The data is a single shot gather, with a source at coordinates $x_s=z_s=3000$ m (units of length are meters in all cases). The receiver line occupies $1500 \le x_r \le 5500$ m, with receiver depth $z_r=1000$ m.  Extended sources occupy $1500 \le x_s \le 5500$ m, with the same depth $z_s=3000$ m as the ``physical'' source used to generate the data. This region turns out to be adequate to represent the extended sources that approximately invert the data, for the cases examined below. An algorithm to automatically identify an appropriate region can be based on the ideas developed by \cite{Fu:Geo17}.

I have used absorbing boundary conditions (split-field PML) on all four sides of the 4000 m (vertical) $\times$ 8000 m (horizontal) simulation domain. Evidently inclusion of a free surface is important to the application of the ideas explained here to diving wave marine data, and I have not addressed the necessary modifications here. 

The Dirichlet-to-Neumann operator $\Lambda$ is an essential part of the inner problem preconditioner just presented. Implementation can be accomplished in several ways:
\begin{itemize}
\item \cite{tenKroode:12} suggests using a one-way operator;
\item if both pressure and normal partical velocity are measured (or simulated), then the two are related by $\Lambda$ and the velocity component can simply be used as the output;
\item presence of a free surface implies all of the usual problems, such as the need for removal of receiver-side ghosts. On the other hand, if the free surface is within a quarter-wavelength throughout the useful bandwidth of the data, then the ghosted data differs from $\Lambda p$ by a time integration and a scale factor, a fact used to good effect by \cite{HouSymes:15}.
\end{itemize}

In the examples, I have used the second observation. With a finite difference implementation of the pressure-velocity system \ref{eqn:awe}, velocity components are available ``for free'', short-circuiting explicit computation of $\Lambda$.

The  {\tt project/SConstruct} script is set up to carry out the necessary computations on grids with spacings $\Delta x = \Delta z = $ 20, 10, and 5 m, with a jump of roughly 8 in computation time resulting from each refinement. For present purposes, the coarsest (20 m) grid seems to be sufficient, and that is the grid used in the examples presented below. The source pulses are chosen so that the computation is reasonably accurate. For the 20 m grid case, I use a zero-phase trapezoidal bandpass filter source with corner frequencies of 1.0, 2.0, 7.5, and 12.5 Hz.

The IWAVE asg driver has been set up to recognize the case {\tt deriv=0} as defining the map from source (right-hand side in the pressure equation) to data (pressure) traces. The adjoint to this map, as explained above, is reverse-time propagation of the data traces as pressure sources, followed by scaling (formula \ref{eqn:sadj}). The approximate inverse is computed by application of the Dirichlet-to-Neumann operator to the pressure traces to produce corresponding velocity traces, followed by injection as pressure sources and reverse time propagation, followed by another application of the Dirichlet-to-Neumann map and scaling (formulas \ref{eqn:appinv}, \ref{eqn:adj}).

The script implements these operatations step-by-step via calls to IWAVE, Madagascar, and SU commands. A peculiarity of the {\tt asg.x} driver needs to be mentioned: it is based on an un-scaled version of the constitutive law defect source representation, that is, {\tt asg.x} approximately computes the solution of the system \ref{eqn:awedata} with the first equation replaced by
\begin{equation}
\label{eqn:asgdata}
\frac{\partial p}{\partial t}  =  - \kappa \nabla \cdot \bv +
h \delta(z-z_s).
\end{equation}
Denote by $S_{\rm asg}[c]$ the forward map produced by {\tt asg.x}. Then comparison of \ref{eqn:awedata} and \ref{eqn:asgdata} reveals that
\begin{equation}
\label{eqn:sreln}
S[c]=S_{\rm asg}[c]\kappa
\end{equation}
where $\kappa$ is shorthand for the operator of multiplication by $\kappa$. Accordingly, and approximate inverse for $S_{\rm asg}[c]$ is
\[
I \approx S_{\rm asg}[c]^{\dagger}S_{\rm asg}[c] = S_{\rm asg}[c]^{\dagger}S[c]\kappa^{-1}
\]
Since $S[c]S[c]^{\dagger} \approx I \approx S[c]^{\dagger}S[c]$, it follows that
\[
S_{\rm asg}[c]^{\dagger}=\kappa S[c]^{\dagger} = \kappa\Lambda[c]_s S[c]^T \Lambda[c]_r 
\]
\begin{equation}
\label{eqn:asginv}
= \kappa \Lambda[c]_s \kappa S_{\rm asg}[c]^T\Lambda[c]_r
\end{equation}
This is the approximate inverse computed in the examples. Note that
only the values of $\kappa$ near the source datum $z=z_s$ play a role
in the relation \ref{eqn:sreln} or in the definition \ref{eqn:asginv}
of the IWAVE ASG approximate inverse.


\section{Fields, Traces and Sources}
In this section, $(p^{\pm}_0,\bv^{\pm}_0)$ denotes an acoustic field,
continuous in the region $z_s \le z $, obeying the homogeneous wave
equation in the interior $z_s < z $:
\begin{eqnarray}
\label{eqn:awe1}
  \frac{1}{\kappa}\frac{\partial p^{\pm}_0}{\partial t} & = & - \nabla \cdot \bv^{\pm}_0, \nonumber \\
  \rho\frac{\partial \bv^{\pm}_0}{\partial t} & = & - \nabla
                                                    p^{\pm}_0, \nonumber \\
  p^{\pm}_0 & =& 0,  \mbox{ for } t \ll 0, \nonumber\\ 
  \bv^{\pm}_0 & = & 0 \mbox{ for } t \ll 0.
\end{eqnarray}
I will show how to view $(p^{\pm}_0,\bv^{\pm}_0)$ as the $z>z_s$ part
of a solution of the system \ref{eqn:awepm} in two different ways,
corresponding to two different extensions to the whole space. An
important relation between the components of the operator
${\cal S}^{\pm}_{z_s,z_r}$ follows (equation \ref{eqn:lamnull} below).

Given a function $\phi$ on $z=z_s$, define $(p^{\pm}_1,\bv^{\pm}_1)$
as the solution in the two half-spaces $z <z_s, z>z_s$ of
\begin{eqnarray}
\label{eqn:awe1}
  \frac{1}{\kappa}\frac{\partial p_1}{\partial t} & = & - \nabla \cdot \bv_1, \nonumber \\
  \rho\frac{\partial \bv_1}{\partial t} & = & - \nabla p_1,\nonumber \\
  p^{\pm}_1 & =& 0,  \mbox{ for } \pm t \ll 0,\nonumber\\ 
  \bv^{\pm}_1 & = & 0 \mbox{ for } \pm t \ll 0,\nonumber\\
  \lim_{z \rightarrow z_s} P_sp^{\pm}_1& =& \phi.
\end{eqnarray}
By construction, $p^{\pm}_1$ is continuous across $z=z_s$. The
discontinuity $[v^{\pm}_{1,z}]|_{z=z_s}$ is a linear function of
$\phi$. Define the {\em pressure-to-source map} $\Lambda^{\pm}_{z_s}$ by
\begin{equation}
  \label{eqn:deflam}
  \Lambda^{\pm}_{z_s}\phi = [v^{\pm}_{1,z}]|_{z=z_s}
\end{equation}
Then as a distribution on all of $\bR^4$, $(p^{\pm}_1,\bv^{\pm}_1)$ solves the
inhomogeneous system
\begin{eqnarray}
\label{eqn:sawe1}
  \frac{1}{\kappa}\frac{\partial p^{\pm}_1}{\partial t} & = &
                                                        (\Lambda^{\pm}_{z_s}\phi)\delta(z-z_s) - \nabla \cdot \bv^{\pm}_1, \nonumber \\
  \rho\frac{\partial \bv^{\pm}_1}{\partial t} & = & - \nabla p^{\pm}_1,\nonumber \\
  p^{\pm}_1 & =& 0,  \mbox{ for } \pm t \ll 0,\nonumber\\ 
  \bv^{\pm}_1 & = & 0 \mbox{ for } \pm t \ll 0.
\end{eqnarray}

This is precisely the causal option of system \ref{eqn:awepm} for
$f_s=0$ and $h_s=\Lambda^{\pm}_{z_s}\phi$. From \ref{eqn:awe1} and the definition
\ref{eqn:fwd},
\begin{equation}
  \label{eqn:lamfwd}
  \phi = S^{\pm}_{z_s,z_s}\Lambda^{\pm}_{z_s}\phi,
\end{equation}
that is,
\begin{equation}
  \label{eqn:lamfwd1}
  S^{\pm}_{z_s,z_s} = (\Lambda^{\pm}_{z_s})^{-1}.
\end{equation}

Similarly, suppose $\psi$ is a function on $z=z_s$, and that
$(p^{\pm}_2,\bv^{\pm}_2)$ solves
\begin{eqnarray}
\label{eqn:awe2}
  \frac{1}{\kappa}\frac{\partial p^{\pm}_2}{\partial t} & = & - \nabla \cdot \bv^{\pm}_2, \nonumber \\
  \rho\frac{\partial \bv^{\pm}_2}{\partial t} & = & - \nabla p^{\pm}_2,\nonumber \\
  p^{\pm}_2 & =& 0,  \mbox{ for } \pm t \ll 0,\nonumber\\ 
  \bv^{\pm}_2 & = & 0 \mbox{ for } \pm t \ll 0,\nonumber\\
  \lim_{z \rightarrow z_s} P_sv^{\pm}_{2,z} &=& \psi.
\end{eqnarray}
in the two half-spaces $z<z_s, z > z_s$. In this case, $v^{\pm}_{2,z}$
is continuous at $z=z_s$ by construction. Viewed as a
distribution on all of $\bR^4$, $(p^{\pm}_2,\bv^{\pm}_2)$ solve the
system \ref{eqn:awepm} with $h_s=0$, and the discontinuity in pressure
across $z=z_s$ as the inhomogeneous term in Newton's law:
\begin{eqnarray}
\label{eqn:sawe2}
  \frac{1}{\kappa}\frac{\partial p^{\pm}_2}{\partial t} & = &
                                                        - \nabla \cdot \bv^{\pm}_2, \nonumber \\
  \rho\frac{\partial \bv^{\pm}_2}{\partial t} & = & 
                                                    [p^{\pm}_2]|_{z=z_s}{\bf e_z}\delta(z-z_s) - \nabla p^{\pm}_2,\nonumber \\
  p^{\pm}_2 & =& 0,  \mbox{ for } \pm t \ll 0,\nonumber\\ 
  \bv^{\pm}_2 & = & 0 \mbox{ for } \pm t \ll 0.
\end{eqnarray}
This relation defines a map from the velocity boundary data $\psi$ to
the source in the velocity equation, having units of force/volume,
similar to $\Lambda^{\pm}_{z_s}$. As it's not needed in the sequel, I
won't name this map.

A closer relation between traces and sources follows from the
geometric optics analysis of the half-space solution, presented
below. I will show that, for propagating wave data,
\begin{equation}
  \label{eqn:odd1}
  \lim_{z\rightarrow z_s^+} v_{1,z} \approx - \lim_{z\rightarrow z_s^-}
  v_{1,z}
\end{equation}
Similarly,
\begin{equation}
  \label{eqn:odd2}
  \lim_{z\rightarrow z_s^+} p_{2} \approx - \lim_{z\rightarrow z_s^-}
  p_{2}
\end{equation}
for the velocity field solving \ref{eqn:awe1}, or equivalently
\ref{eqn:awe1}. Here ``$\approx$'' means in the sense of high
frequency asymptotics, that is, that the difference between the two
sides is relatively smooth, hence small if the data is highly
oscillatory. Similarly, for the pressure field solving \ref{eqn:awe2}
or \ref{eqn:sawe2}, the jump at $z=z_s$ is twice the limiting value
(trace) from above, up to a smooth error. Therefore
\begin{eqnarray}
  \Lambda^{\pm}_{z_s}\phi = [v^{\pm}_{1,z}]|_{z=z_s} &\approx& 2
                              \lim_{z\rightarrow z_s}v^{\pm}_{1,z},\\  \label{eqn:tracejump1}
  [p^{\pm}_2]|_{z=z_s} &\approx& 2 \lim_{z\rightarrow z_s^+}p^{\pm}_{2},\\  \label{eqn:tracejump2}
\end{eqnarray}

%%%%%%%%%%%%%%%%%%%
These apparently harmless relations have a profound consequence,
because of a final step: identify the boundary data $\phi$ and $\psi$,
so far arbitrary, as the traces of $p^{\pm}_0$ and $v^{\pm}_{0,z}$ on $z=z_s$,
that is, set
\begin{eqnarray}
  \phi & = & P_s p^{\pm}_0, \nonumber \\
  \psi & = & P_s v^{\pm}_{0,z}.
             \label{eqn:bv0}
\end{eqnarray}
Since the traces on $z=z_s$ of the pressure fields $p^{\pm}_0$ and $p^{\pm}_1$ are
the same, and both are pressure components of solutions of the same homogeneous
acoustic system in $z>z_s$, the standard energy identity implies that
$(p^{\pm}_0,\bv^{\pm}_0) = (p^{\pm}_1,\bv^{\pm}_1)$ in
$z > z_s$. Similarly, $P_sv^{\pm}_{0,z}$ and $P_sv^{\pm}_{2,z}$ are
the same, whence $(p^{\pm}_0,\bv^{\pm}_0) = (p^{\pm}_2,\bv^{\pm}_2)$ in
$z>z_s$.

Therefore the assumption \ref{eqn:bv0} implies that
\begin{equation}
  \label{eqn:jump0}
  \Lambda^{\pm}_{z_s}\phi = [v^{\pm}_{1,z}]_{z=z_s} \approx 2P_s v^{\pm}_{0,z}  = 2\psi,
\end{equation}
and
\begin{equation}
  \label{eqn:jump1}
[p_2^{\pm}]_{z=z_s} \approx  2P_s p^{\pm}_{0}= 2 \phi.           
\end{equation}
In the notation of the last section, set $2 \phi = 2P_sp^{\pm}_0= f_s,
2\psi = 2P_s v^{\pm}_{0,z}= h_s$. Then relation \ref{eqn:jump0}
implies that
\begin{equation}
  \label{eqn:lamfh}
  \Lambda^{\pm}_{z_s}f_s \approx h_s.
\end{equation}

From the definitions in the last section and the system \ref{eqn:sawe1},
\begin{equation}
  \label{eqn:pident0}
P_r p^{\pm}_0 = \Pi_0{\cal S}^{\pm}(h_s,0)^T = \Pi_0{\cal
  S}^{\pm}\Pi_0^Th_s,
\end{equation}
and
\begin{equation}
  \label{eqn:pident1}
P_r p^{\pm}_0 \approx \Pi_0{\cal S}^{\pm}_{z_s,z_r} (0,f_s)^T = \Pi_0{\cal
  S}^{\pm}_{z_s,z_r} \Pi_1^Tf_s,
\end{equation}
Similarly, from the relations above and \ref{eqn:sawe2},
\begin{equation}
  \label{eqn:vident0}
  P_r  v^{\pm}_{0,z}  \approx \Pi_1{\cal S}^{\pm}_{z_s,z_r}(h_s,0)^T  =
  \Pi_1{\cal S}^{\pm}_{z_s,z_r} \Pi_0^Th_s 
\end{equation}
and
\begin{equation}
  \label{eqn:vident1}
  P_r  v^{\pm}_{0,z}  = \Pi_1{\cal S}^{\pm}_{z_s,z_r}(0,f_s)^T  =
  \Pi_1{\cal S}^{\pm}_{z_s,z_r} \Pi_1^Tf_s 
\end{equation}
Together with \ref{eqn:lamfh}, these equations imply that
\begin{equation}
  \label{eqn:lamnull}
{\cal  S}_{z_s,z_r}^{\pm}\Pi_0^T\Lambda^{\pm}_{z_s} \approx {\cal S}^{\pm}_{z_s,z_r}\Pi_1^T.
\end{equation}
Consequently, ${\cal S}_{z_s,z_r}$ has an ``approximate null space'',
in the sense that its composition with the block operator
$(\Lambda^{\pm}_{z_s},-I)^T$ is asymptotically negligible.

\section{Parametrix via Time Reversal}


\section{Approximate Unitarity}
Recall that the fields $(p^{\pm},\bv^{\pm})$ in $z>z_s$ can be viewed
as either the solution of the system \ref{eqn:awepm} with pressure source $h_s=0$ and
a suitable choice of velocity source $f_s$, or of the same system with
suitable choice of $h_s$ and $f_s=0$, and that these choices are
related by equation \ref{eqn:lamfh}: $\Lambda^{\pm}f_s = h_s$. The
pressure traces at $z=z_r>z_s$ are the same: from the definitions
\ref{eqn:sdef}, \ref{eqn:vdef},
\begin{eqnarray}
  \label{eqn:sandv}
  S^{\pm}_{z_s,z_r}h_s& =& P_rp^{\pm},\nonumber \\
  V^{\pm}_{z_s,z_r}f_s &=& P_rv_z^{\pm}.
\end{eqnarray}
Applying the same reasoning to traces on $z=z_r$: create pressure and
velocity sources $h_r$ and $f_r$ at $z=z_r$ by odd extension of the
velocity, respectively pressure, fields, and using the jumps as source
coefficients. Then as before
$\Lambda^{\pm}_{z_r}f_r=h_r=-P_rv^{\pm}_z$ (the sign change is due to
$Lambda$ producing the jump, whereas $P_rv^{\pm}_z$ is the limit from
the left of the field $v^{\pm}_{1,z}$ of system \ref{eqn:awe1}), hence
$\Lambda^{\pm}_{z_r}P_rp^{\pm} = P_rv_z^{\pm}$. Combining this
observation with equations
\ref{eqn:lamfh}, \ref{eqn:sandv}, obtain (with roles of $z_s,z_r$ interchanged)
\[
V^{\pm}_{z_r,z_w}(-2P_rp^{\pm}) =
  \Lambda^{\pm}_{z_s}S^{\pm}_{z_r,z_s}(-2P_rv_z^{\pm})  =
  \Lambda^{\pm}_{z_s}S^{\pm}_{z_r,z_s}(-2\Lambda^{\pm}_{z_r}P_rp^{\pm})
\]
whence
\begin{equation}
  \label{eqn:preunit}
 V^{\pm}_{z_r,z_w}
 =\Lambda^{\pm}_{z_s}S^{\pm}_{z_r,z_s}\Lambda^{\pm}_{z_r}.
\end{equation}

In view of the relations \ref{eqn:trtrcomp}, equation
\ref{eqn:preunit} implies that
\[
  V^-_{z_r,z_s}=-(V^+_{z_s,z_r})^T = -\Lambda^{-}_{z_s}(S^+_{z_s,z_r})^T\Lambda^{-}_{z_r}.
\]
\[
  =-(\Lambda^{+}_{z_s})^T(S^+_{z_s,z_r})^T(\Lambda^{+}_{z_r})^T.
\]
Therefore the first equation in display \ref{eqn:approxinv} can be re-written
\[
  -4V^-_{z_r,z_s}S^+_{z_s,z_r} =
\]
\begin{equation}
  \label{eqn:baseunit}
 4 (\Lambda^{+}_{z_s})^T(S^+_{z_s,z_r})^T(\Lambda^{+}_{z_r})^T
  S^+_{z_s,z_r} \approx I
\end{equation}

\section{Derivations}
\subsection{Extension to whole space}
If $\phi$ is smooth and vanishes for large negative $t$, then
$(p_1,\bv_1)$ can be constructed by pasting together the smooth
solutions of the inhomogeneous Dirichlet problems in the half-spaces
$\pm z > z_s$. Construct these by extending $\phi$ smoothly to
$\pm z>z_s$, and solve systems similar to \ref{eqn:awe1} for the
difference with $(p_1,\bv_1)$ by quoting the main results from
\cite{BlazekStolkSymes:13}. Since the right hand side in this
difference system is smooth, and the system is causal, the solution is
smooth and vanishes on $z=z_s$. Adding back the extension of $\phi$
produces smooth solutions in $\pm z > z_s$; define $(p_1,\bv)_1$ to be
these smooth functions in $z \ne z_s$. Since the boundary values for
pressure are the same, $p_1$ extends continuously to $z=z_s$. 
The pressure boundary value $\phi$ uniquely determines the solutions
in the two half-spaces, by standard energy estimates,

\subsection{Adjoint Computation}
The adjoint of ${\cal S}^+_{z_s,z_r}$ can be computed by a variant of
the adjoint state method, in this case a by-product of the
conservation of energy. Suppose that $p^-,\bv^-$ solve \ref{eqn:awem}
with $(h_s,f_s\bf{e}_z)\delta(z-z_s)$ replaced by
$ (h_r,f_r\bf{e}_z) \delta(z-z_r)$. Then
\[
0 = 
\left(\int\, dx\,dy\,dz\, \frac{p^+ p^-}{\kappa} +  
\rho \bv^+ \cdot \bv^- \right)|_{t \rightarrow \infty}
-
\left(\int\, dx\,dy\,dz\, \frac{p^+ p^-}{\kappa} +  \rho \bv^+ \cdot \bv^- \right)|_{t \rightarrow -\infty}
\]
\[
= 
\int_{-\infty}^{\infty} \,dt\, \frac{d}{dt}\left(\int\, dx\,dy\,dz\, \frac{p^+ p^-}{\kappa} +  \rho \bv^+ \cdot \bv^- \right)
\]
\[
= 
\int_{-\infty}^{\infty} \,dt\, \left(\int\, dx\,dy\,dz\, \frac{1}{\kappa} \frac{\partial p^+}{\partial t} p^- +  p^+ \frac{1}{\kappa}\frac{\partial p^-}{\partial t} \right.
\]
\[
+
\left. \rho \frac{\partial \bv^+}{\partial t} \cdot \bv^- + \rho \bv^+ \cdot \frac{\partial \bv^-}{\partial t} \right)
\]
\[
= 
\int_{-\infty}^{\infty} \,dt\, \left(\int\, dx\,dy\,dz\, \left(- \nabla \cdot \bv^+ + 
 h_s \delta(z-z_s)\right) p^- + p^+ \left(- \nabla \cdot \bv^- + 
 h_r \delta(z-z_r)\right) \right.
\]
\[
+
\left.  (- \nabla p^++f_s\bf{e}_z) \cdot \bv^- + \bv^+ \cdot (-\nabla
  p^- + f_r \bf{e_z}) \right)
\]
\[
= 
\int_{-\infty}^{\infty}\,dt\, \left(\int\, dx\,dy\,dz\, \left(- \nabla \cdot \bv^+ + 
 h_s \delta(z-z_s)\right) p^- + p^+ \left(- \nabla \cdot \bv^- + 
 h_r \delta(z-z_r)\right) \right.
\]
\[
+
\left.  p^+ (\nabla \cdot \bv^-) + (\nabla \cdot \bv^+) p^- 
  +f_s \delta(z-z_s) v_z^- + v_z^+f_r \delta(z-z_r) \right)
\]
after integration by parts in the last two terms. Most of what is left cancels, leaving 
\[
0 = \int_{-\infty}^{\infty}\,dt\,dx\,dy\, (h_sP_sp^-+f_zP_sv_z^-) +
( h_rP_rp^++f_rP_rv_z^+) = \langle (h_s,f_s), {\cal S}^-(h_r,f_r) \rangle+ \langle (h_r,f_r), {\cal S}^+_{z_s,z_r}(h_s,f_s) \rangle
\]
whence
\begin{equation}
\label{eqn:sadj}
 ({\cal S}^+_{z_s,z_r})^T = -{\cal S}^{-}_{z_r,z_s}.
\end{equation}

The systems \ref{eqn:awep} and \ref{eqn:awem} differ only in the direction of time evolution (causal vs. anti-causal). Define $R$ to be the {\em time-reversal operator} on functions of space-time,
$Rf(\bx,t) = f(\bx,-t)$, and ${\cal R}$ to be the {\em acoustic field time-reversal operator} 
\begin{equation}
  \label{eqn:trdef}
  {\cal R} \left(
    \begin{array}{c}
      p\\
      \bv
    \end{array}
  \right) =
  \left(
    \begin{array}{c}
      Rp\\
      -R\bv
    \end{array}
  \right)
\end{equation}
Then ${\cal R}(p^-,\bv^-)$ solves \ref{eqn:awep} with
$(h_s\delta(z-z_s),f_s\bf{e_z}\delta(z-z_s))$ replaced by
$(-Rh_r\delta(z-z_r), Rf_r\bf{e_z}\delta(z-z_r))$. That is,
\begin{equation}
  \label{eqn:trsadj}
  {\cal R}{\cal S}^- = -{\cal S}^+_{z_r,z_s}{\cal R}
\end{equation}
Since $R^2 = I$ and ${\cal R}^2 = I$, the identities \ref{eqn:sadj} and \ref{eqn:trsadj} imply that

\section{Surface source-to-pressure operator}

A particular case of the relation \ref{eqn:trtr} is important in its own right, when $z_r=z_s$. Define surface source-to-pressure operator $\Lambda_{z_s}$ by
\begin{equation}
  \label{eqn:ntoddef}
  \Lambda^{\pm}_{z_s} = S^{\pm}_{z_s,z_s}.
\end{equation}
Then \ref{eqn:trtr} reads
\begin{equation}
  \label{eqn:ntodtr}
  (\Lambda^+_{z_s})^T = R\Lambda_{z_s}^+R = -\Lambda^-_{z_s}
\end{equation}
This operator is closely related to the operator mapping $\lim_{z\rightarrow z_s^{\pm}} \bv_z^+$ to $\lim_{z\rightarrow z_s^{\pm}}  p^+$, often called the {\em Neumann-to-Dirichlet} operator. The connection is explained below.

The quadratic form defined by $\Lambda^{\pm}_{z_s}$ has fundamental physical
significance. Define the total acoustic energy $E^{\pm}(t)$ of the field $(p^{\pm},\bv^{\pm})$, at time $t$ by
\begin{equation}
  \label{eqn:defae}
  E^{\pm}(t) = \frac{1}{2} \int \,d\bx \, \left(\frac{(p^{\pm})^2}{\kappa} + \rho |\bv^{\pm}|^2\right)
\end{equation}
Then for $t_{\min}<t_{\max}$,
\[
  E^{\pm}(t_{\rm max})-E^{\pm}(t_{\rm min}) = \int_{t_{\rm min}}^{t_{\rm max}} \frac{dE^{\pm}}{dt}
\]
\[
  = \int_{t_{\rm min}}^{t_{\rm max}} \int\,d\bx\, \left(p^{\pm} \frac{1}{\kappa}\frac{\partial p^{\pm}}{\partial t} + \bv^{\pm} \cdot \rho \frac{\partial \bv^{\pm}}{\partial t}\right)
\]
\[
  =\int_{t_{\rm min}}^{t_{\rm max}} \int\,d\bx\,\left(p^{\pm}(-\nabla \cdot \bv^{+} + h_s\delta(z-z_s)) + \bv^{\pm} \cdot \rho \frac{\partial \bv^{\pm}}{\partial t}\right)
\]
\[
=  \int_{t_{\rm min}}^{t_{\rm max}}\int\,d\bx\, \left(\left(\nabla p^{\pm}+\rho \frac{\partial \bv^{\pm}}{\partial t}\right) + p^{\pm}h_s\delta(z-z_s)\right)
\]
\begin{equation}
  \label{eqn:eident}
  =  \int_{t_{\rm min}}^{t_{\rm max}}\int\,dxdy h_s p^{\pm}|_{z=z_s} = \int_{t_{\min}}^{t_{\max}}\langle h_s(t), (\Lambda^{\pm}_{z_s} h_s)(t) \rangle_{L^2(\bR^2)}.
\end{equation}
That is, $\langle h_s(t), (\Lambda^{\pm}_{z_s} h_s)(t)
\rangle_{L^2(\bR^2)}$ is the rate of energy transfer from source to
the fluid at time $t$.

Assume that $h_s$ has compact support in time. Then $E^{\pm}
\rightarrow 0$ as $t \rightarrow \mp \infty$, and is eventually
constant as $t \rightarrow \pm \infty$. Then the energy identity
\ref{eqn:eident} implies that
\begin{equation}
  \label{eqn:etot}
  \pm \lim_{t \rightarrow \pm \infty} E^{\pm}(t) = \int \,dt\, \langle
  h_s(t), (\Lambda^{\pm}_{z_s} h_s)(t)\rangle_{L^2(\bR^2)} = \langle
  h_s,\Lambda^{\pm}_{z_s} h_s\rangle
\end{equation}
By virtue of identity \ref{eqn:ntodtr},
\[
  \lim_{t \rightarrow -\infty}E^{-}(t) =-\langle
  h_s,\Lambda^{-}_{z_s} h_s\rangle
\]
\[
  =  \langle  h_s,(\Lambda^{^+})^T_{z_s} h_s\rangle
\]
\[
  =  \langle  h_s,\Lambda^{^+}_{z_s} h_s\rangle
\]
so
\begin{equation}
  \label{eqn:surfenergy}
  = \lim_{t \rightarrow \infty}E^{+}(t)
\end{equation}
That is, the energy total energy transferred from the source to the
causal acoustic field as it evolves forward in time is the same as the
total energy transferred to the anti-causal acoustic field as it
evolves backwards in time.

Also onclude from identity \ref{eqn:etot} that the symmetric part of $\Lambda^+_{z_s}$ is positive semidefinite:
\begin{equation}
  \label{eqn:lamsd}
  \frac{1}{2}((\Lambda_{z_s}^+)^T + \Lambda^+_{z_s}) \ge 0.
\end{equation}

%Observe that the causal acoustic system \ref{eqn:awep} implies directly that
%\begin{equation}
%  \label{eqn:sourcevjump}
%  h_s= [v_z]|_{z=z_s}.
%\end{equation}

%######################################
\section{Propagating Waves}
Define $H$ to be the subspace $\{\bu \in (L^2(\bR^3))^4: \bu=0,z\ge
z_s\}$, by analogy with \cite{Lax:PDENotes}, p. 122.
%Note: must add microlocal $H^1$ constraint per
%\cite[]{BaoSy:91b} so that trace operator is well-defined - postpone
%this.

Given $u=(\bar{p}_0,\bar{\bv}_0) \in H$, extend in $t$ by solving the
acoustic initial value problem
\begin{eqnarray}
\label{eqn:awe0}
  \frac{1}{\kappa}\frac{\partial p_0}{\partial t} & = & - \nabla \cdot \bv_0, \nonumber \\
  \rho\frac{\partial \bv_0}{\partial t} & = & - \nabla p_0,\nonumber \\
  p_0 & =& \bar{p}_0, t=0,\nonumber\\ 
  \bv_0 & = & \bar{\bv}_0, t=0,\nonumber\\
\end{eqnarray}
Standard results on hyperbolic systems assure existence of a unique
square-integrable (weak) solution, which is smooth if $(p_0,\bv_0)$ is
smooth at $t=0$ \cite[]{CourHil:62,LaxPDENotes}.

Note that $(p_0,\bv_0)=0$ near $t=0$ in the half-space $z>z_s$.

For any suitable space-time function $u$, define
\begin{equation}
  \label{eqn:postrace}
  P^+_{z_s}u(x,y,t) =
  \left\{
    \begin{array}{c}
      u(x,y,z_s,t), t>0,\\
      0, t\le 0.
    \end{array}
  \right\}
\end{equation}
(Here, ``suitable'' means: the trace on $z=z_s$ makes sense. More on
this later.)


Since $p_0$ also solves the acoustic system \ref{eqn:awe1} in $z>z_s$
and has the same data at $t=0, z>z_s$ and trace on $z=z_s$, standard
uniqueness theorem shows that $(p_1,\bv_1) = (p_0, \bv_0)$ in $z>z_s$.



\section{Progressing Waves and Symbols}

The symbol $\sigma(L[c])$ of a matrix operator $L[c]$ is a complex scalar-valued matrix of the same size. In terms of its product with an arbitrary vector $\bu \in \bR^{d+1}$, it is
\[
  \sigma(L[c]) \bu = e^{-i(\omega t + \bk \cdot \bx)} (L[c]e^{i(\omega t + \bk \cdot \bx)}\bu).
\]
For $L[c]$ given by the definition \ref{eqn:awemat}, obtain
\begin{equation}
  \sigma(L[c])(t,\bx,\omega,\bk) = i\left(
    \begin{array}{cc}
      \frac{\omega}{\kappa(\bx)} & \bk^T \\
      \bk & \rho(\bx) \omega)
    \end{array}
  \right)
\end{equation}

The characteristic equation of $L[c]$ is $\det \sigma(L[c]) = 0$. The significance of the characteristic equation is its role in constraining the phase $\psi(\bx)$ in the progressing wave ansatz,
\begin{equation}
  \label{eqn:go}
  \bu(\bx,t) = e^{i\omega (t-\psi(\bx))} \ba (\bx).
\end{equation}
Applying $L[c]$ to this ansatz, obtain
\[
  L[c]\bu(\bx,t)=i \omega
  \left(
    \begin{array}{c}
      \frac{1}{\kappa}a_p -  \nabla \psi \cdot \ba_v \\
      -a_p \nabla \psi + \rho \ba_v
    \end{array}
  \right) + ...
\]
\[
  = i \omega \sigma(L[c])(t,\bx,1,- \nabla \psi(\bx))ba + ...
\]
where ``$...$'' denotes terms of lower order in $\omega$. Thus $\bu$ as defined in equation \ref{eqn:go} is a high-frequency asymptotic solution, $L[c]\bu \approx 0$, if
\begin{itemize}
\item
  \begin{equation}
    \label{eqn:eik}
    \det(\sigma(L[c])(t,\bx,1,-\nabla \psi(\bx))) = 0,
  \end{equation}
  (eikonal equation), and
\item
  $\ba$ is a null vector of $\sigma(L[c])(t,\bx,1,-\nabla \psi)$.
\end{itemize}
Explicitly, the eikonal equation reads
\[
  0 = \rho^2\left(\frac{\rho}{\kappa}-|\nabla \psi|^2\right),
\]
equivalent to the familar form
\begin{equation}
  \label{eqn:usualeik}
  c|\nabla \psi| = 1,
\end{equation}
in which $c = \rho/\kappa$ is the wave velocity. Solution via the method of characteristics gives
\[
  \psi(\bx(t)) = \psi(\bx(0)) + t
\]
for a ray $t \mapsto \bx(t)$ satisfying
\[
  \frac{d\bx}{dt}(t) = -c(\bx)^2\nabla \psi(\bx(t))
\]
Finally, $\ba$ is a null vector of $\sigma(L[c])(\cdot,\cdot,1,-\nabla \psi)$ iff
\begin{equation}
  \label{eqn:godton}
  \ba_v = \frac{a_p}{\rho}\nabla \psi
\end{equation}
The scalar $a_p$ evolves along the ray according to the transport equation, which I will not derive here; suffice it to say that $a_p(t) = a_p(0) \gamma(t)$, where $\gamma > 0$ is smooth and $\gamma(0)=1$.

Suppose that $\bx_0 = (x,y,z_s)^T$ and $(\xi_x,\xi_y) \in \bR^2$ satisfies $c(\bx_0)|(\xi_x,\xi_y)|<1$. Then there is a neighborhood $\Omega$ of $\bx_0$ and $\xi_z \in C^{\infty}(\Omega), \xi_z>0$, so that $\bxi^{\pm}(\bx)=(\xi_x,\xi_y,\pm \xi_z(\bx))$ satisfy $c(\bx)|\bxi^{\pm}(\bx)|=1$ for $\bx \in \Omega$.

A standard argument shows that there exists $\epsilon > 0$ for which a solution $\psi^{\pm}$ of the eikonal equation exists in $\Omega \cap \{\bx: |z-z_s|<\epsilon\}$ with
\begin{equation}
  \label{eqn:eikic}
  \psi^{\pm}(x,y,z_s,t) = t -x \xi_x - y \xi_y, \, \frac{\partial \psi}{\partial z}(x,y,z_s,t) = \pm \xi_z(x,y,z_s).
\end{equation}
Let $\chi \in C_0^{\infty}, \chi(\bx_0)=1$, and set $p_0^{\pm}(x,y,t) = \chi(x,y,z_s)e^{i\omega(t-x\xi_x - y\xi_y)}$. Define $p_{\rm go}^{\pm} \in C^{\infty}(\Omega \cap \{\bx: |z-z_s|<\epsilon\})$ by
\[
  p_{\rm go}^{\pm}(\bx,t) = a^{\pm}_p(\bx) e^{i\omega (t-\psi^{\pm}(\bx))}.
\]
where $a^{\pm}_p$ solves the transport equation with $a^{\pm}_p|_{z=z_s} = \chi$. Set
\[
  \bv^{\pm}_{\rm go}(\bx,t) = \frac{a^{\pm}_p(\bx)}{\rho(\bx)}\nabla \psi^{\pm}(\bx) e^{i\omega (t-\psi^{\pm}(\bx))}
\]
Then $\bu^{\pm}_{\rm go} = (p^{\pm}_{\rm go}, \bv^{\pm}_{\rm go})$ is an asymptotic solution of $L[c]\bu = 0$ with $p^{\pm}_{\rm go}|_{z=z_s} = \chi e^{i\omega(t-x\xi_x - y\xi_y)}$.

From equations \ref{eqn:godton} and \ref{eqn:eikic},
\[
  v_{z,{\rm go}}^{\pm}(x,y,z_s,t) = \pm \frac{\chi(x,y,t)}{\rho(x,y,z_s)}\xi_z  e^{i\omega(t-x\xi_x - y\xi_y)}
\]
\begin{equation}
  \label{eqn:asymdton}
= \pm \frac{\chi(x,y,t)}{\rho(x,y,z_s)c(x,y,z_s)}\sqrt{1-c(x,y,z_s)^2(\xi_x^2+\xi_y^2)}
\end{equation}

Integration over frequency produces a progressing wave traveling (at least near $z=z_s$) downwards, that is, increasing $z$ with increasing $t$, for the choice of sight $\pm = +$, and upwards, that is, decreasing $z$ with increasing $t$, for the other sign. Therefore a causal progressing wave with $p$ given on $z=z_s$ is obtained by synthesizing distorted plane waves as just described with $\pm = +$ in $z>z_s$, $\pm=-$ in $z<z_s$. From equation \ref{eqn:asymdton}, the normal components of $\bv$ have limits equal in magnitude but of opposite sign as $z\rightarrow 0^{\pm}$, up to a lower frequency error. That is, the jump in $\bv$ satisfies
\begin{equation}
  \label{eqn:asymjump}
  [v_z]\approx 2\lim_{z\rightarrow 0^+} v_{z}
\end{equation}
where $\approx$ denotes ``up to a low frequency error'', from here on.

This construction also implies an approximation of the source-to-pressure operator as a pseudodifferential operator. Choose a partition of unity in $z=z_s$, for which the support of every member is a domain $\Omega$ for which the construction above can be carried out: this implies that there be a uniform $\delta > 0$ so that
\begin{equation}
  \label{eqn:nongraze}
  1>\delta + c(x,y,z_s)\sqrt{\xi_x^2+\xi_y^2}
\end{equation}
for all $(x,y,t,1,\xi_x,\xi_y) \in WF(p_0)$. Denote by $\Gamma$ the subset of $(x,y,t,k_x,k_y,\omega) \in T^*(\bR^3)$ for which $x,y,\xi_{xy}=k_{xy}/\omega$ satisfy the inequality \ref{eqn:nongraze}.

Then near $z=z_s$, $p$ is approximately a sum of geometric optics solutions of the type displayed above, integrated over $\omega,k_x,k_y$, with $k_{x,y} = \omega \xi_{x,y}$.  Thus
\[
  p(x,y,z_s,t) \approx \frac{1}{2}\int_{\Gamma(x,y)} \,d\omega\,dk_x\,dk_y \hat{[v_z]}\rho(x,y,z_s)c(x,y,z_s)
\]
\[
   \times \left(1-c^2(x,y,z_s)\left(\left(\frac{k_x}{\omega}\right)^2 + \left(\frac{k_y}{\omega}\right) ^2\right)\right)^{-1/2}e^{\i(\omega t + k_x x+ k_y y)}
\]
Since $p$ is the pressure component of a solution of \ref{eqn:awep}, equation \ref{eqn:sourcevjump} implies that
\begin{equation}
  \label{eqn:psidodton}
  \Lambda^+_{z_s} h_s(x,y,t) \approx \frac{1}{2}\int_{\Gamma(x,y)} \,d\omega\,dk_x\,dk_y
  \lambda_s(x,y,t,k_x,k_y,\omega) \hat{h_s}(k_x,k_y,\omega) e^{\i(\omega t + k_x x+ k_y y)}
\end{equation}
with
\begin{equation}
  \label{eqn:symboldton}
  \lambda^+_{z_s}(x,y,t,k_x,k_y,\omega) =\frac{1}{2}\rho(x,y,z_s)c(x,y,z_s)\left(1-c^2(x,y,z_s)\left(\left(\frac{k_x}{\omega}\right)^2 + \left(\frac{k_y}{\omega}\right) ^2\right)\right)^{-1/2}
\end{equation}

Restricted to $\Gamma$, $\lambda^+_{z_s} \in S^0_{0,1}(\bR^2)$ is real-valued, hence defines an essentially symmetric operator. That is,
\begin{equation}
  \label{eqn:approxsymm}
  (\Lambda^+_{z_s})^T \approx \Lambda^{+}_{z_s}.
\end{equation}

Note that the defintion \ref{eqn:appinv} of $ (S^+_{z_s,z_r})^{\dagger} $ is that of
the transpose of $S^+_{z_s,z_r}$ in the sense of weighted norms, as
follows. Define trace spaces $W=L^2(\bR^2 \times [0,T])$, with inner product
\[
  \langle h_1,h_2\rangle_s = \langle
  h_1,\frac{1}{2}(\Lambda^+_{z_s}+(\Lambda^+_{z_s})^T)h_2\rangle
\]
and $D=L^2(\bR^2 \times [0,T])$ with inner product
\[
  \langle d_1,d_2\rangle_r = \langle
  d_1,\left(\frac{1}{2}(\Lambda^+_{z_s}+(\Lambda^+_{z_s})^T)\right)^{-1}d_2\rangle
\]
(These definitions must be supplemented with constraints on
out-of-aperture smoothness). Then the adjoint of $S^+_{z_s,z_r}: W
\rightarrow D$ is
\[
 \left(\frac{1}{2}
   (\Lambda^+_{z_s}+(\Lambda^+_{z_s})^T)\right)^{-1}(S^+_{z_s,z_r})^T
 \left(\frac{1}{2}(\Lambda^+_{z_r}+(\Lambda^+_{z_r})^T)\right)^{-1}
\]
 By virtue of the approximation \ref{eqn:approxsymm}, this is
\[
   \approx
   (\Lambda^{+}_{z_s})^{-1}(S^{+}_{z_r,z_s})^T((\Lambda^{+}_{z_r})^{-1})^T
   = (S^+_{z_s,z_r})^{\dagger}.
\]
which from the last section is approximately inverse to
$S^+_{z_s,z_r}$. That is, with the domain and range described above,
$S^+_{z_s,z_r}$ is approximately unitary.

%%%%%%%%%%%%%%%%%%%%%%%%%%%%%%%%%%%%%%%%%%%%%%%%%%%%%%%

In this section I will explain how to build an approximate inverse.
The key ingredient in this {\em time reversal} construction is a
{\em local energy decay} assumption. Define the local energy in the
acoustic field $(p,\bv)$ at
time $t$ in a region $\Omega \subset \bR^3$ by limiting the integration
in the definition of total energy \ref{eqn:defae0} to $\Omega$:
\begin{equation}
  \label{eqn:defaeloc}
  E_{\Omega}[(p,\bv)](t) = \frac{1}{2} \int_{\Omega} \,d\bx \, \left(\frac{p(\bx,t)^2}{\kappa(\bx)} + \rho(\bx) |\bv(\bx,t)|^2\right)
\end{equation}
The necessary assumption is: local energy decay in the half-space
$\Omega = \{(x,y,z): z \le z_r\}$. The local energy in $\Omega$ of the
causal solution $(p^+,\bv^+)$ of system \ref{eqn:awepm} cannot decay,
since the source on $z=z_s<z_r$ radiates in both positive and negative
$z$ in forward time. That obstacle is resolved by a redefinition of
the ``downgoing'' concept. I will call  downgoing if there is a
source-free acoustic field $(\tilde{p}^+,\tilde{\bv}^+)$ in all of space-time for which
\begin{itemize}
\item[1. ] $p^+ \approx \tilde{p}^+, \bv^+ \approx \tilde{\bv}^+$ in
  the slab $\{(x,y,z,t): z_s \le z \le z_r\}$, and
\item[2. ] The energy in $(\tilde{p}^+,\tilde{\bv}^+)$ decays in
  forward time, in the half-space $\Omega$:  $E_{\Omega}[(\tilde{p}^+,\tilde{\bv}^+)](t) \rightarrow 0$
  as $t \rightarrow \infty$.
\end{itemize}
Conditions for local energy decay have been widely studied in various
settings, for instance obstacle scattering
\cite[]{MorawetzPhillipsLax:05}. \cite{Pauen2000NonTr-6079 gives a
comprehensive overview of this subject. For smooth (non-scattering)
acoustic parameters, as is assumed in this paper, the crux of the
matter is that rays carrying high-frequence energy exit the half-space
$\Omega$ in finite time. For the half-space
geometries considered here, local energy does not decay in general,
because of the existence of ``trapped'' rays traveling parallel to the
boundary and remaining interior to $\Omega$ for all time. If an
acoustic field is downgoing in the usual sense, that is, its
high-frequency energy is concentrated on rays that exit $\Omega$
within some common maxiumum time, then the results cited above show
that local energy in $\Omega$ decays (exponentially for $n=3$).

Starting from this observation, it is possible to show that if the
source vector $(h_s,f_s)^T$ has the property termed ``downgoing'' in
preceding sections,
\section{Efficient Preconditioned CG}
