\title{Solution of an Acoustic Transmission Inverse Problem by Extended Inversion}
\author{William W. Symes, Orcas Island, Huiyi Chen* and Susan E. Minkoff, The University of Texas at Dallas}

\lefthead{Symes}

\righthead{Discrepancy}

\maketitle
\begin{abstract}
  Study of an extremely simple single-trace transmission example shows
  how an extended source formulation of full waveform inversion can
  produce an optimization problem without spurious local minima
  (``cycle skipping''), hence efficiently solvable via Newton-like
  local optimization methods. The data consist of a single trace
  recorded at a given distance from a point source. The velocity or
  slowness is presumed homogeneous, and the target source wavelet is
  presumed quasi-impulsive or localized near zero time lag. The source
  is extended by permitting energy to spread in time, and the spread
  is controlled by adding a weighted mean square of the extended
  source wavelet to the data misfit, to produce the extended inversion
  objective. The objective function and its gradient can be computed
  explicitly, and it is easily seen that the error in any local
  minimizer is bounded by a multiple of the permitted energy spread,
  which may be interpreted as a wavelength. The derivation shows several
  important features of all similar extended source algorithms. For
  example, nested optimization, with the source estimation in the
  inner optimization (variable projection method), is essential. The
  choice of the penalty operator, controlling the extended source
  degrees of freedom, is critical: in order to produce an objective
  immune from cycle-skipping, the penalty operator must be
  (pseudo-)differential. The penalty weight is chosen dynamically,
  according to a version of the Discrepancy Principle, in order to
  accelerate convergence to a near-global minimum. Like many other
  extended formulations, the example problem considered here permits
  the discrepancy criterion to be satisfied with zero penalty
  weight. This unusual feature removes a common obstacle to the use of
  the discrepancy principle, namely the determination of a
  satisfactory initial estimate of the penalty weight: the initial weight may
  be set to zero, and the dynamic algorithm updates the weight until
  convergence is achieved.

\end{abstract}

\section{Introduction}
Model-based parameter estimation by least squares data fitting is a
widely used and successful technique for data analysis in science and
engineering \cite[]{Bertero:98,Vogel:02}, particularly in the
geosciences \cite[]{Parker:94,Tarantola:05}. In its application to
seismology, least-squares data fitting has come to be known as Full
Waveform Inversion (FWI), and is now well-established as a useful tool
for probing the earth's subsurface
\cite[]{VirieuxOperto:09,Fichtner:10,Schuster:17}. However, in its
application to seismology, FWI encounters a serious
practical obstacle, so-called ``cycle-skipping''. This term signifies
the tendency of the mean-square error function to exhibit many
stationary points far from the global minimizer (we
show an explicit example of this phenomenon later in this paper). Because of the
typical dimensions of field FWI problems, only local gradient-based
minimization algorithms are computationally practical. These find
stationary points (or local minima), so may stagnate at suboptimal and geologically
uninformative earth models. Local descent methods avoid suboptimal
stagnation only if initial models are already quite close to optimal,
in the sense of predicting the arrival times of seismic events to
within a small multiple of the dominant data wavelength
\cite[]{GauTarVir:86,VirieuxOperto:09,Plessix:10}.

This paper examines one of a number of proposed alternatives to least
squares FWI, based on an extension of the model-to-data mapping
(``extended inversion''), in the context of 
a very simple acoustic transmission inverse problem: the sound
velocity in a fluid is to be determined from a single recording of a
pressure wave propagated over a known distance, originating at a known
location around a known time. We show precisely
how descent-based FWI fails to solve this problem absent a very good
initial model estimate, and precisely why extended inversion succeeds, using the same type
of optimization, and from essentially arbitrary initial model
estimate. We give rigorous estimates for the accuracy of the sound
velocity determination in terms of the time extent of the energy
excitation and the signal-to-noise ratio of the data. We also illustrate
these mathematical conclusions with numerical
examples.

The single-trace transmission problem studied in this paper is a
drastically simplified cartoon of actual seismic prospecting. It seems
worth careful study nonetheless for two reasons: it exhibits the
phenomenon of cycle-skipping, and its simplicity allows the properties
of both FWI and extended inversion approaches to be established
rigorously. The underlying mathematical structures are common to many
inverse problems in wave propagation in which wave velocities are
amongst the unknowns. For example, we offer an ad hoc calculation of
the gradient of the extended inversion objective, taking advantage of
the simplicity of the problem context. However we also show, in an
appendix, how that calculation hides an abstract structure shared by
many other problems.% - see for example
%\cite{Symes:IPTA14,tenKroode:IPTA14,Symes:EAGE15,HuangSymes:SEG15a}.

In the next (``Overview'') section, we describe the conceptual
framework of this inverse problem, review the related literature, and
state the essence of our main results, leaving out a few technical
details that are supplied later. Numerical illustrations of these
results occupy the following (``Examples'') section. The fourth
(``Findings'') section of the paper contains precise statements of the
mathematical results, derivations, and supporting discussion. Three
appendices describe the algorithm used to control the penalty weight
in our formulation of extended inversion, further refinement of error
estimates based ond the Heisenberg inequality, and the abstract
structure of extended inversion implicit in the derivations presented
in the ``Findings'' section.

\section{Overview}

The physical setting for our discussion is the propagation of
small-amplitude pressure waves through a
homogeneous acoustic fluid occupying $\bR^3$ \cite[]{Frie:58}. The
fluid's salient characteristic for present purposes is the
slowness, or reciprocal wave velocity, $m > 0$. A source of acoustic
energy localized at a (``source'') point $\bx_s \in \bR^3$ radiates uniformly in all
directions with time-dependent intensity $w(t)$, generating a pressure
field $p(\bx,t)$. Both the 
pressure field $p$ and the intensity (``wavelet'') $w$ vanish at large
negative time. A receiver records the pressure values at a point
$\bx_r \in \bR^3$, a distance
$r>0$ from the source position $\bx_s$, between the times $t_{\rm min}$ and
$t_{\rm max}$. According to linear acoustics, the pressure field
solves the the wave equation \cite[]{Frie:58}:
\begin{eqnarray}
  \label{eqn:awe}
  \left(m^2\frac{\partial^2 p}{\partial t^2} - \nabla^2\right) p(\bx,t) &=&
                                                                         w(t)\delta(\bx-\bx_s) \nonumber\\
  p(\bx,t)&=&0, t\ll 0.
\end{eqnarray}
As the system \ref{eqn:awe} is linear and autonomous, the solution is
the convolution of the source
wavelet $w$ with the Green's function or fundamental solution. For
constant $m$, the result is the well-known expression
(\cite{CourHil:62}, Chapter VI, section 12, equation 47):
\begin{equation}
  \label{eqn:homsol}
  p(\bx,t) = \frac{1}{4\pi |\bx-\bx_s|}w\left(t-m|\bx-\bx_s|\right),
\end{equation}
so the predicted pressure at the receiver point $\bx_r$ is
\begin{equation}
\label{eqn:mod}
p(\bx_r,t) = \frac{1}{4\pi r}w\left(t-mr\right)  = F[m]w(t)
\end{equation}
The mapping $F[m]$ so defined is (ignoring the amplitude factor
$1/(4\pi r)$), an $m$-dependent time shift. This time shift operator is the basis of many illustrations of
the cycle-skipping phenomenon (for example, \cite{VirieuxOperto:09},
Figure 7), so it is unsurprising that an analysis of cycle-skipping
can be based on the simple modeling operator described above.

If the wavelet $w$ is square-integrable, so is the predicted
recorded pressure $F[m]w$ - its $L^2$ norm is the cumulative power trasferred
from the fluid to the receiver \cite[]{SantosaSymes:00}. Thus this relation defines a map
$F: \bR^+ \times L^2(\bR) \rightarrow
L^2([t_{\rm min},t_{\rm max}])$, linear in its second argument.

A naive version of the inverse problem investigated in this paper
would be: given recorded data
$d \in L^2([t_{\rm min},t_{\rm max}])$, find $m \in \bR^+$ and $w \in L^2(\bR)$ so that
$F[m]w \approx d$. Clearly this problem statement is not interesting as it
stands, since $F[m]$ is surjective for every $m \in \bR^+$: fitting
the data does not constrain the slowness $m$ at all.

To be useful, the problem statement must be augmented with a
constraint on the model $(m,w)$.  One natural constraint is to assume
that the support of $w$ is limited: choose $\lambda > 0$ and search
for $w$ with $\mbox{supp }w \subset [-\lambda,\lambda]$. The quantity
$\lambda$ is to be considered ``small''. Under some circumstances, it
is related to a mean wavelength of $w$ (Appendix B).

An assumption of small support may be justified as
follows. In practical seismic data processing, actual sources may act
over considerable time intervals. An example from exploration
seismology is the airgun, the most commonly used
marine source: an airgun (or array of airguns, the usual apparatus)
produces an oscillating pressure pulse that dies away
slowly. It is commonplace to estimate this pulse, usually by recording
the emitted pressure field at a position where it should nominally be isolated from other
signal, then deconvolve it
from the data by safeguarded Fourier division or other means. This
so-called signature deconvolution results in modified data
corresponding to a source equal to signature deconvolution of the
pulse estimate itself. The deconvolved source pulse approximates an
impulse ($\delta(t)$): it cannot have point support, due to the finite
frequency nature of seismic data and the Heisenberg principle (see
Appendix B), but its amplitudes are typically small outside of a small
much smaller time interval than is the case for the original pulse
estimate. Thus after a small modification, the deconvolved source
pulse has small support.

For a contemporary description of view on signature deconvolution as
practiced in the seismic industry, see \cite{SheriffGeldart:1995},
section 9.5 and \cite{Yil:01}, Chapter 2.

Abusing notation slightly, we identify $L^2([-\lambda,\lambda])$ with
a subspace of $L^2(\bR)$ via extension by $0$. Denote by $\lF$
restriction of $F$ to $\bR^+ \times L^2([-\lambda,\lambda])$ defined by equation
\ref{eqn:mod}. Then the FWI problem with support constraint,
described in the last paragraph, can be crudely phrased: given $d \in
L^2([t_{\rm min},t_{\rm max}])$, find $(m,w)$ in the domain of $\lF$ so that $\lF[m,w]
\approx d$.

Despite its simplicity, this problem exhibits the cycle-skipping
pathology characteristic of field-scale FWI. Specifically, we show
that the reduced mean square error function, the objective function of
FWI for this problem,
\begin{equation}
  \label{eqn:redms}
  \lerr[m,w;d]=\frac{1}{2}\|\lF[m]w-d\|^2/\|d\|^2,
\end{equation}
has local minimizers far from any global minimizer, even for noise-free
data (Theorem \ref{thm:fwi}).

This fact implies that local optimization must fail to solve a precise
version of the FWI problem:
\begin{quote} given $\epsilon, \lambda > 0$, fine $(m,w)$ in the
domain of $\lF$ so that
\begin{equation}
  \label{eqn:probstat}
  \|\lF[m]w-d\| \le \epsilon\|d\|.
\end{equation}
\end{quote}
In fact, we show
that``large residual'' local minima of $\lerr[\cdot,\cdot,d]$ exist,
at which its value is much greater than its global minimum
value. Therefore if $1/2\epsilon^2$ is near the global minimum value
of $\lerr[\cdot,cdot;d]$, then
local minimization of $\lerr[\cdot,\cdot;d]$ can fail to find a
solution of the inverse problem. This phenomenon is what is meant by
``cycle-skipping''.

Extended inversion involves an extension of the model-to-data map, in
this case $\lF$.  The extension used here consists in dropping the
support constraint on $w$, so the extended map is simply $F$ as
defined above ($\lF$ is a restriction of $F$, so $F$ is an extension
of $\lF$).

To compensate for the resulting indeterminacy in the
estimation of $m$, we add a penalty term to the objective, defined by
a penalty operator $A$ with domain and range $L^2(\bR)$: for $m \in
\bR^+,w\in L^2(\bR), d \in L^2([t_{\rm min},t_{\rm max}])$, define
\begin{eqnarray}
  r[m,w;d] & = & \|F[m]w-d\|/\|d\|, \label{eqn:rmserr}\\
  e[m,w;d] & = & \frac{1}{2}r[m,w;d]^2, \label{eqn:mserr}\\
  g[w] & = & \frac{1}{2}\|Aw\|^2/\|d\|^2, \label{eqn:pen} \\
  \Ja[m, w;d] &=& e[m,w;d] + \alpha^2 g[w] \nonumber\\
  &=& \frac{1}{2}(\|F[m]w-d\|^2 + \alpha^2 \|Aw\|^2)/\|d\|^2.
\label{eqn:penerr}
\end{eqnarray}

The normalizations by the data norm $\|d\|$ in the definitions
\ref{eqn:rmserr}, \ref{eqn:mserr}, \ref{eqn:pen}, and \ref{eqn:penerr} make the values
of $r$, $e$, $\alpha^2 g$, and $\Ja$ nondimensional, and simplifies
both the statements of theoretical results and the interpretation of
numerical examples.

Apart from dimensional correctness, the only obvious requirement on the penalty operator $A$ is that it
penalize 
nonzero values of $w$ for large $|t|$, so that minimizers $(m,w)$ of
$\Ja$ should have $w$
concentrated near $t=0$. Thus the penalty is a soft stand-in for the
support constraint. We
show that a good choice for $A$ is given (essentially) by
$(Aw)(t)=tw(t)$ (Proposition \ref{thm:rampgood}), and the
remainder of the discussion in this section makes this assumption. The
penalty weight $\alpha$ must be chosen somehow; we review a simple
algorithm for controlling $\alpha$ later in the paper.

Assume that $\mu>0$ and $d = F[m_*]w_* + n$, with $\mbox{supp }w_*
\subset [-\mu,\mu]$, and denote by $\eta = \|n\|/\|F[m_*]w_*\|$
the noise-to-signal ratio. We show that any local minimizer $(m,w)$
of $\Ja$ satisfies 
\begin{itemize}
\item If $\eta \le 0.6$,
  \begin{equation}
    \label{eqn:mbnd}
    |m-m_*| \le (1+f(\eta))\frac{\mu}{r},
  \end{equation}
  where $f(\eta)=2\eta+O(\eta^2)$ (Theorem \ref{thm:mnoiseres});
\item 
  $(m,{\bf 1}_{[-\lambda,\lambda]}w)$ is a solution of the FWI problem
  as stated above, that is,
  \begin{equation}
    \label{eqn:success}
    \lerr[m, {\bf 1}_{[-\lambda,\lambda]}w;d] \le \frac{1}{2}\epsilon^2,
  \end{equation}
  so long as
  \begin{eqnarray}
    \lambda & \ge & (2+f(\eta))\mu,\nonumber\\
    \epsilon &\ge & (4\pi r \alpha \lambda)^2 +\eta.
    \label{eqn:solcond}
  \end{eqnarray}
  (Theorem \ref{thm:ipnoisesuf})
\end{itemize}.
That is, the minimization of the extended objective $\Ja$ provides a
solution of the FWI problem, with constraints on the achievable
performance and bounds on the errors. All of these
bounds are shown to be sharp, and the constraints cannot be
significantly weakened. For example, in conclusion \ref{eqn:solcond}, the support radius $\lambda$ in
the projection onto $L^2([-\lambda,\lambda])$ must in general be at least twice the
support radius of the ``noise free'' wavelet $w_*$, plus an additional
allowance for noise.

Use of model extension to reveal the influence of earth mechanical
parameters on data is an old idea, and underlies several classes of
seismic data processing methods in widespread use for decades
\cite[]{geoprosp:2008}. Use of model extension to create optimization
principles for inversion is a more recent development
\cite[]{SymesCar:91,Plessix:00a,Symes:09,LuoSava:11,BiondiAlmomin:SEG12,LeeuwenHerrmannWRI:13,LiuSymesLi:14,Warner:14,ChaurisGP:14,Warner:16,LeeuwenHerrmann:16,Herve2017,HouSymes:Geo18}. Extended
inversion methods differ by the choice of additional degrees of
freedom, and by choice of penalty applied to eliminate them in the
final result. The approach studied in this paper belongs to the {\em
  source extension} variety: the extra parameters are provided by
permitting the energy source component of the experimental model to
have more, or less constrained, parameters than the experimental
design suggests. \cite{HuangNammourSymesDollizal:SEG19} give an
overview and taxonomy of source extension methods, with extensive
references.

Extended inversion is not the only alternative to straightforward
least-squares data fitting that may overcome cycle-skipping. For
example, evidence has recently emerged that the Wasserstein metric
from arising in the theory of optimal transport may provide a measure
of error between model-predicted and observed seismic data less
oscillatory than the mean-square
\cite[]{EngquistYang:GEO18,Metivier:GEO18}. 

%In many cases, the constraint posed by membership of the second component $w$ in the subspace $\lW$ is an approximation to an idealized limit $\lambda \rightarrow 0$.
%It may be that $W_0 = \cap_{\lambda > 0}\lW$ is non-trivial, in which
%case $F$ may be viewed as an extension of $F_0 = F|_{ M \times
%  W_0}$. In that case, $F_0$ and $F$ realize the idealized extension
%structure discussed in many prior works on extended inversion (for
%example, \cite[]{geoprosp:2008}): $F_0$ is the physical modeling
%operator, and $W_0$ consists of second components of physical or
%feasible models. However, in many of these examples, $W_0$ is not
%computationally accessible, or - even if it is - gives results that
%closely approximate those obtained with $\lW$ for small $\lambda$. For
%other examples, for instance that  discussed later in this paper (and
%implicitly in \cite[]{Warner:16}, for instance), $W_0=\{0\}$ and the
%family $\{\lW: \lambda > 0\}$ actually embodies all of the necessary
%physics.

\section{Examples}
This section presents numerical examples of the estimates of $m$ and
$w$ obtained by minimization of $\Ja$. We have selected these examples
to illustrate the conclusions described in the Overview section.

While minimization of $\Ja$ might be tackled directly - by
alternating minimizations over $m$ and $w$, or by computing updates
for $m$ and $w$ simultaneously - such joint minimization performs
poorly, as \cite{YinHuang:16} has shown. The reason for this poor
performance is that $\Ja$ has dramatically different
sensitivity to $m$ versus $w$, as observed in the cited reference and elsewhere.
Instead, a nested approach, in which $w$ is
eliminated in an inner optimization,
generally gives far better numerical performance. This approach was
formalized by \cite{GolubPereyra:73,GolubPereyra:03} as the {\em
  Variable Projection Method} (``VPM''). Besides being the main
computational device used in our examples, it also plays a central
role in the theory developed in the next section.

The minimization of $\Ja$ over $w$ yields a function $\tJa$ of $m$
alone:
\begin{equation}
  \label{eqn:redexp0}
  \tJa[m;d] = \inf_w \Ja[m,w;d]
\end{equation}
Under appropriate conditions, met in our examples (Theorem
\ref{thm:norminv}, $\tJa$ has an alternate expression
\begin{equation}
  \label{eqn:redexp10}
  \tJa[m;d] = \Ja[m,\aw[m;d];d],
\end{equation}
in which $\aw[m;d] \in W$ is the solution of the normal
equation
\begin{equation}
  \label{eqn:norm0}
  (F[m]^TF[m]+\alpha^2A^TA)w= F[m]^Td.
\end{equation}

[ examples and discussion - Huiyi]

[ may need to repeat some of the explicit formulas for $\aw, e, g,
\tJa$ from the next section - refer to theorems]



%\section{Theory}

%\cite{FuSymes2017discrepancy} introduce a version of the discrepancy
%principle and its application to separable nonlinear least squares
%problems in penalty form \ref{eqn:basic}. The algorithm developed
%there is incomplete, however: it lacks  termination criteria. In this
%section, I review the discrepancy-based algorithm and introduce an
%appropriate stopping rule, thus completing the development of the
%algorithm presented in \cite{FuSymes2017discrepancy}.

\section{Findings}
This section describes the mathematical infrastructure necessary to
make precise the conclusions described in the Overview section, then
gives complete statements and proofs, along with those of
necessary auxiliary results. 

We shall use the abbreviations ${\cal B}(X,Y)$ and ${\cal I}(X,Y)$ for 
the algebra of bounded linear operators from the Hilbert space $X$ to 
the Hilbert space $Y$, and its subalgebra of invertible
operators. All of the operators appearing in this discussion are
members of ${\cal B}(X,Y)$ for some choice of Hilbert spaces $X$ and
$Y$, and superscript $T$ denotes the transpose or adjoint of a such an
operator in the sense specified by its domain and range Hilbert structure.

\subsection{An Acoustic transmission Inverse Problem}
The data prediction or modeling operator for the single-trace acoustic
transmission problem has been defined above, in equation
\ref{eqn:mod}, which we restate here for convenience:
\begin{equation}
\label{eqn:mod1}
F[m]w(t)  = p(\bx_r,t) = \frac{1}{4\pi r}w\left(t-mr\right) 
\end{equation}
The slowness $m$ must be positive, as follows from basic acoustics,
and in fact reside in a range characteristic of the
material model: for crustal rock, a reasonable choice would be
$m_{\rm min}=0.125, m_{\rm max}=0.6$ s/km.

Natural choices for domain and
range of $F$ are thus
\begin{itemize}
\item $M=(m_{\rm min}, m_{\rm max}),\,0 < m_{\rm min} \le m_{\rm
    max}$;
\item $W = L^2(\bR)$;
\item $D=L^2([t_{\rm min},t_{\rm max}]),\, t_{\rm min}<t_{\rm max}$;
\item $F: M \times W \rightarrow D$ as specified in \ref{eqn:mod}.
\end{itemize}
It is immediately evident from these choices and from the definition
\ref{eqn:mod} that
\begin{equation}
  \label{eqn:mapprop}
  \mbox{for }m \in M, F[m] \in {\cal B}(W,D),\mbox{ and }\|F[m]\| =
  \frac{1}{4\pi r}.
\end{equation}
Note also that $F[m]$ is surjective for every $m \in M$.

\noindent{\bf Remark:} In computational practice, $W$ will have to be replaced by a
finite-dimensional subspace of $L^2(\bR)$. Many such choices will
implicitly limit the support of $w \in W$ to a bounded interval, say
$[T_{\rm min},T_{\rm max}]$. To maintain the surjective property,
these bounds should be chosen so that $[t_{\rm min}, t_{\rm max}]
\subset [T_{\rm min}+mr,T_{\rm max}+mr]$ for all $m \in M$, that is,
\begin{eqnarray}
  \label{eqn:dombounds}
  t_{\rm min} &\ge & T_{\rm min}+m_{\rm max}r,\nonumber\\
  t_{\rm max} &\le & T_{\rm max}+m_{\rm min}r.
\end{eqnarray}
We will ignore these computational necessities in this work,
maintaining the definition $W=L^2(\bR)$.

As mentioned earlier, $F[m]$ is surjective for every
$m \in M$. Since all possible data lie in the range of $F[m]$ for any
$m \in M$, some restriction of the domain
of $F$ is necessary in order that fitting the data constrain
$m$. The constraint proposed in the
Overview section involves a selection of a maxium support radius
$\lambda_{\rm max} >0$. Then define for $\lambda \in (0,\lambda_{\rm max}]$:
\begin{itemize}
\item $\lW = \{w \in W:
  \mbox{supp }w \subset [-\lambda,\lambda]\}$;
\item $\lF = F|_{M \times \lW}$.
\end{itemize}

In terms of this infrastructure, the inverse problem studied in 
this paper may be stated as a slightly refined version of the problem statement
\ref{eqn:probstat}: 

\begin{quote}
\noindent {\bf Problem Statement:}
  given data $d \in D$, relative error level $\epsilon \in
  [0,1)$, and support radius $\lambda \in (0, \lambda_{\rm
    max}]$, find $(m,w) \in M \times \lW$ for which 
\begin{equation}
  \label{eqn:probstat0}  \|\lF[m]w-d\| \le \epsilon\|d\|,
\end{equation}
or equivalently
\begin{equation}
  \label{eqn:probstat1}
  \lerr[m,w;d] \le \frac{1}{2}\epsilon^2,
\end{equation}
with $\lerr$ defined in equation \ref{eqn:redms}.
\end{quote}

\noindent{\bf Remark:} The constraint $\epsilon < 1$ imposed on the
target noise level is eliminates the obvious choice $(m,0)$, which
satisfies the data misfit constraint for any $m \in M$. 

\noindent{\bf Remark:} The support constraint is closely linked to the
folk theorem about FWI noted many times in the literature: convergence
of a descent method requires that the initial slowness must be known
to ``within a (fraction of a) wavelength''. The relation is a
consequence of Heisenberg's inequality, and will be reviewed in
Appendix B.

The best case for data fitting
is clearly the one in which the data can be fit precisely: that is,
there exists $(m_*,w_*) \in M\times \lW$ so that
\begin{equation}
  \label{eqn:defdatanonoise}
  d=\lF[m_*]w_*.
\end{equation}
Such data $d$ is {\em noise-free}, in the range of the map $\lF$. For
such data a solution of the Problem Statement \ref{eqn:probstat0}
exists with arbitrarily small $\epsilon>0$.


\subsection{Full Waveform Inversion}
While $F[m]$ is surjective for every $m \in M$, as noted above, it is
very far from injective. On the other hand, under a constraint that
will be assumed throughout, $\lF[m]$ is injective (in fact, $4 \pi r
\lF[m]$ is an isometry):
\begin{proposition}
  \label{thm:fullrec}
  Suppose that 
  \begin{equation}
    \label{eqn:fullrec}
    [ m_{\rm min}r-\lambda_{\rm max}, m_{\rm max}r+\lambda_{\rm max}]
    \subset [t_{\rm min},t_{\rm max}].
  \end{equation}
  Then $\lF[m]\in {\cal B}(\lW,D)$ is coercive for every $m \in M, \lambda \in
  (0,\lambda_{\rm max}]$.
\end{proposition}

\noindent{\bf Remark:} A useful consequence of the condition 
\ref{eqn:fullrec}: for every $m \in M$, 
\begin{equation}
  [-\lambda_{\rm max}, -\lambda_{\rm max}] \subset[ t_{\rm min}-mr , 
  t_{\rm max}-mr]. 
  \label{eqn:zeroinc}
\end{equation}

The first main result establishes the existence of large (100 \%)
residual local minimizers for the basic FWI objective $\lerr$, even
for noise-free data.
\begin{theorem}
  \label{thm:fwi}
  Suppose that $0 <\lambda\le \lambda_{\rm max}$,  $m_* \in M, w_*
  \in \lW, d=\lF[m_*]w_*$ is noise-free data per definition \ref{eqn:defdatanonoise},
  Under assumption \ref{eqn:fullrec}, for any $m \in M$ with $|m-m_*|r>2\lambda$,
\begin{equation}
  \label{eqn:isovpm}
 \min_w \lerr[m,w;d]=\lerr[m,0;d] = \frac{1}{2},
\end{equation}
and any such $(m,0)$ is a local minimizer of $\lerr$ with relative RMS
error = 1.0.
\end{theorem}

\begin{proof} From the definition \ref{eqn:mod},
\[
 \lerr[m,w;d] =  \frac{1}{32\pi^2
    r^2\|d\|^2}\int_0^T\,dt\,\left|w\left(t-mr\right)-w_*\left(t-m_*r\right)\right|^2
\]
Since $w_*, w$ vanish for $|t|>\lambda$,
$\lF[m_*]w_*(t)$ vanishes if $|t-m_*r|>\lambda$ and $\lF[m]w$ vanishes if $|t-mr|>\lambda$. So if $|mr-m_*r|
= |m-m_*|r > 2\lambda$, then $|t-mr|+|t-m_*r| \ge |mr-m_*r| >
2\lambda$ so either $|t-mr|>\lambda$ or $|t-m_*r|>\lambda$, that is,
either $\lF[m]w(t)=0$ or $\lF[m_*]w_*=0$. Therefore $\lF[m]w$ and
$\lF[m_*]w_*$ are orthogonal in the sense of the $L^2$ inner product
$\langle \cdot,\cdot \rangle_D$ on $D$:
\begin{equation}
  \label{eqn:ortho}
  |m- m_*|r > 2\lambda \,\, \Rightarrow \,\, \langle F[m]w,
  F[m_*]w_*\rangle_D = 0
\end{equation}
But $d = \lF[m_*]w_*$, so this is the same as saying that $d$ is
orthogonal to $F[m]w$. So conclude after a minor manipulation that
\[
  |m- m_*|r > 2\lambda \,\, \Rightarrow \,\, \lerr[m,w;d]=\frac{1}{32\pi^2 
    r^2\|d\|^2}(\|w\|^2 + \|w_*\|^2).
\]
\begin{equation}
  \label{eqn:iso}
  = \frac{1}{2}\left(\frac{\|w\|^2}{\|w_*\|^2} + 1 \right)
\end{equation}
That is, for slowness $m$ in error by more than $2\lambda/r$ from the 
target slowness $m_*$, the means square error (FWI objective) $\lerr$ is independent of
$m$, and its minimum over $w$ is attained for $w=0$
\end{proof}

Therefore local minimizers of $\lerr$ abound, as far as you like from the
global minimizer $(m_*,w_*)$. Local exploration of the FWI objective
$e$ gives no useful information whatever about constructive search
directions, and descent-based optimization tends to fail if the
initial estimate $m_0$ is in error by more than $2\lambda/r$
(``further than a multiple of a wavelength'', per discussion
in the second Appendix). In fact the actual behaviour of FWI itererations is worse
(failure if $m_0$ is in error by ``half a wavelength''), as follows
from a more refined analysis of the ``cycle-skipping'' local behaviour of $\lerr$ near its
global minimizer.

\subsection{Variable Projection}
The main theoretical device used in the proofs of our main results on 
extended inversion is Variable Projection reduction of the penalty objective $\Ja$
(equation \ref{eqn:penerr}) to a function $\tJa$ of 
$m$ alone, by minimization over $w$, as described in the Examples
section. We repeat the definition here for the reader's convenience:
\begin{equation}
  \label{eqn:redexp}
  \tJa[m;d] = \inf_w \Ja[m,w;d]
\end{equation}

A minimizer $w$ on the right-hand side of definition
\ref{eqn:redexp} must solve the {\em normal equation}
\begin{equation}
  \label{eqn:norm}
  (F[m]^TF[m]+\alpha^2A^TA)w= F[m]^Td, 
\end{equation}

The choices for the penalty operator $A$ considered here are scalar 
multiplication operators on $W$ defined by a choice of multiplier $a \in L^{\infty}(\bR)$:
\begin{equation}
  \label{eqn:annmult}
  A w(t)= a(t)w(t), \, t\in \bR.
\end{equation}
With $A$ of this form, $\tilde{J}_{\alpha}$ is explicitly
computable. First observe that apart from amplitude, $F[m]$ is
unitary: for $g \in D$,
\begin{equation}
\label{eqn:tran}
F[m]^T g (t) =
\left\{
  \begin{array}{c}
    \frac{1}{4\pi r}g\left(t+mr\right), \, t \in [t_{\rm min}-mr,
    t_{\rm max}-mr],\\
    0, \mbox{ else.}
  \end{array}
\right.
\end{equation}
so
\begin{equation}
  \label{eqn:unit}
  F[m]^TF[m] = \frac{1}{(4\pi r)^2}{\bf 1}_{[t_{\rm min}-mr,  
    t_{\rm max}-mr]}
\end{equation}
in which ${\bf 1}_{S}$ denotes
multiplication by the characteristic function of a measurable 
$S \subset \bR$.

Therefore the normal equation for the minimizer on the RHS of equation \ref{eqn:redexp} is
\begin{equation}
  \label{eqn:norm1}
  \left(\frac{1}{(4\pi r)^2} {\bf 1}_{[t_{\rm min}-mr,  
      t_{\rm max}-mr]} + \alpha^2 A^TA\right)w= F[m]^Td.
\end{equation}

With these choices, the normal equation \ref{eqn:norm1} becomes
\begin{equation}
\label{eqn:norm2}
\left(\frac{1}{(4\pi r)^2}  {\bf 1}_{[t_{\rm min}-mr,  
      t_{\rm max}-mr]} + \alpha^2a^2\right)w= F[m]^Td.
\end{equation}

\begin{proposition}
  \label{thm:norminvexp}
  Assume the conditions \ref{eqn:mod}, \ref{eqn:fullrec},
  \ref{eqn:annmult}. Also assume that $\lambda \in (0,\lambda_{\rm
    max}], \alpha > 0,$ and that  $C>0$ exists so that $a \in L^{\infty}(\bR)$
  mentioned in condition \ref{eqn:annmult} satisfies condition
  \ref{eqn:abnd}.
%  \begin{equation}
%   \label{eqn:abnd}
%    |t| > \lambda \Rightarrow a(t) \ge C,
%  \end{equation}
%  in which $C>0$ may depend on $\lambda$.
  Then
  \begin{itemize}
  \item[1. ]the normal operator $F[m]^TF[m] + \alpha^2A^TA$ is
    invertible for any $m \in M$, $\alpha > 0$;
  \item[2. ]the solution $\aw[m;d]\in W$ of the normal equation
    \ref{eqn:norm} is given by
    \begin{equation}
      \label{eqn:normsol}
      \aw[m;d](t) = \left\{
        \begin{array}{c}
          \left(\frac{1}{(4\pi r)^2} + \alpha^2
          a^2(t)\right)^{-1}\frac{1}{4 \pi r}d(t+mr), t \in [t_{\rm
          min}-mr, t_{\rm max}-mr];\\
          0, \mbox{ else;}
        \end{array}
      \right.
    \end{equation}
  \item[3. ]if in addition $d=F[m_*]w_*, w_* \in \lW$ is noise-free, as in equation
    \ref{eqn:defdatanonoise},
    \begin{equation}
      \label{eqn:solnonoise}
      \aw[m,d](t)= \left(1+ (4\pi r)^2\alpha^2 a(t)^2\right)^{-1}w_*\left(t+(m-m_*)r\right).
    \end{equation}
  \end{itemize}
\end{proposition}

\begin{proof} 
  \begin{itemize}
  \item[1. ]Note that thanks to \ref{eqn:zeroinc}, if $|t|\le
    \lambda \le \lambda_{\rm max}$, then ${\bf 1}_{[t_{\rm min}-mr,  
      t_{\rm max}-mr]}(t) = 1$, whereas if $|t|>\lambda$,
    then $a(t) \ge C$, whence
    \[
      \frac{1}{(4\pi r)^2}  {\bf 1}_{[t_{\rm min}-mr,  
        t_{\rm max}-mr]} + \alpha^2a^2  \ge \min\{(4\pi r)^2,
      \alpha^2\min\{1/(4\pi r)^2,C^2\} \}> 0.
    \]
    Therefore the normal operator is invertible under the stated
    conditions.

  \item[2. ]From the identity \ref{eqn:tran}.
    \[
      \mbox{supp }F[m]^Td \subset [t_{\rm min}-mr,t_{\rm max}-mr].
    \]
    Define $w_{\rm tmp}$ to be the right-hand side of equation \ref{eqn:normsol}. Then
    from the previous observation and identity \ref{eqn:tran},
    \[
      \mbox{supp }w_{\rm tmp} \subset [t_{\rm min}-mr,t_{\rm max}-mr].
    \]
    From the identity \ref{eqn:unit}, for any $w \in W$,
    \[
      t \in [t_{\rm min}-mr,t_{\rm max}-mr] \Rightarrow F[m]^TF[m]w(t)
      = \frac{1}{(4 \pi r)^2}w(t).
    \]
    It follows from this and the previous two observations that
    $w_{\rm tmp}$ solves the normal equation \ref{eqn:norm}, and
    therefore that $\aw[m;d]=w_{\rm tmp}$.

  \item[3. ]Follows by inserting the definition
    \ref{eqn:defdatanonoise} of $d$ in \ref{eqn:normsol} and
    rearranging.
  \end{itemize}
\end{proof}

\begin{theorem}
  \label{thm:norminv}
  Assume the condition \ref{eqn:fullrec}, $C>0$, and suppose that $A$ is
  given by equation \ref{eqn:annmult} for $a \in L^{\infty}(\bR)$
  satisfying
  \begin{equation}
    \label{eqn:abnd} 
    a \ge 0; \, a(t) \ge C\mbox{ for }|t| \ge \lambda_{\rm max}.
  \end{equation}
  Then \begin{itemize}
  \item[1. ]the reduced objective $\tJa$ is given by
    \begin{equation}
      \label{eqn:redexp1}
      \tJa[m;d] = \Ja[m,\aw[m;d];d],
    \end{equation}
    in which $\aw[m;d] \in W$ is the unique solution of the normal
    equation \ref{eqn:norm}.
  \item[2. ]The following are equivalent:
    \begin{itemize}
    \item[i. ]$(m,w) \in M \times W$ is a local minimizer of
      $\Ja[\cdot,\cdot;d]$, and
    \item[ii. ]$m$ is a local minimizer of $\tJa[\cdot;d]$ and
      $w=\aw[m;d]$.
    \end{itemize}
  \end{itemize}
\end{theorem}

\begin{proof} These conclusions follow immediately from Proposition
  \ref{thm:norminvexp}.
\end{proof}

\noindent{\bf Remark:} If $\Ja[\cdot,\cdot;d]$ and $\tJa[\cdot;d]$
were differentiable, then ``local minimizer'' in the conclusion of the
preceding theorem could be replaced by ``stationary point''. However,
for the problem addressed in this paper, $\Ja[\cdot,\cdot;d]$ {\em is
  not} differentiable without added smoothness constraints on $w$,
whereas $\tJa[\cdot;d]$ {\em is} differentiable for proper choice of
penalty operator $A$. 

This conclusion follows from properties of the modeling operator $F$
shared with many other inverse problems in wave propagation, as
explained in the third appendix. Here, we derive it from explicit
expressions for $\tJa$ and its components.

\begin{proposition}
  \label{thm:epjgen}
  Assume the hypotheses of Proposition \ref{thm:norminvexp}. Then
  \begin{equation}
  \label{eqn:residnormgen}
  e[m,\aw[m,d];d] = \frac{1}{2\|d\|^2}\int_{t_{\rm min}}^{t_{\rm max}} \,dt\,(4\pi r \alpha a(t-mr))^4(1 +
  (4\pi r \alpha a(t-mr))^2)^{-2}d(t)^2
\end{equation}
\begin{equation}
  \label{eqn:anninormgen}
  p[m,\aw[m,d];d] = \frac{1}{2\|d\|^2}\int_{t_{\rm min}}^{t_{\rm max}} \,dt\,(4\pi r a(t-mr))^2(1 +
  (4\pi r \alpha a(t-mr))^2)^{-2}d(t)
\end{equation}

\begin{equation}
  \label{eqn:expjgen}
\tJa[m;d] = \frac{1}{2\|d\|^2}\int_{t_{\rm min}}^{t_{\rm max}}\,dt\,(4\pi r \alpha a(t-mr))^2(1+(4\pi r \alpha 
a(t-mr))^2)^{-1}d(t)^2. 
\end{equation}
\end{proposition}

\begin{proof}
  From equation \ref{eqn:normsol},
  \[
    F[m]\aw[m;d](t) = 
    \frac{1}{4 \pi r}\left(\frac{1}{(4\pi r)^2} + \alpha^2
      a^2(t-mr)\right)^{-1}\frac{1}{4 \pi r}d(t),
  \]
  so
  \[
    (F[m]\aw[m;d]-d)(t) = (1 + (4 \pi r\alpha
    a(t-mr))^2)^{-1}-1)d(t)
  \]
  \[
    = -(4 \pi r\alpha a(t-mr))^2(1 + (4 \pi r\alpha
    a(t-mr))^2)^{-1}d(t).
  \]
  Half the integral of the square of this data residual is
  $e[m,\aw[m;d],d]$, which proves identity \ref{eqn:residnormgen}.

  To compute $p[m,\aw[m;d],d]$, note that
  \[
    A\aw[m;d](t)=a(t) \left(\frac{1}{(4\pi r)^2} + \alpha^2
      a^2(t)\right)^{-1}\frac{1}{4 \pi r}d(t+mr)
  \]
  \[
    = 4\pi r a(t) (1 + (4\pi r \alpha a(t))^2)^{-1}d(t+mr)
  \]
  for $ t \in [t_{\rm min}-mr, t_{\rm max}-mr]$, so squaring,
  integrating, and changing integration variables $t \mapsto t-mr$
  gives the result \ref{eqn:anninormgen}

  That the VPM objective $\tJa$ is given by \ref{eqn:expjgen} follows from equations \ref{eqn:pen},
  \ref{eqn:redexp}, \ref{eqn:residnormgen}, and
  \ref{eqn:anninormgen}.
\end{proof}


\begin{theorem}
  \label{thm:diffobj}
  Suppose that in addition to the hypotheses of Theorem
  \ref{thm:norminv}, $a \in W^{1,\infty}_{\rm loc}(\bR)$, then $\tJa[\cdot;d]
  \in C^1(M)$.
\end{theorem}

\begin{proof}
Suppose first that $a \in C^1(\bR)$. Differentiation under the integral sign  
  yields the expression for its derivative:
\begin{equation}
  \label{eqn:dexpjgen}
  \frac{d}{dm}\tJa[m;d] = -\frac{(4 \pi r \alpha)^2}{\|d\|^2} \int_{t_{\rm min}}^{t_{\rm max}} \,dt \, 
  \left(a\frac{da}{dt}\right)(t-mr)(1+(4\pi r \alpha 
  a(t-mr))^2)^{-2}d(t)^2. 
\end{equation}
For $a \in W^{1,\infty}_{\rm loc}(\bR)$ a limiting argument shows that the
same expression gives the derivative of $\tJa$.
\end{proof}

It will be useful to record expressions for the various componenets of
$\tJa$ when the data is noise-free, that is, the context of
Proposition \ref{thm:norminvexp}, item 3.

\begin{corollary}
  \label{thm:epjnonoise}
  Assume the hypotheses of Proposition \ref{thm:norminvexp}, item
  3. Then noting that $\|d\| = \|w_*\|/(4 \pi r)$
\begin{equation}
  \label{eqn:residnorm}
  e[m,\aw[m,d];d] 
= \frac{\alpha^4}{2\|w_*\|^2}\int\,dt\,a(t-(m-m_*)r)^4(1+(4\pi r)^2 \alpha^2 
    a(t-(m-m^*)r)^2)^{-2}w_*(t)^2.
\end{equation}
\begin{equation}
  \label{eqn:anninorm}
  p[m,\aw[m,d];d] = \frac{(4\pi r)^2}{2\|w_*\|^2}\int \,dt\,  a(t-(m-m_*)r)^2 \left(1+ (4\pi r)^2\alpha^2
    a(t-(m-m_*)r)^2\right)^{-2}w_*(t)^2.
\end{equation}
so 
\begin{equation}
\label{eqn:expjnonoise}
\tJa[m;d] = \frac{(4\pi r \alpha)^2}{2\|w_*\|^2}\int\,dt\,a(t-(m-m_*)r)^2(1+(4\pi r)^2 \alpha^2 
  a(t-(m-m_*)r)^2)^{-1}w_*(t)^2. 
\end{equation}
Finally, if $a \in W^{1,1}_{\rm loc}(\bR)$, then $\tJa[\cdot;d]$ is differentiable, and 
\begin{equation}
  \label{eqn:dexpjnonoise}
  \frac{d}{dm}\tJa[m;d] = -\frac{r (4\pi r \alpha)^2}{\|w_*\|^2} \int \,dt \, 
  \left(a\frac{da}{dt}\right)(t-(m-m_*)r)(1+(4\pi r)^2 \alpha^2 
  a(t-(m-m_*)r)^2)^{-2}w_*(t)^2. 
\end{equation}
\end{corollary}

\subsection{Choice of Penalty Operator}

We examine two choices for $A$. For each choice, we ask first whether local minimizers 
of the resulting VPM objective $\tJa[\cdot;d]$ occur far from a slowness $m_*$. 

A penalty operator $A$ of which $\lW$ is the null space would be a
natural choice. Such operators have come to be called
``annihilators'', since they map all members of the constraint
subspace $\lW$ to zero.  A simple example is
\begin{eqnarray}
  A = E^c_{\lambda}&=&I - E_{\lambda},\mbox{ where } \nonumber \\
  E_{\lambda}w(t) &=&{\bf 1}_{[-\lambda,\lambda]}(t)w(t). 
                      \label{eqn:ann0}
\end{eqnarray}
That is, $E_{\lambda}$ is the orthogonal projector onto $\lW$,
and $E_{\lambda}^c$ is the orthogonal projector onto its
orthocomplement, an operator of the form \ref{eqn:annmult} with $a
= 1 - {\bf 1}_{[-\lambda,\lambda]}$. %Note that this choice of $a$
%satisfies condition \ref{eqn:abnd} with $C \equiv 1$.

\begin{theorem}
  \label{thm:boxcarbad}
  Suppose that
  \begin{itemize}
  \item[1. ] $\lambda \in (0,\lambda_{\rm max}]$;
  \item[2. ] $m_*\in M, w_*\in \lW$, and $d=F[m_*]w_*$ (noise-free
    data);
  \item[3. ] $A=E^c_{\lambda}$, that is, $a=1-{\bf 1}_{-\lambda,\lambda]}$ in
    the definition \ref{eqn:annmult}.
  \end{itemize}
  Then if $|m-m_*| >  2\lambda/r$,
  \[
    \tJa[m;d] = \frac{(4\pi r \alpha)^2}{2(1+(4 \pi r \alpha)^2)}.
  \]
\end{theorem}

\noindent {\bf Remark.}
One might have thought that $A=E^c_{\lambda}$ would be a better choice
of annihilator, as for noise-free
data, the solution set defined by the problem statement
\ref{eqn:probstat0} is the same as the set of global minimizers of
$\Ja$ in this case. However, for this choice of annihilator,
$t\Ja$ exhibits the same feature as the mean square error $e$, namely
a continuum of local minimizers at any distance from the global
minimizer $m_*$ greater than a multiple of $\lambda$. Therefore the
extended inversion with this choice of annihilator is no more amenable
to local optimization than is FWI.

A second possible 
penalty operator penalizes energy away from 
$t=0$: choose $\tau > 0$ and set 
\begin{equation}
  \label{eqn:ann}
  a(t) = \min(|t|, \tau). 
\end{equation}
Note that $a \ge 0$ and $a \in L^{\infty}(\bR) \cap W^{1,1}_{\rm
  loc}(\bR)$. The cutoff $\tau$ will be chosen large enough to be effectively inactive: 
specifically, hindsight suggests 
\begin{equation}
  \label{eqn:taudef}
  \tau = \max\{|t_{\rm min}-m_{\rm min}r|,|t_{\rm min}-m_{\rm max}r|, |t_{\rm max}-m_{\rm min}r|, |t_{\rm max}-m_{\rm max}r|\}. 
\end{equation}
This 
particular annihilator has been employed in earlier papers on extended 
source inversion 
\cite[]{Plessix:00a,LuoSava:11,Warner:14,HuangSymes:SEG15a,Warner:16,HuangSymes:GEO17}.

\begin{theorem}
  \label{thm:mnoiseres}
  Asumme that
  \begin{itemize}
  \item[1. ] $\alpha, \mu> 0$,
  \item[2. ] $m_* \in M, w_* \in W_{\mu}$,
  \item[3. ] $d_* = F[m_*]w_*$,
  \item[4. ] $n \in D$ and $d = d_* + n$.
  \end{itemize}
  Set $\eta = \|n\|/\|d_*\|$, and assume further that
  \begin{itemize}
  \item[5. ]
    \begin{equation}
      \label{eqn:mnoisecond}
      \eta < \frac{\sqrt{5}-1}{2}, and
    \end{equation}
  \item[6. ] $m$ is a stationary point of $\tJa[\cdot;d]$.
  \end{itemize}
  Then
  \begin{equation}
    \label{eqn:mnoisesuff}
    |m-m_*| \le \left(1+\frac{2\eta(1+\eta)}{1-\eta(1+\eta)}\right)\frac{\mu}{r}.
  \end{equation}
\end{theorem}

\noindent {\bf Remark.} That is, {\em with the choice of penalty
  multiplier $a$ given in equation \ref{eqn:ann}, and support radius
  $\mu$ of the ``noise-free'' wavelet $w_*$, $\Ja$ has no local
  minima with slownesses further than $(1+O(\eta))\mu/r)$ from the slowness used to
  generate the data}.

\noindent {\bf Remark.} The estimate $|m-m_*|r<\mu(1+O(\eta))$ for
local minima of $\Ja$ is sharp: it is possible to choose $w_* \in W_{\mu}$
so that $\mu - |m-m_*|r$ is as small as you like. In particular,
the ``exact'' or ``true'' slowness $m_*$ is not necessarily the only
slowness component of a local minimizer, or even the slowness
component of any local minimizer, and in particular
is not (necessarily) the slowness component of a global minimizer of $\Ja$.

\noindent {\bf Remark.} No similar bound could hold for much larger
noise levels than specified in condition \ref{eqn:mnoisecond}, the
right-hand side of which is a bit larger than 0.6. For example, if the noise is
the predicted data for the same wavelet $w_*$ with a substantially different
slowness $m_{\flat}$, that is, $n=F[m_{\flat}]w_*$, then a simple
symmetry argument shows that if there is a local minimizer of $\Ja[\cdot,\cdot;d_*+n]$. with
slowness near $m_*$,
there must also be a minimizer with slowness near $m_{\flat}$.
so that the difference with $m_*$ is not constrained at
all by the assumed support radius of $w_*$. So for this example with 100\% noise, no
bound of the type given by conclusion 2 could possibly hold. We will
illustrate this phenomenon numerically later in the paper.

\noindent{\bf Remark.} Note that $\alpha$ plays no role in the
conclusions of this theorem. It is only required that $\alpha >0$.

Unless the data is noise-free, there is no reason to suppose that the
estimated wavelet $\aw[m;d]$ (Theorem \ref{thm:norminv}) will lie in
$\lW$, unless the support of the noise $n$ is not restricted. In order
to construct a solution of the inverse problem \ref{eqn:probstat0}, we
project $\aw[m;d]$ onto $\lW$. For sufficiently large $\lambda,
\epsilon$, the result is a solution of the inverse problem:

\begin{theorem}
  \label{thm:ipnoisesuf}
  Assume the hypotheses of Theorem \ref{thm:mnoiseres}, and that
  inequality \ref{eqn:mnoisecond} holds, and $\mu \in
  (0,\lambda_{\rm max}]$. Then the pair
  \[
    (m,{\bf 1}_{[-\lambda,\lambda]}\aw[m,d])
  \]
  solves the inverse problem as stated in \ref{eqn:probstat0} if
  \begin{equation}
    \label{eqn:ipnoiselam}
    \left( 2+\frac{2\eta(1+\eta)}{1-\eta(1+\eta)}\right)\mu \le \lambda
    \le \lambda_{\rm max}, 
  \end{equation}
  and
  \begin{equation}
    \label{eqn:ipnoiseeps}
    \epsilon \ge (4 \pi r \alpha\lambda)^2+\eta. 
  \end{equation}    
\end{theorem}

\noindent {\bf Remark:} Note that the sufficient condition \ref{eqn:ipnoiselam} for
$\lambda$ is independent of $\alpha$. It follows that for any choice
of $\lambda$ consistent with this bound, $(m,{\bf
  1}_{[-\lambda,\lambda]}\aw[m,d])$ is a solution of the inverse
problem for any $\epsilon > 0$ provided that $\alpha$ is chosen
sufficiently small ($O(\sqrt{\epsilon})$).



\subsection{Proof of Theorem \ref{thm:fwi}}


%A natural formulation of the Problem Statement \ref{eqn:probstat0} is
%via least squares: that is, given $d \in D$, $\lambda >0$, find
%$(m,w) \in M \times \lW$ to minimize
%$\lerr: M \times W \rightarrow \bR$, defined in display
%\ref{eqn:redms}.  A pair $(m,w) \in M\times \lW$ satisfies the
%condition \ref{eqn:probstat0} iff $\lerr[m,w;d] \le \epsilon^2/2$, so a
%solution exists iff a global minimizer of $\lerr$ satisfies this
%inequality.

%In this discussion, I will presume that $d$ is 
%noise-free, as defined at the end of the last section (equation
%\ref{eqn:defdatanonoise}). Thus {\em a priori} a global minimizer of
%$e$ is a solution of the Problem Statement for any
%$\epsilon > 0$. The only
%question is how to find such a global minimizer. As mentioned in the
%introduction, the simple model problem studied here is a proxy for
%large-scale data fitting problems for which the only computationally
%feastible approach to minimization of analogs of $e$
%is some form of descent-based iterative local optimization. Any such
%algorithm requires the specification of an initial iterate
%$(m_0,w_0)$. Therefore the key question is: what
%constraints must be imposed on $(m_0,w_0)$ in order that the resulting
%sequence of iterates convertes to a global minimizer?  


%\begin{theorem}
%  \label{thm:fwi}
%  Suppose that $0 <\lambda\le \lambda_{\rm max}$,  $m_* \in M, w_*
%  \in \lW, d=\lF[m_*]w_*$ is noise-free data per definition \ref{eqn:defdatanonoise},
%  Under assumption \ref{eqn:fullrec}, for any $m \in M$ with $r|m-m_*|>2\lambda$,
%\begin{equation}
%  \label{eqn:isovpm}
% \min_w e[m,w;d]=e[m,0,d]=\frac{1}{32\pi^2 r^2}\|w_*\|^2,
%\end{equation}
%and any such $(m,0)$ is a stationary point of $e$.
%\end{theorem}





%\begin{theorem}
%\label{thm:red}
%$\Ja[m,\cdot;d]$ is invertible:
%\begin{equation}
%\label{eqn:norminv}
%F[m]^TF[m] +\alpha^2 A^TA \in {\cal I}(W,W).
%\end{equation}
%(The superscript $T$ denotes the adjoint with respect to the inner products 
%in $W=L^2(\bR)$ and $D=L^2([t_{\rm min},t_{\rm max}])$.)
%Then the unique stationary point $\aw[m;d] \in W$ of
%$\Ja[m,\cdot;d]$ is a global minimizer, and is the solution of the {\em normal equation}:
%\begin{equation}
 % \label{eqn:norm}
  %(F[m]^TF[m]+\alpha^2A^TA)w= F[m]^Td, 
%\end{equation}
%The reduced or variable projection objective $\tJa$ defined in
%equation \ref{eqn:redexp} is also given by
%\begin{equation}
 % \label{eqn:redexp1}
 % \tJa[m;d] =\Ja[m,w_{\alpha}[m;d];d] = \min_w \Ja[m,w;d]
%\end{equation}
%Then $(m,w)$ is a local minimizer of $\Ja[\cdot,\cdot;d]$ if and only
%if $m$ is a local minimizer of $\tJa[\cdot;d]$ and $w=\aw[m;d]$.
%\end{theorem}

%\noindent{\bf Remark:} The choice
%of penalty weight $\alpha$ has a significant influence on the
%performance of numerical optimization algorithms in minimizing
%$\Ja$ and/or $\tJa$, but relatively little influence on the error estimates to
%follow. For the discussion of these estimates, we assume only that
%$\alpha>0$. Appendix A contains a description of an algorithm for
%constructive determination of $\alpha$.

\subsection{Proof of Theorem \ref{thm:norminv}}



%\begin{proof} of Theorem \ref{thm:norminv}: Since the normal operator
%  is invertible (item 1 of Proposition \ref{thm:norminvexp}), the
%  conclusions of this theorem follow from the Lax-Milgram Theorem
%  (\cite{Yosida}, III.7).
%\end{proof}

\subsection{Proof of Theorem \ref{thm:diffobj}}



\subsection{Proof of Theorem \ref{thm:boxcarbad}}


\begin{proof} of Theorem \ref{thm:boxcarbad}
  
 This is clear from the definition $a = 1-{\bf 1}_{-\lambda,\lambda]}$ and equation \ref{eqn:expjnonoise}, as the supports of $w_*$ and ${\bf
    1}_{[-\lambda,\lambda]}(t-(m-m_*))$ are disjoint for the range of
  $m$ identified in the theorem.
\end{proof}

\subsection{Proof of Theorem \ref{thm:mnoiseres}}
\begin{proposition}
  \label{thm:rampgood}
  Suppose that
  \begin{enumerate}
  \item $m_* \in M$;
  \item $0 < \mu \le \lambda$, and $w_* \in W_{\mu}$;
  \item $d_* = F[m_*]w_*$;
  \item $a(t)=\min\{|t|,\tau\}$ in the definition \ref{eqn:annmult},
    with $\tau$ given by equation \ref{eqn:taudef}; and
  \item $\alpha > 0$.
  \end{enumerate}
  Then for any $m \in M$, 
  \begin{equation}
    | (m - m_*)r| > \lambda  \Rightarrow  \left|\frac{d}{dm}\tJa[m;d_*]\right| > \alpha^2 
    \frac{\lambda-\mu}{(1+(4\pi r)^2\alpha^2 
      (\lambda+\mu)^2)^{2}} \|w_*\|^2 
    \label{eqn:gradbndnonoise}
  \end{equation}
\end{proposition}
\begin{proof}
  As observed before, $\mbox{supp }\aw[m;d_*] \subset [t_{\rm
    min}-mr,t_{\rm max}-mr]\subset [-\tau,\tau]$, with $\tau$ defined
  in \ref{eqn:taudef}. Therefore, $a(t) = |t|$, $a a'(t) = t$ in the
  support of the integrand on the RHS of equation
  \ref{eqn:dexpjnonoise}, which therefore 
  becomes (after change of integration variable)
  %%%%%%%%%%%%%%%%%%%%%%%%%
  \begin{equation}
    \label{eqn:gradfinal}
    \frac{d}{dm}\tJa[m;d_*] = -r \alpha^2 \int \,dt \, 
  t(1+(4\pi r)^2 \alpha^2 
  t^2)^{-2}w_*(t+(m-m_*))^2.
  \end{equation}
  Recall that $w_*(t+(m-m_*)r)$
  vanishes if $|t+(m-m_*)r| > \lambda$. Therefore the integral on the
  RHS of equation \ref{eqn:gradfinal} can be re-written
  \[
    = -r\alpha^2\int_{-(m-m_*)r-\lambda}^{-(m-m_*)r+\lambda}
    \,dt\, t(1+(4\pi r)^2\alpha^2 t^2)^{-2}w_*\left(t+(m-m_*)r\right)^2
  \]
  Suppose that $\mu \le \lambda$ and $w_* \in W_{\mu}$. 
  If $m > m_*+\lambda/r$, then $t+(m-m_*)r \in \mbox{supp }w_*$
  implies $-\mu - \lambda < t < \mu-\lambda<0$, so 
  \[
    t(1+(4\pi r)^2\alpha^2 t^2)^{-2} < (\mu-\lambda)(1+(4\pi r)^2\alpha^2 (\mu+\lambda)^2)^{-2}<0
  \]
  in the support of the integrand in equation
  \ref{eqn:gradfinal}. Arguing similarly for $m<m_*-\lambda/r$, obtain
  a similar inequality, implying the conclusion \ref{eqn:gradbndnonoise}.
\end{proof}

\begin{corollary}
  \label{thm:rampreallygood}
  Suppose that
  \begin{enumerate}
  \item $m_* \in M$;
  \item $0 <  \lambda$, and $w_* \in \lW$;
  \item $d_* = F[m_*]w_*$;
  \item $a(t)=\min\{|t|,\tau\}$ in the definition \ref{eqn:annmult},
    with $\tau$ given by equation \ref{eqn:taudef}; 
  \item $\alpha > 0$; and
  \item$m \in M$ is a stationary point of $\tJa[\cdot;d_*]$.
  \end{enumerate}
  Then $|m-m_*| < \lambda /r$.
\end{corollary}

\begin{proof} Follows directly from Theorem \ref{thm:rampgood} by
  taking $\mu=\lambda$.
\end{proof}

The preceding theorem established that a proper choice of annihilator
leads to a reduced penalty objective all of whose stationary points
are within $O(\lambda)$ of the target slowness $m_*$, provided that
the data are noise-free in the sense of equation
\ref{eqn:defdatanonoise}. This result leaves open two questions:
\begin{itemize}
\item how does one use this reduced penalty minimization to produce
  a solution of the inverse problem as in problem statement
  \ref{eqn:probstat0}? 
\item how does one answer the same question for noisy data?
\end{itemize}

The next result answers the first question, in the case of noise-free data:
\begin{proposition}
  \label{thm:ipnonoisesuf}
  Suppose that $a$ is given by definition \ref{eqn:ann}, $\alpha$,
  $\mu \in (0,\lambda_{\rm max}]$,
  $d$ is given by
  \ref{eqn:defdatanonoise} with $w_* \in W_{\mu}$, and  $m_{\infty}$ is a stationary
  point of $\tJa[\cdot;d]$. Then $(m_{\infty},\aw[m_{\infty};d])$ is a
  solution of the inverse problem \ref{eqn:probstat0} for any $\lambda
  \ge 2\mu$ and $\epsilon \ge (4\pi r \lambda \alpha)^2$.
\end{proposition}

\begin{proof}
  From the assumption $w_* \in W_{\mu}$ and Corollary
  \ref{thm:rampreallygood}, $|(m_{\infty}-m_*)r|\le \mu$. From the
  identity \ref{eqn:solnonoise},
  $\mbox{supp }\aw[m_{\infty};d] \subset
  [(m_{\infty}-m_*)r-\mu,(m_{\infty}-m_*)r+\mu] \subset
  [-2\mu,2\mu]$. Because of the support limitation, $a(t)=|t|$ in the
  interval of integration appearing in \ref{eqn:residnorm}, so
\[
  e[m_{\infty},\aw[m_{\infty},d];d] 
= 8 \pi^2 r^2 \alpha^4\int^{\mu}_{-\mu}\,dt\,|t-(m_{\infty}-m_*)r|^4(1+(4\pi r)^2 \alpha^2 
|t-(m_{\infty}-m^*)r|^2)^{-2}w_*(t)^2.
\]
and therefore
\begin{equation}
  \label{eqn:estresidnorm}
e[m_{\infty},\aw[m_{\infty};d];d] \le 8 \pi^2 r^2 (2\mu\alpha)^4 \|w_*\|^2 =
\frac{1}{2}(8 \pi r \mu \alpha)^4 \|d\|^2
\end{equation}
\end{proof}

The inequality \ref{eqn:estresidnorm} can be interpreted as a bound 
on $\alpha$, given $\epsilon$ and $\lambda$, for a
stationary point of $\tJa$ to yield a solution of the inverse
problem: one obtains a solution, provided that $\alpha$ is
sufficiently small. On the other hand, it is clear that $\alpha$
cannot be too large if stationary points of $\tJa$ are to yield
solutions: the integrand in \ref{eqn:residnorm} is increasing in
$\alpha$ for every $t$ and $m$, and the multiplier
\[
t \mapsto (4\pi r \alpha(t-(m_{\infty}-m_*)r))^4(1+(4\pi r)^2 \alpha^2 
|t-(m_{\infty}-m^*)r|^2)^{-2}
\]
tends monotonically to $1$ as $\alpha \rightarrow \infty$, uniformly
on the complement of any open interval containing
$t=(m_{\infty}-m_*)r$. Therefore
\begin{equation}
  \label{eqn:elimit}
  \lim_{\alpha \rightarrow \infty}e[m_{\infty},\aw[m_{\infty};d];d] =
  \frac{1}{2}\frac{1}{(4 \pi r)^2}\|w_*\|^2 = \frac{1}{2}\|d\|^2.
\end{equation}
Consequently, there exists $\alpha_{\rm max}(\epsilon,\lambda,d)$ so
that
\[
  e[m_{\infty},\aw[m_{\infty};d];d]  \le \frac{1}{2}\epsilon^2\|d\|^2
  \Rightarrow \alpha \le \alpha_{\rm max}(\epsilon,\lambda,d).
\]
The existence of this limiting penalty weight has been inferred
indirectly; the next section is devoted to a constructive algorithm
for its approximation.

We turn now to the second issue identified above, the effect of
noise. Suppose that the data trace $d$ takes the form
\begin{equation}
  \label{eqn:defdatanoisy}
  d = F[m_*]w_* + n = d_*+n,
\end{equation}
with $m_* \in M, w_* \in W_{\mu}$, $0<\mu<\lambda$, and noise trace $n \in
D$. Since no support assumptions can be made about $n$, equation
\ref{eqn:normsol} implies that $\aw[m;d] \notin \lW$ for any values of
$\alpha$ and $\lambda$.  Therefore minimization of $\tJa$ cannot by itself yield a
solution of the inverse problem as defined in the problem statement
\ref{eqn:probstat0}. In this section, we explain how a solution may
nontheless be constructed from a stationary point of $\tJa$.

First we examine the effect of additive noise on the estimation of the
slowness $m$. In expressing the result, we use the dimensionless
relative data error
\begin{equation}
  \label{eqn:defeta}
  \eta = \frac{\|n\|}{\|d_*\|}. 
\end{equation}

\begin{proposition}
  \label{thm:mnoise}
  Assume the hypotheses of Proposition \ref{thm:rampgood}, and that $d$ is
  given by definition \ref{eqn:defdatanoisy}. Suppose that $m \in M$
  is a stationary point of $\tJa[\cdot;d]$, and that
  \begin{equation}
    \label{eqn:mnoisebnd}
    \eta(1+\eta) \le \frac{16}{3\sqrt{3}}\frac{4\pi r \alpha
      (\lambda-\mu)}{(1+(4\pi r\alpha(\lambda+\mu))^2)^2}
  \end{equation}
  Then
  \begin{equation}
    \label{eqn:mnoisebndfin}
    |m-m_*| \le \frac{\lambda}{r}
  \end{equation}.
\end{proposition}

\begin{proof}
  From equation \ref{eqn:dexpjgen}, $d \tJa / dm$ is the
value of a quadratic form in $d$ with (indefinite) symmetric operator
$B = $ multiplication by
\[
  b(t;m,\alpha)  = -\frac{(4 \pi r \alpha)^2 (t-mr)}{(1+(4\pi r \alpha (t-mr))^2)^{2}}
\]
Therefore
\begin{equation}
  \label{eqn:gradlip}
  \left|\frac{d}{dm}\tJa[m;d]-\frac{d}{dm}\tJa[m;d_*]\right| =
  |\langle (d+d_*),B(d-d_*)\rangle| \le \max_{t \in
    \bR}|b(t;m,\alpha)|\eta(1+\eta)\|d_*\|^2
\end{equation}
A straightforward calculation shows that
\[
  \max_{t \in \bR} b(t;m,\alpha) = \frac{3\sqrt{3}}{16} 4\pi r\alpha.
\]
For a stationary point $m$ of
$\tJa[\cdot;d]$, the inequality \ref{eqn:gradlip} implies
\[
  \left|\frac{d}{dm}\tJa[m;d_*]\right| \le \frac{3\sqrt{3}}{16} 4\pi
  r\alpha \eta(1+\eta)\|d_*\|^2
\]
On the other hand, the conclusion \ref{eqn:gradbndnonoise} of Proposition
\ref{thm:rampgood} implies that if also
\[
  \frac{3\sqrt{3}}{16} 4\pi r\alpha \eta(1+\eta)\|d_*\|^2 \le (4 \pi r
  \alpha)^2 \frac{\lambda-\mu}{(1+(4\pi r)^2\alpha^2
    (\lambda+\mu)^2)^{2}} \|d_*\|^2
\]
then $|m-m_*|\le \lambda/r$. Rearranging, obtain the conclusion.
\end{proof}

\begin{corollary}
  \label{thm:mnoisecor}
  Asumme the hypotheses of Theorem \ref{thm:mnoise}, in particular
  that $m$ is a stationary point of $\tJa[\cdot;d]$, $d=d_*+n$. Then
  \begin{equation}
    \label{eqn:mnoisecor}
    |m-m_*| \le \frac{\mu}{r} + \frac{\eta}{\alpha} \left(\frac{3\sqrt{3}(1+\eta)}{64\pi r^2}(1+(8\pi r \alpha
      \lambda_{\rm max})^2)^2\right)
  \end{equation}
\end{corollary}

\begin{proof} Assume that $\lambda$ is chosen to obtain equality in
  the condition \ref{eqn:mnoisebnd}, substitute the bound $2
  \lambda_{\rm max}$ for $\lambda + \mu$ in the denominator, solve for
  $\lambda$ and substitute in inequality\ref{eqn:mnoisebndfin}.
\end{proof}

\noindent {\bf Remark:} This is the only error bound on $m$ in which
the size of $\alpha$ plays a significant role. It
suggests that error in the estimate of $m$ due to data noise is a
decreasing function of $\alpha$, at least for small $\alpha$. This result is intuitively
appealing, and is supported by numerical evidence, as shown below.

%%%%%%%%%%%
\begin{proposition}
  \label{thm:wnoiseres}
  Asumme that
  \begin{itemize}
  \item[1. ] $\alpha, \mu> 0$,
  \item[2. ] $m_* \in M, w_* \in W_{\mu}$,
  \item[3. ] $d_* = F[m_*]w_*$,
  \item[4. ] $n \in D$ and $d = d_* + n$.
  \end{itemize}
  Set $\eta = \|n\|/\|d_*\|$, and assume further that
  \begin{itemize}
  \item[5. ]
    \begin{equation}
      \label{eqn:mnoisecond}
      \eta < \frac{\sqrt{5}-1}{2}, and
    \end{equation}
  \item[6. ] $m$ is a stationary point of $\tJa[\cdot;d]$.
  \end{itemize}
  Then
  \begin{equation}
    \label{eqn:mnoisesuff}
    |m-m_*| \le \left(1+\frac{2\eta(1+\eta)}{1-\eta(1+\eta)}\right)\frac{\mu}{r}.
  \end{equation}
\end{proposition}

\begin{proof}
  From equation \ref{eqn:normsol},
  \[
    \aw[m;d](t) = 
    \left(\frac{1}{(4\pi r)^2} + \alpha^2 t^2)\right)^{-1}\frac{1}{4 \pi r}d(t+mr)
  \]
  \[
    =4\pi r(1 + (4\pi r \alpha t)^2)^{-1} (\frac{1}{4\pi r}
    w_*(t+(m-m_*)r) + n(t+mr) 
  \]
  so
  \begin{equation}
    \label{eqn:wres}
    \aw[m;d](t) - w_*(t) = (4\pi r \alpha t)^2 (1+ (4\pi r \alpha t)^2)^{-1}w_*(t+(m-m_*)r)
    +w_*(t+(m-m_*)r)-w_*(t) + 4 \pi r (1 + (4\pi r \alpha t)^2)^{-1}n(t+mr)
  \end{equation}
\end{proof}
%  \begin{itemize}
%    \item[1. ] the bound \ref{eqn:mnoisebnd} holds only if 
%      noise level $\eta=\|n\|/\|d_*\|$ satisfies
%      \begin{equation}
%        \label{eqn:mnoisecond}
%        \eta < \frac{1+\sqrt{5}}{2} < 1;
%      \end{equation}
%    \item[2. ] if condition \ref{eqn:mnoisecond} holds, then
%      $|m-m_*|<\lambda/r$ provided that
%      \begin{equation}
%        \label{eqn:mnoisesuff}
%        \frac{\lambda}{\mu} \ge 1+\frac{2\eta(1+\eta)}{1-\eta(1+\eta)}. 
%      \end{equation}
%      \begin{equation}
%        \label{eqn:mnoisesuff}
%        |m-m_*| \le \left(1+\frac{2\eta(1+\eta)}{1-\eta(1+\eta)}\right)\frac{\mu}{r}. 
%      \end{equation}
%    \end{itemize}
%\end{corollary}

\begin{proof} of Theorem \ref{thm:mnoiseres}:
  Write $\lambda = (1+\delta)\mu$, and $x=4 \pi r \alpha \mu$. Then
  the right-hand side of equation \ref{eqn:mnoisebnd} may be written as
  \begin{equation}
    \label{eqn:mnoisebndrev}
    \frac{16}{3\sqrt{3}}\frac{4\pi r \alpha
      (\lambda-\mu)}{(1+(4\pi r\alpha(\lambda+\mu))^2)^2} = D
    \frac{x}{(1+C^2 x^2)^2},
  \end{equation}
  where
  \[
    D=\frac{16}{3\sqrt{3}}\delta,\,C=2+\delta.
  \]
  The positive stationary point of the quantity on the right-hand side
  of \ref{eqn:mnoisebndrev} is a maximum, and occurs at
  $x=1/(\sqrt{3}C)$, that is
  \[
    4 \pi r \alpha \mu = \frac{1}{\sqrt{3}(2+\delta)}.
  \]
  Thus
  \[
    1+C^2x^2 = \frac{4}{3}
  \]
  hence the maximum value is
  \[
    \frac{D3\sqrt{3}}{16C} = \frac{\delta}{2+\delta}.
  \]
  This maximum value must be larger than the left hand side of inequality
  \ref{eqn:mnoisebnd}, that is,
  \[
    \eta(1+\eta) \le \frac{\delta}{2+\delta},
  \]
  in order that there be any solutions at all, but the right hand side
  is less than $1$. This observation establishes the necessity of
  hypothesis \ref{eqn:mnoisecond} of the
  theorem. Solving this above inequality for $\delta$ and unwinding
  the definitions, one finds that the right-hand side of inequality
  \ref{eqn:mnoisebnd} is bounded by \ref{eqn:mnoisesuff}, so appeal to
  Proposition \ref{thm:mnoise} finishes the proof.
\end{proof}

\noindent {\bf Remark:} We emphasize that Theorem \ref{thm:mnoiseres} states {\em sufficient} conditions for a bound on
the slowness error $|m-m_*|$ in terms of the relative data noise level $\eta$,
giving an additional ``fudge factor'' beyond the support size $\mu$
of the noise-free wavelet $w_*$ for an interval within which the slowness error is
guaranteed to lie.

Conclusion 1 in Theorem \ref{thm:mnoiseres} constrains the range of
noise level to which these results apply to a bit more than 60\%. That
is, the bound given by conclusion 2 is useful only for small noise. In
the limit as $\eta \rightarrow 0$, conclusion 2 becomes
$\lambda/\mu \gtrsim 1 + 2\eta$, that is, the ``fudge factor'' beyond
the noise-free bound is approximatly twice the noise level.

On the other hand, stronger bounds than given by Theorem
\ref{thm:mnoiseres} are possible, given additional constraints on the
noise $n$. A natural example is uniformly distributed random noise,
filtered to have the same spectrum as the source. The expression
\ref{eqn:dexpjgen} implies that the interaction of noise $n$ and
signal $d_*$ in the derivative of $\tJa$ is local, so that the
coefficient of $\eta$ on the left-hand side of inequality
\ref{eqn:mnoisebnd} is effectively much less that 1, resulting in a
larger range of allowable $\eta$. While we will not formulate such a
result, one of the numerical examples below suggests its feasibility.
  
    % Recall that $\lE$ is the projection of $W$ onto $\lW$
%(i.e. multiplication by ${\bf 1}_{[-\lambda,\lambda]}$.

\subsection{Proof of Theorem \ref{thm:ipnoisesuf}}
\begin{proof} of Theorem \ref{thm:ipnoisesuf}:
From Theorem \ref{thm:norminv}, 
\[
  \aw[m;d](t) = (1+ (4 \pi r \alpha t)^2)^{-1}(w_*(t+(m-m_*)) + 4\pi r 
  n(t+mr)) 
\]
\[
  = \aw[m;d_*] + (1+ (4 \pi r \alpha t)^2)^{-1}4\pi r n(t+mr)
\]

From Theorem \ref{thm:mnoiseres},
\[
  |m-m_*| \le \left(1+\frac{2\eta(1+\eta)}{1-\eta(1+\eta)}\right)\frac{\mu}{r} 
\]
which from assumption \ref{eqn:ipnoiselam} is
\[
  =\left(2+\frac{2\eta(1+\eta)}{1-\eta(1+\eta)}\right)\frac{\mu}{r}
  -\frac{\mu}{r} \le \left(\frac{\lambda}{\mu}-1\right)\frac{\mu}{r}
\]
That is,
\[
  |m-m_*|r \le \lambda-\mu.
\]
From Theorem \ref{thm:norminv},
\[
  \mbox{supp }\aw[m;d_*] \subset [-\mu-(m-m_*)r, \mu-(m-m_*)r]
  \subset [-\lambda,\lambda]
\]
so
\[
  E_{\lambda}\aw[m;d](t) = \aw[m;d_*](t) + E_{\lambda}(1+ (4 \pi r \alpha
  t)^2)^{-1}4\pi r n(t+mr)
\]
From the definition of $F[m]$, for any $t_1<t_2$, $w \in W$,  
\[
  F[m]{\bf 1}_{[t_1.t_2]}w = {\bf 1}_{[t_1+mr,t_2+mr]}F[m]w  
\]
Thus the data residual after projection is
\[
  F[m]E_{\lambda}\aw[m,d](t) -d(t) = F[m]\aw[m,d_*](t) -d_*(t)  
\]
\[
  + {\bf 1}_{[-\lambda+mr,\lambda+mr]}(4 \pi r \alpha (t-mr))^2 (1+ (4 \pi
  r \alpha (t-mr))^2)^{-1} n(t)
\]
\[
  -(1- {\bf 1}_{[-\lambda+mr,\lambda+mr]})n(t)
\]
From \ref{eqn:residnorm} and the bound on $m-m_*$,
\[
  \| F[m]E_{\lambda}\aw[m,d_*] -d_*\|^2 = (4 \pi r \alpha)^4
  \int_{-\lambda+(m-m_*)r}^{\lambda+(m-m_*)r}\,dt\, (t-(m-m_*)r)^4
\]
\[
  \times (1+(4\pi r \alpha)^2 
  (t-(m-m^*)r)^2)^{-2}d_*(t)^2
\]
\[
  \le (4 \pi r \alpha \lambda)^4\|d_*\|^2
\]
Similarly, the norm squared of the sum of the last two terms is 
\[
  \le (4 \pi r \alpha \lambda)^2 \|{\bf 1}_{[-\lambda+mr,\lambda+mr]}
  n\|^2 + \|(1 - {\bf 1}_{[-\lambda+mr,\lambda+mr]}n\|^2
\]
Without additional hypotheses to outlaw the accumulation of $n$ near
$t=mr$, all that can be said is that this is
\[
  \le \max \{(4 \pi r \alpha \lambda)^2, 1\} \|n\|^2
\]
Putting this all together,
\[
  \|F[m]E_{\lambda}\aw[m,d]-d\| \le (4 \pi r \alpha
  \lambda)^2\|d_*\| + \max \{4\pi r \alpha \lambda, 1\}\|n\|
\]
\[
  \le ((4 \pi r \alpha \lambda)^2 +\max \{4\pi r \alpha \lambda, 1\}
  \eta)\|d_*\|.
\]
If the right-hand side is to be less than $\|d_*\|$ as required by the
definition \ref{eqn:probstat0} of the inverse problem, then
necessarily $4\pi r\alpha \lambda < 1$, so the right hand side in the
preceding inequality is bounded by the right hand side of assumption
\ref{eqn:ipnoiseeps} of the theorem. Therefore this assumption implies
that the relative residual is $\le \epsilon$.
\end{proof}

\bibliographystyle{seg}
\bibliography{../../bib/masterref}

\append{Numerics and penalty selection via the Discrepancy Principle}
This appendix describes a method for controlling the penalty
weight $\alpha$ based on the Discrepancy Principle
\cite[]{EnglHankeNeubauer,FuSymes2017discrepancy}.

Modified Problem Statement:
\begin{quote}
  \label{eqn:probmod}
  given $d \in D$ and $0 < e_- < e_+$,  find  $m \in M, \alpha \in \bR^+$ so that
  \begin{itemize}
  \item[(i) ]$m$ is a stationary point of $\tJa[\cdot;d]$, and
  \item[(ii) ]$e[m,\aw[m;d];d] \in [e_-,e_+]$.
  \end{itemize}
\end{quote}

\noindent {\bf Remark:} Given a solution $m_{\infty}, \alpha_{\infty}$, the pair
$(m_{\infty},w_{\alpha_{\infty}}[m_{\infty};d])$ is a local minimizer
of $J_{\alpha_{\infty}}[\cdot,\cdot;d]$ (Theorem \ref{thm:norminv}).
%.
%so is also a minimizer of $g[w]$ over the set
%\[
%  \{(m,w): e[m,w;d] \ge
%  e[m_{\infty},w_{\alpha_{\infty}}[m_{\infty};d]]\}
%\]
%or equivalently a maximizer of $e[m,w,d]$ over the set
%\[
%  \{(m,w): m \in M, g[w] \le g[w_{\alpha_{\infty}}[m_{\infty};d]]\}
%\]
%according to standard Lagrange multiplier theory. That is, 

We describe an alternating, or coordinate search, algorithm for
solution of problem \ref{eqn:probmod}, combining a local optimization
algorithm for updating $m$, and a second algorithm for updating
$\alpha$. A first version of this algorithm appeared in \cite{FuSymes2017discrepancy}.

Given an
objective function $\Phi: M \rightarrow \bR$, a local optimization algorithm generates 
generates a map $G[\Phi,...]: M \rightarrow M$, mapping a current estimate
of $m$ to an updated estimate. The update rule may depend only on the
current estimate, as is the case for steepest descent, the
Gauss-Newton method, or Newton's method, or may also depend on
information generated during earlier updates, as for secant-type
methods such as BFGS. The notation is intended to allow for this
latter possibility. Under ``standard conditions'' on $\Phi$ and the initial estimate for
$m$, and equipped with so-called globalization safeguards to ensure
satisfaction of sufficient decrease conditions
(see \cite{NocedalWright}), the iteration produces a sequence in $M$
converging to a local minimizer of $\Phi$. An example of such a local
optimization algorithm is steepest descent with backtracking line
search.

A normal stopping criterion for such an algorithm would be a tolerance
test for the norm of $\nabla \Phi$ (and a limit on iteration count, of
course). In our application to $\Ja = e + \alpha^2 g$, we add another
stopping criterion:
\begin{quote}
  stop if $e \notin [e_-,e_+]$.
\end{quote}
Since $e$ is a summand in $\tJa$, $e$ will typically decrease along a
minimizing sequence, so the expected condition invoking this stopping
rule is $e<e_-$. We denote a update rule with this enhancement,
applied to $\tJa$, by $G[\alpha,e_-,e_+,...]$, still allowing the
possibility of information in addition to the current iterate.

We will also require a rule for updating $\alpha$, defining a map $H: M \times E
\rightarrow \bR^+$, in which $E=\{(e_-,e_+) \in \bR^2:
e_-<e_+\}$. For any $m \in M$ the rule is required to produce an $\alpha$ for which
the error bounds are satisfied: If $(m, e_-, e_+) \in M \times E$ and $\alpha = H(m, e_-,
e_+)$, then $e[m,\aw[m,d];d] \in [e_-,e_+]$. We will describe two
such rules below.

In outline, the algorithm is as follows:
\begin{algorithm}[H]
\caption{Scheme for updating $m, \alpha$}
\begin{algorithmic}[1]
  \State Choose $m\in M$
  \Repeat
  \State $\alpha \gets H[m,e_-,e_+]$
  \Repeat
  \State $m \gets G[\alpha,e_-,e_+,...](m)$
  \Until{$e[m,\aw[m,d];d] \notin [e_-,e_+]$ or $\|\nabla \tJa[m;d]\|$
    sufficiently small, or...}
  \Until{$e[m,\aw[m,d];d] \in [e_-,e_+]$} 
\end{algorithmic}
\end{algorithm}
Note that after step 3, the error bounds are satisfied, that is,
$e[m,\aw[m;d];d] \in [e_-,e_+]$, but after step 4, that is likely not
to be the case: the alteration of $m$ is likely to reduce $e$, as it is a
summand in the definition of $\tJa$. If the $e$ is reduced below
$e_-$, or if an approximate local minimizer is detected, then the condition in step 6 is satisfied, so control returns
to step 7. If it is satisfied, the algorithm terminates, elsewhich
control loops back to step 2, $\alpha$ is updated, and the inner $m$
update loop is entered again. Termination requires that the bounds on
$e$ are satisfied (step 7), hence that the $m$ update loop terminates
by finding a local min which satisfies these bounds. 




Accordingly, regard $m$ as fixed and suppress it from the notation for the remainder of ;dthis section, and introduce the abbreviations
\begin{eqnarray}
\label{eqn:erralph}a
e(\alpha) &=& e[m,\aw[m;d],d]\\
\label{eqn:penalph}
p(\alpha) &=& p[m,\aw[m;d],d]
\end{eqnarray}
Since the normal matrix $F^T F + \alpha^2 A^T A$ has been assumed
invertible (condition \ref{eqn:nopco}), the solution of $\aw$ of the
normal equation \ref{eqn:condition} is smooth in $\alpha^2$.
Differentiating equation \ref{eqn:condition} with respect to  $\alpha^2$ leads to the relation
\begin{equation}
(F^T F + \alpha^2 A^T A ) \frac{dw}{d\alpha^2} = -A^T A w
\label{eqn:dnorm}
\end{equation}
whence
\begin{align}
\frac{de}{d\alpha^2} 
&=\left\langle\frac{dw}{d\alpha^2},F^T(Fw-d) \right\rangle \nonumber \\
&=-\alpha^2\left\langle\frac{dw}{d\alpha^2},A^TAw\right\rangle \nonumber \\ 
&=\alpha^2 \langle A^TAw,(F^TF + \alpha^2 A^TA)^{-1}A^TAw\rangle \nonumber \\
&\ge 0
\label{eqn:de}
\end{align}
Note that the inequality in equation \ref{eqn:de} is {\em strict}  if $p > 0$ hence $A^TAw \ne 0$, since the normal operator is assumed to be positive definite.
The derivative of $p(\alpha^2)$ with respect to $\alpha^2$ is
\begin{align}
\frac{dp}{d\alpha^2} &=  -\langle A^T Aw,(F^TF + \alpha^2 A^TA)^{-1}A^TAw \rangle \nonumber \\
&\leq 0
\label{eqn:dp}
\end{align}
similarly a strict inequality if $p > 0$.

Equations \ref{eqn:de} and \ref{eqn:dp} show that increasing $\alpha^2$ implies increasing $e$ while decreasing $p$, and
\begin{align}
&\langle A^TA w,(F^TF + \alpha^2 A^TA)^{-1} A^TA w \rangle \nonumber \\ 
=& \langle (A^TA)^{1/2}w,[(A^TA)^{-1/2}F^TF(A^TA)^{-1/2} + \alpha^2 I]^{-1}(A^TA)^{1/2}w \rangle \nonumber \\ 
\le& \frac{1}{\alpha^2} \langle A^TA w, w\rangle = \frac{2}{\alpha^2}p
\end{align}
In view of equation \ref{eqn:de},
\begin{equation}
\label{eqn:lep}
\frac{de}{d\alpha^2}  \le 2p.
\end{equation}
with this inequality also being strict if $p > 0$.

Suppose the current weight is $\alpha^2_c$ and denote a candidate for an updated weight by $\alpha^2_+$. Then from inequality \ref{eqn:lep},
\begin{equation}
e(\alpha^2_+)-e(\alpha^2_c) \le \int_{\alpha^2_c}^{\alpha^2_+} 2p d\alpha^2
\end{equation}
If $\alpha^2_+ \ge \alpha^2_c$, then in view of inequality \ref{eqn:dp}, the above is
\begin{equation}
\label{eqn:basic}
\le 2 p(\alpha^2_c) (\alpha^2_+-\alpha^2_c)  
\end{equation}

Let us suppose that $e(\alpha^2_c) < e_{+}$. Then setting 
\begin{equation}
\label{eqn:alphasecant}
\alpha^2_+ = \alpha^2_c + \frac{e_{+}-e(\alpha^2_c)}{2p(\alpha^2_c)} 
\end{equation}
implies via inequality \ref{eqn:basic} that 
\begin{equation}
e(\alpha^2_+)-e(\alpha^2_c) \le e_{+}-e(\alpha^2_c)
\end{equation}

Assuming that $p(\alpha^2_+) > 0$, hence $p(\alpha^2)>0$ for $\alpha^2_c \le \alpha^2 \le \alpha^2_+$,  we conclude that if $\alpha^2_+$ is given by the rule \ref{eqn:alphasecant},
then
\begin{equation}
\label{eqn:assert}
e(\alpha^2_c) < e(\alpha^2_+) \le e_{+}
\end{equation}
That is, unless $p(\alpha^2_+) = 0$ (in which case a physical solution of the inverse problem has been reached),  $e(\alpha^2_+)$ is larger than $e(\alpha^2_c)$ but in any case does not exceed $e_+$. The rule \ref{eqn:alphasecant} therefore provides a feasible updated $\alpha^2$ consistent with the upper bound in the discrepancy principle.


\append{Heisenberg, support, and wavelength}
Without further contraints on data or solution, nothing more can be
said about bounds on $\alpha$. If the data $d$ (and the target wavelet
$w_*$) are assumed to have a square integrable derivative, then a
necessary condition follows from the Heisenberg
inequality (see for example \cite{Folland:07}, p. 255). To formulate
this result in its most general form, introduce the Hilbert subspaces
$V^0 \subset L^2(\bR), V^1 \subset H^1(\bR)$:
\begin{eqnarray}
  V^0 & = & \{ f \in L^2(\bR): Af \in L^2(\bR)\}, \nonumber\\
  V^1 & = & V^0 \cap H^1(\bR).
            \label{eqn:vdef}
\end{eqnarray}
$V^0$ is the domain of $A$, and equipped with the graph norm of $A$. A
natural norm in $V^1$ is
\[
  \|f\|^2_{V^1} = \|f\|_{V^0}^2 + \|f\|_{H^1}^2.
\]
$V^j$ is the completion of $C_0^{\infty}(\bR)$ in the corresponding
norm, j=0,1.

\begin{proposition}
  \label{thm:heis}
For $w \in V^1$,
  \begin{equation}
    \label{eqn:heis}
    \|Aw\|\|w'\| \ge \frac{1}{2}\|w\|^2
  \end{equation}
\end{proposition}

\begin{proof}
  For $w \in C_0^{\infty}(\bR)$,
  \[
    \int w^2 = \left|\int\,dt\, t (w(t)^2)' \right|= \left|2\int\,dt\,tw(t)w'(t)\right| \le
    2\|Aw\| \|w'\|
  \]
  by the Cauchy-Schwarz inequality. Since $C_0^{\infty}(\bR)$ is dense
  in $V^1$, the conclusion follows by continuity.
\end{proof}

In the conventional formulation of the Heisenberg inequality, the $L^2$ norm of
$w'$ is replaced by the its equivalent in terms of the Fourier
transform $\hat{w}$. Adopting temporarily the use of dummy variables
in the expression of functions, the identity \ref{eqn:heis} turns into
the usual form of the Heisenberg inequality: for $w \in V^1$,
\begin{equation}
\label{eqn:fheis}
\|tw(t)\|\|k\hat{w}(k)\| \ge \frac{1}{4\pi}\|w\|^2.
\end{equation}

Define $\krms[w]$, the root mean square estimator of frequency of $w
\in V^1$, by
\begin{equation}
  \label{eqn:krms}
  \krms[w]=\frac{1}{2\pi}\frac{\|w'\|}{\|w\|} = \left(\int
    \,dk\,\frac{|\hat{w}(k)|^2}{\|\hat{w}\|^2} k^2\right)^{1/2}.
\end{equation}
Then the inequalities \ref{eqn:heis}, \ref{eqn:fheis} can be rewritten as
\begin{equation}
  \label{eqn:frmsheis}
  \|Aw\| \ge \frac{\|w\|}{4\pi \krms[w]}.
\end{equation}

For $\lambda >0$, define
\begin{equation}
  \label{eqn:wlam1}
  \lW^1 = \lW \cap H^1(\bR).
\end{equation}
Note that $\lW^1 \subset V^1$ is a closed subspace for any
$\lambda>0$.

\begin{proposition}
  \label{thm:klam}
  For $w \in \lW^1$,
  \[
    \krms[w] \ge \frac{1}{4 \pi \lambda}
  \]
\end{proposition}

\begin{proof}
  Follows directly from inequality \ref{eqn:frmsheis} and the obvious bound
  $\|A|_{\lW}\|\le \lambda$.
\end{proof}

\noindent{\bf Remark:} This result is the link mentioned earlier between the support
constraint and the well-known frequency-based criteria for success of
FWI. The result of Theorem \ref{thm:fwi} can be rephrased as showing the
existence of many stationary points of the mean-square error function
for which the travel time is in error by more than
$\lambda \ge 1/{4 \pi \krms[w]|}$. In fact, the usual error criterion mentioned in the
literature is that the initial estimate of travel time must be in error by at most ``half a
wavelength'' if FWI is to converge reliably. This is correct in some
circumstances, depending on features of the target wavelet $w_*$ of
which the arguments in this paper do not take account.

\begin{proposition}
  \label{thm:heis2}
  For $\lambda>0, w \in \lW^1$,
  \begin{equation}
    \label{eqn:heis2}
    \|A^2w\| \ge \frac{\|w\|}{24 \lambda (2\pi\krms[w])^3}.
  \end{equation}
\end{proposition}

\begin{proof}
  Since $A$  preserves $\lW^1$, inequality \ref{eqn:heis} implies
  \begin{equation}
    \label{eqn:heissq}
    \|A^2w\| \ge \frac{\|Aw\|^2}{2\|(Aw)'\|}.
  \end{equation}
  From the definition of $A$, $(Aw)'=w +Aw'$, so for $w \in \lW^1$,
  \[
    \|(Aw)'\| \le \|w\|+ \lambda\|w'\|.
  \]
  Applying \ref{eqn:heis} again,
  \[
    \|w\|^2 \le 2\|w'\|\|Aw\|\le 2\lambda \|w'\|\|w\|,
  \]
  so obtain the 1D Poincar\'{e} inequality: for $w \in \lW^1$,
  \[
    \|w\| \le 2\lambda\|w'\|.
  \]
  Thus
  \[
    \|(Aw)'\| \le 3\lambda\|w'\|.
  \]
  Apply this estimate together with the basic Heisenberg estimate
  \ref{eqn:heis} to the inequality \ref{eqn:heissq} to obtain
  \[
    \|A^2w\| \ge \frac{\|w\|^4}{24\lambda \|w'\|^3}
  \]
  Rearranging and using the definition \ref{eqn:krms} of $\krms[w]$,
  arrive at inequality \ref{eqn:heis2}.
\end{proof}

\begin{theorem}
  \label{thm:ipnonoisenec}
  Suppose that $a$ is given by definition \ref{eqn:ann}, $\alpha$,
  $\mu \in (0,\lambda_{\rm max}/2]$, $d$ is given by
  \ref{eqn:defdatanonoise} with $w_* \in W_{\mu}^1$, and that a
  stationary point $m_{\infty}$ of $\tJa[\cdot;d]$ yields a solution
  of the inverse problem \ref{eqn:probstat0} for $\lambda \ge 2\mu$,
  $\epsilon>0$. Then
\begin{equation}
  \label{eqn:epsalpha}
  3 (4\pi \lambda \krms[w_*])^3 \epsilon \ge \frac{(4\pi  r\alpha\lambda)^2}
  {(1+(4\pi r \alpha \lambda)^2)}
\end{equation}
\end{theorem}

\begin{proof}
  Rearranging the identity \ref{eqn:residnorm}, and observing as in
  the proof of Theorem \ref{thm:ipnonoisesuf} that the support of the
  integrand is contained in $[-2\mu, 2\mu] \subset [\lambda,\lambda]$,
  \[
  e[m_{\infty},\aw[m_{\infty},d];d] 
= 8 \pi^2 r^2 \alpha^4\int\,dt\,t^4(1+(4\pi r)^2 \alpha^2 t^2)^{-2}w_*(t+(m_{\infty}-m_*))^2
\]
\[
  \ge \frac{8\pi^2r^2\alpha^4}{(1+(4\pi r)^2 \alpha^2
    \lambda^2)^{2}}\int\,dt\,t^4w_*(t+(m_{\infty}-m_*))^2
\]
\begin{equation}
  \label{eqn:residnormbis}
  \ge \frac{8\pi^2r^2\alpha^4}{(1+(4\pi r)^2 \alpha^2
    \lambda^2)^{2}}\|A^2w_*(\cdot + (m_{\infty}-m_*)r)\|^2.
 \end{equation}
Since $\krms[w]$ is invariant under translation of $w \in V^1$,
Proposition \ref{thm:heis2} implies that this is
\[
  \ge \frac{8\pi^2r^2\alpha^4}{(1+(4\pi r)^2 \alpha^2
    \lambda^2)^{2}}\frac{\|w_*\|^2}{(24)^2 \lambda^2 (2\pi\krms[w_*])^6}
\]
\[
  = \frac{1}{2}\frac{(4\pi r\alpha\lambda)^4}{(1+(4\pi r \alpha
    \lambda)^2)^{2}}\frac{\|d\|^2}{(24)^2 (2\pi
    \lambda \krms[w_*])^6}
\]
The pair $(m_{\infty}, \aw[m_{\infty},d])$ is presumed to solve the inverse
problem as stated in \ref{eqn:probstat0}, in particular
\[
  \epsilon \ge (2 e[m_{\infty}, \aw[m_{\infty},d],d])^{1/2}/\|d\|
\]
the inequality \ref{eqn:epsalpha} follows.

\end{proof}

Inequality \ref{eqn:epsalpha} couples the dimensionless quantites
$\epsilon$,  $4\pi r \lambda \alpha$, and $4 \pi \lambda
\krms[w_*]$. Proposition \ref{thm:klam}, implies that the left
hand side is $\ge 3\epsilon$. Since the right hand side is $\le 1$,
the inequality implies no limitation on $\alpha$ if the left hand side
is $\ge 1$. For small $\epsilon$, the the largest permissible $\alpha$
is $O(\sqrt{\epsilon})$. The permissible range of $\alpha$ increases with
nondimensionalized RMS frequency $4\pi \lambda \krms[w_*]$. Since
$\krms$ is invariant under translation and scaling of its argument,
$\krms[w_*]=\krms[d]$, that is, the nondimensionalized RMS frequency
is an observable property of the data, in the noise-free case at
least.

