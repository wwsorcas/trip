\title{Example Plots}
\author{William. W. Symes \email{\tt symes@rice.edu}}

\lefthead{WWS}

\righthead{HC Examples}

\maketitle
\begin{abstract}
Examples from HC, 2020.10.08.
\end{abstract}

\section{Introduction}
Data process:

For example {\tt n, n=1, 3, 4} and {\tt type} = {\tt data}, {\tt residual}:
\begin{itemize}
\item read HC file {\tt file.txt} = ex{\tt n\_type\_}(appropriate figure number){\tt.txt}, use {\tt grep -v} to remove comment ilne
\item use SU utility {\tt a2b} to convert ascii file {\tt file.txt} to binary file {\tt file.bin}
\item use SU utility {\tt suaddhead} to add headers {\tt ns} and {\tt dt} to produce SEGY file {\tt file.su}
\item use Madagascar utility {\tt sfsuread} to convert SEGY file {\tt file.su} to RSF file {\tt file.rsf} for plotting
\item use IWAVE utility {\tt SEGYNorm} to output L2 norm of the data in {\tt file.su} to terminal.
\end{itemize}

For example  {\tt n, n=1, 3, 4}, use Madagascar utility {\tt sfcat} to concatenate the data and residual {\tt .rsf} files, then use {\tt sfgraph} to plot two curves, with data in blue and residual in red.

{\tt SEGYNorm} computes the scaled $l^2$ norm of a vector $x$ (interpreted as the vector of samples of a function of a real variable on a regular grid) with length $ns$ and sample interval $dt$:
\[
  \|x\| = \left(dt \sum_{i=0}^{ns-1} |x[i]|^2\right)^{1/2}
\]
Note that if the extreme samples are $=0$, then this expession is equivalent to the trapezoidal rule.

There are (at least) two obvious ways of defining SNR. By ``SNR'' in the captions below, I mean the norm of the data divided by the norm of the residual. The meaning in the current draft paper might be called ``ideal SNR'', meaning the norm of the {\em noise-free} data (the data of Example 1 in all cases), divided by the difference of the predicted data minus the noise-free data (rather than the input data, which the algorithm is trying to fit). There shouldn't be much difference between these two, though they are certainly not the same. I have only given the straightforward SNR since in any non-synthetic situation you would not know the ``noise-free'' data.

\section{Examples}

\inputdir{hc}

\plot{ex1}{width=\textwidth}{Example 1 (combines Figures 3 and 5 in HC notes of 2020.10.05): data = blue curve, residual = red curve. Data norm = 1.3765, residual norm = 0.724783, achieved signal-noise ratio = 1.899.}

\plot{ex3}{width=\textwidth}{Example 3 (combines Figures 18 and 21 in HC notes of 2020.10.05): data = blue curve, residual = red curve. Data norm = 2.01366, residual norm = 0.595673, achieved signal-noise ratio = 3.38.}

\plot{ex4}{width=\textwidth}{Example 4 (combines Figures 27 and 30 in HC notes of 2020.10.05): data = blue curve, residual = red curve. Data norm = 1.60526, residual norm = 0.685794, achieved signal-noise ratio = 2.341.}



