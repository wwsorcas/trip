\section{Structure of the Gradient}
As was mentioned in the introduction, proper choice of norms in domain
and range of the example modeling operator $F$ described there renders
it unitary. If $F$ were also differentiable (rather than
``differentiable with loss of a derivative'', as explained above),
$Q[m,\delta m] = F[m]^TDF[m]\delta m$ would factor $DF[m]\delta m =
F[m]Q[m,\delta m]$, and $Q[m,\delta m]$ is skew-symmetric. This
section presents a precise variant of this structure that is shared by
the simple example studied in this paper and by many other inverse
problems in wave propagation. 

The substitute for unitarity that can be achieved in for many such
problems is

\noindent {\bf Assumption D:} For each $m \in M$, $F[m]$ is
{\em approximately unitary}: there exist
\begin{eqnarray}
  W_X&\in& C^{\infty}(M,\cap_{s=0}^{\infty}{\cal I}(X^s,X^s)),\\\nonumber
  W_D&\in& C^{\infty}(M,\cap_{s=0}^{\infty}{\cal I}(D^s,D^s))
  \label{eqn:w}
\end{eqnarray}
for all $s \ge 0$, so that
\begin{equation}
\label{eqn:appinvdef}
F^{\dagger} = W_X^{-1}F^TW_D: M \rightarrow {\cal B}(D^s,X^s),
\end{equation}
satisfies
\begin{equation}
  \label{eqn:appinvprop}
  F^{\dagger}F = I +R_X,\,FF^{\dagger}=I+R_D
\end{equation}
with
\begin{eqnarray}
   R_X&\in& C^{\infty}(M,\cap_{s,t=0}^{\infty}{\cal B}(X^s,X^{s+t})),\\\nonumber
  R_D&\in& C^{\infty}(M,\cap_{s,t=0}^{\infty}{\cal B}(D^s,D^{s+t}))
  \label{eqn:r}
\end{eqnarray}

Note that the Assumption D together with condition \ref{eqn:reg1} from
Assumption C imply that $F^{\dagger}$ is also ``differentiable with loss of a derivative'':
\begin{equation}
\label{eqn:appinvderiv}
F^{\dagger} \in C^k(M, {\cal B}(D^s,X^t)) \mbox{ provided that
}s > t+k, t.
\end{equation}
$F^{\dagger}$ is the adjoint of $F$ with respect to the $M$-dependent
norms
\begin{eqnarray}
  \|x\|^2_m &=& \langle x, W_X[m]x\rangle_0,\\ \nonumber
  \|d\|^2_m &=& \langle d, W_D[m]d\rangle_0
  \label{eqn:unit}
\end{eqnarray}

\noindent {\bf Remark:} As mentioned earlier, in concrete instances,
$X^s,D^s$ are$L^2$-based  Sobolev spaces, for
which the index $s$ indicates the number of square-integrable
derivatives. The assertion \ref{eqn:r} thus means that the
``remainder'' operators $R_X,R_D$ are smoothing: outputs are
arbitrarily smoother than inputs. For the very simple example
described in the introductions, in fact $R_X=0=R_D$.

\begin{theorem}
  \label{thm:q} Under Assumptions A-D, 
the derivative $DF$ admits the factorization
\begin{equation}
  \label{eqn:decomp}
  DF[m]\delta m = F[m]Q[m,\delta m] + E[m,\delta m],
\end{equation}
in which $ Q \in \cap_{s \ge 0}C^{\infty}(M \times TM, {\cal
  B}(X^{s+1},X^s))$ and $E \in C^{\infty}(M \times TM, ,\cap_{s,t=0}^{\infty}{\cal B}(X^s,D^{s+t}))$
are linear in their second arguments. Also, $Q$ is {\em essentially skew-adjoint}, in the sense that for any $m \in M,  \delta m \in TM$, and $x_1,x_2 \in X^1$,
\begin{equation}
  \label{eqn:essskew}
  \langle Q[m,\delta m] x_1,x_2 \rangle_{X^0} = -\langle x_1, Q[m,\delta
  m]x_2\rangle_{X^0}
  + \langle R[m,\delta m]x_1,x_2\rangle_{X^0},
\end{equation}
with $R \in \cap_{s,t \ge 0}C^{\infty}(M \times TM, {\cal B}(X^s,X^{s+t}))$,
also linear in its second argument.
\end{theorem}

\begin{proof} Define $Q = F^{\dagger}DF$. Then
  \[
    FQ = (I+R_D)DF = DF + E
  \]
  \[
    D (F^{\dagger}F) = DF^{\dagger}F+F^{\dagger}DF 
  \]
  \[
    = D(W_X^{-1}F^TW_D)F +Q = -W_X^{-1}DW_X W_X^{-1}F^TW_DF
  \]
  \[+
  W_X^{-1}(Q^TF^T-E)W_DF + W_X^{-1}F^TW_D W_D^{-1}DW_DF +Q
\]
\[
=-W_X^{-1}DW_XF^{\dagger}F + Q^{\dagger}F^{\dagger}F - E^{\dagger}F + F^{\dagger}W_D^{-1}DW_DF+Q
\]
\[
  = -W_X^{-1}DW_X(I+R_X) + Q(I+R_X) - E^{\dagger}F +
  F^{\dagger}W_D^{-1}DW_DF+Q
\]
\[
  = Q^{\dagger} + Q -W_X^{-1}DW_X(I+R_X) + QR_X - E^{\dagger}F +
  F^{\dagger}W_D^{-1}DW_DF
  \]
\end{proof}

From equations \ref{eqn:grad1} and \ref{eqn:decomp},
\[
    D\tJa[m;d]\delta m = \langle 
  (DF[m]\delta m)\ax[m,d],F[m]\ax[m;d]-d\rangle_{D^0}
\]
\[
  =\langle F[m]Q[m,\delta m]\ax[m;d],F[m]\ax[m;d]-d\rangle_{D^0}
\]
\[
  = \langle Q[m,\delta m]\ax[m;d],F[m]^T(F[m]\ax[m;d]-d)\rangle_{D^0}
\]
\[
  = -\langle Q[m,\delta m]\ax[m;d],A^TA\ax[m;d]\rangle_{X^0}
\]
\[
  = -\alpha^2\langle A^TA Q[m,\delta m]\ax[m;d],\ax[m;d]\rangle_{X^0}
\]
\[
  =\alpha^2\langle [Q[m,\delta m],A^TA]\ax[m;d],\ax[m;d]\rangle_{X^0} -\alpha^2\langle Q[m,\delta m]A^TA\ax[m;d],\ax[m;d]\rangle_{X^0}
\]
\[
  =\alpha^2\langle [Q[m,\delta m],A^TA]\ax[m;d],\ax[m;d]\rangle_{X^0}-\alpha^2\langle Q[m,\delta m]^T\ax[m;d], A^TA\ax[m;d]\rangle_{X^0}
\]

