Speaker: William W. Symes

Title: Extended Inversion: when it works, and why

Abstract: Model extension is the basis for several classic seismic velocity analysis techniques. In the last couple of decades, various inversion methods based on model extension have been suggested as approaches to eliminating or reducing the tendency of Full Waveform Inversion to cycle-skip. This talk will illustrate the extended inversion concept by applying it to perhaps the simplest waveform inversion problem imaginable: to estimate the sound velocity in a homogeneous fluid from recording of a transient pulse propagated from a single source to a single receiver at known offset.  While cartoonishly simple, this single trace transmission problem exhibits key difficulties of industrial scale inversion - in particular, it provides an often-used illustration of cycle-skipping. I will present complete analyses of Full Waveform Inversion, Wavefield Reconstruction Inversion, and another extended inversion method similar to Adaptive Waveform Inversion, applied to single trace transmission. The mathematics is straightforward enough that the analysis can be performed ``on paper'': it is possible to determine precisely why each method overcomes cycle-skipping - or does not. In fact, only one of these approaches ``works'', that is, does not cycle-skip. To conclude, I'll present a few preliminary results from a generalization of the ``good'' method to a 2D crosswell velocity estimation problem, and suggest that this type of extended inversion may eventually be feasible on an industrial scale, with cost comparable to FWI's.

Acknowledgements: I thank many students, colleagues and collaborators who have helped me learn whatever it is that I know about this subject. The work reported in this talk is in part the result of a collaboration with Prof. Susan E. Minkoff and graduate assistant Huiyi Chen at University of Texas - Dallas.