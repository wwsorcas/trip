\title{Implementation of Adaptive Waveform Inversion via a Penalty Function}
\author{Huiyi Chen, Susan E. Minkoff, and William W. Symes}

\begin{abstract}
goes here.
\end{abstract}

\section{Introduction}
\section{A Penalty Version of AWI}
The adaptive kernel $u_0^{\epsilon}$ appearing in the definition \ref{eqn:tJz} is the solution of a regularized least squares problem. This observation suggests augmenting the regularized least squares objective by a penalty proportional to the right-hand side of \ref{eqn:tJz}:
\begin{equation}
  \label{eqn:Ja}
  \Ja[m,u;d] = \frac{1}{2}\left(\|S[m]u-d\|_d^2 + \epsilon^2\|u\| + \alpha^2\|tu\|_m\right).
\end{equation}
Here $\|u\|^2 = \langle u, u \rangle$ is the (unweighted) $L^2$ norm, and $\|\cdot\|_m$ is an $(m,d)$-dependent (weighted) norm: with
\[
  u_0^{\epsilon}[m,d] = \mbox{arg min}_u \Jz[m,u;d],
\]
define
\[
  Wu (\bx_r,t;\bx_s) = \sum_{\bx_r,\bx_s}\frac{1}
  {\int\,ds\,|u_0^{\epsilon}[m,d](\bx_r,s;\bx_s)|^2}u(\bx_r,t;\bx_s)
\]
and
\begin{equation}
  \label{eqn:defmnorm}
  \|u\|^2_m = \langle u, W u \rangle.
\end{equation}
With this notation,
\[
  \tJz[m;d] = \frac{1}{2}\|tu^{\epsilon}_0[m;d]\|^2_m
\]
In the definition of $\tJz$, a quadratic is evaluated at the minimizer of $\Jz[m,u;d]$. This suggests the possibility of evaluating the quadratic objective $\Ja[m,u;d]$ at its own minimizer:
\begin{equation}
  \label{eqn:tJa}
  \tJa[m;d] = \min_u \Ja[m,u,d] = \Ja[m,\ua[m;d];d]
\end{equation}
This is the so-called variable projection reduction of the quadratic $\Ja$.
Note that the right-hand side of the definition of $\tJz$ is the last term on the right-hand side of $\Ja$, scaled by $\alpha^{-2}$ and evaluated at the minimizer of the sum of the first two terms, that is, $\Ja$ for $\alpha=0$.

\begin{theorem}
  \label{thm:basic}
Suppose that $S \in {\cal L}(U,D)$, the is a bounded linear maps from (Hilbert space) domain $U$ (``model space'') with inner product $\langle \cdot, \cdot \rangle_m$ to range $D$ with inner product $\langle \cdot, \cdot \rangle_d$. Assume that $S$ is coercive, that is, there exists $\epsilon >0$ so that $\|Su\|_d \ge \epsilon\|u\|_m$ for $u \in U$.
Suppose also that $A \in {\cal L}(U,U)$, and that $d \in D$. Define the function $J_{\alpha}$ on $U$ by 
\begin{equation}
  \label{eqn:eq1}
  J_{\alpha}(u) = \frac{1}{2}(\|Su-d\|^2_d + \alpha^2\|Au\|_m^2).
\end{equation}
Then for $\alpha \ge 0$, $J_{\alpha}$ possesses a unique minimizer $u_{\alpha} \in U$, for which
\begin{equation}
  \label{eqn:eq0}
  \tilde{J}_{\alpha} = \min_{u \in U} J_{\alpha}(u) = J_{\alpha}(u_{\alpha}),
\end{equation}
and
\begin{equation}
  \label{eqn:eq4}
  \lim_{\alpha \rightarrow 0} \frac{1}{\alpha^2}  (\tilde{J}_{\alpha}-\tilde{J}_0).= \frac{1}{2}\|Au_0\|^2
\end{equation}
This limit is uniform over bounded sets of $(S, A) \in {\cal L}(U,D) \times {\cal L}(U, U)$.
\end{theorem}

For completeness, the proof is given in an appendix.

This relation will be applied in the context of a model space $U$ possessing a reference Hilbert norm $\|\cdot\|$ and a family of equivalent norms $\|\cdot\|_m$ depending on a parameter $m \in M$, in which $M$ is an open subset of a Banach space $X$. The dependence is through a an operator-valued function $W$ on $M$, taking values in cone ${\cal B}(U)$ of bounded self-adjoint positive-definite operators on $U$. On the other hand, we assume that the norm on $D$ is fixed, independent of $M$, and treat it as a reference norm.

\begin{theorem}
  \label{thm:uniform}
Assume that $W: M \rightarrow {\cal B}(U)$ is of class $C^1$ and uniformly positive definite, and that the norm $\|\cdot\|_m$ is given by
\begin{equation}
  \label{eqn:normdef}
  \|u\|_m^2 = \langle u, W[m] u \rangle.
\end{equation}
Suppose also that $S: M \rightarrow {\cal L}(U,D)$ is of class $C^1$ and uniformly coercive, and that $A \in {\cal L}(U,U)$ as before. For $\alpha \ge 0$, define $\tJa: M \rightarrow \bR$ by
\[
  \tilde{J}_{\alpha}[m] = \min_{u \in U}\frac{1}{2}(\|S[m]u-d\|^2_d + \alpha^2\|Au\|_m^2)
\]
\begin{equation}
  \label{eqn:tjadef}
  = \frac{1}{2}(\|S[m]u_{\alpha}[m]-d\|^2_d + \alpha^2\|Au_{\alpha}[m]\|_m^2)
\end{equation}
where
\begin{equation}
  \label{eqn:ua}
  u_{\alpha}[m] = \mbox{arg min}_{u \in U}\frac{1}{2}(\|S[m]u-d\|^2_d + \alpha^2\|Au\|_m^2)
\end{equation}
for $m \in M$. Then $\tilde{J}_{\alpha}$ is of class $C^1(M)$, and
\begin{equation}
  \label{eqn:lim}
  \lim_{\alpha \rightarrow 0} \frac{1}{\alpha^2}  (\tJa[m]-\tJz[m]).= \frac{1}{2}\|A\uz[m]\|_m^2.
\end{equation}
The limit is in the sense of $C^1(M)$, that is, both the left-hand side and its $m$ derivative converge to the right-hand side and its $m$ derivative, respectively. Moreover $u_{\alpha}$ is of class $C^1(M,U)$.
\end{theorem}

\begin{proof} All conclusions but the last follow immediately from Theorem \ref{thm:basic} and the uniformity assumptions on $S$ and $W$.

Since the norm in $D$ is fixed for the discussion, drop the subscript and treat it as a reference norm. Denote the adjoint with respect to the norms $\|\cdot\|_m$ on $U$, $\|\cdot\|_d = \|\cdot\|$ on $D$ by a superscript $*$, and with respect to $\|\cdot\|$ on $U$ by a superscript $T$. The normal equation defining $\ua$ is
  \[
    (S[m]^*S[m] + \alpha^2 A[m]^*A) \ua[m] = S[m]^*d
  \]
  \begin{equation}
    \label{eqn:normal}
= (W[m]^{-1}S[m]^TS[m] + \alpha^2 W[m]^{-1}A^TW[m]A)\ua[m]= W[m]^{-1}S[m]^Td
\end{equation}
or equivalently
\begin{equation}
  \label{eqn:normal1}
  (S[m]^TS[m] + \alpha^2 A^TW[m]A)\ua[m] = S[m]^Td
\end{equation}
Since $S$ is uniformly coercive, and both $S$ and $W$ are of class $C^1$, the normal operator on the left hand side is boundedly invertible and of class $C^1$. Therefore the solution $\ua$ is of class $C^1$.

[need to add proof of $C^1$ convergence.]
\end{proof}

Apply this relation in the context of the AWI penalty problem defined earlier. $M$ is an open set in $C^k(\Omega, \bR^2)$. $U$ is $\oplus_{\bx_r,\bx_s}L^2([-T_u,T_u])$, $D$ is the data space $\oplus_{\bx_r,\bx_s}L^2([0,T])$,. $S$ is the regularized modeling operator $(S[m],\epsilon I)^T$, $d$ is replaced by the $(d,0) \in D \times U$,  $W$ is the weight operator on $U$ defined in \ref{eqn:defmnorm}.

As noted before [but it wasn't - fix!], stationary points of $\tJz$
are $\lambda(w_*)$-close to stationary points of the travel time
mean-square error, in the single-arrival case. The result of this
section implies that stationary points of $\tJa$ are close to those of
$\tJz$. Therefore, at least for small $\alpha$, minimization of $\tJa$
produces acoustic models that closely match the traveltimes predicted
in single-arrival data.



\section{Preliminaries}


$Q \subset \bR^3 \times \bR^3$ is the finite ``acquisition'' set of source-receiver location pairs $(\bx_r,\bx_s)$, assumed disjoint from the diagonal (sources and receivers non-coincident). $Q_s$ denotes projection onto second factor (source locations occurring in $Q$).

%$P = \bR^{|Q|}$ = real functions on $Q$, with Euclidean norm (no attempt to scale for geometry of acquisition).

The data to be inverted consists of $|Q|$ functions of time (``traces''). The time interval of recording is assumed to be the same for all traces, namely $[0,t_D]$. Traces are assumed square-integrable, members of $ L^2([0,t_D])$ - indeed the square of the trace $L^2$ norm is essentially the energy transferred from the surrounding acoustic fluid to the recording instrument \cite[]{SantosaSymes:00}. The set of traces forms the Hilbert space $D$ = data Hilbert space = $\bigoplus_{(\bx_r,\bx_s) \in Q} L^2([0,t_D])$. Note that no attempt is made to weight the traces for the spatial distribution of the acquisition coordinates.

The pressure field $p$, sampled at the receiver locations, is a component of the solution ${p,\bv}$ of the linear acoustic wave equation
\begin{eqnarray}
  \label{eqn:awe}
  \frac{\partial p}{\partial t} & = & - \kappa \nabla \cdot \bv +
                                      w_*(t) \delta(\bx-\bx_s); \\
  \frac{\partial \bv}{\partial t} & = & - \beta \nabla p; \\
  p, \bv & = & 0 \mbox{ for }  t \ll 0.
\end{eqnarray}
The right-hand side is localized in space at a source position $\bx_s \in Q_s$, with time dependence given by a {source wavelet} $w_* \in C_0^{\infty}(\bR)$. Details about the source mechanism are in fact difficult to measure directly and could justifiably be included in the parameters to be estimated in the solution of the inverse problem, but for the purposes of this discussion the wavelet $w_*$ is assumed known.

The acoustic parameter fields, bulk modulus ($\kappa$) and buoyancy ($\beta$) appearing in the system \ref{eqn:awe} are assumed smooth and uniformly bounded.
Since the speed of propagation is bounded as is the set of source locations $Q_s$ and the interval of propagation $[0,t_D]$, the values of $\kappa(\bx)$ and $\beta(\bx)$ for sufficiently large $|\bx|$ have no effect on the solution of the system \ref{eqn:awe}, so $\log \kappa$ and $\log \beta$ can be assumed square-integrable without loss of generality. The set of feasible models $M$ is a  bounded open subset of ${\cal M} = C^{\infty}(\bR^3) \times C^{\infty}(\bR^3) \cap L^{\infty}(\bR^3) \times L^{\infty}(\bR^3) \cap L^2(\bR^3) \times L^2(\bR^3)$.

Under these conditions, the following conclusions may be drawn about the system \ref{eqn:awe}:
\begin{itemize}
  \item[A1. ] For $m = (\log \kappa, \log \beta) \in M$, shot location ${\bf x}_s\in Q_s$, there exist distributions (pressure and velocity fields) $p(\cdot,\cdot;\bx_s)$, ${\bf v}(\cdot,\cdot;\bx_s)$ satisfying the system \ref{eqn:awe};
  \item[A2. ] $p(\cdot,\cdot,t;\bx_s)$, ${\bf v}(\cdot,\cdot;\bx_s)$ are smooth in the punctured space-time $\bR^4 \setminus \{(\bx_s,t): t \in \bR\}$;
    \item[A3. ] For $(\log \kappa, \log \beta) \in M$, define 
\[
  F[m](\bx_r,t;\bx_s) = p(\bx_r,t;\bx_s), \, (\bx_r,\bx_s) \in Q, \, t \in [0,t_D],
\]
for the pressure field $p$ solving \ref{eqn:awe}. The mapping ({\em forward map}, {\em modeling operator},...) $F: M \rightarrow D$ so defined satisfies
\begin{itemize}
  \item[A3.1 ] $\|F[m](\bx_r,\cdot;\bx_s)\|_{L^2([0,t_D])}$ is uniformly bounded over $m \in M, (\bx_r,\bx_s) \in Q$.
  \item[A3.2 ] $F$ has a directional derivative $DF[m]: {\cal M} \rightarrow D$ for every $m \in M$;
  \item[A3.3 ] for every $m \in M$, $DF[m]$ extends to a bounded map on ${\cal H} = L^2(\bR^3) \times L^2(\bR^3)$.
  \end{itemize}
\end{itemize}
See \cite{Symes:23a} for detailed statements and justifications of these assertions.

Adaptive filters are collections of traces defined on a (possibly)
different time interval, usually conveniently chosen to be
time-symmetric, that is, $[-t_U,t_U]$. The trace geometry of adaptive
filters is otherwise the same as that of the data traces. $U$ will denote the  adaptive
filter Hilbert space: $U= \bigoplus_{(\bx_r,\bx_s) \in Q} L^2([-t_U,t_U])$.

The central object in AWI is the adaptive-filter-to-data operator $S:
M \rightarrow {\cal B}(U,D)$, computed by trace-by-trace convolution
with predicted data: $S[m]u = F[m] * u$.
This notation for convolution in time hides a number of important details: none of the arguments are defined on the entire real line, so extension and restriction must be involved in the definition. Define $\Pi_D: \bigoplus_{(\bx_r,\bx_s) \in Q} L^2(\bR) \rightarrow D$ to be the restriction operator: that is, for $t \in [0,t_D]$, $\Pi_Df(\bx_r,t;\bx_s) = f(\bx_r,t;\bx_s)$. Then its adjoint $\Pi_D^T$ is the zero extension operator: that is, $\Pi_D^Tf(\bx_r,t;\bx_s) = f(\bx_r,t;\bx_s)$ if $t \in [0,t_D]$, $= 0$ else. Define $\Pi_U:  \bigoplus_{(\bx_r,\bx_s) \in Q} L^2(\bR) \rightarrow U$ similarly.

The meaning of convolution in the definition of $S$ is: extension followed by (trace-by-trace) convolution followed by restriction. That is, for $u \in U, f \in D$,
\begin{equation}
  \label{eqn:convdef}
  K[u]f(\bx_r,t;\bx_s) = \Pi_D\left(\int\,ds\,(\Pi_U^Tu)(\bx_rt-s;\bx_s) (\Pi_D^Tf)(\bx_r,s;\bx_s)\right)
\end{equation}
So defined, $K:U \rightarrow {\cal B}(D,D)$ is continuous and bilinear thanks to Young's inequality.

Convolution is commutative, so the right-hand side of the definition \ref{eqn:convdef} could also be viewed as defining $L: D \rightarrow {\cal B}(U,D)$, for $u \in U, f \in D$,
\begin{equation}
  \label{eqn:altconvdef}
  L[f]u = K[u]f
\end{equation}

A proper definition of $S[m]$ is thus
\begin{equation}
  \label{eqn:sdef}
  S[m]u = K[u]F[m] = L[F[m]]u.
\end{equation}
$S$ has a directional partial derivative with respect to its first argument, given by
\begin{equation}
  \label{eqn:dsdef}
  D_m(S[m]u) \delta m = K[u]DF[m] \delta m,\,m\in M, u \in U, \delta m \in {\cal M}.
\end{equation}
Since $DF[m]$ extends to a continuous linear map: ${\cal H} \rightarrow D$, the same is true of $D_mS[m]u$, $m \in M, u \in U$..

Since operation is trace-by-trace, that is block-diagonal, $S[m] = \mbox{diag}_{(\bx_r,\bx_s) \in Q}S[m]_{\bx_r,\bx_s}$. Fact A3.1 implies that $\|S[m]_{\bx_r,\bx_s}\|_{{\cal B}(L^2([-t_U,t_U]),L^2([0,t_D]))}$ is also uniformly bounded over $m \in M, (\bx_r,\bx_s) \in Q$. A similar statement applies to the derivative.

No-zero-traces Assumption: there exists $C>0$ so that for all $(\bx_r,\bx_s) \in Q$,
$\|d(\bx_r,\cdot;\bx_s)\|_{L^2([0,t_D])} \ge C.$

Observable Data Assumption: there exists $C>0$ so that for all $m \in M, (\bx_r,\bx_s) \in Q$,
\begin{equation}
  \label{eqn:obsdata}
  \|S[m]_{\bx_r,\bx_s}^Td(\bx_r,\cdot;\bx_s)\|_{L^2[-t_U,t_U]} \ge C \|d(\bx_r,\cdot;\bx_s)\|_{L^2[0,t_D]}
\end{equation}.
That is, the projection of each trace onto the range of convolution of the corresponding predicted data trace is coercive.  

$\sigma > 0$ = Tihonov regularization weight

$\alpha \ge 0$ = penalty weight

Definition of AWI penalty objective divides into two parts:

\subsection{ Unpenalized adaptive filter}
The unpenalized filter $u_{0,\sigma}[m]$ solves the regularized least squares data fitting problem:  given $m \in M$, minimizes
\[
 J_{0,\sigma}[u,m] = \frac{1}{2}(\|S[m]u - d\|_D^2 + \sigma^2 \|u\|^2_U).
\]
Since $\sigma > 0$, the normal operator is uniformly bounded below and of class $C^1$ in $M$ and $D$. So a unique solution $u_{0,\sigma}[m] \in C^1(M, U)$ exists.

Also, the normal operator is block-diagonal, so each trace solves a least-squares problem:
\begin{equation}
  \label{eqn:blocknormal}
  (S[m]_{\bx_r,\bx_s}^TS[m]_{\bx_r,\bx_s} + \sigma^2 I)u_{0,\sigma}[m](\bx_r,\cdot; \bx_s)= S[m]_{\bx_r,\bx_s}^Td(\bx_r,\cdot;\bx_s)
\end{equation}

According to Fact A3.1 and the Observable Data Assumption, the left-hand side of equation \ref{eqn:blocknormal} is bounded by a multiple of $\|u_{0,\sigma}[m,d](\bx_r,\cdot; \bx_s)\|_{L^2[-t_U,t_U]}$, uniformly in $m \in M, (\bx_r,\bx_s) \in Q$. According to the Observable Data Assumption, the right-hand side is bounded below by a multiple of $\|d(\bx_r,\cdot;\bx_s)\|_{L^2[0,t_D]}$, uniform in the same sense. Conclude that there exists $C>0$ so that
\begin{equation}
  \label{eqn:u0lower}
  \|u_{0,\sigma}[m,d](\bx_r,\cdot; \bx_s)\|_{L^2[-t_U,t_U]} \ge C \|d(\bx_r,\cdot;\bx_s)\|_{L^2[0,t_D]}
\end{equation}
uniformly in $m \in M, (\bx_r,\bx_s) \in Q$.

\subsection{AWI Penalty Operator}
The penalty operator is also block diagonal, that is, operates independently on each trace. For the trace at $(\bx_r,\bx_s) \in Q$, 
\begin{equation}
  \label{eqn:ppf}
  T_{\sigma}[m]u (\bx_r,t;\bx_s) = \frac{t u(\bx_r,t;\bx_s)}{\|u_{0,\sigma}[m](\bx_r,\cdot; \bx_s)\|_{L^2[-t_U,t_U]} }
\end{equation}
That is, this penalty operator scales the trace by time, so that the output is small if the nonzero samples are concentrated near $t=0$. The second scaling, by the reciprocal of norm of the trace $u_{0,\sigma}[m](\bx_r,\cdot;\bx_s)$, plays an essential role in linking the AWI objective to the travel-time residual, as is explained by \cite{HCSMWS:23a}
This operator is well-defined and $C^1$ as function of $m$ due to the
bound \ref{eqn:u0lower} and the the No-zero-trace assumption.

To express the penalty operator $T_{\sigma}[m]$ conveniently, along
with similar operators appearing in the expression for the gradient
developed below, introduce functions
\begin{enumerate}
\item $R: U \rightarrow ({\bf R}^{+})^{|Q|}$,
  \[
    Ru((\bx_r,\bx_s)) = \|u(\bx_r, \cdot;\bx_s)\|
  \]
\item $W: {\bf R}^{|Q|} \times {\bf R} \times {\bf R}^{+}
  \rightarrow {\cal B}(D,D)$,
  \[
    W(q,p_s, p_t)f (\bx_r,t;\bx_s) = t^{p_t}
    q(\bx_r,\bx_s)^{p_s}f(\bx_r,t;\bx_s)
  \]
\end{enumerate}
Use the same notation for the similar function with $D$ replaced by
$U$. Note that $W$ is only well-defined if either $p_s \ge 0$ or
$q>0$.

For convenience, note these obvious properties of $W$:
\begin{itemize}
\item[W1. ] $W(\cdot,\cdot,0)$ is block-scalar, hence commutes with any
  block-diagonal operator;
\item[W2. ] $W(q,p_{s,1},p_{t,1})W(q,p_{s,2},p_{t,2}) = W(q,p_{s,1}+p_{s,2},p_{t,1}+p_{t,2})$;
\item[W3. ] $W(q_1,p_s,p_{t,1})W(q_2,p_s,p_{t,2}) = W(q_1,q_2,
  p_s,p_{t,1}+p_{t,2})$
\end{itemize}

With these notations, $T_{\sigma}[m]$ may be expressed as
\begin{equation}
  \label{eqn:talt}
  T_{\sigma}[m] = W(Ru_{0,\sigma}[m],-1, 1)
\end{equation}

\section{Objectives}
Define for $\alpha \ge 0$
\begin{equation}
  \label{eqn:j}
 J_{\alpha,\sigma}[u,m,d] = \frac{1}{2}(\|S[m]u - d\|_D^2 + \alpha^2\|T_\sigma[m]u\|_U^2 + \sigma^2 \|u\|^2_U).
\end{equation}
Note that the notation is consistent, that is, reduces to $J_{0,\sigma}$ as defined above for $\alpha=0$.

The variable projection reduction of $J_{\alpha,\sigma}$ is
\begin{equation}
  \label{eqn:jtilde}
 \tilde{J}_{\alpha,\sigma}[m,d] = \frac{1}{2}(\|S[m]u_{\alpha,\sigma}[m,d] - d\|_D^2 + \alpha^2\|T_{\sigma}[m]u_{\alpha,\sigma}[m,d]\|_U^2 + \sigma^2 \|u_{\alpha,\sigma}\|^2_U).
\end{equation}
in which $u_{\alpha,\sigma}[m,d]$ is the minimizer of $J_{\alpha\sigma}[u,m,d]$ over $u \in U$, that is, the solution of the normal equation
\begin{equation}
  \label{eqn:normal}
  (S[m]^TS[m] + \alpha^2T_{\sigma}[m]^TT_{\sigma}[m] + \sigma^2I)u_{\alpha,\sigma}[m] = S[m]^Td
\end{equation}
Note that $u_{0,\sigma}[m]$ solves the above system for $\alpha=0$, and that the left-hand side is block-diagonal, similar to the left-hand side of equation \ref{eqn:blocknormal}.

Computing $\tilde{J}_{\alpha,\sigma}$ requires the following steps:
\begin{enumerate}
\item given $m$, $d$, $\alpha$, and $\sigma$, compute $u_{0,\sigma}[m]$ by solving the normal equations \ref{eqn:normal} for $\alpha=0$;
\item use $u_{0,\sigma}[m]$ to construct $T_{\sigma}[m]$ by means of
  equation \ref{eqn:talt};
\item use $S[m]$, $T_{\sigma}[m]$ to compute $u_{\alpha,\sigma}[m]$ by solving the normal equations \ref{eqn:normal} with the given value of $\alpha$;
\item assemble $\tilde{J}_{\alpha,\sigma}[m]$ according to equation \ref{eqn:jtilde}.
\end{enumerate}

Therefore evaluating $\tilde{J}_{\alpha,\sigma}$ involves solution of
two least-squares problems, in sequence.

\section{Derivatives}
The fundamental result of variable projection theory
\cite[]{GolubPereyra:73} applies under the conditions presented here:
the stationary points of $J_{\alpha,\sigma}$ and
$\tilde{J}_{\alpha,\sigma}$ are in bijective correspondence. Moreover,
\[
D_m\tilde{J}_{\alpha,\sigma}[m]\delta m = D_mJ[u,m]\delta m|_{u =u_{\alpha,\sigma}[m,d]}
\]
\[
  = \langle D_m(S[m]u)\delta m_{u =  u_{\alpha,\sigma}[m]} (S[m]u_{\alpha,\sigma}[m]-d) \rangle
\]
\begin{equation}
+ \alpha^2 \langle D_m(T_{\sigma}[m]u)\delta m|_{u=u_{\alpha,\sigma}[m]},T_{\sigma}[m]u_{\alpha,\sigma}[m]\rangle_U.
\label{eqn:basederiv}
\end{equation}
Note that the regularization term does not appear in the VPM gradient expression, since it does not depend explicitly on $m$.

This section presents the derivative of $\tilde{J}_{\alpha,\sigma}$, expressed in terms of
\begin{itemize}
  \item trace-by-trace convolution and cross-correlation operators,
  \item the derivative $DF[m]$ and its adjoint, and
  \item trace-by-trace scaling via the operators $R$ and $W$.
\end{itemize}
These the first two items are common components of typical FWI implementations. The
third is necessary to express the AWI penalty operator.

\subsection{Single Trace}

Since both $S$ and $T$ are block-diagonal, acting trace-by-trace, I
will first compute the derivatives in the single-trace case. For the
remainder of this section, $F$, $S$, $T$, $d$, $u$, and $u_0$ denote
the restriction of the corresponding operators and vectors to a trace subspace, that is, for some $(\bx_r,\bx_s) \in Q$,
\begin{eqnarray*}
  F[m] = F[m]_{\bx_r,\bx_s}&: & M \rightarrow D_{\bx_r,\bx_s} =
                 L^2[0,t_D] \\
  S[m] = S[m]_{\bx_r,\bx_s}&: & L^2[-t_U,t_U] \rightarrow L^2[0,t_D] \\
  T[m] = T[m]_{\bx_r,\bx_s}&: & L^2[-t_U,t_U] \rightarrow L^2[-t_U,t_U] \\
  u_{0,\sigma}[m] &=& u_{0,\sigma}[m](\bx_r,\cdot;\bx_s).
\end{eqnarray*}
Inner products and transposes are implicit in these definitions.

The derivative of $S$ is related to the derivative of $F$ by
convolution (definition \ref{eqn:sdef}), so the first term in the
right-hand side of equation \ref{eqn:basederiv} is
\[
  \langle D_m(S[m]u)\delta m, S[m]u-d \rangle = \langle
  K[u]DF[m]\delta m, S[m]u-d \rangle
\]
\begin{equation}
  \label{eqn:resderiv}
  = \langle DF[m]\delta m, K[u]^T(S[m]u-d) \rangle
\end{equation}

With the notational conventions listed above, the normal equation \ref{eqn:blocknormal} for $u_{0,\sigma}[m]$ implies
\[
  D_mu_{0,\sigma}[m] \delta m = -(S[m]^TS[m] + \sigma^2I)^{-1}
\]
\begin{equation}
  \label{eqn:blockderiv}
  \times ((DS[m]^{T}\delta m) (S[m]u_{0,\sigma}[m]-d) + S[m]^T(DS[m]\delta m) u_{0,\sigma}[m])
\end{equation}
The single-trace version of the definition \ref{eqn:ppf} is
\[
  T_{\sigma}[m]u)(t) = \frac{t u(t)}{\|u_{0,\sigma}[m]\|},\,u \in L^2[-t_U,t_U],
\]
so
\[
D(T_{\sigma}[m] u)\delta m = -\frac{t u}{\|u_{0,\sigma}[m]\|^3}\langle D_mu_{0,\sigma}[m]\delta u, u_{0,\sigma}[m]\rangle.
\]
\[
  =-\frac{t u}{\|u_{0,\sigma}[m]\|^3}\langle  (S[m]^TS[m] +
  \sigma^2I)^{-1}((DS[m]^{T}\delta m) (d - S[m]u_{0,\sigma}[m])
\]
\[
   - S[m]^T(DS[m]\delta m) u_{0,\sigma}[m]), u_{0,\sigma}[m] \rangle
\]
Define
\begin{equation}
  \label{eqn:defv0tilde}
  \tilde{v}_{0,\sigma}[m] = (S[m]^TS[m]+ \sigma^2I)^{-1} u_{0,\sigma}[m] \in
  L^2[-t_U,t_U].
\end{equation}
Then
\[
  D(T_{\sigma}[m] u)\delta m =
  \frac{t u}{\|u_{0,\sigma}[m]\|^3}[\langle (DS[m]\delta
  m)\tilde{v}_{0,\sigma}[m], S[m]u_{0,\sigma}[m]-d\rangle
\]
\[+
  \langle S[m]\tilde{v}_{0,\sigma}[m], (DS[m]\delta m) u_{0,\sigma}[m] \rangle ]
\]
and
\[
  \langle D(T_{\sigma}[m]u)\delta m, T_{\sigma}[m]u\rangle =
\frac{\langle t u, tu \rangle}{\|u_{0,\sigma}[m]\|^4}
\]
\[
\times [\langle (DS[m]\delta
  m)\tilde{v}_{0,\sigma}[m], S[m]u_{0,\sigma}[m]-d\rangle
+
  \langle S[m]\tilde{v}_{0,\sigma}[m], (DS[m]\delta m) u_{0,\sigma}[m] \rangle ]
\]
\[
  = \frac {\|t u\|^2}{\|u_{0,\sigma}[m]\|^4}
\]
\[
  \times [\langle DF[m]\delta m, K[\tilde{v}_{0,\sigma}[m]]^T(S[m]u_{0,\sigma}[m]-d)
+
K[u_{0,\sigma}[m]]^T S[m]\tilde{v}_{0,\sigma}[m]\rangle ]
\]

\[
  = \langle DF[m]\delta m, W(R(W(Ru_{0,\sigma}[m],-2,1)u),2,0)
\]
\begin{equation}
  \label{eqn:penderiv}
  \times ( K[\tilde{v}_{0,\sigma}[m]]^T(S[m]u_{0,\sigma}[m]-d)
+
K[u_{0,\sigma}[m]]^T S[m]\tilde{v}_{0,\sigma}[m])\rangle
\end{equation}

It will turn out to be convenient to take advantage of the {\em
  block-scalar} property of the operator
$W(R(W(Ru_{0,\sigma}[m],-2,1)u),2,0)$. The other operators on the RHS
of equation \ref{eqn:penderiv} are all block-diagonal, and the
right-hand side is linear in $\tilde{v}_{0,\sigma}[m]$. Thus the
$W...$ operator can be combined with the definition of $\tilde{v}...$:
define
\[
  v_{0,\sigma}[m,u]=W(R(W(Ru_{0,\sigma}[m],-2,1)u),2,0) \tilde{v}_{0,\sigma}[m]
\]
to be the solution of the system
\begin{equation}
  \label{eqn:defv0}
  (S[m]^TS[m]+ \sigma^2I) v_{0,\sigma}[m,u] = W(R(W(Ru_{0,\sigma}[m],-2,1)u),2,0) u_{0,\sigma}[m].
\end{equation}

Then equation \ref{eqn:penderiv} can be re-written as
\[
  \langle D(T_{\sigma}[m]u)\delta m, T_{\sigma}[m]u\rangle = \langle DF[m]\delta m, 
\]
\begin{equation}
  \label{eqn:penderiv1}
   K[v_{0,\sigma}[m,u]]^T(S[m]u_{0,\sigma}[m]-d)
+
K[u_{0,\sigma}[m]]^T S[m]v_{0,\sigma}[m,u]\rangle
\end{equation}
Combine equations \ref{eqn:resderiv} and \ref{eqn:penderiv1} with
equation \ref{eqn:basederiv} to obtain
\[
  \nabla \tilde{J}_{\alpha,\sigma}[m] =
  DF[m]^T\{ K[u_{\alpha,\sigma}[m]]^T (S[m]u_{\alpha,\sigma}[m]-d)
\]
\begin{equation}
  \label{eqn:tildejgrad}
  + \alpha^2 (
  K[v_{0,\sigma}[m, u_{\alpha,\sigma}[m]]]^T(S[m]u_{0,\sigma}[m]-d)
+ K[u_{0,\sigma}[m]]^T S[m]v_{0,\sigma}[m,u_{\alpha,\sigma}[m]]) \}.
\end{equation}

\noindent{\bf Remark:} ``Gradient'' should really be in quotes, as $F$ is not actually differentiable, or even well-defined, in any open subset of ${\cal H}$. It is however well-defined in the intersection of $M$ with any ${\cal H}$-closed subspace of ${\cal M}$. The latter is necessarily finite-dimensional, but all feasible calculations are carried out in finite-dimensional settings, so that's OK.
 
\subsection{Multiple Traces}
The general case follows immediately from the single trace case, since
every operator ($S[m]$, $K[...]$, $W$, $R$) ,appearing in the right-hand side of expression
\ref{eqn:tildejgrad} is block-diagonal (trace-by-trace), with the exception
of $DF[m]^T$. This latter is a reduction operation, i.e. a block-row
operation, so its expression is also the same. That is, equation
\ref{eqn:tildejgrad} also correctly expresses the gradient in the
multi-trace case.

The gradient is unlikely to be needed without the value
of $\tilde{J}_{\alpha,\sigma}$, so the vector of trace norms
$Ru_{0,\sigma}[m]$ will have been computed, as it is a key ingredient
in the definition of the penalty
$T_{\sigma}[m]u_{\alpha,\sigma}$, and can be
cached.

\begin{enumerate}
\item given $m$, $d$, $\alpha$, and $\sigma$, compute
  $u_{0,\sigma}[m]$ by solving the normal equations \ref{eqn:normal}
  for $\alpha=0$, also $Ru_{0,\sigma}$ (this has
  likely already been done to compute the value of $\tilde{J}_{\alpha,\sigma}$);
\item compute $u_{\alpha,\sigma}[m]$ by
  solving the normal equations \ref{eqn:normal}, and record
  $R(W(Ru_{0,\sigma}[m],-2,1)u_{\alpha,\sigma}[m])$ as a by-product;
\item compute $v_{0\sigma}[m, u_{\alpha,\sigma}[m]] \in U$ by solving the linear system
  \ref{eqn:defv0};
\item compute $r_0 = S[m]u_{0,\sigma}[m]-d,\,r_{\alpha} = S[m]u_{\alpha,\sigma}[m]-d$ (these may be by-products of steps 1 and 3;
\item compute four vectors in $D$:
  \begin{enumerate}
  \item $G_1 = K[u_{\alpha,\sigma}[m]]^T r_{\alpha}$
  \item $G_2 = K[v_{0,\sigma}[m, u_{\alpha,\sigma}[m]]]^Tr_0$
  \item $G_3 = K[u_{0,\sigma}[m]]^T S[m]v_{0,\sigma}[m, u_{\alpha,\sigma}[m]]$
  \item $G_0 = G_1 + \alpha^2 (G_2 + G_3)$
  \end{enumerate}
\item assemble
  $\nabla \tilde{J}_{\alpha,\sigma}[m] = DF[m]^TG_0$
\end{enumerate}

\subsection{Warner-Guasch AWI}
The (``forward'') AWI objective defined in \cite{Warner:16} is
the limit
\begin{equation}
  \label{eqn:wglim}
  \tilde{J}_{\sigma}[m] = \lim_{\alpha \rightarrow 0}
  \frac{1}{\alpha^2}(\tilde{J}_{\alpha,\sigma}[m]-\tilde{J}_{0,\sigma}[m])
\end{equation}
\begin{equation}
  \label{eqn:wg}
  = \frac{1}{2} \|T_{\sigma}[m]u_{0,\sigma}\|^2
\end{equation}

In the following paragraphs, I will compute the $m$ derivative of
$\tilde{J}_{\sigma}$ directly from the right-hand side of the
definition \ref{eqn:wg}. Then I will compute it as the limit of
derivatives as suggested in the identity \ref{eqn:wglim}, and show
that these are the same.  A by-product of this comparison is the
observation that the limit \ref{eqn:wglim} is in the sense of
$C^1(L^{\infty}(\bR^3) \times L^{\infty}(\bR^3))$.

The derivative of the expression on the right-hand side of
\ref{eqn:wg} is the sum of two terms, the first of which is given by
\ref{eqn:penderiv1} with $u = u_{0,\sigma}[m]$. The second is
\[
  \langle T_{\sigma}[m]u_{0,\sigma}[m], T_{\sigma}[m]D_m
  u_{0,\sigma}[m]\delta m \rangle
\]
which from identity \ref{eqn:blockderiv} is
\[
  = -\langle  T_{\sigma}[m]^TT_{\sigma}[m]u_{0,\sigma}[m], (S[m]^TS[m] + \sigma^2I)^{-1}
\]
\[
  \times ((DS[m]^{T}\delta m) (S[m]u_{0,\sigma}[m]-d) +
  S[m]^T(DS[m]\delta) m u_{0,\sigma}[m])
\]
Since
$T_{\sigma}[m]^TT_{\sigma}[m] = W(Ru_{0,\sigma}[m],-2,2)$, this is
\[
  = -\langle (DS[m]\delta m) (S[m]^TS[m] +
  \sigma^2I)^{-1}W(Ru_{0,\sigma}[m],-2,2)u_{0,\sigma}[m],
  S[m]u_{0,\sigma}[m] - d \rangle
\]
\begin{equation}
  \label{eqn:term2}
  - \langle S[m] (S[m]^TS[m] +
  \sigma^2I)^{-1}W(Ru_{0,\sigma}[m],-2,2)u_{0,\sigma}[m], (DS[m]\delta
  m)u_{0,\sigma}[m] \rangle
\end{equation}
Introduce the abbreviation
\begin{equation}
  \label{eqn:defv1}
  v_{1,\sigma}[m] = (S[m]^TS[m] + \sigma^2I)^{-1}W(Ru_{0,\sigma}[m],-2,2)u_{0,\sigma}[m].
\end{equation}
Then the RHS of equation \ref{eqn:term2bis} may be re-written as
\begin{equation}
  \label{eqn:term2bis}
  = -\langle DF[m]\delta m, K[v_{1,\sigma}[m]]^T (S[m]u_{0,\sigma}[m]-d) + 
  K[u_{0,\sigma}[m]]^T S[m] v_{1,\sigma}[m] \rangle.
\end{equation}
Recall that we have just computed the second of two terms, the sum of
which is the derivative of $\tilde{J}_{\sigma}[m]$. The first is
\ref{eqn:penderiv1} with $u = u_{0,\sigma}[m]$. Combine the two
terms to obtain
\[
  D\tilde{J}_{\sigma}[m] \delta m = \langle DF[m]\delta m,
\]
\[
  K[v_{0,\sigma}[m,u_{0,\sigma}[m]]]^T(S[m]u_{0,\sigma}[m]-d)
+
K[u_{0,\sigma}[m]]^T S[m]v_{0,\sigma}[m,u_{0,\sigma}[m]]\rangle
\]
\[
  -\langle DF[m]\delta m, K[v_{1,\sigma}[m]]^T (S[m]u_{0,\sigma}[m]-d) + 
  K[u_{0,\sigma}[m]]^T S[m] v_{1,\sigma}[m] \rangle
\]
\[
  = \langle DF[m]\delta
  m,(K[v_{0,\sigma}[m,u_{0,\sigma}[m]]]-K[v_{1,\sigma}[m]])^T(S[m]u_{0,\sigma}[m]-d)
\]
\[
  + K[u_{0,\sigma}[m]]^T S[m] (v_{0,\sigma}[m,u_{0,\sigma}[m]] -
  v_{1,\sigma}[m]) \rangle.
\]
To simplify this expression further, and reduce the computational
complexity of the derivative expression, note that the
operator-valued function $K$ is linear in its argument, according to
its definition \ref{eqn:convdef}. So the above is
\begin{equation}
  \label{eqn:wgderiv}
  = \langle DF[m]\delta m, (K[v_{\sigma}[m]]^T(S[m]u_{0,\sigma}[m]-d) 
  + K[u_{0,\sigma}[m]]^T S[m] v_{\sigma}[m] )\rangle.
\end{equation}
in which
\[
  v_{\sigma}[m] = v_{0,\sigma}[m,u_{0,\sigma}[m]] -  v_{1,\sigma}[m].
\]
The definitions \ref{eqn:defv0} and \ref{eqn:defv1} imply that
$v_{\sigma}[m]$ is the solution of the linear system
\[
    (S[m]^TS[m]+ \sigma^2I) v_{\sigma}[m] =
\]
\begin{equation}
  \label{eqn:defvsigma}
  (W(R(W(Ru_{0,\sigma}[m],-2,1)u_{0,\sigma}),2,0) - W(Ru_{0,\sigma}[m],-2,2))
  u_{0,\sigma}[m].
\end{equation}
That is, computation of $v_{\sigma}[m]$, hence of $D
\tilde{J}_{\sigma}[m]\delta m$, requires the solution of only one additional
linear system, beyond that necessary to compute the value of $\tilde{J}_{\sigma}[m]$.

\noindent {\bf Remark:} The expression \ref{eqn:wgderiv} immediately
yields a formal prescription for the gradient of
$\tilde{J}_{\sigma}[m]$ as the output of the adjoint modeling operator
$DF[m]^T$ applied to an ``adjoint source'' (the right-hand side of the
inner product \ref{eqn:wgderiv}. \cite{Warner:16} point out this
similarity with FWI, in which the adjoint source (in the simplest
version) is the residual $F[m]-d$. Consequently, FWI implementations
already contain a great deal of the code necessary to implement
AWI. \cite{Warner:16} also presume that the regularization weight
$\sigma$ is sufficiently small that the adaptive filter $u_{0,\sigma}$
effectively fits the data. Thus $S[m]u_{0,\sigma}-d$ is negligible,
regardless of $m$, and the first term on the right-hand side of the
inner product \ref{eqn:wgderiv} drops out, further simplifying the AWI
adjoint source.

It remains to establish that the same result (expression
\ref{eqn:wgderiv}) for the derivative can be derived from the
limit \ref{eqn:wglim}. From from equation \ref{eqn:basederiv},
\[
  \frac{1}{\alpha^2}(D \tilde{J}_{\alpha,\sigma}[m] \delta m - D
  \tilde{J}_{0,\sigma}[m]\delta m) =
\]
\[
  = \frac{1}{\alpha^2} \left(\langle D_m(S[m]u)\delta m_{u =  u_{\alpha,\sigma}[m]}, (S[m]u_{\alpha,\sigma}[m]-d) \rangle\right.
\]
\[
  - \left.\langle D_m(S[m]u)\delta m_{u =  u_{0,\sigma}[m]}, (S[m]u_{0,\sigma}[m]-d) \rangle\right)
\]
\begin{equation}
  \label{eqn:limdiff}
  +  \langle D_m(T_{\sigma}[m]u)\delta
  m|_{u=u_{\alpha,\sigma}[m]},T_{\sigma}[m]u_{\alpha,\sigma}[m]\rangle_U
\end{equation}
Since the normal operator \ref{eqn:normal} is uniformly positive
definite hence boundedly invertible uniformly in $\alpha \ge 0$,
$u_{\alpha,\sigma}[m] \rightarrow u_{0,\sigma}[m]$ in $U$. All of the
operators appearing in the third term are $U-$continuous, so  the
third term tends to
\[
  \langle D_m(T_{\sigma}[m]u)\delta
  m|_{u=u_{0,\sigma}[m]},T_{\sigma}[m]u_{0,\sigma}[m]\rangle_U
\]
which is identixal to the left-hand side of equation
\ref{eqn:penderiv1} with $u=u_{0,\sigma}[m]$ and is the first term in the direct computation of
the $\tilde{J}_{\sigma}$ derivative, analyzed above.

It remains to understand
\[
\frac{1}{\alpha^2} \left(\langle D_m(S[m]u)\delta m_{u =  u_{\alpha,\sigma}[m]}, (S[m]u_{\alpha,\sigma}[m]-d) \rangle\right.
\]
\[
  - \left.\langle D_m(S[m]u)\delta m_{u =  u_{0,\sigma}[m]}, (S[m]u_{0,\sigma}[m]-d) \rangle\right)
\]
\[
  =\frac{1}{\alpha^2} \left(\langle D_m(S[m]u)\delta m_{u =  u_{\alpha,\sigma}[m]-u_{0,\sigma}[m]}, (S[m]u_{\alpha,\sigma}[m]-d) \rangle\right.
\]
\begin{equation}
  \label{eqn:resterms}
  + \left.\langle D_m(S[m]u)\delta m_{u =  u_{0,\sigma}[m]},
    (S[m](u_{\alpha,\sigma}[m]-u_{0,\sigma}[m]) \rangle\right)
\end{equation}
Observe that the normal equation \ref{eqn:normal} implies that
\[
  (S[m]^TS[m] + \sigma^2 I)(u_{\alpha,\sigma}[m]-u_{0,\sigma}[m]) =
  -\alpha^2 T_{\sigma}[m]^TT_{\sigma}[m]u_{\alpha,\sigma}[m]
\]
Since $u_{\alpha,\sigma}[m] \rightarrow u_{0,\sigma}$ in the sense of
$U$ as $\alpha \rightarrow 0$, it follows from definition \ref{eqn:defv1} that
\[
  \frac{1}{\alpha^2 }u_{\alpha,\sigma}[m]-u_{0,\sigma}[m] \rightarrow
  - v_{1,\sigma}[m], \, \alpha \rightarrow 0
\]
Writing as before $D_m(S[m]u)\delta m_{u =  ...} =
(DS[m]\delta m)...$,  the quantity on the right-hand side of equation
\ref{eqn:resterms}, obtain
\[
  \lim_{\alpha \rightarrow 0} 
\frac{1}{\alpha^2} \left(\langle D_m(S[m]u)\delta m_{u =  u_{\alpha,\sigma}[m]}, (S[m]u_{\alpha,\sigma}[m]-d) \rangle\right.
\]
\[
  - \left.\langle D_m(S[m]u)\delta m_{u =  u_{0,\sigma}[m]}, (S[m]u_{0,\sigma}[m]-d) \rangle\right)
\]
\[
  = - \langle (DS[m]\delta m)v_{1,\sigma}[m], S[m]u_{0,\sigma}[m]-d
  \rangle
\]
\[
  - \langle (DS[m]\delta m) u_{0,\sigma}[m],S[m]v_{1,\sigma}[m] \rangle
\]
\[
  =-\langle DF[m]\delta m, K[v_{1,\sigma}[m]]^T (S[m]u_{0,\sigma}[m]-d) + 
  K[u_{0,\sigma}[m]]^T S[m] v_{1,\sigma}[m] \rangle.
\]
which is precisely the right-hand side of equation
\ref{eqn:term2bis}. That is, the limit of the first two terms on the
right-hand side of equation \ref{eqn:limdiff} as $\alpha \rightarrow
0$ is exactly the second term of the direct derivative
computation. Since we have already shown that the third term in
\ref{eqn:limdiff} limits to
the first term in the direct computation, a second derivation of the
derivative $D\tilde{J}_{\sigma}[m]$ has been accomplished.

We have also shown that the limit of the scaled difference of the
derivatives is the derivative of the limit of the scaled difference,
pointwise in $m \in M$. In fact the limit is locally uniform in a
suitable sense, but that is a topic for another discussion.

\append{Unpreconditioned AWI: Matched Source Waveform Inversion}
Matched Source Waveform Inversion (MSWI) is a close relative to
AWI, differing in the use of a simpler penalty operator $T$, defined
by
\begin{equation}
  \label{eqn:upf}
  Tu (\bx_r,t;\bx_s) = t u(\bx_r,t;\bx_s).
\end{equation}
Similarly to AWI, $m$ is estimated by minimizing
\begin{equation}
  \label{eqn:junp}
 J^{\rm unp}_{\alpha,\sigma}[u,m,d] = \frac{1}{2}(\|S[m]u - d\|_D^2 + \alpha^2\|Tu\|_U^2 + \sigma^2 \|u\|^2_U).
\end{equation}
or equivalently
\begin{equation}
  \label{eqn:jtildeunp}
 \tilde{J}^{\rm unp}_{\alpha,\sigma}[m,d] = \frac{1}{2}(\|S[m]u_{\alpha,\sigma}[m,d] - d\|_D^2 + \alpha^2\|Tu_{\alpha,\sigma}[m,d]\|_U^2 + \sigma^2 \|u_{\alpha,\sigma}\|^2_U).
\end{equation}
in which $u_{\alpha,\sigma}$ is the solution of the normal equation
\begin{equation}
  \label{eqn:normalunp}
  (S[m]^TS[m] + \alpha^2T^TT + \sigma^2I)u_{\alpha,\sigma}[m] = S[m]^Td
\end{equation}
The only difference between AWI and MSWI lies in the absence of
scaling by the reciprocal of $Ru_{0,\sigma}[m]$ in the definition of the MSWI
penalty operator.

MSWI was introduced in Hua Song's thesis
\cite[]{Song:94c} using a different definition of $T$ (but also
$m$-independent). The penalty operator defined in \ref{eqn:upf}, and
the associated inversion algorithm, was explored by Huang and Symes
\cite[]{HuangSymes2015SEG,HuangSymes:Geo17}, who show that at least to
some extent it avoids the cycle-skipping pathology of FWI. Note that
the division by $Ru_{0,\sigma}[m]$ in the definition of
$T_{\sigma}[m]$ is responsible for the global relation between AWI and
travel time tomography, within a limited class of problems, and this
relation cannot be asserted for MSWI.

Since the MSWI penalty operator $T$ is $m$-independent, it does not
appear explicilly in the the gradient
of the reduced MSWI objective $\tilde{J}^{\rm unp}_{\alpha,\sigma}$, as
given by the VPM gradient formula:
\[
  \nabla \tilde{J}^{\rm unp}_{\alpha,\sigma}[m] =
  D_m(S[m]u)^T_{u=u_{\alpha,\sigma}[m]}(S[m]u_{\alpha,\sigma}[m]-d)
\]
\begin{equation}
  \label{eqn:tildejunpgrad}
  = DF[m]^TK[u_{\alpha,\sigma}[m]]^T(S[m]u_{\alpha,\sigma}[m]-d).
\end{equation}

Note that the right-hand side of equation \ref{eqn:tildejunpgrad} is
exactly the first term in the right-hand side of equation
\ref{eqn:tildejgrad}. That is, the gradient of the AWI objective
function is the gradient of the MSWI objective function, modified by
terms that account for the $m$-dependence of the AWI penalty operator $T_{\sigma}[m]$.

\append{Remarks on VCL Implementation}
\noindent {\bf 1.} $F$ is the mapping implemented (in finite difference approximation) in {\tt asg.fsbop}, with the additional constraint that the buoyancy field is fixed so that the computational version of ${\cal M}$ consists of (gridded) bulk modulus fields.

\noindent {\bf 2.} The convolution operator implemented in the module {\tt segyvc}
actually implements a time-discrete version of $K$ (or $L$). Note that
the transpose $K[u]^T$ is a restricted version of cross-correlation,
implemented as the transpose of {\tt segyvc.ConvolutionOperator}.

\noindent {\bf 3.} IWAVE implements the mappings $R$ and $W$ in the form of
the commands \\{\tt iwave/trace/main/rms.x} and {\tt
  iwave/trace/main/txgain.x}. The natural python implementation of $R$
would output the vector of trace norms as a python array (or a NumPy
ndarray). However that would require extracting the number of traces
from the input, and checking that the inputs are compatible. This is
already implicitly done in the IWAVE manipulations of SU trace
data. So {\tt iwave/trace/main/rms.x} stores the vector of trace norms
as the 0th data sample in a set of traces, one for each trace of the
input data: all header words are the same except for {\tt ns}, set to 1.

\noindent {\bf 4.} The python functions {\tt awi.setrms} and {\tt awi.applytxgain} call
the two IWAVE commands, manipulating the data as SU data files, rather
than as VCL vectors. That is, these are simply functions, not {\tt
  vcl.Function} objects. Since VCL vectors defined on a {\tt segyvc.Space}
have access to their data files (keyword {\tt data}), these are
convenient to use to express the action of $R$ and $W$ in AWI penalty operator,
gradient, and so on.

\noindent {\bf 5.} The implementation {\tt vcalg.conjgrad} of the Conjugate Gradient algorithm for the normal equations optionally returns the residual vector (parameter {\tt e}). The obvious way to use this is to create a 3-component product space for the output of the block-column operator $(S[m],\alpha T[m], \sigma I)^T$, and set the right-hand side vector (parameter {\tt b})
to $(d,0,0)^T$. Then on return one-half the norm-squared of {\tt e} is
precisely the (approximate) value of $\tilde{J}_{\alpha,\sigma}[m]$.

\append{$\alpha \rightarrow 0$ scaled limit of penalty function}
Define
\begin{equation}
  \label{eqn:eq0}
  \tJa = \min_u J_{\alpha}(u) = J_{\alpha}(\ua)
\end{equation}
Since $S$ is positive definite, the minimum is well-defined for any $\alpha \ge 0$, and the minimizer $\ua \in U$ satisfies the normal equation.

The claim to be established is that
\begin{equation}
  \label{eqn:eq4}
  \lim_{\alpha \rightarrow 0} \frac{1}{\alpha^2}  (\tJa-\tJz).= \frac{1}{2}\|A\uz\|_m^2
\end{equation}

To see this, use the normal equation to write
\[
  \ua = (S^TS + \alpha^2 A^TA)^{-1}S^Td = (S^TS + \alpha^2 A^TA)^{-1}S^TS \uz
\]
\begin{equation}
  \label{eqn:eq2}
  = \uz - \alpha^2 (S^TS + \alpha^2 A^TA)^{-1}A^TA\uz = \uz-\alpha^2 \va
\end{equation}
Note that $\va = (S^TS + \alpha^2 A^TA)^{-1}A^TA\uz$ is uniformly bounded in $\alpha \ge 0$.

Stuff the RHS of equation \ref{eqn:eq2} into the definition \ref{eqn:eq1} to obtain
\begin{equation}
  \label{eqn:eq3}
  \tJa = \frac{1}{2}(\|S\uz-d\|_d^2 - 2 \alpha^2\langle S\uz-d, S\va\rangle_d + \alpha^2\|A\uz\|_m^2 + O(\alpha^4))
\end{equation}
The second term on the RHS of equation \ref{eqn:eq3} vanishes thanks to the normal equation, and the first term is precisely $\tJz$. The conclusion \ref{eqn:eq4} follows.


\bibliographystyle{seg}
\bibliography{../../bib/masterref}