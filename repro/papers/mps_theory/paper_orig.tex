
\title{Discretization of Multipole Sources in a Finite Difference Setting for Wave Propagation Problems}
\date{}
\address{
        \footnotemark[1]Institute for Computational and Experimental Research in Mathematics,\\ Brown University,\\ Providence, RI 02912 USA\\
        \footnotemark[2]The Rice Inversion Project,\\ Rice University,\\ Houston, TX
        77005-1892 USA
}
\author{Mario J. Bencomo\footnotemark[1] and William Symes\footnotemark[2]}

\righthead{Multipole Source Discretizations}

\maketitle
\parskip 12pt


%%%%%%%%%
\begin{abstract}
%%%%%%%%%

Seismic sources are commonly idealized as point-sources due
to their small spatial extent relative to seismic wavelengths.
The acoustic isotropic point-radiator is inadequate as a model 
of seismic wave generation for seismic sources that are known to exhibit directivity. 
Therefore, accurate modeling of seismic wavefields must include source 
representations generating anisotropic radiation patterns. 
Such seismic sources can be modeled as linear combinations of 
{\em multipole point-sources}.
In this paper we present a method for discretizing multipole sources 
in a finite difference setting, an extension of the moment matching 
conditions developed for the Dirac delta function in other applications.
We also provide the necessary analysis and numerical evidence 
to demonstrate the accuracy of our singular source approximations.
In particular, we develop a weak convergence theory for the discretization of 
a family of symmetric hyperbolic systems of first-order partial differential equations, with singular source terms, solved via staggered-grid finite difference methods.
Numerical experiments demonstrate a stronger result than 
what is presented in our convergence theory, namely, optimal convergence rates 
of numerical solutions are achieved point-wise in space away from the source if an 
appropriate source discretization is used.

\end{abstract}


%%%%%%%%%%%
\section{Introduction}
%%%%%%%%%%%

Seismic sources are commonly idealized as concentrated at a source point due
to their small spatial extent relative to seismic wavelengths.
A simple and familiar example of a wave propagation model with
spatially concentrated source is the isotropic point-radiator problem
for the acoustic wave equation in Euclidean 3-space \cite[]{CourHil:62}:
\begin{equation}\label{eq:isorad}
\begin{split}
        \frac{\partial^2}{\partial t^2} p(\mathbf x,t) - c^2\nabla^2
  p(\mathbf x,t) & = f(\mathbf x,t) \equiv w(t)\delta(\mathbf x),\\ 
p(\mathbf x,t) & =  0,  \quad t<<0,
\end{split}
\end{equation}
$\delta(\mathbf x)$ being the Dirac delta function.
The solution $p$ is spherically symmetric:
\begin{equation}
\label{eq:green3d}
	p(\mathbf x,t) = \frac{w\left(t-\frac{r}{c}\right)}{4\pi c^2 r}, \quad 
	r= \sqrt{\mathbf x^T \mathbf x}.
\end{equation}
Both active and earthquake seismic sources, however, generate
spatially asymmetric wavefields (see for example \cite{Shearer:2009}, \cite{Yil:01}). 
The acoustic isotropic point-radiator 
is therefore inadequate as a model of seismic wave generation
and propagation because of its prediction of spatial symmetry.
The symmetry of the solution arises in part from the spherical symmetry of
the right-hand side in equation \ref{eq:isorad}. Therefore, accurate
modeling of seismic wavefields must include energy source (right-hand
side) representations generating anisotropic radiation patterns.

A {\em multipole}, or {\em multipole source} in the 
context of source modeling, is a finite linear combination of partial 
derivatives of the spatial Dirac delta function. Such sources combine
localization of energy and anisotropic radiation pattern. In fact,
Peetre's Theorem \cite[]{Horm:69} implies that any function of space
and time $f(\mathbf x,t)$ concentrated entirely at a point in space
(of {\em point support}) is a multipole of finite order $N \ge 0$,
\begin{equation}\label{eq:MPSappx}
        f(\mathbf x,t) = \sum_{|\mathbf s|=0}^N w_{\mathbf s}(t) \; D^{\mathbf s}\delta(\mathbf x-\mathbf x^*),
\end{equation}
in which we have introduced {\em multi-index notation}: for spatial
dimension $d=\{1,2,3\}$ and multi-index (integer d-tuple) ${\bf s} =
(s_1,...,s_d)$, the $\mathbf s$-mixed partial derivative operator, denoted $D^{\bf s}$,
and its (total) order $|{\bf s}|$, are defined as
\begin{equation}\label{eq:PDO}
        D^{\mathbf s} = \prod_{i=1}^d \left( \frac{\partial}{\partial x_i}\right)^{s_i}, \quad |\mathbf s| = \sum_{i=1}^{d} s_i.
\end{equation}
The coefficient time functions $w_{\mathbf s}(t)$ may be scalar-, vector-, or
tensor-valued, according to the nature of the quantity (pressure, velocity, or stress) 
in the equation in which
$f(\mathbf x,t)$ appears as right-hand side.  Multipoles may
approximate arbitrary sources highly localized on the wavelength scale, in
the sense of generating approximately the same field away from the
source location, and for this reason have enjoyed widespread use in
the representation of seismic sources \cite[]{Shearer:2009}.

While finite element methods are also used in earthquake seismic modeling and inversion \cite[]{KomTromp:00,Cohen:01,Ghattas:IP25}, this paper focuses on regular (rectangular) grid finite difference methods, in particular the {\em staggered-grid} variety, which are widely used for basin- and exploration-scale modeling; see \cite{moczoetal:06} for an excellent overview and many older references. 
Such methods pose an immediate problem for singular source models such as multipoles: finite
difference algorithms ``know'' only gridded fields, so a source located at an arbitrary point $\mathbf x^*$ in space must be represented somehow by virtual sources at nearby grid points. 
This task is complicated by the nature of the field, as is evident for instance from inspection of the 3-D analytical function \ref{eq:green3d}: solutions of the acoustic equations with singular right-hand sides are generally themselves singular at the source point, so the Taylor-series based analysis of finite difference accuracy does not apply.
%Imitating finite-element singular source representation, for example by using adjoint interpolation, may reduce the accuracy of the modeled field, in the sense of convergence order: visually, large errors can pollute the field near the source, and propagate throughout the simulation domain. 

The focus of this paper is to study singular source discretizations, particularly in the context of staggered-grid finite difference solvers for a family of wave propagation differential equations (e.g., linear acoustics and elasticity in first-order form) with multipole sources.
Specifically, we are interested in the following class of symmetric hyperbolic systems of first-order partial differential equations: find vector-valued fields $(u,v)$ such that
\begin{equation}\label{eq:hyper_fam}
\begin{split}
	A \frac{\partial}{\partial t}u + P^T v &= f \\
	B \frac{\partial}{\partial t}v  - P u & = g
\end{split}
\end{equation}
for given source terms $(f,g)$.
Operators $A$ and $B$ are bounded symmetric matrix-valued and represent medium parameters.
$P$ is a first-order spatial differential operator with constant coefficients, e.g., $P$ is the gradient in the acoustic equations \ref{eq:acoustics} with $P^T$ being the negative of the divergence.
We directly discretize multipole sources using {\em discrete moment conditions} as developed by \cite{Walden:1999}, and elaborated by \cite{TorEng:04} for multi-dimensional approximations.
Let $(u_h,v_h)$ denote the finite difference solution to \ref{eq:hyper_fam} via staggered-grid finite differences, with discretized source terms $(f_h,g_h)$, where $h>0$ denotes the grid size.
The class of staggered-grid finite difference methods we deal with are second-order in time and $2p$-order in space accurate, and we refer to such a scheme as being of order $2$-$2p$.

Our main contribution is a weak convergence theory, a novel approach to the singular source approximation problem, that is applicable to \ref{eq:hyper_fam} in that $(u_h,v_h)$ converge to $(u,v)$ given that source approximations $(f_h,g_h)$ converge to $(f,g)$ as $h\to 0$, both in a weak sense which we make clear in the theory section.
In particular, we show
\begin{equation}\label{eq:result}
	\langle (u,v),(\tilde f,\tilde g) \rangle - \langle (u_h,v_h), (\tilde f,\tilde g)\rangle = O(h^q + \Delta t^2 + (h^{2p}+\Delta t)h^{-N^*-d/2})
\end{equation}
given smooth test functions $(\tilde f,\tilde g)$. 
Inner products $\langle\cdot,\cdot\rangle$ are interpreted as standard $L^2$ space-time inner products in a continuum or discrete sense depending on their arguments.
Estimate \ref{eq:result} thus depends on singular source approximation order $q$, spatial half-order $p$ for a staggered-grid finite difference scheme of order $2$-$2p$, the maximum multipole order between $f$ and $g$ denoted by $N^*$, and spatial dimension $d$.
Numerical results presented here, consistent with other similar works (e.g., \cite{Petersson:2010}), appear to indicate, however, stronger convergence results: optimal convergence point-wise away from source.
In particular, we report second and fourth order rates when studying the spatial convergence of 2-2 and 2-4 finite difference methods respectively, where the source approximation order matched the spatial order of the numerical scheme.

%Though numerical results, both reported in other works (e.g., \cite{Petersson:2010}) and in this paper, demonstrate stronger results, mainly optimal convergence point-wise away from source, a weak convergence error estimate is an appropriate first step towards a complete theory. 


%There are, of course, many ways to discretize/regularize singularities.
%However, we are primarily concerned with direct discretization of the Dirac delta function and its derivatives through the use of discrete moment conditions, as done by \cite{Walden:1999} in a finite difference and finite element setting for source terms in the 1-D Helmholtz equation.
%His analysis and numerical examples, though limited to his particular problem, demonstrated point-wise convergence of numerical solutions with optimal convergence rates (as suggested by the numerical scheme) away from the source location when appropriately discretizing the singular source term.

\subsection{Literature Review} %ADDED

The analysis and numerical examples presented by \cite{Walden:1999}, though limited to the 1-D Helmholtz equation, demonstrated point-wise convergence of numerical solutions with optimal convergence rates (as suggested by the numerical scheme) away from the source location when appropriately discretizing the singular source term.
The theory of singular source approximations has been further extended to a range of applications, most notably for the Dirac delta function in the context of the {\em immersed boundary method} \citep{Pes:02}. 
Several authors have addressed questions regarding the convergence of source approximations and subsequently their effect on solutions to more complicated differential equations.
Consider the following abstract problem: find $u$ such that
\begin{equation}\label{eq:pde}
	\mathcal L u = f,
\end{equation}
for some differential operator $\mathcal L$ and singular source term $f$.
Define the regularization of problem \ref{eq:pde} by replacing $f$ with some regular (at least piecewise continuous) function $f^\epsilon$ parameterized by regularization parameter $\epsilon>0$, that is, find $u^\epsilon$ such that
\begin{equation}\label{eq:pde_reg}
	\mathcal L u^\epsilon = f^{\epsilon}.
\end{equation}
Regularized source term $f^\epsilon$ is said to approximate $f$ in that $f^\epsilon \to f$ as $\epsilon \to 0$ in some sense.
The end goal is of course to have $u^\epsilon$ approximate true solution $u$, i.e.,
\[
	\lim_{\epsilon\to 0}\|u^\epsilon-u\|_X = 0
\]
under some suitable norm $\|\cdot\|_X$.

\cite{TorEng:03} have studied the regularization error $\|u^\epsilon - u\|_X$, point-wise away from the source location.
Their analysis is based on a simple ODE case where they prove convergence of regularized solutions $u^\epsilon$ if $f^{\epsilon}$ satisfy what we call the {\em continuum moment conditions}. 
We use the qualifier ``continuum'' to differentiate at times between the discrete moment conditions.
Recent work by \cite{hoss:16} addresses the mode of convergence of $f^\epsilon\to f$ subject to regularized source terms satisfying the continuum moment conditions, mainly convergence in a weak-$*$ topology (distribution sense) and in a weighted Sobelev norm.
Both \cite{TorEng:03} and \cite{hoss:16} argue that $f^{\epsilon}\to f$ implies $u^\epsilon\to u$ as $\epsilon \to 0$, point-wise away from the source location, in particular for elliptic operators $\mathcal L$. 
This argument hinges on the integral representation of elliptic operators and the smoothness of their kernel (i.e., Green's functions) away from source location.

Suppose that the regularized problem \ref{eq:pde_reg} is discretized with mesh or cell size $h>0$;
\begin{equation}\label{eq:pde_disc}
	\mathcal L_h u^{\epsilon}_h = f^{\epsilon}_h.
\end{equation}
In the context of finite difference methods $\mathcal L_h$ is the finite difference approximation of differential operator $\mathcal L$.
The discrete source term $f^\epsilon_h$ can be interpreted as the discretization 
(e.g., sampling over grid points) of the regularized source term $f^\epsilon$  generated by the continuum moment conditions, or as the direct discretization of $f$ through the discrete moment conditions.
In practice the regularization parameter $\epsilon$ is related to the discretization parameter $h$, that is, $\epsilon = \epsilon(h)$ such that
\[
	\lim_{h\to 0} \epsilon(h) = 0.
\]
\cite{TorEng:04} provided insight into the convergence of $f^\epsilon_h$ as $h\to 0$ for direct discretizations of $f$ via the discrete moment conditions, in particular for discretizations whose support is proportional to $\epsilon$ and $\epsilon(h) = O(h)$.
Consistent with results by \cite{Walden:1999}, \cite{TorEng:04} demonstrated the convergence of numerical solutions $u^\epsilon_h$ for problem \ref{eq:pde_disc}, in particular
\[
	\| u^\epsilon_h - u\|_X = O(h^p)
\]
where $p$ is the convergence rate of the numerical scheme and $f^{\epsilon}_h$ satisfies a sufficient number of moment conditions. 
The norm $\|\cdot\|_X$ in this case coincides with the sup-norm with a deleted neighborhood containing the source location.
Theory presented by \cite{TorEng:04} is based on analysis of the 1-D Poisson equation discretized by second- and fourth-order finite difference approximations.%, in particular, studying the numerical Green's functions.

Green's functions for hyperbolic problems are singular at the wavefront they propagate as well as at the source location, thus the analysis of \cite{TorEng:04} does not apply.
Recent work by \cite{Petersson:2016} presented some of the most relevant analysis based on centered difference approximations to the 1-D advection equation with singular source term.
They show that the discrete moment conditions are necessary but not sufficient for the convergence of numerical solutions at optimal rates away from the source location.
They demonstrated that {\emph smoothness conditions} are also required to achieve convergence; these conditions are based on Fourier analysis and the ability of the finite difference operators at play to resolve spurious modes injected by the singular source approximation. 
%The main result of \cite{Petersson:2016} is however suboptimal, in that
%\[
%	\|u^\epsilon_h - u\|_2 \le Ch^{p-\frac{1}{2m}}
%\]
%where $p$ is the order of the finite difference scheme and the number of moment and smoothness conditions satisfied by the source discretization, and integer $m$ is related to the regularity of the time-dependent component of the singular source term.

%%%
%Our main contribution is a weak convergence theory of $u^\epsilon_h\to u$ that is, unlike current convergence analysis that is limited by its particular application, applicable to a large set of symmetric hyperbolic problems solved via staggered grid finite difference methods.
%%Though numerical results, both reported in other works and in this paper, demonstrate stronger results, mainly point-wise convergence away from source, we believe a weak convergence error estimate is an appropriate first step towards a complete theory.
The remainder of the paper is organized as follows:
In the theory section we begin by presenting an overview of the singular source approximation method via continuum and discrete moment conditions.
%perhaps include regularity error estimates for 3-D acoustics?
We focus primarily on source approximations of narrow support, as discussed in \cite{TorEng:04}, though we provide explicit formulas for approximations of arbitrary order.
Moreover, we show that the discrete moment conditions in fact define a sequence of continuum functions that converge to target distributions in a weak sense, a new result.
The theory section also covers known convergence results of staggered-grid finite difference methods via energy estimates, applied to our set of differential equations under smooth coefficients and smooth source terms.
At this point we present our weak convergence theory.
The last section covers numerical results, mainly for staggered-grid finite difference solutions to the 2-D acoustic equations in first-order form with multipole sources.
Consistent with numerical results presented in the literature (see for example \cite{Petersson:2010}), we observe optimal convergence rates of our numerical solutions when the source discretization satisfied the proper number of moment conditions.


\newpage

%%%%%%%%%%%%%%%%%%%%%%%%%
\section{Theory}
%%%%%%%%%%%%%%%%%%%%%%%%%

%%%%
\subsection{Singular Source Approximation}
%%%%



%%%%%%%
\subsubsection{(Continuum) Moment Conditions}
%%%%%%%

%
We begin by noting that the delta function and its derivatives (and thus multipoles) are not actually functions but rather so-called distributions, functionals that return a real number when applied to a test function. 
Let $\mathcal D$ denote the {\em space of test functions} over $\mathbb R^d$, that is, the space of $C^\infty_0(\mathbb R^d)$ endowed with the standard topology of test functions.
The set of \emph{distributions} is given by the dual of the space of test functions, denoted by $\mathcal D'$.
It is conventional to represent the application of a distribution on a function by the integral of the product, even when the distribution is not actually a function that can be integrated in the usual sense.
For example, given multi-index $\mathbf s=(s_1,...,s_d)$, the $\mathbf s$-mixed partial derivative of the Dirac delta function, shifted by $\mathbf x^*\in\mathbb R^d$, is defined by
\[
	\int_{\mathbb R^d} D^{\mathbf s} \delta(\mathbf x-\mathbf x^*) \; \psi(\mathbf x)\; d\mathbf x =
	(-1)^{|\mathbf s|} D^{\mathbf s} \psi(\mathbf x^*), \quad \forall \psi \in\mathcal D.
\]
%The distribution $D^{\mathbf s}\delta(\cdot;\mathbf 0)$ is simply denoted by $D^{\mathbf s}\delta$.
%In the remainder of the paper we focus on deriving the approximation theory for $\mathbf x^*=\mathbf 0$ (assuming $\mathbf 0\in\Omega$) since the general case follows by simply shifting the approximation accordingly.

%
The key idea for constructing approximations to $D^{\bf s}\delta(\mathbf x-\mathbf x^*)$ is based on mimicking the behavior of the target distribution on polynomials, reminiscent of finite difference approximations for differential operators.
Consider $\psi(\mathbf x) = (\mathbf x - \mathbf x^*)^{\alpha}$, with multi-index $\alpha=(\alpha_1,...,\alpha_d)$, where multi-indexed monomials are interpreted as the product of monomials in each dimension,
\[
       {\bf x}^{\alpha} = \prod_{k=1}^d x_k^{\alpha_k}.
\] 
It can be shown that
\[
	\int_{\mathbb R^d} D^{\mathbf s}\delta(\mathbf x-\mathbf x^*)\; \psi(\mathbf x) \; d\mathbf x = \mathbf s! (-1)^{|\mathbf s|} \delta_{\mathbf s \alpha}
\]
where $\delta_{\mathbf s \alpha}$ is the Kronecker delta, defined as follows for multi-indexes,
\[
	\delta_{\mathbf s \alpha} := \prod_{k=1}^d \delta_{s_k \alpha_k}.
\]

%
Given $\eta\in L^1_0(\mathbb R^d)$ (i.e., integrable function of compact support) and multi-index $\alpha$, the {\em $\alpha$-moment} of $\eta$ centered at $\mathbf x^*\in\mathbb R^d$, denoted by $M^\alpha(\cdot,\mathbf x^*)$, is defined as
%definition of moment
\begin{equation}\label{eq:def_mom}
	M^\alpha(\eta,\mathbf x^*) := \int_{\mathbb R^d} \eta(\mathbf x) \; (\mathbf x-\mathbf x^*)^{\alpha} \; d\mathbf x.
\end{equation}
%Note that integration in equation \ref{eq:def_mom} is translation invariant hence $M_{\alpha}(\eta,\mathbf x^*)$ is constant with respect to $\mathbf x^*$, which I denote by $M_{\alpha}(\eta)$.
For given nonnegative integer $q$ and multi-index $\mathbf s$, the function $\eta$ is said to satisfy the {\em continuum $(q,\mathbf s)$-moment conditions} at $\mathbf x^*\in \mathbb R^d$ if
%definition of moment conditions
\begin{equation}\label{eq:momcond}
	M^\alpha(\eta,\mathbf x^*) = \mathbf s! (-1)^{|\mathbf s|} \delta_{\mathbf s \alpha}, \quad \forall |\alpha|=0,...,q+|\mathbf s|-1.
\end{equation}

If $\eta$ satisfies the $(q,\mathbf s)$-moment conditions at $\mathbf x^*$, then its associated distribution, that is
\[
	\int_{\mathbb R^d} \eta(\mathbf x) \; \psi(\mathbf x)\; d\mathbf x, \quad \forall \psi\in\mathcal D,
\]	
is an approximation to $D^{\mathbf s}\delta(\mathbf x-\mathbf x^*)$ in that it is exact on polynomials of order $q+|\mathbf s|-1$.
The following theorem states that a sequence of (regular) distributions of compact support, satisfying the $(q,\mathbf s)$-moment conditions, will converge in the weak-$*$ topology at a rate $q$ to the target distribution as the width of the supports approach zero. 
Let $B(\mathbf x^*,\epsilon)$ denote the $d$-dimensional ball of radius $\epsilon$ centered at $\mathbf x^*$.

%theorem weak convergence of approximations to distribution
\begin{theorem}\label{thm:weakconv}
	Let nonnegative integer $q$, multi-index $\mathbf s$, and $\mathbf x^*\in\mathbb R^d$ be given.
	Suppose $\{\eta^\epsilon\}\subset L^1_{0}(\mathbb R^d)$ is a sequence of functions as $\epsilon\to 0$, where $supp(\eta^\epsilon)\subset B(\mathbf x^*,\epsilon)$.
	Furthermore, suppose that there exists a constant $K>0$ independent of $\epsilon$ such that 
	\begin{equation}\label{eq:boundH}
		\int_{\mathbb R^d} |\eta^\epsilon(\mathbf x)|\; |(\mathbf x-\mathbf x^*)^{\alpha}| \; d\mathbf x \le K, \quad \forall |\alpha| = |\mathbf s|.
	\end{equation}
	If $\{\eta^\epsilon\}$ satisfy the $(q,\mathbf s)$-moment conditions at $\mathbf x^*$, equation \ref{eq:momcond}, then the sequence of distribution they generate converges to $D^{\mathbf s}\delta(\mathbf x-\mathbf x^*)$ in the weak-$*$ topology as $\epsilon\to 0$.
	In particular, if $\psi$ is of class $C^{q+|\mathbf s|}$ over $B(\mathbf x^*,\epsilon)$, then
	\begin{equation}\label{eq:delta_error}
		E:= \left| \int_{\mathbb R^d} D^{\mathbf s}\delta(\mathbf x-\mathbf x^*)\;\psi(\mathbf x) \; d\mathbf x\; - 
		\int_{\mathbb R^d} \eta^\epsilon(\mathbf x) \; \psi(\mathbf x)\; d\mathbf x \right| = O(\epsilon^q).
	\end{equation}
\end{theorem}

%PROOF
\begin{proof}
We first apply multi-variate Taylor's theorem to $\psi$, centered at $\mathbf x^*$ and truncated after $N=q+|\mathbf s|-1$ terms \cite[]{konigsberger2013}, assuming $\psi$ is $C^{q+|\mathbf s|}$ over $B(\mathbf x^*,\epsilon)$,
\begin{align*} 
	\int_{\mathbb R^d} \eta^\epsilon (\mathbf x) \; \psi(\mathbf x) \; d\mathbf x
	&= \int_{\mathbb R^d} \eta^\epsilon(\mathbf x) \left( \sum_{|\alpha|=0}^{N} \frac{D^{\alpha} \psi(\mathbf x^*)}{\alpha!} \; (\mathbf x - \mathbf x^*)^\alpha 
	+ \sum_{|\beta|=N+1}R_{\beta}(\mathbf x) \; (\mathbf x - \mathbf x^*)^\beta \right)
	d\mathbf x \\
	&= \sum_{|\alpha|=0}^N \frac{D^{\alpha}\psi(\mathbf x^*)}{\alpha!}  \left( \int_{\mathbb R^d}  \eta^\epsilon(\mathbf x) \; (\mathbf x -\mathbf x^*)^\alpha \;d\mathbf x\right) 
	+  \sum_{|\beta|=N+1} \int_{\mathbb R^d} \eta^{\epsilon}(\mathbf x) R_{\beta}(\mathbf x) \; (\mathbf x-\mathbf x^*)^\beta \; d\mathbf x,
\end{align*}
where $R_{\beta}$ is the remainder term,
\[
	R_\beta(\mathbf x) = \frac{|\beta|}{\beta!} \int_{0}^1 (1-t)^{|\beta|-1} D^\beta \psi(\mathbf x^* + t(\mathbf x-\mathbf x^*)) \; dt.
\]
Note that the term in the parenthesis in the bottom equation corresponds to the $\alpha$-moment centered at $\mathbf x^*$ with $|\alpha|\le q+|\mathbf s|-1$, hence the ($q,\mathbf s$)-moment conditions apply;
\begin{align*}
	\int_{\mathbb R^d} \eta^\epsilon(\mathbf x) \; \psi(\mathbf x)\;d\mathbf x
	&= \sum_{|\alpha|=0}^{N} \frac{1}{\alpha!} D^{\alpha}\psi(\mathbf x^*) \Big( \mathbf s! (-1)^{|\mathbf s|}\delta_{\mathbf s \alpha} \Big) 
	+  \sum_{|\beta|=N+1} \int_{\mathbb R^d} \eta^\epsilon(\mathbf x) R_{\beta}(\mathbf x)\; (\mathbf x-\mathbf x^*)^\beta \; d\mathbf x\\
	&= (-1)^{|\mathbf s|}D^{\mathbf s} \psi(\mathbf x^*) 
	+  \sum_{|\beta|=N+1} \int_{\mathbb R^d} \eta^\epsilon (\mathbf x) R_{\beta}(\mathbf x) \; (\mathbf x-\mathbf x^*)^\beta \; d\mathbf x.
\end{align*}

The remainder term is bounded uniformly over $B(\mathbf x^*,\epsilon)$,
\[
	\underset{\mathbf x\in B(\mathbf x^*,\epsilon)}{\sup} |R_\beta(\mathbf x)| \le C(\beta,\psi) := \frac{1}{\beta!} \max_{|\alpha|=|\beta|} \; \max_{\mathbf y\in B(\mathbf x^*,\epsilon)} |D^{\alpha}\psi(\mathbf y)|,
\]
using the fact that $\psi\in C^{N+1}$ in $B(\mathbf x^*,\epsilon)$.
This gives the following error estimate,
\begin{align*}
	 E
	 & \le \sum_{|\beta|=N+1} C(\beta,\psi) \int_{B(\mathbf x^*,\epsilon)} 
	  |\eta^\epsilon(\mathbf x)| \; |(\mathbf x-\mathbf x^*)^\beta| \; d\mathbf x.
\end{align*}
Let $\gamma$ be a multi-index such that $|\gamma|=q$, thus $|\beta-\gamma| = |\mathbf s|$.
This yields,
\begin{align*}
	E
	&\le \sum_{|\beta|=N+1} C(\beta,\psi) \left( \sup_{\mathbf x\in B(\mathbf x^*,\epsilon)} \left|(\mathbf x-\mathbf x^*)^{\gamma}\right|\right) 
	\int_{B(\mathbf x^*,\epsilon)} |\eta^\epsilon(\mathbf x)| \; |(\mathbf x-\mathbf x^*)^{\beta-\gamma}| \; d\mathbf x \\
	&\le \sum_{|\beta|=N+1} C(\beta,\psi) \left( \sup_{\mathbf x\in B(\mathbf x^*,\epsilon)} \left|(\mathbf x-\mathbf x^*)^{\gamma}\right|\right)  K\\
	&= O(\epsilon^q)
\end{align*}
\end{proof}
%END PROOF

Theorem \ref{thm:weakconv} and moment conditions given in equation \ref{eq:momcond} are extensions of what is presented in \cite{hoss:16} for $|{\bf s}|\neq0$.
Given equation \ref{eq:delta_error}, we refer to $q$ as the \emph{singular source approximation order} and $\eta^\epsilon$ as being a $q$-order approximation of $D^{\mathbf s}\delta(\mathbf x-\mathbf x^*)$.


%%%%%%%
\subsubsection{Discrete Moment Conditions}
%%%%%%%

We define the regular grid centered at $\mathbf x_0\in\mathbb R^d$ with cell size $\mathbf h = (h_1,...,h_d)>\mathbf 0$, as the collection of points denoted by $\mathcal G(\mathbf x_0,\mathbf h)$,
\[
	\mathcal G(\mathbf x_0,\mathbf h) = \{ \mathbf x_n = (x_{1,n_1},...,x_{d,n_d}), \; \mathbf n \in\mathbb Z^d\},
\]
where
\[
	x_{k,n_k} = x_{0,k} + h_kn_k, \quad \forall k=1,...,d.
\]
%The dependence of real-valued grid functions with respect to a given regular grid $\mathcal G(\mathbf x_0,\mathbf h)$, say $\eta^h: \mathcal G(\mathbf x_0,\mathbf h)\to \mathbb R$, is made implicit through the use of multi-indexing notation.
%For example, given $\mathbf x_{\mathbf n}\in \mathcal G(\mathbf x_0,\mathbf h)$ for some multi-index $\mathbf n=(n_1,...,n_d)$,
%\[
%	\eta^h_{\mathbf n}= \eta^h(\mathbf x_{\mathbf n}).
%\]
Note that there is no reason to assume that the grid cell is cubical, it may have different lengths along different axes.
However, we assume that the grid is refined by scaling a characteristic grid cell size $h$ (e.g., $h=\max_{k} h_k$) and which we will use to denote grid functions and other grid-dependent quantities with subscript $h$ even in dimensions higher than one.

The obvious definition of the \emph{discrete $\alpha$-moment}, centered at $\mathbf x^*\in\mathbb R^d$ (note that $\mathbf x^*$ need not coincide with a grid point in $\mathcal G(\mathbf x_0,\mathbf h)$), of a grid function $\eta_h:\mathcal G(\mathbf x_0,\mathbf h)\to \mathbb R$ is given as follows: 
\begin{equation*}\label{eq:def_discmom}
        M_h^\alpha(\eta_h,\mathbf x^*) := \left(\prod_{k=1}^{d} h_k \right) \sum_{\mathbf x\in\mathcal G(\mathbf x_0,\mathbf h)}  \eta_h(\mathbf x)\; ({\bf x}-{\bf x}^*)^{\alpha}.
\end{equation*}
It is worth pointing out that the discrete moment defined above is dependent on choice of grid, in particular dependent on the source location $\mathbf x^*$ relative to the grid.
Similar to the continuum moment conditions (equation \ref{eq:momcond}), grid function $\eta_h$ is said to satisfy the \emph{discrete $(q,\mathbf s)$-moment conditions} at $\mathbf x^*\in\mathbb R^d$ if
\begin{equation}\label{eq:discmomcond}
	M_h^\alpha(\eta_h,\mathbf x^*) = \mathbf s! (-1)^{|\mathbf s|} \delta_{\mathbf s \alpha}, \quad \forall |\alpha|=0,...,q+|\mathbf s|-1.
\end{equation}
The following theorem is a discrete analogue of theorem \ref{thm:weakconv}.
%the discrete $(q,\mathbf s)$-moment conditions imply convergence of gridded functions $\{S^h\}$ as the characteristic grid size $h$ is refined, assuming the support of $S^h$ is proportional to $h$, in a ``discrete'' weak-$*$ topology.
%theorem discrete weak convergence for approximations to distribution

\begin{theorem}\label{thm:discweakconv}
Let nonnegative integer $q$, multi-index $\mathbf s$, and $\mathbf x^*\in\mathbb R^d$ be given.
Suppose $\{\eta_h^\epsilon\}$ is a sequence of grid functions $\eta_h^\epsilon:\mathcal G(\mathbf x_0,\mathbf h)\to\mathbb R$ as $\epsilon\to 0$.
Furthermore, assume that the support of $\eta_h^\epsilon$ is contained in $B(\mathbf x^*,\epsilon)$ with $\epsilon=O(h)$, and that there exists constant $K>0$ independent of $\epsilon$ such that
\begin{equation*}\label{eq:boundh}
	\left(\prod_{k=1}^d h_k \right) \sum_{\mathbf x\in\mathcal G(\mathbf x_0,\mathbf h)} |\eta_h^\epsilon(\mathbf x)|\; |(\mathbf x-\mathbf x^*)^{\alpha}| \le K, \quad \forall |\alpha| = |\mathbf s|.
\end{equation*}
If $\{\eta_h^\epsilon\}$ satisfy the discrete $(q,\mathbf s)$-moment conditions at $\mathbf x^*$ (equation \ref{eq:discmomcond}) and $\psi$ is of class $C^{q+|\mathbf s|}$ over $B(\mathbf x^*,\epsilon)$, then 
\[
	\left| \int_{\mathbb R^d} D^{\mathbf s}\delta(\mathbf x- \mathbf x^*) \;\psi(\mathbf x)\; d\mathbf x - \left(\prod_{k=1}^d h_k\right) \sum_{\mathbf x\in\mathcal G(\mathbf x_0,\mathbf h)}\eta_h^\epsilon (\mathbf x) \; \psi(\mathbf x)\right| = O(h^q).
\]
\end{theorem}

%%
\begin{proof}
The proof of this theorem is omitted since it is nearly identical to that of the continuum case (theorem \ref{thm:weakconv}), replacing integrals with summations over grid points.
The jump from $O(\epsilon^q)$ to $O(h^q)$ follows from $\epsilon=O(h)$.
\end{proof}

Theorem \ref{thm:discweakconv} and discrete moment conditions \ref{eq:discmomcond} are generalizations of work by \cite{TorEng:04} for $|\mathbf s|\neq 0$. 
The discrete moment conditions are also an extension of \cite{Walden:1999} for dimensions higher than one.
We now discuss with more detail how to construct said sequences of gridded and continuum functions starting in 1-D.

%%%%
\subsubsection{1-D Constructions}
%%%%%

We make the choice of having 1-D gridded approximations be centered at source location $x^*\in\mathbb R^d$, and define them to be zero outside the interval $[-\epsilon+x^*,\epsilon+x^*)$, with $2\epsilon=Nh$ for some positive integer $N$.
In other words, there exists $N$ grid points, denoted by $\{\tilde x_\ell\}_{\ell=1}^{N}$, such that they are contained in the interval $[-\epsilon+x^*,\epsilon+x^*)$ for a given grid $\mathcal G(x_0,h)$.
These grid points $\{\tilde x_\ell\}_{\ell=1}^N$ are referred to as the \emph{stencil points} of the approximation.
We assume that the stencil points are ordered, i.e., $\tilde x_1<\tilde x_2<\cdots<\tilde x_{N}$.
The discrete $(q,s)$-moment conditions thus results in a $N\times(q+s)$ system of equations for the grid function $\eta_h^\epsilon$ evaluated at stencil points,
\[
	\mathbf A \mathbf d = \mathbf b
\]
with 
\[
	\{\mathbf A\}_{k\ell} = (\tilde x_\ell-x^*)^{k-1}, \quad \{\mathbf d\}_{\ell}=\eta_h^\epsilon(\tilde x_\ell), \quad \{\mathbf b\}_{k}= \frac{s! (-1)^s}{h} \delta_{s,k-1},
\]
for $\ell=1,...,N$ and $k=1,...,q+s$.
Note that $\mathbf A$ is a Vandermonde matrix of full rank and is guaranteed a solution if $N\ge q+s$ and no solution for $N<q+s$ under general $x^*\in\mathbb R$.
%It will be of benefit to pick grid functions of minimal support, that is $N=q+s$; this will be more apparent when coupling source approximations with finite difference schemes.

We choose the case where $N=q+s$, which we refer to as grid functions of narrow support.
The system above will result in a unique solution for a given $x^*\in\mathbb R$.
In fact the inverse matrix for $\mathbf A$ can be written explicitly using the following Vandermonde matrix inverse formula:
%VANDERMONDE INVERSE
\begin{equation*}\label{eq:vandinv}
	\{\mathbf A^{-1} \}_{\ell k} = \left\{ \begin{array}{cl}
		\displaystyle
		(-1)^{N-k} \; 
		\frac{
			\displaystyle
			\sum_{\underset{m_1,...,m_{N-k}\neq \ell}{1\le m_1<\cdots<m_{N-k}\le N}}
			 (\tilde x_{m_1}-x^*) \cdots (\tilde x_{m_{N-k}}-x^*)
		}
		{
			\displaystyle
			\prod_{\underset{m\neq \ell}{1\le m\le N}} (\tilde x_\ell - \tilde x_m)
		}
		,& \text{for} \; 1\le k\le N  \vspace{10pt} \\ 
		\displaystyle
		\frac{
			1
		}
		{
			\displaystyle
			\prod_{\underset{m\neq \ell}{1\le m\le N}} (\tilde x_\ell - \tilde x_m)
		}
		,& \text{for}\; k=N.
	\end{array}\right.
\end{equation*}
Given the particular form of right-hand side vector $\mathbf b$, it follows that $\mathbf d$ is simply the scaled $(s+1)$-column of $\mathbf A^{-1}$, whence
%GRID SOLUTION
\begin{equation}\label{eq:gridfunsol}
	\eta_h^\epsilon(\tilde x_\ell) = \left\{ \begin{array}{cl}
		\displaystyle
		s! (-1)^{N-1} \; 
		\frac{
			\displaystyle
			\sum_{\underset{m_1,...,m_{q-1}\neq \ell}{1\le m_1<\cdots<m_{q-1}\le N}}
			 (\tilde x_{m_1}-x^*) \cdots (\tilde x_{m_{q-1}}-x^*)
		}
		{
			\displaystyle
			h^{N} \prod_{\underset{m\neq \ell}{1\le m\le N}} (\ell - m)
		}
		,& \text{for} \; q>1  \vspace{10pt} \\ 
		\displaystyle
		\frac{
			s!(-1)^{s}
		}
		{
			\displaystyle
			h^N \prod_{\underset{m\neq \ell}{1\le m\le N}} (\ell - m)
		}
		,& \text{for}\; q=1.
	\end{array}\right.
\end{equation}

Note that the equation above above is dependent on $x^*$.
To be more precise, approximation $\eta_h^\epsilon$ is actually dependent on the relative position of $x^*$ with respect to the stencil points $\{\tilde x_\ell\}_{\ell=1}^N$, or equivalently dependent on $x_0$ and $h$.
For example, shifting the source location by $h$ would result in the same grid function $\eta_h^\epsilon$ though shifted by a grid point.
However, arbitrary shifts in $x^*$ yield approximations that may vary by more than a simple translation.
%Intuitively enough, the same cannot be said of the continuum case. 
%In other words, if $\eta(x)$ is a $q$-order approximation of $D^{s}\delta(x)$ then it follows that $\eta(x-x^*)$ is also a $q$-order approximation of $D^{s}\delta(x-x^*)$, for all $x^*\in\mathbb R$.
Given that equation \ref{eq:gridfunsol} is unique for a particular source location, we show that imposing the discrete moment conditions over all $x^*\in\mathbb R$ indeed defines a function over the reals.
Moreover, we show that these continuum functions satisfy the continuum moment conditions and thus define a sequence of distributions that converge to $D^{s}\delta$ in the weak-$*$ topology, a result previously not known.

%Equation \ref{eq:gridfunsol} is used to generate gridded functions $\eta^{h_k}$ satisfying the discrete $(q,s_k)$-moment conditions at $x^*_k\in\mathbb R$ on each dimension $k=1,...,d$.
%Approximations on higher dimension are then constructed by taking the tensor product of the 1-D approximations.
%Note, however, that in order for theorem \ref{thm:discweakconv} to apply it is required that the multi-variate grid function, that is $\eta^h(\mathbf x) = \eta^{h_1}(x_1) \cdots \eta^{h_d}(x_d)$, satisfy the discrete $(q,\mathbf s)$-moment conditions.
%Though it may not appear obvious, the proposition below shows that my tensor construction indeed satisfies the appropriate discrete moment conditions.
%
%%theorem proving tensor prod appx satisfy discrete moment conditions
%\begin{theorem}\label{thm:prodmomcond}
%	Let $q\in\mathbb N$, multi-index $\mathbf s$, and $\mathbf x^*\in\mathbb R^d$ be given. 
%Suppose $\eta^h:\mathcal G(\mathbf x_0,\mathbf h)\to \mathbb R$ is a multi-variate grid function given by the tensor product of 1-D grid functions $\eta^{h_k}:\mathcal G(x_{0,k},h_k)\to\mathbb R$.
%If $\eta^{h_k}$ satisfy the discrete $(q,s_k)$-moment conditions at $x^*_k$ for all $k=1,...,d$, then it follows that $\eta^h$ satisfies the discrete $(q,\mathbf s)$-moment conditions at $\mathbf x^*$.
%\end{theorem}
%
%%%
%\begin{proof}
%Suppose for each $k=1,...,d$ that $\eta^{h_k}$ satisfies the discrete $(q,s_k)$-moment conditions at $x^*_k$. 
%Let $\alpha$ be some multi-index with $0\le |\alpha|\le q+|\mathbf s|-1$. 
%Note that,
%\[
%	M^{h}_{\alpha}(S^{h},\mathbf x^*) = \prod_{k=1}^d M^{h_k}_{\alpha_k}(\eta^{h_k},x^*_k).
%\]
%Clearly, if $\alpha_k\le q+s_k-1$ for all $k=1,...,d$, then the result follows from the supposition. 
%Same applies for $\alpha=\mathbf s$.
%Suppose then, that there exists index $\ell$ such that $\alpha_\ell>q+s_\ell-1$, that is $a_\ell=q+s_\ell -1 +i$ for some $i\in\mathbb N$. Thus,
%\begin{align*}
%	& |\alpha| = \sum_{k\neq \ell} \alpha_k + \alpha_\ell = \sum_{k\neq \ell} \alpha_k + q+s_\ell-1+i \le q + |\mathbf s| -1\\
%\Longrightarrow	&  \sum_{k\neq\ell} \alpha_k + i \le \sum_{k\neq \ell} s_k.
%\end{align*}
%which implies that $\alpha_k<s_k$ for at least one $k\neq\ell$;
%for this particular $k$, it follows that 
%\[
%	M^{h_k}_{\alpha_k}(\eta^{h_k},x^*_k) = s_k!(-1)^{s_k} \delta_{s_k \alpha_k} = 0
%\]
%since it has been established that $\alpha_k\neq s_k$, i.e., the product over $k$ is zero if $\mathbf s\neq \alpha$.
%\end{proof}
%
%It is important to note that grid functions $\eta^h$ as computed by equation \ref{eq:gridfunsol}, will be dependent on three things:
%\begin{description}
%	\item{1.} source location $\mathbf x^*$,
%	\item{2.} choice of grid $\mathcal G(\mathbf x_0,\mathbf h)$,
%	\item{3.} and choice of support of $\eta^h$.
%\end{description}
%The last point refers to the fact that the center of the support, that is $a^*$, was never specified above; different choices of $a^*$ will result in different grid functions.
%In practice, however, I will consider only grid functions that are centered at source location $x^*$, as shown in section \ref{sc:examples}.

%%%%%%%%
%\subsection{Relating Discrete and Continuous Moments Conditions}
%%%%%%%%

%The discrete $(q,\mathbf s)$-moment conditions, for a given $\mathbf x^*$, yield gridded approximations from an explicit formula (equation \ref{eq:gridfunsol}) assuming that the diameter of the support of the grid function is equal to $(q+s_k)h_k$ in each dimension.

%%%%%%
\subsubsection{Connection between Continuum and Discrete Moment Conditions}

We first focus on the $x^*=0$ case and define our continuum approximation $\eta^\epsilon$ to be zero outside $[-\epsilon,\epsilon)$, with $2\epsilon=Nh$ for $N=q+s$. 
Furthermore, we define $\eta^\epsilon$ to be piecewise polynomial over $N$ \hl{intervals}: %MOD
%ETA
 \begin{equation}\label{eq:eta}
 	\eta^\epsilon(x) = \left\{ \begin{array}{rl}
		P_\ell(x),& \quad x\in[a_\ell,a_{\ell+1}), \; \text{for}\; \ell=1,...,N\\
		0,& \quad \text{otherwise}
	\end{array}\right.
 \end{equation}
where $P_\ell$ is some polynomial over the considered interval, and $a_{\ell}=-\epsilon+ (\ell-1)h$ for $\ell=1,...,N+1$.
Given the support of our approximation, and a regular grid $\mathcal G(x_0,h)$, it follows that there are $N$ grid points contained in the interval $[-\epsilon,\epsilon)$, again denoted by $\{\tilde x_\ell\}_{\ell=1}^N$.
In fact,
\[
	\tilde x_\ell \in [a_\ell,a_{\ell+1}), \quad \forall \ell=1,...,N.
\]
Let $\ell^*$ be the index such that $0\in[a_{\ell^*},a_{\ell^*+1})$ and define $\zeta\in(0,h]$ by $\zeta=a_{\ell^*+1}-\tilde x_{\ell^*}$.
Thus, if we vary $x^*$ within the interval $(-\zeta,h-\zeta]$ if follows that
\[
	\tilde x_\ell - x^* \in [a_\ell,a_{\ell+1}), \quad \ell=1,...,N.
\]


Let $\eta_h^\epsilon(\cdot;x^*)$ denote the grid function that satisfies the discrete $(q,s)$-moment conditions for a given $x^*\in(-\zeta,h-\zeta]$.
Then $\eta^\epsilon(\tilde x_\ell - x^*) := \eta_h^\epsilon(\tilde x_\ell; x^*)$  defines $\eta^\epsilon$ over $[a_\ell,a_{\ell+1})$ by allowing $x^*$ to vary over the prescribed interval. 
Moreover, slightly modifying equation \ref{eq:gridfunsol} as a function of $x=\tilde x_\ell - x^*$ defines the $P_\ell(x)$ polynomials of $\eta^\epsilon$,
%POLY P_L
\begin{equation}\label{eq:poly}
	 P_\ell(x) = \left\{ \begin{array}{cl}
		\displaystyle
		s! (-1)^{N-1} \; 
		\frac{
			\displaystyle
			\sum_{\underset{m_1,...,m_{q-1}\neq \ell}{1\le m_1<\cdots<m_{q-1}\le N}}
			 ( h(m_1- \ell) + x) \cdots ( h(m_{q-1} - \ell) + x)
		}
		{
			\displaystyle
			h^{N}\prod_{\underset{m\neq \ell}{1\le m\le N}} ( \ell - m)
		}
		,& \text{for} \; q>1  \vspace{10pt} \\ 
		\displaystyle
		\frac{
			s! (-1)^{s}
		}
		{
			\displaystyle
			h^{N} \prod_{\underset{m\neq \ell}{1\le m\le N}} (\ell - m)
		}
		,& \text{for}\; q=1.
	\end{array}\right.
\end{equation}
Inspection of equation \ref{eq:poly} reveals that $P_\ell$ is a polynomial of degree $q-1$.

%By construction of our approximations, it follows that $\eta_H(x-x^*)$ satisfies the discrete $(q,s)$-moment conditions at $x^*$ in a manner that is independent of how the grid $\Omega^h$ is centered, or grid origin in other words.

%%%
\begin{theorem}\label{thm:DCmomcond}
	Let nonnegative integer $q$, positive integer $s$, and $x^*\in \mathbb R$ be given. 
	Suppose $\eta^\epsilon$ is constructed according to equations \ref{eq:eta} and \ref{eq:poly} for a given $h>0$.
	Then it follows that $\eta^\epsilon(x)$ is a $q$-order approximation of $D^s\delta(x)$. 
\end{theorem}

%%%
\begin{proof}
	In order to apply theorem \ref{thm:weakconv} we need to verify that $\eta^\epsilon$ indeed satisfies the (continuum) $(q,s)$-moment conditions at $0$ as well as the estimate given by equation \ref{eq:boundH}.
	We first evaluate the $\alpha$-moment of $\eta^\epsilon$ for $\alpha=0,...,q+s-1$;
	\begin{subequations}
	\begin{align*}
		M^\alpha(\eta^\epsilon,0) &= \int_{\mathbb R} \eta^\epsilon(x) \;x^\alpha\; dx \\
			&= \sum_{\ell=1}^N \int_{a_\ell}^{a_{\ell+1}} P_\ell(x) \;x^\alpha \; dx.
	\end{align*}
	\end{subequations}
	Applying the following change of variables, $x= a_\ell + \xi$ with $\xi\in[0,h)$, over each interval yields
	\begin{subequations}
	\begin{align*}
		M^\alpha(\eta^\epsilon,0) &= \sum_{\ell=1}^N \int_{0}^h P_\ell(a_\ell+\xi) (a_\ell+\xi)^\alpha \; d\xi \\
			&= \int_{0}^h \frac{1}{h} \left[ h \sum_{\ell=1}^N P_\ell(a_\ell+\xi) (a_\ell+\xi)^\alpha \right]\; d\xi.
	\end{align*}
	\end{subequations}
	Note that the term in the bracket coincides with the discrete $\alpha$-moment of $\eta^\epsilon$ with respect to a uniform grid $\mathcal G(a_1,h)$ (containing stencil points $\{a_\ell\}_{\ell=1}^{N+1}$) for a source located at $x^*=-\xi$.
	In other words,
	\[
		h \sum_{\ell=1}^N P_\ell(a_\ell+\xi) (a_\ell+\xi)^\alpha = M_h^\alpha(\eta_h^\epsilon,-\xi)
	\]
	where $\eta_h^\epsilon:\mathcal G(a_1,h)\to \mathbb R$ defined by $\eta_h^\epsilon(x) = \eta^\epsilon(x+\xi)$ satisfies the discrete $(q,s)$-moment conditions at $-\xi$ by construction.
	We can conclude
	\begin{align*}
		M^\alpha(\eta^\epsilon,0) &= \int_{0}^h  \frac{1}{h} \left[ s! (-1)^{s} \delta_{s\alpha}\right]  \; d\xi \\
		&= s! (-1)^{s} \delta_{s\alpha}.
	\end{align*}
	Lastly, since $\eta^\epsilon$ consist of piecewise polynomials of order $q-1$ divided by a factor of $h^N$, where $N=q+s$, and $\epsilon=O(h)$, we have that
\[
	\sup_{x\in B(0,\epsilon)} |\eta^\epsilon(x)| = O(\epsilon^{-s-1}).
\] 
Thus
\[
	\int_{B(0,\epsilon)} dx\; |\eta^\epsilon(x)| \; |x^s| \le \sup_{B(0,\epsilon)} |\eta^\epsilon(x)|  \; \int_{B(0,\epsilon)} dx\; |x^s| = O(1),
\]
as required for estimate \ref{eq:boundH}

\end{proof}


%%%%%
\subsubsection{Constructions in Higher Dimensions via Tensor Products}
%%%%%

We construct approximations in general $d$-dimension by taking tensor products of 1-D approximations, similar to work by \cite{TorEng:04} for $\mathbf s=\mathbf 0$.
Namely, given approximation order $q$ and multi-index $\mathbf s$, multivariate continuum approximation $\eta:\mathbb R^d\to \mathbb R$ is given by
\begin{equation}\label{eq:tensor_eta}
	\eta(\mathbf x) = \prod_{k=1}^{d} \eta^k(x_k) 
\end{equation}
where $\eta^k:\mathbb R\to \mathbb R$ is a continuum function over the $k$-th axis as given by equations \ref{eq:eta} and \ref{eq:poly}, satisfying the $(q,s_k)$-moment conditions for each $k=1,...,d$.
It follows that these tensor approximations are indeed approximations to multipoles in the higher spatial dimension.

\begin{theorem}\label{thm:prodmomcond}
	Let nonnegative integer $q$, multi-index $\mathbf s$, and $\mathbf x^*\in\mathbb R^d$ be given. 
Suppose $\eta:\mathbb R^d\to\mathbb R$ is a multi-variate grid function given by the tensor product of 1-D approximations $\eta^{k}:\mathbb R\to\mathbb R$.
If $\eta^{k}$ satisfy the discrete $(q,s_k)$-moment conditions at $x^*_k$ for each $k=1,...,d$, then it follows that $\eta$ satisfies the discrete $(q,\mathbf s)$-moment conditions at $\mathbf x^*$.
\end{theorem}

%%
\begin{proof}
Suppose for each $k=1,...,d$ that $\eta^{k}$ satisfies the $(q,s_k)$-moment conditions at $x^*_k$. 
Let $\alpha$ be some multi-index with $|\alpha|\le q+|\mathbf s|-1$. 
Note that,
\[
	M^\alpha(\eta,\mathbf x^*) = \prod_{k=1}^d M^{\alpha_k}(\eta^{k},x^*_k).
\]
Clearly, if $\alpha_k\le q+s_k-1$ for all $k=1,...,d$, then the result follows from the supposition. 
Same applies for $\alpha=\mathbf s$.
Suppose then, that there exists index $\ell$ such that $\alpha_\ell>q+s_\ell-1$, that is $a_\ell=q+s_\ell -1 +i$ for some $i\in\mathbb N$. Thus,
\begin{align*}
	& |\alpha| = \sum_{k\neq \ell} \alpha_k + \alpha_\ell = \sum_{k\neq \ell} \alpha_k + q+s_\ell-1+i \le q + |\mathbf s| -1\\
\Longrightarrow	&  \sum_{k\neq\ell} \alpha_k + i \le \sum_{k\neq \ell} s_k.
\end{align*}
which implies that $\alpha_k<s_k$ for at least one $k\neq\ell$;
for this particular $k$, it follows that 
\[
	M^{\alpha_k}(\eta^{k},x^*_k) = s_k!(-1)^{s_k} \delta_{s_k \alpha_k} = 0
\]
since it has been established that $\alpha_k\neq s_k$, i.e., the product over $k$ is zero if $\mathbf s\neq \alpha$.
\end{proof}


%%%%%%%%
\subsubsection{1-D and 2-D Examples}
%%%%%%%%

%MOD
%Numerical convergence rate tests in the following section will employ the singular source approximation discussed here, replacing multipole terms with grid functions to be used in finite difference schemes.
\hl{Numerical examples in this paper will employ the singular source approximation discussed here, replacing multipole terms with grid functions to be used in finite difference schemes.}
Equation \ref{eq:gridfunsol} gives an explicit formula for such grid functions, depending on the source location $\mathbf x^*$, multi-index $\mathbf s$, approximation order $q$, and of course the underlying finite difference grid.
Alternatively, equations \ref{eq:eta} and \ref{eq:poly} define the continuum form of the approximations, where grid functions are obtained by shifting and sampling over the grid.
The following figures plot 1-D and 2-D continuum approximations $\eta^\epsilon$,
in particular, we plot second- and fourth-order approximations of 
%MOD
$D^s\delta(\mathbf x)$ for $s=0,1,2$ in 1-D \hl{(figure 1)} and $\mathbf s=(0,0),(0,1),(0,2)$ in 2-D \hl{(figure 2)}, with $h=1$.
Two dimensional approximations are constructed via tensor product of 1-D approximations as discussed.
In particular, the $q=2$ approximation for the 1-D Dirac delta function $(s=0)$ is none other than the well known hat/triangular function of unit mass.


%%%%%%%%%
\subsection{Convergence Theory of Wave Equation Solutions with
  Multipole Sources}
%%%%%%%%%

We prove weak convergence of finite difference solutions with discrete multipole
sources satisfying the discrete moment conditions discussed above. 
Convergence theory for smooth solutions is in some sense standard. 
%However 
The easily available results (those based on Lax's Equivalence Theorem) 
pertain to $L^2$ convergence for smooth solutions and smooth source terms,
which we begin with.
%whereas weak convergence of multipole solutions depends on uniform 
%($L^{\infty}$) error estimates for smooth solutions. 
%Therefore we begin with a full account of the convergence of smooth solutions.

Theoretical results presented here pertain to the family of symmetric hyperbolic
systems of  first-order partial differential equations for vector-valued fields $(u,v)$,
defined over a domain $\Omega\subset \mathbb R^d$ in $d=\{1,2,3\}$ dimension
space and some interval in time, given by \ref{eq:hyper_fam} which we restate 
with greater detail.
Let $k_1\in\mathbb N$ and $k_2\in\mathbb N$ denote the dimension of 
vector-fields $u$ and $v$ respectively. 
The systems we are interested in are of the following form:
\begin{equation}\label{eq:cont_system}
\begin{split}
	A(\mathbf x)\frac{\partial}{\partial t}u(\mathbf x,t) + P^T v(\mathbf x,t) 
		&= f(\mathbf x,t) \\
	B(\mathbf x)\frac{\partial}{\partial t}v(\mathbf x,t) - Pu(\mathbf x,t) 
		&= g(\mathbf x,t)
\end{split}
\end{equation}
where
\begin{itemize}
	\item $u\in L^2_{loc}(\mathbb R,\mathcal H_1)$ and 
		$v\in L^2_{loc}(\mathbb R,\mathcal H_2)$ with Hilbert spaces 
		$\mathcal H_1, \mathcal H_2$;
	\item ``coefficient'' operators $A$ and $B$ are $k_1\times k_1$ 
		and $k_2\times k_2$ matrix-valued functions respectively; 
		we assume they are symmetric for all $\mathbf x\in\Omega$ 
		and uniformly-positive definite , i.e., there exists constants 
		$0< A_*\le A^*$ and $0< B_*\le B^*$ such that 
	\begin{equation}
	\begin{split}
		A_*I \le A(\mathbf x) \le A^*I, \quad \forall \mathbf x\in\Omega,\\
		B_*I \le B(\mathbf x) \le B^*I, \quad \forall \mathbf x\in\Omega;
	\end{split}
	\end{equation}
	
	\item $P:\mathcal H_1 \to \mathcal H_2$ is a constant coefficient, 
		first-order differential operator of the form
		\[
			Pu = \sum_{j=1}^d P_j \frac{\partial u}{\partial x_j}, \quad P_j\in \mathbb R^{k_2\times k_1},
		\]
		%ADDED
		\hl{satisfying the following skew-adjoint relation}
		\begin{equation}\label{eq:skew}
			\langle P u, v \rangle = - \langle u, P^T v\rangle
		\end{equation}
		for all $u\in \mathcal H_1$ and $v\in\mathcal H_2$;
	\item source terms $(f,g)$ are for the time being assumed to be smooth, i.e., 
		 $f\in L^2_{loc}(\mathbb R,\mathcal H_1)$ 
		and $g\in L^2_{loc}(\mathbb R,\mathcal H_2)$ ;
	\item we assume solution $(u,v)$ and source terms $(f,g)$ are causal, i.e., for $t < 0$
		\[
			u(\mathbf x,t) = f(\mathbf x,t) = 0, \quad 
			v(\mathbf x,t) = g(\mathbf x,t) = 0.
		\]
\end{itemize}
See \cite{BlazekStolkSymes:13} for proof of existence of solutions, stability, and other mathematical properties for a larger class of partial (integro-)differential systems for wave modeling, applicable to system \ref{eq:cont_system}.

%As an example, consider the acoustic equations in first-order (pressure-velocity) form:
%\begin{equation}\label{eq:acoustics}
%\begin{split}
%	\frac{1}{\kappa(\mathbf x)} \frac{\partial}{\partial t} p(\mathbf x,t) + \nabla\cdot \mathbf v(\mathbf x,t) 
%		&= f(\mathbf x,t)\\
%	\rho(\mathbf x) \frac{\partial}{\partial t} \mathbf v(\mathbf x,t) + \nabla p(\mathbf x,t) 
%		&= \mathbf g(\mathbf x,t)
%\end{split}
%\end{equation}
%where $u=p$ is the scalar pressure field and $v=\mathbf v\in\mathbb R^d$ is the vector particle velocity field; we take $\mathcal H_1=H^1_0(\Omega)$ and $\mathcal H_2=H^1(\Omega)^d$. %MOD
%Coefficient operator $A(\mathbf x)=1/\kappa(\mathbf x)$ and $B(\mathbf x)=\rho(\mathbf x) I$, with $\kappa$ denoting bulk-modulus and $\rho$ density of the medium; here $I\in\mathbb R^{d\times d}$ is the identity matrix.
%Lastly, the differential operator $P$ coincides with the gradient and its adjoint with the negative of the divergence,
%\[
%	P = \mat{\frac{\partial}{\partial x_1}\\ \vdots \\ \frac{\partial}{\partial x_d}}, \quad 
%	P^T = - \mat{\frac{\partial}{\partial x_1},...,\frac{\partial}{\partial x_d}}.
%\]
%%ADDED
%Given homogenous boundary conditions for the pressure field, it can be shown that $P$ given as the gradient operator satisfies the skew-adjoint relation \ref{eq:skew}.


The discretization of continuum problem \ref{eq:cont_system} is given by the following abstract staggered grid dynamical system:
\begin{equation}\label{eq:sg}
\begin{split}
	A_h(u^{n+1}_h - u^n_h)  + P_h(r)^T v^{n+1/2}_h &= rh\,   f_h^{n+1/2} \\
	B_h(v^{n+1/2}_h-v^{n-1/2}_h) -  P_h(r) u^n_h &=  rh\, g_h^{n} 
\end{split}
\end{equation}
where
\begin{itemize} 
	\item $u_h^n\in H_{1,h}$ and $v_h^{n+1/2}\in H_{2,h}$
	for a 	family of Hilbert spaces $H_{1,h}, H_{2,h}$ (``space of spatial-grid functions''), with $h>0$ ("cell size"). Superscript indexes in $u_h^{n}$ and $v_h^{n+1/2}$ refer to the discretized time axis at times $n\Delta t$ and $(n+1/2)\Delta t$ respectively, with time step size $\Delta t = rh$; $r$ will play the role of the CFL constant, crucial for stability of the discretization. Note that the fields $u_h$ and $v_h$ are staggered in time;

	\item bounded self-adjoint positive-definite operators (discretizations of $A$ and $B$)
\[
	A_h: H_{1,h} \rightarrow H_{1,h},\quad B_h: H_{2,h} \rightarrow H_{2,h}
\]
with upper and lower bounds uniform in $h$: there exists constants $0<A_*\le A^*$ and $0<B_*\le B^*$ such that
\begin{equation*}
\begin{split}
	A_*\|u_h\|^2 \le \langle A_h u_h,u_h \rangle \le A^*\|u_h\|^2, 
	\quad \forall u_h\in H_{1,h}, \\
	B_*\|v_h\|^2 \le \langle B_h v_h,v_h \rangle \le B^*\|v_h\|^2,
	\quad \forall v_h\in H_{2,h}; \\
\end{split}
\end{equation*}
\item bounded finite difference operators $P_h(r), Q_h:H_{1,h} \rightarrow H_{2,h}$ such that there is a constant $Q^*>0$ where
\[
	\|Q_h\| \le Q^* \mbox{ for all } h>0,  \quad P_h(r) = rQ_h
\]
%ADDED
\hl{and $P_h(r)$ satisfies the following skew-adjoint relation}
\begin{equation}\label{eq:skew_h}
	\langle P_h(r) u_h, v_h \rangle = - \langle u_h, P_h(r)^T v_h\rangle
\end{equation}
for all $u_h\in H_{1,h}$ and $v_h\in H_{2,h}$;
\item discrete source terms $f_h^{n+1/2}\in H_{1,h}$ and $g_h^n\in H_{2,h}$; if $(f,g)$ are smooth source terms, then
\[
	f_h^{n+1/2} = S_{1,h} f(\cdot,(n+1/2)\Delta t), \quad g_h^n = S_{2,h} g(\cdot,n\Delta t),
\]
where we define bounded operator $S_{1,h}:\mathcal H_1\to H_{1,h}$, and $S_{2,h}:\mathcal H_2\to H_{2,h}$ for ``sampling'' continuum fields onto the spatial grids.
\end{itemize} 


%Consider again the acoustic case (related to problem \ref{eq:acoustics}).
%For simplicity we assume we are dealing with rectangular grids, in 2-D for this example, of the form
%\[
%	\mathcal G(\mathbf 0,\mathbf h) = \{ \mathbf x_{i,j} = (ih,jh) \; : \; i,j\in\mathbb Z \}.
%\]
%The simplest staggered-grid finite difference scheme, second-order in time and space, is given by
%\begin{equation}\label{eq:acoustics_sgfd}
%\begin{split}
%	\frac{1}{\kappa(ih,jh)} \frac{1}{\Delta t} \Big[  (p_h)^{n+1}_{i,j} - (p_h)^{n}_{i,j} \Big] + \hspace{7cm}\\
%	\frac{1}{h} \Big[ (v_{1,h})_{i+1/2,j}^{n+1/2} - (v_{1,h})_{i-1/2,j}^{n+1/2} + 
%				(v_{2,h})_{i,j+1/2}^{n+1/2} - (v_{2,h})_{i,j-1/2}^{n+1/2} \Big] 
%		&= (f_h)^{n+1/2}_{i,j}\\
%	\rho((i+1/2)h,jh) \frac{1}{\Delta t} \Big[ (v_{1,h})^{n+1/2}_{i+1/2,j} - (v_{1,h})^{n-1/2}_{i+1/2,j} \Big] +
%	\frac{1}{h} \Big[ (p_h)_{i+1,j}^{n} - (p_h)_{i,j}^{n} \Big] 
%		&= (g_{1,h})_{i+1/2,j}^{n}\\
%	\rho(ih,(j+1/2)h) \frac{1}{\Delta t} \Big[ (v_{2,h})^{n+1/2}_{i,j+1/2} - (v_{2,h})^{n-1/2}_{i,j+1/2} \Big] +
%	\frac{1}{h} \Big[ (p_h)_{i,j+1}^{n} - (p_h)_{i,j}^{n} \Big] 
%		&= (g_{2,h})_{i,j+1/2}^{n}
%\end{split}
%\end{equation}
%where finite difference solution $(p_h,v_{1,h},v_{2,h})$ approximates the continuum fields $(p,v_1,v_2)$,% for system \ref{eq:cont_system}, mainly,
%\begin{equation*}
%\begin{split}
%	(p_h)_{i,j}^{n} &\approx p(ih,jh,n\Delta t), \\
%	(v_{1,h})_{i+1/2,j}^{n+1/2} &\approx v_1((i+1/2)h,jh,(n+1/2)\Delta t),\\
%	(v_{2,h})_{i,j+1/2}^{n+1/2} &\approx v_2(ih,(j+1/2)h,(n+1/2)\Delta t).
%\end{split}
%\end{equation*}
%We emphasize that the velocity fields for finite difference scheme \ref{eq:acoustics_sgfd} are staggered with respect to spatial grids, more specifically each component is shifted by half a cell size in its respective axis.
%
%Relating system \ref{eq:acoustics_sgfd} with \ref{eq:sg}, we see
%\[
%	(u_h)_{i,j} = (p_h)_{i,j}, \quad 
%	(v_h)_{i,j} = \mat{(v_{1,h})_{i+1/2,j}\\ 
%				  (v_{2,h})_{i,j+1/2} }.
%\]
%The space $H_{1,h}$ corresponds to the set of square summable scalar-valued
%%MOD
%functions on rectangular grids that satisfy the homogenous boundary condition, and $H_{2,h}$ is the set of square summable $\mathbb R^2$-valued functions on half-cell shifted grids, both equipped with a discrete $L^2$ inner-products,  
%\begin{equation*}
%\begin{split}
%	\langle u_h,\tilde u_h\rangle = h^2 \sum_{i,j} (u_h)_{i,j} (\tilde u_h)_{i,j} &\quad 	\text{for } u_h,\tilde u_h \in H_{1,h}\\
%	\langle v_h, \tilde v_h \rangle = h^2 \sum_{i,j} (v_h)_{i,j} \cdot (\tilde v_h)_{i,j} &\quad
%	\text{for } v_h,\tilde v_h \in H_{2,h}.
%\end{split}
%\end{equation*}
%Inner-products and norms of the spaces $\mathcal H_1, \mathcal H_2, H_{1,h}, H_{2,h}$ are all denoted by $\langle\cdot,\cdot\rangle$ and $\|\cdot\|$ and interpreted given the context, unless otherwise specified.
%
%Sampling operators $S_{1,h}$ and $S_{2,h}$ coincide with evaluating continuum functions on grid or shifted grid points accordingly,
%\[
%	(S_{1,h} u)_{i,j} = u(ih,jh), \quad 
%	(S_{2,h} v)_{i,j} = \mat{v_{1}((i+1/2)h,jh)\\ 
%				  v_{2}(ih,(j+1/2)h) }.
%\]
%Discretized coefficient operators $A_h$ and $B_h$ are given by,
%\[
%	(A_h u_h)_{i,j} = \frac{1}{\kappa(ih,jh)} (u_h)_{ij}, \quad
%	(B_h v_h)_{i,j} = \mat{ \rho((i+1/2)h,jh) (v_{1,h})_{i+1/2,j} \\
%					 \rho(ih,(j+1/2)h) (v_{2,h})_{i,j+1/2} }.
%\]
%Lastly, $P_h$ and $P_h^T$ correspond to second-order approximations of the gradient and the negative of the divergence respectively;
%\begin{equation*}
%\begin{split}
%	(P_{h}(r) u_h)_{i,j} &= r \mat{ (u_h)_{i+1,j} - (u_h)_{i,j} \\
%					     (u_h)_{i,j+1} - (u_h)_{i,j} }, \\
%	(P_{h}(r)^T v_h)_{i,j} &= -r \Big[ (v_{1,h})_{i+1/2,j} - (v_{1,h})_{i-1/2,j} + (v_{2,h})_{i,j+1/2} - (v_{2,h})_{i,j-1/2} \Big]
%\end{split}
%\end{equation*}
%again, with $r =  \Delta t/h$.
%In general, the differential operator $P$ can be approximated by a family of central difference operators $P_h$ of even order $2p$ for $p\in\mathbb N$, resulting in the 2-$2p$ staggered-grid finite difference schemes.
%%ADDED
%Similar to the continuum case, it can be shown that the $P_h$ indeed satisfy the skew-adjoint relation \ref{eq:skew_h}.

%%%%%%%
\subsubsection{Energy Estimates}

Define the following energy form:
\begin{equation*}
\begin{split}
	E^n := E(u_h^n,v_h^{n-1/2}) &:= \frac{1}{2} \Big( \langle A_h u_h^n,u_h^n\rangle + 
				       \langle B_hv_h^{n-1/2},v_h^{n-1/2}\rangle +
				       \langle P^T_h v_h^{n-1/2},u_h^n\rangle \Big).
%	&= \frac{1}{2} \mat{ u_h^n \\ v_h^{n-1/2}}^T
%			     \mat{ A_h & \frac{1}{2}P^T_h\\
%			     	     \frac{1}{2}P_h & B_h }
%			     \mat{ u_h^n\\ v_h^{n-1/2} }.
\end{split}
\end{equation*}

\begin{theorem}
For a solution $(u_h,v_h)$ of system \ref{eq:sg} with $(f_h, g_h)\equiv 0$, it follows that $E^n$ is independent of time index $n$.
Moreover, for sufficiently small $r$, there exist $0 < C_* \le C^*$ such that 
\begin{equation}
\label{eqn:pd}
	C_* \Big( \|u_h\|^2 + \|v_h\|^2 \Big) \le E(u_h,v_h) \le C^*\Big( \|u_h\|^2 +\|v_h\|^2 \Big).
\end{equation}
\end{theorem}

\begin{proof}
First we show the energy conservation property:
\begin{equation*}
\begin{split}
	E^{n+1} - E^n =\langle A_h( u^{n+1}_h-u^n_h ), (u^{n+1}_h+u^n_h)\rangle 
		&+ \langle B_h( v^{n+1/2}_h-v^{n-1/2}_h ), (v^{n+1/2}_h+v^{n-1/2}_h)\rangle \\
		&+ \langle P_h^T v^{n+1/2}_h, u^{n+1}_h\rangle ) - \langle P_h^T v^{n-1/2}_h, u^n_h\rangle )
\end{split}
\end{equation*}
\begin{equation*}
\begin{split}
	= \langle -P_h^Tv^{n+1/2}_h,(u^{n+1}_h+u^n_h) \rangle 
		&+ \langle P_hu^n_h, (v^{n+1/2}_h+v^{n-1/2}_h) \rangle\\
		&+\langle P_h^Tv^{n+1/2}_h,u^{n+1}_h\rangle ) - \langle P^T_h v^{n-1/2}_h,u^n_h\rangle )
	=0.
\end{split}
\end{equation*}
The upper bound in \ref{eqn:pd} is clear from $h$-uniform bounds on $A_h$, $B_h$,
$P_h$. 
The lower bound follows from
\[
	E(u_h,v_h) \ge A_*\|u_h\|^2 + B_*\|v_h\|^2 - rQ^*\|u_h\|\|v_h\|
\]
which is a positive definite form in $\|u_h\|,\|v_h\|$ when $rQ^* <\sqrt{A^*B^*}$.
In other words, we have established positive definiteness of $E(\cdot,\cdot)$, and more importantly, that the energy form is equivalent to a norm.
\end{proof}

%%%%%%%%
\subsubsection{$L^2$ Convergence of Smooth Solutions}

%Inhomogeneous system:
%\begin{equation}\label{eqn:in}
%\begin{split}
%A(u^{n+1}_h - u^n_h) & =  -D^T v^{n+1/2}_h + hf^{n+1/2}_h \\
%B(v^{n+1/2}_h-v^{n-1/2}_h) & =  D u^n_h + h g^n_h
%\end{split}
%\end{equation}

Given theorem 5, that is, equivalency of between the $L^2$ and energy norms, we focus on proving stability estimates with respect to the energy norm in order to imply $L^2$ convergence. 

\begin{theorem}
For $r$ sufficiently small that $E$ is positive definite, stability follows, i.e., there exists $K \ge 0$, $\lambda > 1$ so that
\[
	E^n \le \lambda^n E^0 + 
	Kh\sum_{m=0}^{n-1}\lambda^{n-m+1}(\|f^{m-1/2}_h\|^2 + \|g^m_h\|^2)
\]
where $(u_h,v_h)$ satisfy the inhomogeneous system \ref{eq:sg} with source terms $(f_h,g_h)$.
\end{theorem}

\begin{proof}
The same arithmetic as used in the homogeneous case leads to
\begin{equation*}
\begin{split}
	E^{n+1}-E^n 
	&= rh\langle f^{n+1/2}_h, u^{n+1}_h+u^n_h\rangle + rh\langle g^n_h,
v^{n+1/2}_h+v^{n-1/2}_h\rangle\\
	&\le \frac{rh}{2\alpha^2} (\|f^{n+1/2}_h\|^2 + \|g^n_h\|^2) +
\frac{\alpha^2rh}{C_*}(E^{n+1}+E^n)
\end{split}
\end{equation*}
for any $\alpha \in (0,1)$. Choose $\alpha$ so that $\alpha^2 rh < C_*$
(that is, can make fixed choice of $\alpha$ for $rh$ small enough), and
set 
\begin{equation}\label{eq:K}
	\lambda = \left(1+\frac{\alpha^2rh}{C_*}\right) \left(1-\frac{\alpha^2rh}{C_*}\right)^{-1}
	\quad
	\text{and}
	\quad
	K = \frac{\frac{r}{2\alpha^2}}{ \left(1+\frac{\alpha^2rh}{C_*}\right) }.
\end{equation}
Then
\begin{equation*}
\begin{split}
	\lambda^{-1}E^{n+1} - E^n &= Kh (\|f^{n+1/2}_h\|^2 + \|g^n_h\|^2)  \\
	\Longrightarrow
	\lambda^{-(n+1)}E^{n+1} - \lambda^{-n}E^n &= \lambda^{-n}Kh (\|f^{n+1/2}_h\|^2 + \|g^n_h\|^2) 
\end{split}
\end{equation*}
whence
\[
\lambda^{-n}E^n = E^0 +
Kh\sum_{m=0}^{n-1}\lambda^{-m+1}(\|f^{m-1/2}_h\|^2 + \|g^m_h\|^2).
\]
\end{proof}

\begin{theorem}
Suppose that $(u^n,v^{n-1/2})\in \mathcal H_1 \times \mathcal H_2$,
and $(u^n_h,v^{n-1/2}_h)$ solves the discretized system \ref{eq:sg} with initial data
\[
	u^0_h = S_{1,h} u^0, \quad v^{-1/2}_h = S_{2,h} v^{-1/2}.
\]
Set
\begin{equation}\label{eqn:tr}
\begin{split}
	\delta f^{n+1/2}_h & = \frac{1}{rh} \Big( A_h( S_{1,h} u^{n+1} - S_{1,h}u^{n}) + 
							   P_h^T S_{2,h} v^{n+1/2} \Big) + f_h^{n+1/2}\\
	\delta g^{n}_h  &= \frac{1}{rh} \Big( B_h (S_{2,h} v^{n+1/2} - S_{2,h} v^{n-1/2}) - 
							P_h S_{1,h}u^n \Big) + g_h^{n}.
\end{split}
\end{equation}
Moreover, let $T>0$ be given, independent of $h$. If $(\delta f_h,\delta g_h)$, as defined above, satisfy the estimate
\begin{equation}\label{eq:estimate}
	\|\delta f_h^{m-1/2}\|^2 + \|\delta g_h^m\|^2 \le L^2 h^{2p}
\end{equation}
	for $0\le m \le N$, $N\Delta t=T$, then 
\begin{equation}
\label{eqn:ee}
	E(u_h^n - S_{1,h}u^n, v_h^{n-1/2} - S_{2,h}v^{n-1/2} ) \le M^2L^2h^{2p}.
\end{equation}
\end{theorem}

\begin{proof}
Define $K, \lambda$ as in \ref{eq:K}, then it follows from previous theorem that
\[
	E(u^n_h - S_{1,h}u^n, v^{n-1/2}_h - S_{2,h} v^{n-1/2}) \le
	Kh\sum_{m=0}^{n-1}\lambda^{n-m+1}(\| \delta f_h^{m-1/2}\|^2 + \|\delta g_h^m\|^2).
\]
which implies
\[
	E(u_h^n - S_{1,h}u^n, v_h^{n-1/2} - S_{2,h}v^{n-1/2} ) \le 
	KL^2 h^{2p+1}\frac{\lambda^N-1}{\lambda-1}
\]
using estimate \ref{eq:estimate}.
Note that if $h$ is sufficiently small, then $\lambda < 1 +
2\frac{\alpha^2rh}{C_*}$ so $\lambda^N \le
exp(2\frac{\alpha^2}{C_*}T)$, where $T=Nrh$. If $h$ is perhaps smaller
yet, then $\lambda > 1+\frac{\alpha^2rh}{2C_*}$. Putting this all
together, there is $M^2$ depending on $T$ and all of the other constants
in the setup, so that if $nrh <T$ so that \ref{eqn:ee} holds.
\end{proof}

%MOD
%If $(u,v)$ is a smooth solution of the acoustic (pressure-velocity) system 
%\ref{eq:acoustics} with smooth right-hand sides $(f,g)$,
%and $(u_h,v_h)$ is the staggered-grid finite difference approximation of 
%order $2$-$2p$, then, from standard truncation error calculations, for 
%each time slice of compact support we have,
\hl{If $(u,v)$ is a smooth solution of the continuum problem} \ref{eq:cont_system}
\hl{with smooth right-hand sides $(f,g)$, and $(u_h,v_h)$ is the finite difference 
solution where $P_h$ is of order $2p$,} then, from standard truncation error 
calculations, for each time slice of compact support we have,
\begin{equation*}
\begin{split}
	\delta f_h^{n+1/2} &= O( \Delta t^2+h^{2p})\\
	\delta g_h^{n} &= O( \Delta t^2+h^{2p})
\end{split} 
\end{equation*}
point-wise and uniformly in $i,j$. 
Since the support is uniformly bounded if $0\le n\Delta t \le T$, 
the $O$ statements apply to the $L^2$ norms as well.
Thus, by equivalency of the energy form with the $L^2$-norm, and the \hl{theorem} above, 
we can conclude
\[
	\|u_h-S_{1,h}u\| + \|v_h-S_{2,h}v\| = O(\Delta t^2+h^{2p}),
\]
that is, $2$-$2p$ order convergence in the $L^2$ sense.

%Let $u$ denote the solution of \ref{eqn:sg} with
%$(u^0,v^{-1/2})=S(h)(U^0,V^{-1/2})$. Then $(u,v)-S(h)(U,V)$ solves
%\ref{eqn:in} with $f,g$ as just given.  Therefore the error estimate
%\ref{eqn:ee} implies that 
%\[
%\|(u,v)-S(h)(U,V)\| = O(h^{2p} + \Delta t^2),
%\]
%that is, $(2,2p)$-th order convergence in the $L^2$ sense.

%As mentioned at the beginning, to establish weak convergence of
%multipole solutions requires $L^{\infty}$, rather than $L^2$,
%convergence of smooth solutions. We obtain the stronger convergence
%theory by imitating the proof of the Sobolev Embedding Theorem.
%
%Sketch:
%
%apply difference operators $D_j$ to difference equations, commute with
%coefficients, note that same error estimates inherited by $D_j(u,v)$
%(with respect to $\partial_j (U,V)$) assuming that data is smooth enough
%
%iterate, addemup to show that discrete Laplacian of $(u,v)$ converges
%at same rate, that is, discrete Laplacian of $(u,v)-S(h)(U,V)$ is
%$O(h^{2p} + \Delta t^2)$ also (in $L^2$ sense)
%
%use discrete FT, Plancherel identity, and C-S inequality to bound
%$(u,v)-S(h)(U,V)$ pointwise. This requires only energy estimate on
%Laplacian for dimensions 1 and 2; for dimension 3, requires square of
%Laplacian for dimension 3.

 
%%%%%%%
\subsubsection{Weak Convergence of Singular Solutions}

Let $n_{\rm min}, n_{\rm max}$ be integers.
Suppose that the grid-function sequence $(u_h^n,v_h^{n+1/2})$, 
satisfies \ref{eq:sg} with source term $(f_h^{n+1/2},g_h^{n})$
and vanish identically for $n<n_{\rm min}$.
Let $(\tilde u_h^{n+1/2},\tilde v_h^{n})$ be another sequence of 
grid functions that vanish for $n>n_{\rm max}$. 
Then,
%$(q^n(h),w^{n+1/2}(h))$ be yet another $G(h)$-valued sequence that
%vanished for $n > n_{\rm max}$. Then (suppressing $h$-dependence for
%the moment)
\[
	\Delta t \sum_{n} \Big( \langle \tilde u_h^{n+1/2}, f^{n+1/2}\rangle +
					   \langle \tilde v_h^n, g_h^n \rangle \Big)
\]
\begin{equation*}
\begin{split}
	= \sum_n \Big( \langle \tilde u_h^{n+1/2}, 
					A_h( u_h^{n+1} - u_h^n) + P_h^T v_h^{n+1/2} \rangle 
			  +  \langle \tilde v_h^n, 
					B_h ( v_h^{n+1/2}-v_h^{n-1/2} ) - P_h u_h^n \rangle \Big)
\end{split}
\end{equation*}
\begin{equation*}
\begin{split}
	= \sum_n \Big( \langle A_h ( \tilde u_h^{n-1/2}-\tilde u_h^{n+1/2} ), u_h^n \rangle 
			       -\langle P_h \tilde u_h^{n+1/2}, v_h^{n+1/2}\rangle 
	 		       +\langle B_h ( \tilde v_h^{n} - \tilde v_h^{n+1} ), v_h^{n+1/2} \rangle
			       +\langle P_h^T \tilde v_h^n, u_h^n\rangle \Big)
\end{split}
\end{equation*}
\begin{equation}\label{eqn:sbp}
\begin{split}
	= \sum_n \Big( \langle A_h (\tilde u_h^{n-1/2} - \tilde u_h^{n+1/2}) - P_h^T \tilde v_h^n, 
				u_h^n\rangle 
	                      +	\langle B_h (\tilde v_h^{n} - \tilde v_h^{n+1}) + P_h \tilde u_h^{n+1/2},
	                      	v_h^{n+1/2}\rangle
\end{split}
\end{equation}

Define $(\tilde f_h^{n},\tilde g_h^{n+1/2})\in H_{1,h}\times H_{2,h}$ by
\begin{equation}\label{eqn:inadj}
\begin{split}
	A_h( \tilde u_h^{n-1/2}-\tilde u_h^{n+1/2} ) - P_h^T \tilde v^n & =  rh\, \tilde f_h^{n} \\
	B_h( \tilde v^{n} - \tilde v^{n+1} ) + P_h \tilde u^{n+1/2} &=  rh \, \tilde g_h^{n+1/2}.
\end{split}
\end{equation}
Then, identity \ref{eqn:sbp} can be re-written as
\begin{equation}
\label{eqn:weak}
	\Delta t \sum_{n} \Big( \langle \tilde u_h^{n+1/2}, f_h^{n+1/2} \rangle +
	 			 	  \langle \tilde v_h^n, g_h^n \rangle \Big)
	= \Delta t \sum_{n} \Big( \langle \tilde f_h^n, u_h^n \rangle + 
					     \langle \tilde g_h^{n+1/2}, v_h^{n+1/2}\rangle \Big).
\end{equation}

%MOD
%Now specialize to the acoustic case - the elastic case is
%similar. In this case $g$ (and $\tilde g$) is vector-valued, and $f$
 %(and $\tilde f$) is scalar-valued. Once again, $P_h$ is a discrete gradient;
%assume that it is accurate of order $2p$, as before. $P_h^T$ is a discrete
%(negative) divergence, also accurate of order $2p$.  
%%\hl{Assuming $P_h$ is accurate of order $2p$, it follows that $P_h^T$ is a $2p$-order discretization of $P^T$.}
The adjoint system
\ref{eqn:inadj} is recognized as another discretization of the
original system \ref{eq:cont_system}, with two differences: 
the time index is dual to the one used in \ref{eq:sg}, that is, 
shifted by $\Delta t/2$, and the inhomogenous terms get a negative sign. Evidently, the truncation error analysis is exactly the same. 

Suppose that $(\tilde f,\tilde g)$ are smooth in both temporal and spatial variables, and that 
\begin{equation}\label{eq:tilde_f}
	\tilde f_h^n = S_{1,h}\tilde f^{n}, \quad
	\tilde g_h^{n+1/2} = S_{2,h} \tilde g^{n+1/2}.
\end{equation}
Denote by $(\tilde u,\tilde v)$ the (weak) solution of the acoustic system \ref{eq:acoustics} with
right-hand sides $(-\tilde f,-\tilde g)$ that vanishes for $t>T$. 
The standard theory for hyperbolic systems shows that $(\tilde u,\tilde v)$ is also smooth. 
Let $(\tilde u_h^{n+1/2},\tilde v_h^n)$ be the solution of the discrete adjoint system
\ref{eqn:inadj} with right-hand sides $(\tilde f_h^n, \tilde g_h^{n+1/2})$. 
Then (as before assuming that $P_h$ is accurate of order $2p$)
\[
	\delta \tilde u_h^{n+1/2} = \tilde u_h^{n+1/2} - S_{1,h} \tilde u^{n+1/2},\quad
	\delta \tilde v_h^{n} = \tilde v_h^{n} - S_{2,h} \tilde v^{n}
\]
solve \ref{eqn:inadj} with right-hand side that is
$O(\Delta t^2+h^{2p})$, so according to the error estimate \ref{eqn:ee}
(which applies ipso facto to the system \ref{eqn:inadj}),
\begin{equation}
\label{eqn:truncadj}
	\| \delta \tilde u_h^{n+1/2}\|^2 + \|\delta \tilde v_h^{n}\|^2 = O(\Delta t^2+h^{2p})
\end{equation}
over any finite range of $n\Delta t$. 

We now have all of the ingredients to prove weak convergence, which we now state:
\begin{theorem}\label{thm:conv}
Let $(u,v)$ be a solution to continuum problem \ref{eq:cont_system} with singular source terms $(f,g)$ of the form
\begin{equation*}
\begin{split}
	f(\mathbf x,t) &=  w(t) D^{\mathbf s_1}\delta(\mathbf x-\mathbf x^*),\\
	g(\mathbf x,t) &= z(t) D^{\mathbf s_2}\delta(\mathbf x-\mathbf x^*),
\end{split}
\end{equation*}
where $w(t)\in\mathbb R^{k_1}$ and $z(t)\in\mathbb R^{k_2}$ are smooth vector-valued functions in time.
Also, let $(u_h,v_h)$ denote the corresponding finite difference solution with singular source approximates $(f_h,g_h)$ of order $q$.
Then, for any smooth test functions $(\tilde f,\tilde g)$, we have the following error estimate
\begin{equation}\label{eq:weak_conv}
\begin{split}
	\mathcal E &:= \left| \int_0^T  \Big\{ \langle u,\tilde f\rangle + \langle v,\tilde g\rangle \Big\} dt -
		\Delta t \sum_{n=0}^N \Big\{ \langle u_h^{n}, \tilde f_h^{n} \rangle +
						 	   \langle v_h^{n+1/2}, \tilde g_h^{n+1/2}  \rangle \Big\} \right|\\
	&= O(\Delta t^2 + h^q + (\Delta t^2+h^{2p})h^{-N^*-d/2}),
\end{split}
\end{equation}
with $N^* = \max\{|\mathbf s_1|,|\mathbf s_2|\}$. 
Again, $(\tilde f_h,\tilde g_h)$ as given by \ref{eq:tilde_f}. 
\end{theorem}

\begin{proof}
Using \ref{eqn:weak}, and its continuum version, error $\mathcal E$ is rewritten in terms of inner products with the singular source terms, both in the continuum and discrete sense, i.e.,
\[
	\mathcal = \left| \int_0^T  \Big\{ \langle f,\tilde u \rangle + \langle g,\tilde v\rangle \Big\} dt -
		\Delta t \sum_{n=0}^N \Big\{ \langle f_h^{n+1/2}, \tilde u_h^{n+1/2} \rangle +
						 	   \langle g_h^{n}, \tilde v_h^{n}  \rangle \Big\} \right|.
\]
%MOD Note $(\tilde p,\tilde v)$
Note \hl{$(\tilde u,\tilde v)$} are solutions to problem \ref{eq:cont_system} with smooth source terms 
$(-\tilde f,-\tilde g)$, and $(\tilde u_h,\tilde v_h)$ is the respective staggered grid finite difference solutions with sources $(-\tilde f_h,-\tilde g_h)$.
Furthermore,
\begin{equation*}
\begin{split}
\mathcal E &= \left| \int_0^T  \Big\{ \langle f,\tilde u \rangle + \langle g,\tilde v\rangle \Big\} dt -
		\Delta t \sum_{n=0}^N \Big\{ \langle f_h^{n+1/2}, \tilde u_h^{n+1/2} \pm S_{1,h}\tilde u^{n+1/2} \rangle +
						 	   \langle g_h^{n}, \tilde v_h^{n} \pm S_{2,h}\tilde v^{n} \rangle \Big\} \right|\\
	& \le \mathcal E_1 + \mathcal E_2.
\end{split}
\end{equation*}
with
\[
\mathcal E_1 := \left|  \int_0^T  \Big\{ \langle f,\tilde u \rangle + \langle g,\tilde v\rangle \Big\} dt
			-\Delta t \sum_{n=0}^{N} \Big\{  \langle f_h^{n+1/2}, S_{1,h}\tilde u^{n+1/2} \rangle +
			\langle g_h^n, S_{2,h} \tilde v^{n} \rangle \Big\} \right|
\]
\[
\mathcal E_2 := \left| \Delta t \sum_{n=0}^N \Big\{ \langle f_h^{n+1/2}, \delta \tilde u_h^{n+1/2} \rangle +
			\langle g_h^n, \delta \tilde v_h^n \rangle \Big\} \right|.
\]

For $\mathcal E_1$, we first focus on the terms involving $f$ and $\tilde u$;
\begin{equation*}
\begin{split}
	\left| \int_0^T \langle f, \tilde u\rangle \;  dt -
		\Delta t \sum_{n=0}^N \langle f_h^{n+1/2}, S_{1,h}\tilde u^{n+1/2}\rangle \right|
	\le \\
	\left| \int_0^T \langle f,\tilde u \rangle \; dt - \Delta t \sum_{n=0}^N \langle f^{n+1/2},\tilde u^{n+1/2}\rangle \right|
	+ \left| \Delta t \sum_{n=0}^N \Big\{ \langle f_h^{n+1/2}, S_{1,h}\tilde u^{n+1/2}\rangle - 
							     \langle f^{n+1/2}, \tilde u^{n+1/2} \rangle \Big\} \right |,
\end{split}
\end{equation*}
which is nothing more than quadrature error for the time integration and the singular source discretization error.
In particular,
\[
	\left| \int_0^T \langle f,\tilde u \rangle \; dt - \Delta t \sum_{n=0}^N \langle f^{n+1/2},\tilde u^{n+1/2}\rangle \right| = O(\Delta t^2)
\]
from standard error estimates of the midpoint rule, and
\[
	\left| \Delta t \sum_{n=0}^N \Big\{ \langle f_h^{n+1/2}, S_{1,h}\tilde u^{n+1/2}\rangle - 
							     \langle f^{n+1/2}, \tilde u^{n+1/2} \rangle \Big\} \right |
	= O(h^q)
\]
from the singular source approximation theory (theorem 2).
Similar error estimates can be derived for the terms involving $g$ and $\tilde v$.\footnote{$O(\Delta t^2)$ quadrature error estimates follow from noting that the summation term coincides with the trapezoidal rule, assuming that $g(t)$ and $\tilde v(t)$ satisfy homogenous initial and final time conditions respectively.}

We bound $\mathcal E_2$ using $L^2$-error estimates of finite difference solutions for smooth problems \hl{(i.e., equation }\ref{eqn:truncadj} \hl{)}, and using $L^2$-bounds of the discrete singular source approximations; one can show
\[
 	\| f_h\|_2 = O(h^{-|\mathbf s_1|-d/2}) \quad {\rm and} \quad 
	\| g_h\|_2 = O(h^{-|\mathbf s_2|-d/2}).
\]
Whence,
\[
	\mathcal E_2 \le \Delta t \sum_{n=0}^N \Big\{ \|f_h^{n+1/2}\|_2 \; \|\delta \tilde u_h^{n+1/2}\|_{2}
								+ \|g_h^{n}\|_2  \; \| \delta \tilde v_h^n\|_2 \Big\}
					= O((\Delta t^2+h^{2p})h^{-N^*-d/2}).
\]
Estimate \ref{eq:weak_conv} thus follows.
\end{proof}

Consider the simpler case where only multipoles of order zero are present, that is $N^*=0$.
According to estimate \ref{eq:weak_conv}, we have weak convergence at rate $2-d/2$ if we choose $q=2p$ and $\Delta t = O(h)$.
In other words, convergence (in the weak sense) is guaranteed though at a suboptimal rate.
For multipoles of order $N^*>0$, to retrieve the smooth solution behavior, we would require $2p > N^*+d/2$ and $\Delta t\to 0$ like a positive power of $h$.



The (weak) error estimate we present here can be improved by making the following observation: the $h^{-d/2}$ factor originates from $L^2$-bounds of multipole source approximations.
In particular, we can remove this factor by using $L^1$-bounds instead, mainly
\[
	\| f_h\|_1 = O(h^{-|\mathbf s_1|}) \quad {\rm and} \quad \|g_h\|_1 = O(h^{-|\mathbf s_2|}).
\] 
Applying H\"older's inequality to $\mathcal E_2$ yields
\begin{equation}
\label{eq:weak_conv_infty}
	\mathcal E_2 \le \Delta t \sum_{n=0}^N \Big\{ \|f_h^{n+1/2}\|_1 \; \|\delta \tilde u_h^{n+1/2}\|_{\infty}
								+ \|g_h^{n}\|_1  \; \| \delta \tilde v_h^n\|_{\infty} \Big\}
					= O((\Delta t^2+h^{2p})h^{-N^*}).
\end{equation}
We conjecture point-wise bounds hold for smooth $\delta \tilde u$ and $\delta \tilde v$, whence
\begin{equation}\label{eq:conjecture}
	\mathcal E = O(\Delta t^2 +h^q + (\Delta t^2+h^{2p})h^{-N^*}),
\end{equation}
yielding optimal rates for the $N^*=0$ case, again if $q\ge2p$.
Note that \ref{eq:weak_conv_infty} follows through if $L^\infty$ error terms $\|\delta \tilde u_h\|_\infty$ and $\|\delta \tilde v_h\|_\infty$ have optimal rates.
The authors are not, however, aware of any such $L^\infty$ error estimates for the types of hyperbolic systems considered here. 
One could potentially attempt to prove stability in $L^\infty$ in order to apply the Lax equivalence theorem and imply $L^\infty$ estimates (see \cite{Brenner:06}) or alternatively  
proof $L^\infty$ estimates requiring only $L^2$ stability (see \cite{Layton:82}).


%see \cite{Brenner:06}, an overview of the theory for $L^p$ estimates and stability for initial value problems, this theory is based on Fourier multipliers and Besov spaces.
%Alternatively, there is a way to produce $L^\infty$ estimates requiring only $L^2$ stability (this is also discussed in \cite{Brenner:06}).
%In particular, we highlight work by \cite{Layton:82}, for his $L^\infty$ estimates (with $L^2$ stability) based on simple $L^2$-techniques demonstrating optimal convergence rates, albeit for a simple hyperbolic initial value problem.
%Again, we reiterate that the current theory on $L^\infty$ error estimate finite difference approximations is not immediately applicable to the class of hyperbolic systems considered here, and thus the improved convergence rates remain as a conjecture. 



%Any discrete multipole of order zero (not a lot of variety there) is
%uniformly bounded in $L^1$. Therefore this last estimate shows that
%the multipole solution is weakly convergent of the same order as a
%smooth solution, mainly $2$-$2p$ convergence if source approximation order $q\ge 2p$.
%For a multipole of order $N^*$, to retrieve the smooth
%solution behaviour must have $2p > N^*$ and
%$\Delta t$ must go to zero like a positive power of $h$.




%\newpage
%So it follows from
%\ref{eqn:weak} that
%\[
%\Delta t \sum_{n} \langle r^n,u^n \rangle + \langle
%z^{n+1/2},v^{n+1/2}\rangle
%\]
%\[
%=\Delta t \sum_{n} \langle S(h)(W((n+1/2)\Delta t),Q(n\Delta t)),
%(f^{n+1/2},g^n)\rangle + O(h^{2p}+\Delta t^2) \times \|(f^{n+1/2},g^n)\|_{L^1}
%\]
%using the $L^{\infty}$ error estimate for the smooth solution $(w,q)$
%proved at the end of the last subsection. 
%
%Any discrete multipole of order zero (not a lot of variety there) is
%uniformly bounded in $L^1$. Therefore this last estimate shows that
%the multipole solution is weakly convergent of the same order as a
%smooth solution. For a multipole of order $k$, to retrieve the smooth
%solution behaviour must have $2p > k$ and
%$\Delta t$ must go to zero like a positive power of $h$.

%ADDED
\subsubsection{Acoustic Case}

As an example, consider the acoustic equations in first-order (pressure-velocity) form:
\begin{equation}\label{eq:acoustics}
\begin{split}
	\frac{1}{\kappa(\mathbf x)} \frac{\partial}{\partial t} p(\mathbf x,t) + \nabla\cdot \mathbf v(\mathbf x,t) 
		&= f(\mathbf x,t)\\
	\rho(\mathbf x) \frac{\partial}{\partial t} \mathbf v(\mathbf x,t) + \nabla p(\mathbf x,t) 
		&= \mathbf g(\mathbf x,t)
\end{split}
\end{equation}
where $u=p$ is the scalar pressure field and $v=\mathbf v\in\mathbb R^d$ is the vector particle velocity field; we take \hl{$\mathcal H_1=H^1_0(\Omega)$ and $\mathcal H_2=H^1(\Omega)^d$}. %MOD
Coefficient operator $A(\mathbf x)=1/\kappa(\mathbf x)$ and $B(\mathbf x)=\rho(\mathbf x) I$, with $\kappa$ denoting bulk-modulus and $\rho$ density of the medium; here $I\in\mathbb R^{d\times d}$ is the identity matrix.
Lastly, the differential operator $P$ coincides with the gradient and its adjoint with the negative of the divergence,
\[
	P = \mat{\frac{\partial}{\partial x_1}\\ \vdots \\ \frac{\partial}{\partial x_d}}, \quad 
	P^T = - \mat{\frac{\partial}{\partial x_1},...,\frac{\partial}{\partial x_d}}.
\]
%ADDED
\hl{
Given homogenous boundary conditions for the pressure field, it can be shown that the gradient operator $P$ satisfies skew-adjoint relation} \ref{eq:skew}.

For simplicity we assume we are dealing with rectangular grids, in 2-D for this example, of the form
\[
	\mathcal G(\mathbf 0,\mathbf h) = \{ \mathbf x_{i,j} = (ih,jh) \; : \; i,j\in\mathbb Z \}.
\]
The simplest staggered-grid finite difference scheme, second-order in time and space, is given by
\begin{equation}\label{eq:acoustics_sgfd}
\begin{split}
	\frac{1}{\kappa(ih,jh)} \frac{1}{\Delta t} \Big[  (p_h)^{n+1}_{i,j} - (p_h)^{n}_{i,j} \Big] + \hspace{7cm}\\
	\frac{1}{h} \Big[ (v_{1,h})_{i+1/2,j}^{n+1/2} - (v_{1,h})_{i-1/2,j}^{n+1/2} + 
				(v_{2,h})_{i,j+1/2}^{n+1/2} - (v_{2,h})_{i,j-1/2}^{n+1/2} \Big] 
		&= (f_h)^{n+1/2}_{i,j}\\
	\rho((i+1/2)h,jh) \frac{1}{\Delta t} \Big[ (v_{1,h})^{n+1/2}_{i+1/2,j} - (v_{1,h})^{n-1/2}_{i+1/2,j} \Big] +
	\frac{1}{h} \Big[ (p_h)_{i+1,j}^{n} - (p_h)_{i,j}^{n} \Big] 
		&= (g_{1,h})_{i+1/2,j}^{n}\\
	\rho(ih,(j+1/2)h) \frac{1}{\Delta t} \Big[ (v_{2,h})^{n+1/2}_{i,j+1/2} - (v_{2,h})^{n-1/2}_{i,j+1/2} \Big] +
	\frac{1}{h} \Big[ (p_h)_{i,j+1}^{n} - (p_h)_{i,j}^{n} \Big] 
		&= (g_{2,h})_{i,j+1/2}^{n}
\end{split}
\end{equation}
where grid functions $\{p_h,v_{1,h},v_{2,h}\}$ approximate the continuum fields $\{p,v_1,v_2\}$ respectively, % for system \ref{eq:cont_system}, mainly,
\begin{equation*}
\begin{split}
	(p_h)_{i,j}^{n} &\approx p(ih,jh,n\Delta t), \\
	(v_{1,h})_{i+1/2,j}^{n+1/2} &\approx v_1((i+1/2)h,jh,(n+1/2)\Delta t),\\
	(v_{2,h})_{i,j+1/2}^{n+1/2} &\approx v_2(ih,(j+1/2)h,(n+1/2)\Delta t).
\end{split}
\end{equation*}
We emphasize that the velocity fields for finite difference scheme \ref{eq:acoustics_sgfd} are staggered with respect to spatial grids, more specifically each component is shifted by half a cell size in its respective axis.

Relating system \ref{eq:acoustics_sgfd} with \ref{eq:sg}, we see
\[
	(u_h)_{i,j} = (p_h)_{i,j}, \quad 
	(v_h)_{i,j} = \mat{(v_{1,h})_{i+1/2,j}\\ 
				  (v_{2,h})_{i,j+1/2} }.
\]
The space $H_{1,h}$ corresponds to the set of square summable scalar-valued
%MOD
functions on rectangular grids \hl{that satisfy the homogenous boundary condition}, and $H_{2,h}$ is the set of square summable $\mathbb R^2$-valued functions on half-cell shifted grids, both equipped with a discrete $L^2$ inner-products,  
\begin{equation*}
\begin{split}
	\langle u_h,\tilde u_h\rangle = h^2 \sum_{i,j} (u_h)_{i,j} (\tilde u_h)_{i,j} &\quad 	\text{for } u_h,\tilde u_h \in H_{1,h}\\
	\langle v_h, \tilde v_h \rangle = h^2 \sum_{i,j} (v_h)_{i,j} \cdot (\tilde v_h)_{i,j} &\quad
	\text{for } v_h,\tilde v_h \in H_{2,h}.
\end{split}
\end{equation*}
Inner-products and norms of the spaces $\mathcal H_1, \mathcal H_2, H_{1,h}, H_{2,h}$ are all denoted by $\langle\cdot,\cdot\rangle$ and $\|\cdot\|$ and interpreted given the context, unless otherwise specified.

Sampling operators $S_{1,h}$ and $S_{2,h}$ coincide with evaluating continuum functions on grid or shifted grid points accordingly,
\[
	(S_{1,h} u)_{i,j} = u(ih,jh), \quad 
	(S_{2,h} v)_{i,j} = \mat{v_{1}((i+1/2)h,jh)\\ 
				  v_{2}(ih,(j+1/2)h) }.
\]
Discretized coefficient operators $A_h$ and $B_h$ are given by,
\[
	(A_h u_h)_{i,j} = \frac{1}{\kappa(ih,jh)} (u_h)_{ij}, \quad
	(B_h v_h)_{i,j} = \mat{ \rho((i+1/2)h,jh) (v_{1,h})_{i+1/2,j} \\
					 \rho(ih,(j+1/2)h) (v_{2,h})_{i,j+1/2} }.
\]
Lastly, $P_h$ and $P_h^T$ correspond to second-order approximations of the gradient and the negative of the divergence respectively;
\begin{equation*}
\begin{split}
	(P_{h}(r) u_h)_{i,j} &= r \mat{ (u_h)_{i+1,j} - (u_h)_{i,j} \\
					     (u_h)_{i,j+1} - (u_h)_{i,j} }, \\
	(P_{h}(r)^T v_h)_{i,j} &= -r \Big[ (v_{1,h})_{i+1/2,j} - (v_{1,h})_{i-1/2,j} + (v_{2,h})_{i,j+1/2} - (v_{2,h})_{i,j-1/2} \Big]
\end{split}
\end{equation*}
with $r =  \Delta t/h$.
In general, the differential operator $P$ can be approximated by a family of central difference operators $P_h$ of even order $2p$ for $p\in\mathbb N$, resulting in the $2$-$2p$ staggered-grid finite difference schemes.
%ADDED
\hl{Similar to the continuum case, it can be shown that the $P_h$ for these type of staggered grid schemes indeed satisfy skew-adjoint relation} \ref{eq:skew_h}.
\hl{It follows that our weak convergence analysis applies to this example.}

\newpage
%%%%%%%%%%%%%%%%
\section{Numerical Tests}
%%%%%%%%%%%%%%%%

We have implemented the singular source approximation as discussed in the theory section
using the C++ packages \emph{IWave} and \emph{Rice Vector Library} (RVL);
\citep{GeoPros:11,RVL_TOMS}.  The IWave package %serves as
is a framework for finite difference solvers over uniform grids while
the RVL package provides a system of classes for expression of
gradient-based optimization algorithms over Hilbert spaces.  IWave and
RVL come together to form a modeling engine for seismic inversion and
migration. Implementation of multipole sources as RVL objects enables
straightforward composition with IWave solvers and inclusion in
inversion algorithms powered by RVL optimization code. Any other wave
equation solver wrapped in the appropriate RVL interfaces could be
coupled to the multipole source objects in the same way.

A convergence rate study is performed to corroborate theoretical results
pertaining to the accuracy of moment-consistent approximations to multipole sources.
In particular, our numerical experiments explore the semi-discrete error of staggered-grid finite difference solutions (time discretization errors are 
minimized by taking sufficiently small time steps).
We used the IWave implementation of staggered-grid finite difference schemes 
for the acoustic system \ref{eq:acoustics} \cite[]{Vir:84}, of order 2 in time and orders 2 and 4 in space - we
refer to these as the 2-2 and 2-4 order schemes respectively. In these
experiments we use only scalar (pressure) sources and pressure trace
data. Similar results are obtained with other choices. Boundary
conditions are of PML type, as described by \cite{Habashy:07},

The numerical experiments carried out concern multipoles in 2-D of the form
\[
	f(\mathbf x,t) = w(t) D^{\mathbf s} \delta(\mathbf x-\mathbf x^*),
\]
for $\mathbf s =(0,0),(0,1),(0,2)$.
The discretizations of $D^{\mathbf s}\delta$ are chosen as to achieve a target order of
convergence for the difference schemes, in most cases the nominal
spatial order ($q=2$ for the 2-2 scheme, $q=4$ for the 2-4 scheme). 
Again, the time step is fixed small enough that the time discretization error plays
essentially no role in the global error - it reflects truncation error
of the spatial derivative and the source approximation only.
Time-dependent function $w(t)$ is chosen to be a Ricker wavelet with peak 
frequency of $5Hz$, see figure \ref{fig:wlt-ricker}.

We approximate the convergence rate $R(\mathbf x)$ at a given location $\mathbf x$ via Richardson extrapolation,
\[
        R(\mathbf x) = \text{log}_2 \left( \frac{\| p_{h}(\mathbf x,\cdot)-p_{h/2}(\mathbf x,\cdot)\| }{\|p_{h/2}(\mathbf x,\cdot)-p_{h/4}(\mathbf x,\cdot)\| } \right)
\] 
where $p_h$ denotes the computed pressure field via finite difference using a grid size $h$ respectively.
The norm $\|\cdot\|$ is chosen to be 
\[
        \|p(\mathbf x,\cdot) \| := \sqrt{ \Delta t \sum_{k} |p(\mathbf x,t_k)|^2},
\]

Coordinates are aligned such that $\mathbf x = (z,x)$, 
where $z$ and $x$ refer to depth and horizontal distance respectively.  
%Similarly for 3-D, $\mathbf x=(z,x,y)$ where $y$ is
%measured along the remaining perpendicular direction.  
The computational domain consists of a $4km$ by $4km$ square medium
with a constant bulk modulus of $9 GPa$ and buoyancy of $1 cm^3/kg$,
thus a constant velocity of $3 km/s$.
%Receivers are placed in a 2-D grid covering the entire domain at every $40 m$ in each direction.
The source is placed slightly off of the center, $(z^*,x^*)=(-2003m,2003m)$, 
as to not coincide with a grid point for any of the computational uniform grids.
The following are some other specifications that apply to all tests carried out here:
\begin{itemize}
        \item total recording time is $1.5s$;
        \item spatial grid sizes $h=40m$, $h/2=20m$, $h/4=10m$;
        \item time step $\Delta t=0.5ms$;
\end{itemize}
Note that the coarsest grid cell size is $40m$ which implies that at least $15$ 
grid {\em points per wavelength (gpw)} at peak frequency of $5Hz$ or $5$ gpw at $15Hz$ 
(see figure \ref{fig:wlt-ricker-fft}), in all numerical experiments, 
thus minimizing the effects of grid dispersion on approximated convergence rates.

%Figures \ref{fig:model-geo-2D} and \ref{fig:model-geo-3D} describe the model 
%and source-receiver geometry for convergence rate tests in 2-D and 3-D respectively.
%The following are some other specifications that apply to all tests carried out here:
%\begin{itemize}
%        \item total recording time is $1.5s$;
%        \item spatial grid sizes $h=40m$, $h/2=20m$, $h/4=10m$;
%        \item time step $\Delta t=0.5ms$;
%\end{itemize}
%Note that the source location is chosen as to not coincide with a grid point for any of the computational uniform grids.
%Moreover, the coarsest grid cell size is $40m$, which implies at least $15$ grid {\em points per wavelength (gpw)}
%at peak frequency of $5Hz$ or $5$ gpw at $15Hz$ (see figure \ref{fig:wlt-ricker-fft}), in all numerical experiments, 
%thus minimizing the effects of grid dispersion on approximated convergence rates.

%%%%%%%%%%%%%%%%
\subsection{Results}


Figures \ref{fig:sq-test-s00q2-fd2-plotsnap} -- \ref{fig:sq-test-s02q2-fd2-plotsnap} 
plot a snapshot of the computed pressure field for different multipole sources.
Consistent with analytical formulas for the homogeneous unbounded medium case, 
the observed pressure field exhibited a polarity reversal (or lack of), a symmetry about the $x$-axis (i.e., $z=2003$), and an overall decrease in amplitude dependent on the multipole.

%Figures \ref{fig:test-s00q2-fd2-data10} -- \ref{fig:test-s02q2-fd2-data10} 
%and \ref{fig:test-s000q2-fd2-data10} -- \ref{fig:test-s020q2-fd2-data10} 
%plot pressure field traces versus horizontal location for different multipole sources in 2-D and 3-D respectively.
%Consistent with analytical formulas for the homogeneous unbounded medium case, 
%traces exhibited a polarity reversal (or lack of) and an overall decrease in amplitude dependent on 
%the multipole.

Convergence rates are plotted for the $\mathbf s=(0,0)$ case in figure 5, 
over the entire domain and at a particular depth of $z=2000$ in the left and right column plots respectively.
The first row of graphs shows results when using the 2-2 finite difference scheme and the second-order source approximation.
We observe that second order convergence is indeed achieved away from the source.
Similarly, fourth order convergence results are observed when using the 2-4 finite difference scheme and the fourth-order source approximation, i.e., second row in figure 5.
Lastly, the third row of plots demonstrates the negative effects of using a lower order source approximation relative to the spatial finite difference order, namely a second-order source approximation with a fourth-order method in space.
Clearly, fourth order rates are not achieved and moreover, there seems to be regions 
where rates dip below or above the expected second order.
Convergence results for $\mathbf s=(0,1)$ and $\mathbf s=(0,2)$ are given in figures 6 and 7 respectively with similar results.
The green line in figure \ref{fig:sq-test-s01q4-fd4-plots-wind} coincides with the nodal plane of the dipole source, that is, a line in the $zx$-plane where the source produces a null response. 



%Convergence rates versus receiver horizontal location for 2-D and 3-D cases, 
%when using the 2-2 finite difference scheme and an
%$O(h^2)$ approximation for the various multipoles, are plotted in figures  
%\ref{fig:test-s00q2-fd2-plots} -- \ref{fig:test-s02q2-fd2-plots} and 
%\ref{fig:test-s000q2-fd2-plots} -- \ref{fig:test-s020q2-fd2-plots}.
%Second-order convergence is observed for all cases in both the $\|\cdot\|_2$ (solid blue curve) and $\|\cdot\|_\infty$ (dashed red curve) norms away from the source location.
%Analogous results for the 2-4 finite difference scheme in conjunction with an $O(h^4)$ singular source approximation 
%are shown in figures 
%\ref{fig:test-s00q4-fd4-plots} -- \ref{fig:test-s02q4-fd4-plots} and 
%\ref{fig:test-s000q4-fd4-plots} -- \ref{fig:test-s020q4-fd4-plots},
%where the target quartic convergence rate is again achieved away from the source.

%Lastly, the last set of tests consider the case where a lower singular source approximation order
%is used, relative to the spatial finite difference order.
%In particular, figures 
%\ref{fig:test-s00q2-fd4-plots} -- \ref{fig:test-s02q2-fd4-plots} and 
%\ref{fig:test-s000q2-fd4-plots} -- \ref{fig:test-s020q2-fd4-plots}
%plot the convergence rates for the case when using the 2-4 
%finite difference scheme with an $O(h^2)$ source approximation.

%The onset distance of optimal order convergence is consistent with the
%theoretical prediction, that is, roughly $h(q+|{\bf s}|)$ for the coarses $h$ used in
%the convergence study ($h=40m$), which is the same as the trace spacing.

%Test 6 demonstates the negative effects of using the wrong multipole
%approximation order, namely $O(h^2)$ (i.e., tensor product of shifted hat functions)
%for a fourth order finite difference method in 2-D. 
%The theory makes no prediction in this case; apparently the
%convergence rate can be lower than would be the case with a difference
%scheme of lower order (second, in this case), even at a substantial
%distance from the source. We interpret this as meaning that 
%higher-order difference operators create larger truncation error when
%applied to discrete multipole sources of lower approximation
%order. This is not what one would expect of a discretely sampled smooth 
%right hand side, and is a matter worth further study.


%%%%%%%%%%%%%%%%%%%%%%
\section{Discussion}
%%%%%%%%%%%%%%%%%%%%%%

Numerical results presented here validate the accuracy of moment-consistent singular source discretizations for controlling the propagation of finite difference truncation error for multipole sources.  
In particular, optimal spatial convergence rates for the 2-2 and 2-4 staggered grid finite difference methods are achieved, point-wise in space away from source location, when the source approximation order is equal to that of the spatial finite difference approximation order.

The onset distance of optimal order convergence is consistent with the support of the source approximation for the coarsest grid used in the convergence study, that is, $h(q+|{\bf s}|)$ with $h=40m$.
The erratic behavior of numerical convergence rates within the support of source approximations can be explained in part by how convergence rates were approximated.
Namely, numerical solutions over consecutively refined grids are essentially compared, though over the coarsest grid.
It follows that the support of $f_{h}$, the source approximation with respect to grid with cell size $h$, will contain the support of $f_{h/2}$ and a region in the $h$-grid outside the direct influence of $f_{h/2}$, resulting in irregular convergence rates.

Work by \cite{Petersson:2016} demonstrated that optimal convergence rates can be achieved if the source discretization satisfied moment as well as smoothness conditions.
Again, their work was centered around central difference approximations of the 1-D advection equation.
As our numerical tests demonstrate, smoothness conditions were not necessary here, which is attributed to the fact that we use staggered-grid finite difference schemes. 
In particular, the central finite difference operator used in \cite{Petersson:2016} have an associated grid spacing of $2h$ with respect to an $h$-sized grid, coupled with a source approximation of narrow support, i.e., $2\epsilon=(q+|\mathbf s|)h$, results in the triggering of spurious modes unresolved by the numerical solution.
The finite difference operators used here however have a grid spacing of $h$ given that they approximate derivatives over staggered grids.
Thus, in essence our narrow source approximations ``seem'' twice as wide, and therefore smoother, from the point of view of the difference operator $P_h$.
For this reason, smoothness conditions are not necessary for staggered-grid finite difference schemes.






%, which is the same as the trace spacing.
%For example, consider test 12 (figure \ref{fig:test12-plots}) 
%where $D^{\bf s}\delta({\bf x}-{\bf x}^*)$ with $|{\bf s}|=2$ is replaced by
%a fourth order approximation in 3-D; the source stencil had a diameter of about
%$q+|{\bf s}|=6$ traces, equivalent to 6 grid points in the $x_2$-direction for the coarsest
%($40m$) mesh.

%the observed region of optimal order convergence is also roughly as predicted.
%We note that accuracy claims made above are in relation to receiver positions away from the source stencil.
%Source approximations with large source stencils are problematic since they produce large regions where the numerical solution is not guaranteed to be as accurate.








%%In particular, we observe good agreement between finite difference and
%%analytical solutions, see figures \ref{fig:comp2D-00} --
%%\ref{fig:comp2D-01} and figures \ref{fig:comp3D-000} --
%%\ref{fig:comp3D-020}; this is to be expected given since our choice of
%%grid cell size $h=40m$, or equivalently $15$ grid points per wavelength at peak frequency of $5Hz$, 
%%and time step size $\Delta t=0.5ms$.
%
%
%%MJB edit
%As mentioned in our discussion of singular source approximation theory,
%current literature fails to provide analysis directly applicable to the solution of 
%acoustodynamics (equations \ref{eq:acous}) or elastodynamics (equations \ref{eq:elas}) with multipole sources.
%The most complete account, as far as we are aware, of the analysis of approximations 
%to PDEs with singular sources is given by \cite{hoss:16} in the context
%of regularizations of the delta distribution.
%\citeauthor{hoss:16} developed a unified framework for analyzing approximation errors 
%that can be readily applied to any numerical scheme.
%Their work in particular addressed two fundamental questions: 
%(i) What form of convergence should be used to examine $f^h\to f$? 
%That is, how does our source approximations converge to multipoles as distributions?
%(ii) What form of convergence should be used to examine $u^h\to u$, 
%convergence of numerical solution $u^h$ with source approximation $f^h$ to true
%solution $u$ with source $f$?
%Authors showed that convergence of $f^h\to f$ in the distribution sense (weak-* topology),
%and in some weighted Sobolev norm, can be achieved at a desired rate if a set of 
%{\em continuous moment conditions} are satisfied by the approximation $f^h$, 
%analogous to the discrete moment conditions discussed here.
%Furthermore, they discussed answers to question (ii) and the interplay of approximation $f^h$ with
%the convergence of $u^h\to u$ for protopical elliptic and hyperbolic PDEs.
%
%%MJB edit
%Numerical results presented here, and by \cite{Petersson:2010}, 
%help motivate the use of approximations to multipoles as suggested by
%\cite{Walden:1999} and \cite{TorEng:04}.
%Moreover, we believe it is possible to provide a formal error analysis of our conjecture 
%by building upon the theoretical framework laid out by \cite{hoss:16}.
%%This would entail answering the following questions:
%%Can the theory be extended to distributions of the form $D^{\bf s}\delta$?
%%If so, do source approximations that satisfy discrete moments (equation \ref{eq:qmom})
%%also satisfy some version of the continuous moments as given by \cite{hoss:16}?
%%Question (ii) and the interplay of approximation $f^h$ with
%This would entail analyzing the interplay of approximations $f^h$ with
%the convergence of $u^h\to u$ for our particular PDEs and choice of 
%numerical scheme.
%
%The RVL-based structure of our MPS framework does not really get a
%solid workout in this study. Inspection of the code and scripts
%shows that the framework is a convenient environment for 
%implementing the numerical examples presented in the last
%section. However the capabilities of this mode of code organization
%emerge much more clearly in inversion applications. For example, we have shown in
%another place REFERENCE??? that a choice of norm in an MPS space, that
%reflects the effect of the spatial delta derivatives on temporal
%frequency, can dramatically accelerate the convergence of
%source recovery by Krylov subspace iteration. For this, it was only
%necessary to pass the {\tt RVL::LinOpValOp} representing the MPS-to-data 
%trace map, and an {\tt RVL:LinearOp}
%representing the Gram operator of the norm,  to an RVL implementation of the
%preconditioned conjugate gradient method, used completely without
%alteration. It will also be possible to incorporate the MPS based
%operator into the RVL implementation of variable projection method
%\cite[]{LiRickettAbubakar:13} for joint source-medium inversion
%without any change either to the MPS code or to the RVL variable
%projection algorithm.
%
%%We would be remised if we did not point out some of the attractive features present in this singular source approximation.
%%For one, incorporating source approximations is rather straight-forward and minimaly invasive to the finite difference implementation, a matter of cooking up the correct collection of RHS sources and injecting them at proper grid points.
%%Moreover, the relationship between our source parameterization via MPS coefficients and RHS source-terms for finite difference solvers is a linear one, aptly implemented by our \texttt{MPS\_to\_RHS} operator.
%%Ultimately the singular source approximation implemented in our framework preserves linearity of the MPS-to-data map in a natural way that will be crucial for source inversion.
%
%We have not addressed two important issues in the work reported here.
%The first is the order of multipole necessary to represent a given
%degree of source anisotropy. Point support is an idealization: active source regions may
%be small or comparable to a wavelength in spatial
%extent. \cite{SantosaSymes:00} showed that multipole approximation
%(equation \ref{eq:MPSappx}) of acoustic sources vanishing outside
%a small region of space
%exhibits a threshhold effect, in that the error in the
%resulting acoustic fields drops abruptly as the length of the series
%($N$ in the expression \ref{eq:MPSappx})
%is increased past a critical value.
%This threshhold in the number of terms necessary for accurate
%approximation depends on the size of the active region and a measure of energy
%output relative to energy input (related to the degree of
%anisotropy).
%The analysis in \cite[]{SantosaSymes:00} pertains to acoustics,
%% the leap to linear elasticity is rather justifiable. (REALLY?)
%however we expect similar results to hold for linear elastodynamics.
%Earthquake seismology has 
%%indeed benefitted from the applicability of
%long used multipole source approximations in elastic media 
%%particularly for the determination of
%to describe earthquake mechanisms. 
%The %basic earthquake representation is  
%%what is referred to as 
%%the 
%seismic moment tensor
%%which is 
%represents earthquake source mechanisms by a combinations of point-dipole body forces, that is, a
%first-order vector-valued multipole \cite[]{Backus:1976a,Shearer:2009}.
%Moment tensors of higher order (that is, higher order multipole series) have been shown to be important in cases where finiteness of the source is of issue, that is the fault size is comparable to propagating wavelengths \cite[]{Stump:1982}.
%
%%WWSAccurate source representation and estimation is crucial to the seismic inversion problem, whos focus is the recovery of medium parameters.
%%The moderate success of
%%Joint determination of medium and source parameters
%% in 
%%for improving seismic inversion 
%%has motivated 
%%motivates our work on MPS representation of seismic sources.
%A second natural question is: how much source anisotropy is really
%necessary to fit field data well?
%%Since source radation pattern anisotropy has a first-order effect on
%%seismic amplitudes, it is natural to expect that 
%%anisotropic source representation, such as multipole series, would be
%%an important ingredient in seismic inversion.
%\cite{SymMink:97} 
%%motivate the need for joint medium-source parameter inversion and a source representation that accounts for anisotropy in the context of plane-wave viscoelastic modeling of marine reflection data.
%%Their results 
%show that inverting marine reflection data for multipole source
%parameters together with a layered viscoelastic model,% as oppose to
%%using a given isotropic modeled source or even inverting for an
%%isotropic source, %allowed them to account 
%resulted in fitting 25\% more of the data than was possible with any
%isotropic source, and allowed
%%were even able to achieve a dramatic 
%90\% data fit to the target portion of the data. Moreover, recovered
%p-wave and s-wave impedance parameters matched closely the expected
%seismic-lithologic signature of the gas sand only when %jointly
%                                %estimating for 
%viscoelastic model and anisotropic source parameters were simultaneously esitmated.
%The order of multipole required to achieve this degree of data fit and
%well log tie was $N=6$. Obviously the necessary order depends on many
%factors, and varies from survey to survey. The framework we have
%provided here offers a platform in which to develop an approach to
%source anisotropy estimation as part of the overall inversion process.

%%%%%%%%%%%%%%%%%%%%%%
\section{Conclusion}
%%%%%%%%%%%%%%%%%%%%%

In this paper we have covered the singular source approximation theory
based on moment conditions, which essentially has approximations mimic 
the behavior of the target distribution $D^{\mathbf s}\delta(\mathbf x-\mathbf x^*)$ on polynomials.
Moreover, we give explicit forms of source approximations with narrow support,
based on the discrete moment conditions in the context of finite difference solvers:
diameter $2\epsilon = (q+|\mathbf s|) h$ for a multipole of order $|\mathbf s|$
over a grid of size $h$.
As a new result, we connect the discrete and continuum singular source approximation
theory by proving that continuum functions generated from the discrete moment conditions 
indeed satisfy the continuum moment conditions.

Our main contribution was the development of a weak convergence theory that is applicable to a large set of wave propagation problems (including acoustics and elasticity in first-order form) solved via a family of staggered-grid finite difference schemes.%MOD, larger than what is reported in the current literature.
Posing the convergence mode of numerical solutions in terms of weak convergence was indeed a natural choice given that source terms are derivatives of the Dirac delta function, that is, distributions.
%ADDED
\hl{
The weak convergence theory relied on the structure of the the general continuum problem} \ref{eq:cont_system} \hl{and the ability for its discretization to preserve this structure, i.e, coefficient operators ($A,B,A_h,$ and $B_h$) are symmetric and bounded and the differential operators $P$ and $P_h$ satisfy a skew-adjoint relation.
The general staggered finite difference scheme} \ref{eq:sg} \hl{is also assumed to be staggered in time, in particular a central difference approximation of the time derivative was applied.
Given the aforementioned assumptions on the continuum problem and its discretization, energy estimates along with the singular source approximation theory (theorem }\ref{thm:discweakconv} \hl{) are the key ingredients that gives us our main weak convergence result, theorem} \ref{thm:conv}.
%END ADDED
Numerical results, however, give evidence of stronger convergence, namely 
optimal convergence rates given by numerical scheme under smooth conditions,
point-wise away from source location for appropriate source discretizations and
in particular for multipoles of order $|\mathbf s |=0,1$ and $2$.






%%%%%%%%%
\section{Acknowledgements}

We are grateful to the sponsors of The Rice Inversion Project for their long-term support,
and to the Rice Graduate Education for Minorities (RGEM) and XSEDE scholarship programs for their support of M. Bencomo's Ph.D. research.
This material is also based upon work supported by the National Science Foundation under Grant No. DMS-1439786 while the author was in residence at the Institute for Computational and Experimental Research in Mathematics in Providence, RI, during the Fall 2017 semester.




%%%%%%%%%
% PLOTS 
%%%%%%%%%
\inputdir{project}

\multiplot{6}{ssappx-s0q2,ssappx-s0q4,ssappx-s1q2,ssappx-s1q4,ssappx-s2q2,ssappx-s2q4}{width=0.33\textwidth}{Plots of $\eta^\epsilon(x)$, 1-D approximations to $D^s\delta(x)$ with $h=1$. (a) $s=0$ and $q=2$, (b) $s=0$ and $q=4$, (c) $s=1$ and $q=2$, (d) $s=1$ and $q=4$, (e) $s=2$ and $q=2$, (f) $s=2$ and $q=4$.}

\multiplot{6}{ssappx-s00q2,ssappx-s00q4,ssappx-s01q2,ssappx-s01q4,ssappx-s02q2,ssappx-s02q4}{width=0.33\textwidth}{Plots of $\eta^\epsilon(\mathbf x)$, 2-D approximations to $D^{\mathbf s}\delta(\mathbf x)$ with $\mathbf h=(1,1)$. (a) $\mathbf s=(0,0)$ and $q=2$, (b) $\mathbf s=(0,0)$ and $q=4$, (c) $\mathbf s=(0,1)$ and $q=2$, (d) $\mathbf s=(0,1)$ and $q=4$, (e) $\mathbf s=(0,2)$ and $q=2$, (f) $\mathbf s=(0,2)$ and $q=4$.}

\multiplot{2}{wlt-ricker,wlt-ricker-fft}{width=0.6\textwidth}{Ricker wavelet with peak frequency $5Hz$: (a) time plot, (b) power spectrum plot.}

%\plot{model-geo-2D}{width=0.8\textwidth}{Model and source-receiver geometry specifications for 2-D convergence rate tests; constant density ($1kg/m^3$) and velocity ($3km/s$), receiver positions $(z_r,x_r)=(-200m,0m:40m:6000m)$, source position $(z^*,x^*)=(-203m,3003m)$.}

%\plot{model-geo-3D}{width=0.8\textwidth}{Model and source-receiver geometry specifications for 3-D convergence rate tests; constant density ($\rho=1kg/m^3$) and velocity ($c=3km/s$), receiver positions $(z_r,x_r,y_r)=(-200m,0m:40m:6000m,100m)$, source position $(z^*,x^*,y^*)=(-203m,3003m,103m)$.}


%%%%%%%
%% 2-D %
%%%%%%%

\multiplot{3}{sq-test-s00q2-fd2-plotsnap,sq-test-s01q2-fd2-plotsnap,sq-test-s02q2-fd2-plotsnap}{width=0.48\textwidth}{Snapshot of pressure field ($t=0.748$ sec) computed using 2-2 finite difference scheme ($h=10m, \Delta t=0.5$ ms) and second-order approximation of multipole source $D^{\bf s}\delta({\bf x}-{\bf x^*})$: (a) ${\bf s}=(0,0)$, (b) ${\bf s}=(0,1)$, and (c) ${\bf s}=(0,2)$.}

\multiplot{6}{sq-test-s00q2-fd2-plots,sq-test-s00q2-fd2-plots-wind,sq-test-s00q4-fd4-plots,sq-test-s00q4-fd4-plots-wind,sq-test-s00q2-fd4-plots,sq-test-s00q2-fd4-plots-wind}
{width=0.33\textwidth}{Convergence rate results for ${\mathbf s}=(0,0)$. Using 2-2 FD scheme and second-order source approximation (a),  2-4 FD scheme and fourth-order source approximation (c), 2-4 FD scheme and second-order source approximation (e). Plots on right column correspond to convergence rates at $z=2000$, related to the left column.}

\multiplot{6}{sq-test-s01q2-fd2-plots,sq-test-s01q2-fd2-plots-wind,sq-test-s01q4-fd4-plots,sq-test-s01q4-fd4-plots-wind,sq-test-s01q2-fd4-plots,sq-test-s01q2-fd4-plots-wind}
{width=0.33\textwidth}{Convergence rate results for ${\mathbf s}=(0,1)$. Using 2-2 FD scheme and second-order source approximation (a),  2-4 FD scheme and fourth-order source approximation (c), 2-4 FD scheme and second-order source approximation (e). Plots on right column correspond to convergence rates at $z=2000$, related to the left column.}

\multiplot{6}{sq-test-s02q2-fd2-plots,sq-test-s02q2-fd2-plots-wind,sq-test-s02q4-fd4-plots,sq-test-s02q4-fd4-plots-wind,sq-test-s02q2-fd4-plots,sq-test-s02q2-fd4-plots-wind}
{width=0.33\textwidth}{Convergence rate results for ${\mathbf s}=(0,2)$. Using 2-2 FD scheme and second-order source approximation (a),  2-4 FD scheme and fourth-order source approximation (c), 2-4 FD scheme and second-order source approximation (e). Plots on right column correspond to convergence rates at $z=2000$, related to the left column.}

%\multiplot{3}{test-s00q2-fd2-data10,test-s01q2-fd2-data10,test-s02q2-fd2-data10}{width=0.48\textwidth}{2-D pressure field data using 2-2 finite difference scheme ($h=10m$) and $O(h^2)$ approximation of multipole source $D^{\bf s}\delta({\bf x}-{\bf x^*})$: ${\bf s}=(0,0)$ in (a), ${\bf s}=(0,1)$ in (b), and ${\bf s}=(0,2)$ in (c).}
%
%\multiplot{3}{test-s00q2-fd2-plots,test-s01q2-fd2-plots,test-s02q2-fd2-plots}{width=0.48\textwidth}{2-D convergence rate results using 2-2 finite difference scheme and $O(h^2)$ approximation for multipole source $D^{\bf s}\delta({\bf x}-{\bf x^*})$. $L^2$- (solid blue) and $L^{\infty}$- (dashed red) norms. (a) ${\bf s} = (0,0)$, (b) ${\bf s}= (0,1)$, and (c) ${\bf s}= (0,2)$.}
%
%\multiplot{3}{test-s00q4-fd4-plots,test-s01q4-fd4-plots,test-s02q4-fd4-plots}{width=0.48\textwidth}{2-D convergence rate results using 2-4 finite difference scheme and $O(h^4)$ approximation for multipole source $D^{\bf s}\delta({\bf x}-{\bf x^*})$. $L^2$- (solid blue) and $L^{\infty}$- (dashed red) norms. (a) ${\bf s} = (0,0)$, (b) ${\bf s}= (0,1)$, and (c) ${\bf s}= (0,2)$.}
%
%\multiplot{3}{test-s00q2-fd4-plots,test-s01q2-fd4-plots,test-s02q2-fd4-plots}{width=0.48\textwidth}{2-D convergence rate results using 2-4 finite difference scheme and $O(h^2)$ approximation for multipole source $D^{\bf s}\delta({\bf x}-{\bf x^*})$. $L^2$- (solid blue) and $L^{\infty}$- (dashed red) norms. (a) ${\bf s} = (0,0)$, (b) ${\bf s}= (0,1)$, and (c) ${\bf s}= (0,2)$.}


%%%%%%
% 3-D %
%%%%%%

%\multiplot{3}{test-s000q2-fd2-data10,test-s010q2-fd2-data10,test-s020q2-fd2-data10}{width=0.48\textwidth}{3-D pressure field data using 2-2 finite difference scheme ($h=10m$) and $O(h^2)$ approximation of multipole sources $D^{\bf s}\delta({\bf x}-{\bf x^*})$: ${\bf s} = (0,0,0)$ in (a), ${\bf s}= (0,1,0)$ in (b), and ${\bf s}=(0,2,0)$ in (c).}
%
%\multiplot{3}{test-s000q2-fd2-plots,test-s010q2-fd2-plots,test-s020q2-fd2-plots}{width=0.48\textwidth}{3-D convergence rate results using 2-2 finite difference scheme and $O(h^2)$ approximation for multipole source $D^{\mathbf s}\delta(\mathbf x-\mathbf x^*)$. $L^2$- (solid blue) and $L^{\infty}$- (dashed red) norms. (a) ${\bf s}=(0,0,0)$, (b) ${\bf s}=(0,1,0)$, and (c) ${\bf s}=(0,2,0)$.}
%
%\multiplot{3}{test-s000q4-fd4-plots,test-s010q4-fd4-plots,test-s020q4-fd4-plots}{width=0.48\textwidth}{3-D convergence rate results using 2-4 finite difference scheme and $O(h^4)$ approximation for multipole source $D^{\mathbf s}\delta(\mathbf x-\mathbf x^*)$. $L^2$- (solid blue) and $L^{\infty}$- (dashed red) norms. (a) ${\bf s}=(0,0,0)$, (b) ${\bf s}=(0,1,0)$, and (c) ${\bf s}=(0,2,0)$.}
%
%\multiplot{3}{test-s000q2-fd4-plots,test-s010q2-fd4-plots,test-s020q2-fd4-plots}{width=0.48\textwidth}{3-D convergence rate results using 2-4 finite difference scheme and $O(h^2)$ approximation for multipole source $D^{\mathbf s}\delta(\mathbf x-\mathbf x^*)$. $L^2$- (solid blue) and $L^{\infty}$- (dashed red) norms. (a) ${\bf s}=(0,0,0)$, (b) ${\bf s}=(0,1,0)$, and (c) ${\bf s}=(0,2,0)$.}



\bibliographystyle{seg}
\bibliography{../../bib/masterref}

