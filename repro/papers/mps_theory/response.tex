\title{Responses to reviewer's comments on ``Discretization of Multipole Sources in a Finite Difference Setting for Wave Propagation Problems''}
\date{}
\address{
        \footnotemark[1]Institute for Computational and Experimental Research in Mathematics,\\ Brown University,\\ Providence, RI 02912 USA\\
        \footnotemark[2]The Rice Inversion Project,\\ Rice University,\\ Houston, TX
        77005-1892 USA
}
\author{Mario J. Bencomo\footnotemark[1] and William Symes\footnotemark[2]}

\righthead{Responses}

\maketitle
\parskip 12pt
\section{Introduction}
We thank the reviewers for insightful and constructive comments. Their insights led to many changes in this manuscript. We trust that the revised manuscript may answer the questions posed, and generally provide a much improved overview of our work.

\section{Reviewer 1}

Comment: The article presents an extension to the theory on moment conditions for finite differences to multipole point sources. The moment conditions are first derived in a continuous setting followed by the derivation of a discrete set of moment conditions. Further, a convergence theory for staggered grid finite-difference schemes on a family of first order systems of hyperbolic PDEs with multipole source terms is presented. The authors conclude the article by testing the convergence by numerical experiments on the acoustic wave equation.

In the introduction the authors claim that the presented work is an extension of the theory on point source discretization in finite differences. The moment conditions in the first section appears as a valuable extension to current theory, but the convergence analysis is restricted to a specific formulation of hyperbolic PDEs solved with staggered grid finite differences. My interpretation is that these results are not general for all stable finite difference discretisations. If the theory is more general than my interpretation, this must be stated more clearly. Otherwise, the reason for the restriction to staggered grid-finite differences should be explained.

Response: The introduction has been rewritten to emphasize the focus on staggered grid discretizations for symmetric hyperbolic systems of ``div-grad'' form.  The latter does not seem a severe constraint to us, as several physically important wave theories can be cast in that form, and others (e.g. several forms of viscoelasticity) can be viewed lower-order perturbations of systems of this type. The focus on staggered grid schemes is a more serious limitation. It is conceivable that results like those proven here could be established for other stable finite difference schemes, or for finite element schemes of various types, however such generalization is beyond the scope of this paper. The key ingredient in our arguments is the discrete energy estimate proved in the theory section. However as is well-known such an estimate is only a sufficient condition for stability, hence convergence.

%I think the for the most part, the paper does a good job in specifying that the analysis and results are limited to staggered-grid FD schemes. Hopefully this is more apparent now that the theory section has been cleaned up. I added some extra sentences in the conclusion section to emphasize the integral components of the theory. The motivation for using staggered-grid methods was briefly mentioned in the intro (page 4, bottom paragraph). Not sure if it is worth mentioning again. Would it be a good idea to “staggered-grid” somewhere in the title? 

Comment: In segments of the article, the line of argument is difficult to follow. For example, the subsection "Examples" on p. 23 appears out of context and the subsection is not well written. In the introduction of this subsection numerical convergence tests in the "next section" are mentioned but the next section does not contain such tests. Figure references are also missing in this section. This subsection must be rewritten and perhaps moved to the section "numerical tests and results".

Response: This section has been cleaned up, and figure references added. 

Comment: The structure of the convergence theory is difficult to follow since the more general system which is actually analyzed (21) is mixed with an example (23). I think the example causes confusion about what is actually proven. I believe that the example would fit better in the section "numerical tests and results”.

Response: The convergence theory section has been extensively reworked. All references to the acoustic case have been moved from the theory section to a subsection titled “Acoustic Case” at the end of the respective section.

Comment: The way of presenting the convergence result is non-standard. I suggest that you perform a convergence study with several grid refinements presented similar to figure 7 in Petterson et al. 2016. If if this is not added, a motivation of the current form of convergence test is needed.

Response: Convergence rates are presented in this manner to demonstrate their spatial dependence, in particular, relative to source location in 2D space. For example, this presentation highlights the degradation in convergence rate far from the source location when using source approximations with the wrong approximation order. The convergence rates reported here are equivalent to the slopes of typical convergence figures in log-log plots.

Comment: In figure 6 (a) a green line which corresponds to third order convergence appears vertically in the figure. In figure 7 (a) one can also detect some green areas surrounding the source. Can you comment on this locally lower convergence rate?

Response: I think the author mean figures 6(c) and 7(c). The green line in fig 6(c) coincides with the nodal plane of the dipole. 


\section{Reviewer 3:} 

Comment: Your analysis is restricted to the Cauchy problem (periodic case), but that is not stated anywhere (please state).  Many applications require treatment of point sources on the boundary. In this case, the moment conditions need to be modified to account for the quadrature rule used by the difference scheme, as shown in Petersson 2016.  

Response: Quite correct, and intentional. We do not limit the analysis to the periodic case, but to the full-space case, assuming always that solutions have compact support in space for each time.  It is certainly possible to extend the results to some combinations of boundary conditions and interior and boundary schemes, so long as the latter admit energy estimates. We know of no such staggered grid boundary schemes for the elastic free surface of order higher than two, for example (though we are aware of widely used approximations, such as Robertsson's vacuum scheme). Also we do not study sources supported on the boundary. All of these other questions are interesting, but beyond the scope of the paper. We include explicit statements to that effect in the rewritten introduction.

%Boundary conditions are not required in the analysis, except implicitly in satisfying the skew-adjoint relations in the continuum and discrete systems. In the numerical examples, we use PML boundaries to simulate an unbounded domain, though we implement reflective boundary conditions at the computational domain, consistent with the theory. Source discretization near boundaries is not considered here. Not sure what could be added to the paper.  

Comment: Before (16) on page 19, invervals $\rightarrow$ intervals 

Response: fixed

Comment: On page 33 "corollary above", what corollary? do you mean theorem? 

Response: fixed

Comment: Page 36, did you mean $\tilde{u}$ and not $\tilde{p}$?  

Response: fixed.  

Comment: Page 37, does $\pm$ indicate addition and subtraction?  

Response: Yes. 

Comment:  You rely on $P_h = -P_h^T$ for your proofs, but this property is never stated (only for the continuous problem).  It is now mentioned explicitly in the presentation of the general continuum problem and its discretization by the skew-adjoint relations (23) and (25) respectively.  

Response: Actually this statement (that $P_h$ must be skew adjoint) was an error on our part, and has been removed entirely from the rewrite - the acoustic example, for instance, does not have this property. The matrix operator $\left(\begin{array}{cc}0 & P^T\\-P & 0\end{array}\right)$ is skew adjoint, for any P of the type considered here (continuous or discrete). 

Comment: Page 38, How can you bound ${\cal E}_2$ using $L^2$-error estimates of finite difference solutions for smooth problems, when the solution to the problem is not smooth? 

Response: $(\tilde{u}^{n-1/2},\tilde{v}^n)$ solves a FD approximation to the adjoint problem with smooth data $\tilde{f},\tilde{g}$, hence satisfies an energy estimate. The point of using the adjoint formulation is that it replaces inner products of  singular solutions and smooth test functions with inner products of singular sources and smooth solutions with the test functions serving as sources.

Comment: At no point in your proof do you make use of the fact that you have a staggered finite difference discretization. 

Response: This is an astonishing statement, in view of the formulation of the abstract staggered grid scheme (equation 24) and its use througout the rest of the paper. We are at a loss as to what the reviewer means.

Comment: However, if the discretization is replaced by the collocated grid, central finite discretizations (e.g., -1/2h, 0,1/2h), then smoothness conditions are necessary for convergence. To prove convergence without imposing smoothness conditions, I believe, at some point you have to invoke the property that the Nyquist mode $(-1)^j$ does not lie in the nullspace of the staggered grid difference operators. 

Response: if the Nyquist mode lay in the nullspace of the finite difference propagator, then an energy estimate of the type we prove would be impossible. Conversely, the energy estimate shows that no such phenomenon can occur, provided that the CFL criterion is satisfied.

%We can apply L_2-error estimates to \mathbcal{E}_2 since \delta{\tilde u} and \delta{\tilde v} refer to the finite difference error for a problem with smooth sources (\tilde f, and \tilde g), i.e., see equation (35).  We did most of the heavy lifting for the proof of theorem 8 (weak convergence) a bit ahead of the actual theorem.  May have contributed to some confusion for the reviewer?  

Comment: Page 42, how do you define the minimum number of grid points per wavelength? Usually, the frequency at 5\% of
peak amplitude in the spectral domain is defined as $f_{max}$ and then the minimum wavelength is estimated by $ \lambda_{min} = c_{min}/f_{max}$, where $c_{min}$ is the minimum wavespeed. For Ricker wavelets, one obtains $f_{max} = 2.5 f_p$ where $f_p$ is the peak frequency.  

%Dr. Symes, not sure how to handle this? Would it be sufficient to reference a paper stating the rule of thumb of 10 grid points per wavelength for 2-2 and 5 grid points per wavelength for 2-4? If so, could you remind me of the reference?  

Response: We have removed this paragraph. In our study of convergence rates, gridpoints per wavelength is not a relevant concept, though it might be used to encapsulate the results of such a study.

Comment: Page 43, You state "... our narrow source approximations seem twice as wide and therefore smoother, from the point of view of the difference operator $P_h$. For this reason, smoothness conditions are not necessary for staggeredgrid finite difference schemes". I do not follow this argument. Why would stencil width imply smoothness? As previously mentioned, smoothness conditions are not necessary for convergence when the Nyquist mode does not lie in the nullspace of the difference operator.  

%Response: The stencil width and smoothness comment was referring to smoothness in Fourier space. For example, the Fourier transform of a wider the hat function is a narrower $sinc^2$ function, thus the contribution from spurious modes is diminished. Incidentally, source discretizations presented by Petersson 2016 that satisfy m-moment and ssmoothness conditions (with m=s) coincide with approximations that satisfy m-moment conditions and have two times the stencil width relative to the finite difference operator. These are the same source approximations used here.

Response: we have removed this discussion, as it is redundant. We have shown in the theory section that for the class of schemes considered here, the moment conditions imply convergence for multipole rhs (in the weak sense, of course), without need to appeal to any additional conditions such as the smoothness condition of  Petersson et al.

Comment: Page 44, Instead of saying convergence theory "applicable to a large set of wave propagation problems", why not
simply say "wave equations"? In other places you say "symmetric hyperbolic systems, which also can change to
"wave equations" to be more specific.

%Dr. Symes, any thoughts on this?

Response: We are not sure why the referee makes these suggestions. Symmetric hyperbolic system are first order systems of PDEs with wave-like solutions. They do not encompass all wave equations (or even the ``classic'' wave equation), but most wave propagation dynamics can be (re)formulated in terms of such systems. So as we say we are mystified, and with apologies for being so obtuse, have left the prose in question intact.
