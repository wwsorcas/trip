\title{Introduction}

\section{Goals}
This book will explore some of the core concepts and methods of active-source seismic imaging and inversion from a mathematical point of view. The content arose from my efforts (and those of many others) to explain {\em precisely} the origin in the physics of waves of many clever and effective processing algorithms, widely used in industrial and scientific practice. Evidently, either such explanation is possible, or our understanding of seismic waves is lacking in important respects. To some extent this project has been successful: wave theory - linear elastodynamics -  actually does explain {\em precisely} why some important seismic imaging and inversion methods work. Even better, the theoretical concepts required to understand modern imaging methods are themselves (relatively) modern - the necessary mathematics has developed in parallel with (and largely independent of) engineering/scientific methodology. Rich and beautiful connections have been drawn in both directions: imaging algorithms based on data-driven insight have inspired new mathematics, which has returned the favor by suggesting improved (or, in some cases, brand-new) imaging algorithms.


Active-source seismic surveys use various human-made and -controlled devices, or {\em sources}, to stimulate stress/displacement (sound, or more properly seismic) waves propagating through the earth, and records the resulting displacement or stress fields at positions on or near the earth's surface. The recordings encode the acoustic structure of the earth through which the waves has passed on their journey from sources to recording devices, or {\em receivers}. Seismic imaging aims to decode the earth's structure, or at least those features of it to which sound waves are sensitive, from the recorded data.

This branch of earth science originated roughly one hundred years ago, in prospecting for ore bodies before the first World War, and in artillery ranging methods developed during it. Since the early successes in locating petroleum reservoirs along the Gulf of Mexico coast in the 1920's, active-source seismology has become the primary exploration tool of the oil industry, and has also supplied useful methodology to civil and environmental engineering. The topics that form the focus of this book, while at least implicit in the early development of the subject, flowered after the introduction of digital computing in the exploration industry, in the 50's and early 60's. Each subsequent twenty-year period saw the development of a distinct collection of topics:
\begin{itemize}
\item 1960's through 1970's: trace-by-trace processing: deconvolution, normal moveout, stack, post-stack migration (``classic imaging'');
\item 1980's through 1990's: pre-stack and reverse-time migration (``linearized inversion'');
\item last 20 years: full waveform inversion (``nonlinear inversion'').
\end{itemize}
This book consists of three parts, each devoted to one of the topics listed above.
The list above is full of jargon; one of the main goals of this book is to explain each term above (and many others used in this subject) in terms of the mathematics of waves. The story told here is not comprehensive by any means. However, in the pages to follow you will find a few complete accounts, starting in basic linear elastodynamics and arriving at some practical and important imaging and inversion algorithms.  

These notes owe a great deal, in structure, content, and spirit, to Jon Claerbout. In preparing the text, I re-read the introduction to Jon's book {\em Image Estimation by Example}. In Jon's terms, you might say that this course is about ``the theory of the similarity between theory and practice'' (an emphasis that would clearly not appeal to Jon!). In particular, it is about both theory and practice, with practice represented by computational examples illustrating the theory, using open source software packages and public domain data, both synthetic and field.

These examples take the form of {\em reproducible computational research}, another pioneering concept of Claerbout's. The reader can rebuild the results from raw data and code by executing a few simple commands, included as auxiliary material with the book. Most importantly, the reader can also modify the examples to explore beyond the content of the text.
This computational infrastructure is described in detail in Appendix A, and rests on three open source software packages (Seismic Unix, Madagascar, and TRIP), which in turn depend on Python, C/C++, and Unix. Madagascar provides the computational ``glue'' tying together tex, data, and code. The chapters form a directory tree of Madagascar papers, each one with examples and resulting figures produced from Python scripts in a project subdirectory. The participants can rebuild every one of these papers from scratch, and the scripts can (and should!) serve as departure points for additional projects.


\section{Overview}

\begin{itemize}
\item {\bf Part 1: Basic Imaging.} Starts with an imaging exercise, using 2D marine survey data: RMS velocity estimation, NMO correction of CMP gathers, stack, and post-stack time and depth migration. The object of this part of the course is to identify the origin in wave theory of each process used in this exercise. Two approximations to solution of the wave equation are central to this story: linearization, aka single scattering, aka the Born approximation, and geometric optics, aka high frequency asymptotics, aka ray theory. Applied to ``locally layered'' modeling, these lead to the convolutional model, basic time processing of seismic reflection surveys (deconvolution, statics, NMO, stack), and NMO based velocity analysis. Applied to models with mild lateral variation, it leads to poststack imaging. NMO based velocity analysis provides a starting point for prestack imaging (the topic of the second part of the course) and migration velocity analysis, and serves as an interesting and accessible test bed for inversion algorithms.  

\item {\bf Part 2: Linear Inversion.} Prestack depth imaging via diffraction sums (Kirchhoff migration), Generalized Radon Transform inversion for acoustics and elasticity, Reverse Time Migration, Least Squares Migration (iterative and true amplitude migration). This time we start with the theory, as the computational examples will use Reverse Time and True Amplitude migration, but the justification for these is essentially the theory of the Generalized Radon Transform. We end by applying these techniques to the 2D marine survey of Part 1.

\item {\bf Part 3: Nonlinear Inversion.} Theory of reflection and transmission tomography, relation to migration velocity analysis in extended time and image domains. Inversion beyond single scattering - full waveform inversion.  Application to the marine survey of segment 1. 

\end{itemize}

\bibliographystyle{seg}
\bibliography{../bib/masterref}